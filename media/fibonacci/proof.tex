\documentclass{article}

\usepackage{amsthm}
\usepackage{amsmath}

\title{\huge{\textbf{A Fibonacci Proof}}}
\author{\large{\textit{Marcel Goh}}}
\date{23 June 2018}

\newtheorem*{theorem}{Theorem}
\DeclareMathOperator{\fib}{fib}

\begin{document}

\maketitle

\begin{theorem}
    For all $n\in\mathcal{N}$, the $n$-th number in the Fibonacci sequence $(1,1,2,3,5\ldots\fib(n))$ is the closest integer to $\phi^n/\sqrt{5}$ where $\phi = (1+\sqrt{5})/2$
\end{theorem}
\begin{proof}
    For natural numbers $n$, the Fibonacci sequence is defined as follows:
    $$\fib(n)=\begin{cases}
                  1 & n=1,2\\
                  \fib(n-1)+\fib(n-2) & n\geq3\\
              \end{cases}$$
    The proof consists of two parts. First we will show that $\fib(n)$ differs from $\phi^n/\sqrt{5}$ by $\psi^n/\sqrt{5}$ where $\psi = (1-\sqrt{5})/2$. Then we will prove that this difference is strictly less than 0.5 for all natural numbers $n$. Observe the following two cases:\\ \\
    For n=1:
    $$\frac{(\frac{1+\sqrt{5}}{2})^1}{\sqrt{5}} - \frac{(\frac{1-\sqrt{5}}{2})^1}{\sqrt{5}} = \frac{(1+\sqrt{5})-(1-\sqrt{5})}{2\sqrt{5}} = \frac{\sqrt{5} + \sqrt{5}}{2\sqrt{5}} = 1$$
    For n=2:
    $$\frac{(\frac{1+\sqrt{5}}{2})^2}{\sqrt{5}} - \frac{(\frac{1-\sqrt{5}}{2})^2}{\sqrt{5}} = \frac{(1+2\sqrt{5}+5)-(1-2\sqrt{5}+5)}{4\sqrt{5}} = \frac{2\sqrt{5} + 2\sqrt{5}}{4\sqrt{5}} = 1$$\\
    So if we set $n=3$, we know the following two identities hold:
    $$\fib(n-1) = \frac{\phi^{n-1}-\psi^{n-1}}{\sqrt{5}} \quad \text{and} \quad \fib(n-2) = \frac{\phi^{n-2}-\psi^{n-2}}{\sqrt{5}}$$
    We want to show that for all $n\in\mathcal{N},n\geq3$, $\fib(n) = fib(n-1)+fib(n-2)$.
    \begin{align} \label{eq:1}
        \frac{\phi^n-\psi^n}{\sqrt{5}} &= \frac{\phi^{n-1}-\psi^{n-1}}{\sqrt{5}} + \frac{\phi^{n-2}-\psi^{n-2}}{\sqrt{5}} \nonumber \\
        \phi^2\cdot\phi^{n-2}-\psi^2\cdot\psi^{n-2} &= \phi\cdot\phi^{n-2}-\psi\cdot\psi^{n-2} + \phi^{n-2} - \psi^{n-2} \nonumber \\
        \phi^2\cdot\phi^{n-2}-\psi^2\cdot\psi^{n-2} &= (\phi+1)\phi^{n-2}-(\psi+1)\psi^{n-2}
    \end{align}
    To prove that identity \eqref{eq:1} holds, we must equate the coefficients and prove that $\phi^2 = \phi+1$ and $\psi^2 = \psi + 1$:\\\\
    For $\phi$:
    \begin{align*}
        (\frac{1+\sqrt{5}}{2})^2 &= \frac{1+\sqrt{5}}{2} + 1\\
        \frac{1+2\sqrt{5}+5}{4} &= \frac{1+\sqrt{5}}{2} + \frac{2}{2}\\
        \frac{6+2\sqrt{5}}{4} &= \frac{1+\sqrt{5}+2}{2}\\
        \frac{3+\sqrt{5}}{2} &= \frac{3+\sqrt{5}}{2}
    \end{align*}
    For $\psi$:
    \begin{align*}
        (\frac{1-\sqrt{5}}{2})^2 &= \frac{1-\sqrt{5}}{2} + 1\\
        \frac{1-2\sqrt{5}+5}{4} &= \frac{1-\sqrt{5}}{2} + \frac{2}{2}\\
        \frac{6-2\sqrt{5}}{4} &= \frac{1-\sqrt{5}+2}{2}\\
        \frac{3-\sqrt{5}}{2} &= \frac{3-\sqrt{5}}{2}
    \end{align*}
    So identity \eqref{eq:1} holds and we have proven that $\fib(n)$ differs from $\phi^n/\sqrt{5}$ by $\psi^n/\sqrt{5}$.\\ \\
    Now we will show that this difference is less than $0.5$. Concretely, this means that for all $n\in\mathcal{N}$, $\left|\phi^n/\sqrt{5}\right| < 0.5$. $\psi \approx -0.618$ so if $n=1$, we have $\psi/\sqrt{5} \approx -0.276\ldots$, the absolute value of which is less than $0.5$. And because $\left|\psi\right| < 1$, $\left|\psi^{n+1}\right| < \left|\psi^n\right|$ and for all $n\in\mathcal{N}$, $\left|\psi^n/\sqrt{5}\right| < 0.5$.\\ \\
    Therefore, for all natural numbers $n$, $\fib(n)$ differs from $\phi^n/\sqrt{5}$ by less than 0.5 and the theorem is proved.
\end{proof}

\end{document}
