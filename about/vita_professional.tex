% Marcel Goh CV

% From cwebmac: Date and time
\def\today{\ifcase\month\or
  January\or February\or March\or April\or May\or June\or
  July\or August\or September\or October\or November\or December\fi
  \space\number\day, \number\year}
\newcount\twodigits
\def\hours{\twodigits=\time \divide\twodigits by 60 \printtwodigits
  \multiply\twodigits by-60 \advance\twodigits by\time :\printtwodigits}
\def\gobbleone1{}
\def\printtwodigits{\advance\twodigits100
  \expandafter\gobbleone\number\twodigits
  \advance\twodigits-100 }

\font\ninesc=cmcsc9
\font\eightrm=cmr8
\font\elevensc=cmcsc10 at 11pt  % For section headers
\font\bigbold=cmb10 at 12pt  % Name
\font\tenb=cmb10
\font\mc=cmr9

\let\tenbf=\tenb

% For publication lists (adapted from Knuth's algorithms)
\newdimen\itemindent
\newif\ifitempar \itempartrue
\def\itemindentset#1{\setbox0\hbox{{\bf #1.\kern.25em}}\itemindent=\wd0\relax}
\def\pubbegin #1{\itemindentset{#11}} % when all have 1 digit
\def\ppubbegin #1{\itemindentset{#111}} % when 10 or more publications
\def\pubitem#1.{\ifitempar\smallskip\noindent\else\itempartrue
  \hskip-\parindent\fi
  \hbox to\itemindent{\bf\hfil #1.\kern.25em}%
  \hangindent=\itemindent\hangafter=1\ignorespaces}

\def\up#1{\leavevmode \raise.16ex\hbox{#1}}
\def\sectheader#1{{\bigskip\elevensc #1}\smallskip\hrule\medskip}
\def\leftright#1#2{\hbox to\hsize{{#1}\hfill{#2}}\par}
\def\starleftright#1#2{\hbox to\hsize{\llap{*}{#1}\hfil{#2}}\par}
\def\CEE{{\mc C}}
\def\UNIX{{\mc UNIX}}
\def\thing{\item{$\bullet$}}

\parindent=0pt  % No indentation

\newcount\papercount
\papercount=1
\def\papitem{\pubitem P\the\papercount. \global\advance\papercount by -1}
\newcount\submittedcount
\submittedcount=1
\def\subitem{\pubitem S\the\submittedcount. \global\advance\submittedcount by -1}
\newcount\reportcount
\reportcount=1
\def\repitem{\pubitem R\the\reportcount. \global\advance\reportcount by -1}

\def\datethis{\eightrm \today}

% =========== START ============ %

% \footline{\phantom{\datethis}\hss\folio\hss\datethis}  % \phantom for centering

\leftright{\bigbold Marcel Kieren Goh
\ifx\Umathchar\undefined\else% Chinese characters do not appear when using pdfTeX
\font\chinese="AR PL UKai HK" at 12pt%
\font\twelverm=cmr12%
\twelverm\up({\chinese 吳麒仁}\up)%
\fi}{\datethis}
\smallskip

\sectheader{Personal and Contact Information}

\leftright{+1\thinspace\up(825\up)\thinspace 440-0681}{3660 Rue Hutchison, Apt.\ 4}
\leftright{Email: marcel\_goh@yahoo.ca}{Montr\'eal, QC H2X 2H3, Canada}
\leftright{Webpage: marcelgoh.ca}{}

\sectheader{Education}

\leftright{\bf M.Sc.\ Mathematics}{\bf May 2021 -- present}
\leftright{\sl McGill University}{\sl Montr\'eal, QC, Canada}
\smallskip
Supervisor: Luc Devroye. CGPA: 4.00/4.
Thesis: ``Structural properties of conditional Galton--Watson trees.''
\medbreak
\leftright{\bf B.Sc.\ Joint Honours Mathematics and Computer Science}{\bf September 2017 -- April 2021}
\leftright{\sl McGill University}{\sl Montr\'eal, QC, Canada}
\smallskip
Minor: Linguistics. CGPA: 3.84/4.
\medbreak
\leftright{\bf Exchange semester}{\bf February -- June 2019}
\leftright{\sl Faculty of Mathematics and Physics, Charles University}{\sl Prague, Czech Republic}
\smallskip
Grade: 1 on all courses (highest score attainable).

\sectheader{Awards and Funding}

\leftright{\bf NSERC Canada Graduate Scholarship -- Master's \rm (\$17,500)}{\bf May 2021 -- April 2022}
\smallskip
\leftright{\bf NSERC Undergraduate Student Research Award \rm (\$7,125)}{\bf May -- August 2020}
\smallskip
\leftright{\bf Governor General's Academic Medal -- Bronze}{\bf June 2015}

\sectheader{Graduate-level Coursework}

Higher algebra (categories, groups, modules), functional analysis, quadratic and modular forms, combinatorics,
topology (including some algebraic topology), probabilistic analysis of algorithms.

\sectheader{Research Interests}

Analysis of random discrete structures, analysis and design of algorithms, enumerative combinatorics,
additive combinatorics.

\sectheader{Papers}

\begingroup\frenchspacing
\papercount=4
\pubbegin P
\papitem (with Luc Devroye and Rosie Y. Zhao)
On the peel number and the leaf-height of a Galton--Watson tree.
To appear in {\sl Combinatorics, Probability and Computing}, 20~pp. [arXiv:2106.14389]
\papitem (with Anna M. Brandenberger, Luc Devroye, and Rosie Y. Zhao)
Leaf multiplicity in a Bienaym\'e--Galton--Watson tree. {\sl Discrete Mathematics
and Theoretical Computer Science} {\bf 24},1 (2022), \#7, 16~pp. [arXiv:2105.12046]
\papitem (with Anna M. Brandenberger and Luc Devroye)
Root estimation in Galton--Watson trees. To appear in {\sl Random Structures and Algorithms}, 24~pp.
[arXiv:2007.05681]
\papitem (with Rosie Y. Zhao)
Arithmetic subsequences in a random ordering of an additive set.
{\sl Integers: Electronic Journal of Combinatorial Number Theory} {\bf 21} (2021), \#A89, 19~pp.
[arXiv:2012.12339]

\sectheader{Submitted Papers}

\begingroup\parindent=10pt
\thing (with Jad Hamdan and Jonah Saks)
The lattice of arithmetic progressions. {\sl arXiv preprint 2106.05949}, 15~pp.
\endgroup
\endgroup%end the frenchspacing

\goodbreak
\sectheader{Reports}

\reportcount=4
\pubbegin R
\repitem Finding regularity in Tlingit verb prefixes. Semester project report, McGill University
(Montr\'eal, Qu\'ebec, April 2021), 7 pp.
\repitem Grid-building algorithms on manifolds. Summer research report, McGill University (Montr\'eal, Qu\'ebec,
August 2020), 10 pp.
\repitem Typechecking proof scripts: making interactive proof assistants robust.
Honours project report, McGill University (Montr\'eal, Qu\'ebec, December 2019), 10 pp.
\repitem The OPythn programming language. Software project report, Charles University (Prague,
Czech Republic, June 2019), 10 pp.

\sectheader{Selected Software Projects}

\leftright{\bf Tlingit verb parsing}{\bf February -- April 2021}
\smallskip
A rule-based program that finds regularity in Tlingit verbs, written in Python.

\noindent[{\tt https://github.com/marcelgoh/tlingit-verb-parsing}]
\medbreak

\leftright{\bf Sorting algorithms on manifolds}{\bf May -- August 2020}
\smallskip
Research project under the supervision of Michael Lipnowski.
Studied algorithms that investigate the topology of group actions on locally symmetric spaces.
Work involved writing of code in OCaml and PostScript that generates illustrations of certain quotient
spaces, in various models of hyperbolic geometry.

\noindent[{\tt https://github.com/marcelgoh/manifold-sorting}]
\medbreak

\leftright{\bf Bytecode compiler and interpreter}{\bf February -- June 2019}
\smallskip
Under the supervision of Adam Dingle, wrote a bytecode compiler and virtual machine for a subset of Python.
OPythn includes support for lists, dictionaries, named and anonymous functions, objects, and classes.
It is implemented in OCaml using Ocamllex for lexing and Menhir for parsing.

\noindent[{\tt https://github.com/marcelgoh/opythn}]
\medbreak

\leftright{\bf Alonzo $\lambda$-calculus interpreter}{\bf May -- June 2019}
\smallskip
Wrote an interpreter for the untyped $\lambda$-calculus that performs $\alpha$-conversion
and $\beta$-reduction. Includes an interactive environment that allows users to bind names to
terms. Implemented in Haskell.

\noindent[{\tt https://github.com/marcelgoh/alonzo}]
\medbreak

\leftright{\bf Escape-time fractals}{\bf May 2019}
\smallskip
Made an interactive graphical program to generate and explore fractals. Implemented using the SDL2
graphics library and written in \CEE. (HackPrague 2019)

\noindent[{\tt https://github.com/mattonicorp/mattoni}]
\medbreak

\leftright{\bf Nim}{\bf January 2019}
\smallskip
Implemented the game of Nim graphically using Processing. Worked on game logic and created the soundtrack
using an $8$-bit sequencer. (ConUHacks IV)

\noindent[{\tt https://github.com/conudihedral4/nim}]
\medbreak

\leftright{\bf CourseTalk}{\bf November 2018}
\smallskip
Created a web chat app using Node.js and React.js. (HackPrinceton Fall 2018)

\noindent[{\tt https://github.com/marcelgoh/hackprinceton-2018}]
\medbreak

\sectheader{Research Experience}

\leftright{\bf Probabilistic analysis of branching processes}{\bf May 2020 -- present}
\leftright{\sl Research group}{\sl McGill University}
\smallskip
Ongoing research on branching processes, headed by Luc Devroye. Studied estimation problems
on Galton-Watson trees. Attended informal seminars, discussing
various topics related to branching
processes and other topics in probability and combinatorics.
(Gave a presentation at six of these: one on root estimation in Galton-Watson
trees, two on generating functions and elementary analytic combinatorics, two on graph regularity, and
one concerning two of Erd\H os's proofs on prime numbers.)
\medbreak

\leftright{\bf Interactive proofs}{\bf September -- December 2019}
\leftright{\sl Honours research project}{\sl McGill University}
\smallskip
Semester-long research project in the Computation and Logic Group, supervised
by Brigitte Pientka.
Proved a theorem in constructive logic concerning the formal verification
of the interactive proof assistant Harpoon and wrote OCaml code as part of
ongoing work on the functional programming language Beluga.
\medbreak

\sectheader{Work and Volunteer Experience {\rm($*$ indicates a paid position)}}

\leftright{\bf Mentor}{\bf January 2021 -- present}
\leftright{\bf Department of Mathematics and Statistics, McGill University}{\sl Montr\'eal, QC, Canada}
\smallskip
Meet with two undergraduate students as part of the Directed Reading Program
on a weekly basis to give them an introduction to research-level
mathematics in a casual setting. Focused on topics in extremal combinatorics related to the increasing
triples problem.
\medbreak

\starleftright{\bf Teaching assistant}{\bf September -- December 2021}
\leftright{\sl School of Computer Science, McGill University}{\sl Montr\'eal, QC, Canada}
\smallskip
Teaching assistant for COMP 690, a graduate-level course on the probabilistic analysis of algorithms.
Hold office hours twice a week and responsible for grading of assignments.
\medbreak

\leftright{\bf First responder}{\bf October 2017 -- September 2021}
\leftright{\sl McGill Student Emergency Response Team}{\sl Montr\'eal, QC, Canada}
\smallskip
On call on a weekly basis to provide emergency medical care at campus residences overnight as well as at
university events such as frosh, sports games, and formals. Attend team training sessions twice a month to keep
first-aid skills up-to-date. Most recent first responder certification: September 2020.
\medbreak

\starleftright{\bf Grader}{\bf September 2019 -- April 2021}
\leftright{\sl Department of Mathematics and Statistics, McGill University}{\sl Montr\'eal, QC, Canada}
\smallskip
Grading of assignments in the following courses:
\begingroup\parindent=10pt
\smallskip
\thing Winter 2021: MATH 457 Honours Algebra 4
\smallskip
\thing Fall 2020: MATH 323 Probability, MATH 456 Honours Algebra 3
\smallskip
\thing Winter 2020: MATH 240 Discrete Structures
\smallskip
\thing Fall 2019: MATH 235 Algebra 1
\endgroup
\medbreak

\starleftright{\bf Helpdesk tutor}{\bf September 2018 -- April 2021}
\leftright{\sl Computer Science Undergraduate Society Helpdesk}{\sl Montr\'eal, QC, Canada}
\smallskip
Hold twice-weekly office hours to tutor students in a variety of undergraduate courses.
Topics covered include elementary data structures and algorithms, command-line scripting,
and functional programming. Recipient of the Tomlinson Engagement Award for Mentoring.
\medbreak

\leftright{\bf Vice President, Academic}{\bf May 2019 -- January 2020}
\leftright{\sl Society of Undergraduate Mathematics Students}{\sl Montr\'eal, QC, Canada}
\smallskip
Oversaw academic affairs within SUMS council and acted as liaison between the undergraduate community and
mathematics faculty. Duties included representing the student body at department meetings,
organising midterm and final review sessions,
and helping students with
academic concerns.
\medbreak

\starleftright{\bf Painter}{\bf May -- August 2018}
\leftright{\sl Bakir Contracting Corp.}{\sl Edmonton, AB, Canada}
\smallskip
Exterior painting (siding, decks, fences, trim, etc.) for residential clients.
\medbreak

\starleftright{\bf Infantryman}{\bf August 2015 -- August 2017}
\leftright{\sl Singapore Armed Forces}{\sl Singapore}
\smallskip
Held appointment of machine-gun team commander in the 3rd Battalion, Singapore Infantry Regiment.
Led a six-person team consisting of a medic, signaller, sensor, and two-machine gunners within a rifle platoon.
\medbreak

\starleftright{\bf Tutor}{\bf September 2014 -- June 2015}
\leftright{\sl \'Ecole Secondaire Beaumont Composite High School}{\sl Beaumont, AB, Canada}
\smallskip
Tutored various students in grades 4 through 11 in chemistry, physics, math, and French.
\medbreak

\starleftright{\bf Summer camp counsellor}{\bf June -- August 2014}
\leftright{\sl YoWoChAs Outdoor Education Centre}{\sl Fallis, AB, Canada}
\smallskip
Led children aged 4--15 through various activities (e.g. archery, canoeing, zipline) at a sleepaway camp.
\medbreak

\sectheader{Skills}

{\bf Programming Languages}\par
\CEE, OCaml, Python, Java, PostScript, Haskell, Scheme Lisp,  Standard ML, {\tt CWEB}, {\tt MIXAL}, MIPS Assembly.
\medbreak

{\bf Technologies}\par
\UNIX, Vim, \TeX, Git.
\medbreak

{\bf Languages}\par
Fluent: English, French. Proficient: Mandarin, Italian.
\medbreak

\sectheader{Other}

\parindent=10pt
\thing Recipient of 0x\$3.40 in Knuth reward cheques.
\smallskip
\thing Contributed sequences A335562, A338550, A338993, A339941, A339942, A341822, and A347580
to the On-line Encyclopedia of Integer Sequences.
\medbreak

\filbreak

\bye

