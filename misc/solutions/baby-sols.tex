\input fontmac
\input mathmac

\font\bigboldsl=cmbxsl12
\font\eightsl=cmsl8

\def\d{\,d}

\maketitlenodate{Answers to Selected Exercises in {\bigboldsl Principles of Mathematical Analysis}\footnote{$^*$}{\frenchspacing \eightpt Walter Rudin. 1986. {\eightsl Principles of Mathematical Analysis, 3rd ed}. McGraw-Hill, Inc., USA.}}{Solutions by}{Marcel K.\ Goh}

\section CHAPTER 2. BASIC TOPOLOGY

\proclaim 1. Prove that the empty set is a subset of every set.

\proof Let $S$ be an arbitrary set. Then every element of $\emptyset$ is an element of $S$. So $\emptyset \subseteq S$.\slug

\proclaim 2. A complex number $z$ is said to be algebraic if there are integers $a_0,\ldots,a_n$, not all zero, such that
$$a_0z^n + a_1z^{n-1} + \cdots + a_{n-1}z + a_n = 0.$$
Prove that the set of all algebraic numbers is countable. {\it Hint:} For every positive integer $N$, there are only finitely many equations with
$$n + |a_0| + |a_1| + \cdots + |a_n| = N$$

\proof Let $A$ denote the set of all algebraic numbers and partition $A$ as follows: For each $z\in A$, calculate the positive integer $N$ that corresponds to its equation and place it in a set $E_N$. So
$$A = \bigcup_{N\in \N} E_N,$$
where each $E_N$ is finite. Then apply the corollary of Theorem 2.12 to find that $A$ is countable.\slug

\proclaim 3. Prove that there exist real numbers which are not algebraic.

\proof Let $A$ denote the set of all algebraic numbers and suppose, towards a contradiction, that all real numbers are algebraic. Then ${\bf R}\subseteq A$. But the set of real numbers is uncountable and we know from Problem 2 that $A$ is countable. The contradiction completes the proof.\slug

\proclaim 4. Is the set of all irrational real numbers countable?

The answer is no.

\proof If $\R \setminus \Q$ is countable, then $\R = \Q \cup (\R\setminus\Q)$, is countable, which we know to be false.\slug

\proclaim 5. Construct a bounded set of real numbers with exactly three limit points.

$\{{1\over n} : n\in \N\} \cup\{{1\over n} + 2: n\in \N\} \cup \{{1\over n} + 4: n\in \N\}$ has limit points 0, 2, and 4. It is bounded above by 6 and below by 0.

\section CHAPTER 11. THE LEBESGUE THEORY

\proclaim 1. If $f\geq 0$ and $\int_E f\d\mu = 0$, prove that $f(x) = 0$ almost everywhere on $E$. {\it Hint:} Let $E_n$ be the subset of $E$ on which $f(x)> 1/n$. Write $A = \bigcup_{n=1}^\infty E_n$. Then $\mu(A) = 0$ if and only if $\mu(E_n) = 0$ for every $n$.

\proof First we claim that $\mu(E_n) = 0$ for every $n$. Suppose, towards a contradiction, that $\mu(E_n) > 0$ for some $n$. Then since $f(x) > 1/n$ for all $x\in E_n$, we find that
$$\int_E f\d\mu \geq \int_{E_n} f \d\mu > {\mu(E_n)\over n} > 0,$$
contradicting our hypothesis. Now since $\mu(E_n) = 0$ for all $n$, $\mu(A) \leq \sum_{n=1}^\infty \mu(E_n) = 0$. So $\mu(A) = 0$. Now we write
$$\int_E f\d\mu = \int_{E\setminus A} f\d\mu + \int_{A} f\d\mu,$$
and $f$ is equal to 0 on $E\setminus A$ by our construction of $A$.\slug

\bye
