% Solutions to exercises in Modern Algebra by
% B. L. van der Waerden, translated by Fred Blum (1949)

\input fontmac
\input mathmac

\font\bigboldsl=cmbxsl12
\font\eightsl=cmsl8

\maketitlenodate{Answers to Selected Exercises in {\bigboldsl Modern Algebra}\footnote{$^*$}{\eightpt B.\ L.\ van der Waerden, {\eightsl Modern Algebra}, translated by Fred Blum, New York: Ungar, 1949.}}{Solutions by}{Marcel K.\ Goh}

\section CHAPTER I: NUMBERS AND SETS

\vskip -\medskipamount

\section 2. Mappings. Cardinality

\proclaim 1. For an arbitrary set $A$, prove that $A\sim A$.

\proof For each element $a\in A$, let $\phi(a) = a$. It is easy to see that this is a one-to-one correspondence.\slug

\proclaim 2. Given sets $A$ and $B$, prove that $A\sim B$ implies $B\sim A$.

\proof Since $A\sim B$, there exists a one-to-one correspondence $\phi$ from $A$ onto $B$. Then $\phi^{-1}$ is a one-to-one correspondence from $B$ onto $A$.\slug

\proclaim 3. For sets $A$, $B$, and $C$, prove that if $A\sim B$ and $B\sim C$, then $A\sim C$.

\proof We have the existence of biunique mappings $\phi : A \rightarrow B$ and $\psi : B\rightarrow C$. Then $\psi\phi$ is a one-to-one correspondence from $A$ to $C$ (with $\psi^{-1}\phi^{-1}$ as its inverse).\slug

\section 3. The Number Sequence

\proclaim 1. Let a property $E$ be true, first for $n=3$, and second for $n+1$ whenever it is true for $n\geq 3$. Prove that $E$ is true for all numbers $\geq 3$.

\proof For a number $k$, let $F$ denote the property ``$E$ is true for $k+2$''. Then $E$ is true for all numbers $n\geq 3$ if and only if $F$ is true for all natural numbers $k$. We find that $F$ is true for $k=1$, since $E$ is true for $n = 3$. Then from the second statement about the property $E$, we can derive that also $F$ is true for $k+1$ whenever it is true for $k\geq 1$. So by the principle of complete induction, we have $F$ true for all natural numbers.\slug

\proclaim 3. The same as Ex.\ 1 with the number 3 replaced by 0.

\proof For a natural number $k$, let $F$ denote the property ``$E$ is true for $k-1$''. Then proceed as in the solution to Exercise 1.\slug

\section CHAPTER II: GROUPS

\vskip -\medskipamount

\section 6. The Group Concept

\proclaim 1. The Euclidean motions of space combined with reflections (i.e.\ those transformations that preserve all distances between pairs of points) form an infinite non-abelian group.

\proof We work in 2-dimensional space, but most of the work generalises to higher dimensions. Any Euclidean motion $M$ can be described as one of the following:
\medskip
\item{a)} a translation $t_{PQ}$ that takes a point $P$ to a point $Q$;
\smallskip
\item{b)} a rotation $s_P(\theta)$ of $\theta$ radians about the point $P$;
\smallskip
\item{c)} a reflection $r_H$ where $H$ is a hyperplane (line).
\medskip

(Note that the same motion may be described in multiple ways. For example, if $PQ$ and $ST$ are parallel line-segments with the same length, then $t_{PQ} = t_{ST}$.) The motion $t_{PP}$ that fixes every point satisfies the requirements of an identity element. Associativity follows from the fact that, for any point $X$ in Euclidean space, multiplication equates to composing motions. Thus $(M_1M_2)M_3(X) = M_1\big(M_2(M_3(X))\big) = M_1(M_2M_3)(X)$; since the two motions act identically on every point in the space, they are the same transformation. To see that every transformation has an inverse, we need only note that $(t_{PQ})^{-1} = t_{QP}$ for all points $P$ and $Q$; $(s_P(\theta))^{-1} = s_P(-\theta)$ for all points $P$ and choices of $\theta$; and $(r_H)^{-1} = r_H$ for every hyperplane $H$. The group is not abelian because rotations do not commute with reflections in general.\slug

\proclaim 2. Prove that the elements $e,a$ form a group (abelian) if the group operation is defined by
$$ee = e, \quad ea = a, \quad ae = a,\quad aa = e.$$

\proof By inspection, we see that $e$ is the identity element, and both elements are their own inverses. Associativity may be checked by hand, examining all eight possible triples. And finally, the group is abelian because $ea = ae = a$.\slug

\proclaim 3. Construct a multiplication table for the group of all permutations on three digits.

\solution For brevity, cycle notation is employed:
$$\vbox{\offinterlineskip
    \halign{ 
        \hfil $#$ \hfil & \hfil\vrule# \quad\hfil &
        \hfil $#$ \hfil & \hfil $#$ \hfil & \hfil $#$ \hfil &
        \hfil $#$ \hfil & \hfil $#$ \hfil & \hfil $#$ \hfil \cr
        \cdot && () & (1\,2) & (1\,3) & (2\,3) & (1\,2\,3) & (1\,3\,2) \cr
        & height2pt &&&&&\cr
        \noalign{\hrule}
        & height2pt &&&&&\cr
        () && () & (1\,2) & (1\,3) & (2\,3) & (1\,2\,3) & (1\,3\,2) \cr
        & height2pt &&&&&\cr
        (1\,2) && (1\,2) & () & (1\,3\,2) & (1\,2\,3) & (2\,3) & (1\,3)\cr
        & height2pt &&&&&\cr
        (1\,3) && (1\,3) & (1\,2\,3) & () & (1\,3\,2) & (1\,2) & (2\,3)\cr
        & height2pt &&&&&\cr
        (2\,3) && (2\,3) & (1\,3\,2) & (1\,2\,3) & () & (1\,3) & (1\,2)\cr
        & height2pt &&&&&\cr
        (1\,2\,3) && (1\,2\,3) & (1\,3) & (2\,3) & (1\,2) & (1\,3\,2) & ()\cr
        & height2pt &&&&&\cr
        (1\,3\,2) && (1\,3\,2) & (2\,3) & (1\,2) & (1\,3) & () & (1\,2\,3)\cr
    }
}\noskipslug
$$

\end
