\input fontmac
\input mathmac

\font\bigboldsl=cmbxsl12
\font\eightsl=cmsl8

\def\d{\,d}

\maketitlenodate{Answers to Selected Exercises in {\bigboldsl Real and Complex Analysis}\footnote{$^*$}{\frenchspacing \eightpt Walter Rudin. 1966. {\eightsl Real and complex analysis}. McGraw-Hill, Inc., USA.}}{Solutions by}{Marcel K.\ Goh}

\section CHAPTER 1. ABSTRACT INTEGRATION

\proclaim 2. Put $f_n = \chi_E$ if $n$ is odd, $f_n = 1 - \chi_E$ if $n$ is even. What is the relevance of this example to Fatou's lemma?

This gives an example of strict inequality arising. Let $X$ be a measure space such that $\mu(X) = 1$ and let $E\subseteq X$ be such that $\mu(E) = 2/3$. Then
$$\int_X \liminf_{n\rightarrow\infty} f_n \d\mu = \int_X 0 \d\mu = 0,$$
while on the other hand
$$\liminf_{n\rightarrow\infty} \int_X f_n\d\mu = \int_X 1 - \chi_E \d\mu = \int_X 1 \d\mu - \int_X \chi_E \d\mu = \mu(X) - \mu(E) = {1\over 3},$$
and we have $0<1/3$.

\proclaim 3. Suppose $f_n : X\rightarrow[0,\infty]$ is measurable for $n=1,2,3,\ldots$, $f_1\geq f_2\geq f_3\geq\cdots\geq 0$, $f_n(x)\rightarrow f(x)$ as $n\rightarrow\infty$ for every $x\in X$, and $f_1\in L^1(\mu)$. Prove that then
$$\lim_{n\rightarrow\infty} \int_X f_n \d\mu = \int_X f\d\mu$$
and show that this conclusion does {\it not} follow if the condition ``$f_1\in L^1(\mu)$'' is omitted.

\proof Consider the sequence of functions $(f_1 - f_n)$. This sequence is nonnegative, nondecreasing and for any $x\in X$, $\lim_{n\rightarrow\infty} (f_1 - f_n)(x) = f_1(x) - f(x)$. So by the monotone convergence theorem,
$$= \lim_{n\rightarrow \infty} \int_X f_1 - f_n \d\mu = \int_X f_1 - f\d\mu.$$
Since the integrals of $f_n$ are finite, we can apply Theorem 1.27 to get
$$\int_X f_1\d\mu - \lim_{n\rightarrow \infty}\int_X f_n \d\mu = \int_X f_1\d\mu -  \int_X f \d\mu,$$
which gives us what we want after subtracting $\int_X f_1 \d\mu$ (which is finite) and multiplying by $-1$.\slug

The condition $f_1\in L^1(\mu)$ is necessary. Let $I = [0,1]$ and $f_n = {1\over nx^2}$, which converges to the constant function 0. Then $\lim_{n\rightarrow \infty} \int_I f_n \d\mu = +\infty$ but $\int_I 0\d\mu = 0$.

\proclaim 4. Prove that if $f$ is a real function on a measurable space $X$ such that $\{x:f(x)\geq r\}$ is measurable for every rational $r$, then $f$ is measurable.

\proof We immediately infer that for rational $r$ and $s$, $f^{-1}\big([r,s)\big)$ is measurable, since
$$f^{-1}\big([r,s)\big) = \{ x : f(x)\geq r\} \cap \{ x : f(x) \geq s\}^c.$$
Next, let $I\subseteq \R$ be an open interval. Setting $J = I\cap \Q$, we can show that $f^{-1}(I)$ is measurable, since
$$f^{-1}(I) = f^{-1}\bigg(\bigcup_{r\in J}\bigcup_{s\in J} [r,s)\bigg) = \bigcup_{r\in J}\bigcup_{s\in J} f^{-1}\big([r,s)\big)$$
is a countable union of measurable sets.

Now let $V\subseteq \R$ be open. By Lindel\"of's lemma, we can express $V$ as a countable union of open intervals $V=\bigcup_{n=1}^\infty I_n$. Then
$$f^{-1}(V) = f^{-1}\bigg(\bigcup_{n=1}^\infty I_n\bigg) = \bigcup_{n=1}^\infty f^{-1}(I_n)$$
is measurable.\slug

\proclaim 5. Prove that the set of points at which a sequence of measurable real-valued functions converges (to a finite limit) is measurable.

\proof Let $(f_n)$ be a sequence of real-valued measurable functions. Then $(f_n)$ converges at a point $x$ if and only if it is Cauchy at $x$, i.e.\ for any $\epsilon > 0$ there exists $N\in \N$ such that for all $m,n\geq N$, $|f_m(x) - f_n(x)| < \epsilon$. By the Archimedean property, we can replace $\epsilon$ with $1/k$ for some $k\in \N$. So the set of all points at which $(f_n)$ converges can be written thus:
$$\bigcap_{k=1}^\infty \bigcup_{N=1}^\infty \bigcap_{m=1}^\infty \bigcap_{n=1}^\infty \big\{x : |f_m(x) - f_n(x)| < 1/k\big\}$$
Set $g_{m,n} = f_m - f_n$; this is a measurable function. Hence the set
$$\big\{x : |f_m(x) - f_n(x)| < 1/k\big\} = \big\{x : |g_{m,n}(x)| < 1/k\big\} = {g_{m,n}}^{-1}\big((-1/k, 1/k)\big)$$
is measurable for every $m$, $n$, and $k$. We have thus proved that the set of all points at which $(f_n)$ converges is a countable union of measurable sets.\slug

\proclaim 6. Let $X$ be an uncountable set, let ${\cal M}$ be the collection of all sets $E\subseteq X$ such that either $E$ or $E^c$ is at most countable, and define $\mu(E) = 0$ in the first case, $\mu(E) = 1$ in the second. Prove that ${\cal M}$ is a $\sigma$-algebra in $X$ and that $\mu$ is a measure on ${\cal M}$. Describe the corresponding measurable functions and their integrals.

\noindent First we prove that ${\cal M}$ is a $\sigma$-algebra and that $\mu$ is a measure.

\proof Since $\emptyset$ is countable, $X\in {\cal M}$. Then for any set $E\in {\cal M}$, either $E$ or $E^c$ is at most countable; in any case $E^c\in {\cal M}$. Lastly, let $\{E_n\}$ be a collection of sets in ${\cal M}$. If every set $E_n$ is at most countable, then $\bigcup_{n=1}^\infty E_n$ is countable and thus in ${\cal M}$. Otherwise, there is some $k$ for which $E_k$ is uncountable. Then $E_k^c\in {\cal M}$ is at most countable and $\left(\bigcup_{n=1}^\infty E_n\right)^c = \bigcap_{n=1}^\infty E_n^c \subseteq E_k^c$ is countable, so $\bigcup_{n=1}^\infty E_n$ is in ${\cal M}$. This shows that ${\cal M}$ is a $\sigma$-algebra.

It is clear that the range of $\mu$ is in $[0,\infty]$. To show that $\mu$ is countably additive, let $\{E_n\}$ be a disjoint collection of sets in ${\cal M}$. If every $E_n$ is at most countable, then $\bigcup_{n=1}^\infty E_n$ is at most countable and $\mu(\bigcup_{n=1}^\infty E_n) = 0 = \sum_{n=1}^\infty \mu(E_n)$. Otherwise, there exists $E_k$ such that $E_k^c$ is countable and thus we have $\mu(E_k) = 1$ and $\mu(E_k^c) = 0$. Since the $E_n$ are pairwise disjoint, $E_n\subseteq E_k^c$ for all $n\neq k$. So $\mu(E_n) = 0$ for all $n\neq k$. So $\mu(\bigcup_{n=1}^\infty E_n) = 1 = \sum_{n=1}^\infty \mu(E_n)$ and $\mu$ is a measure.\slug

\noindent Now we claim that the measurable functions are those that are constant at all but countably many points.

\proof It suffices to prove this for a real-valued function. Suppose $f:X\rightarrow[-\infty,\infty]$ is measurable. For any $a\in \R$, let $E_a = f^{-1}\big([-\infty,a)\big)$. Note that for any $a$, either $E_a$ is countable or $E_a^c$ is countable. Note also that if $a\leq b$, then $E_a \subseteq E_b$. So let there is a constant $c$ such that
$$k = \sup\{a : E_a\ \hbox{is countable}\}.$$
This supremum is not $-\infty$, since if it were, then $E_a^c$ would be countable for all $a\in \R$, and since $\bigcap_{n=0}^\infty E_{-n} = \emptyset$, $X = \bigcap_{n=0}^\infty E_{-n}^c$ is countable, a contradiction. By a similar argument, we know that the supremum is not $\infty$. So $k\in \R$ and there exists a sequence $(a_n)$ all of whose terms are less than $k$ and whose limit is $k$. Hence $E_k = \bigcup_{n=1}^\infty E_{a_n}$ is countable. Now let $(b_n)$ be a sequence that converges to $k$ such that $b_n > k$ for all $n$. Then note that if $f(x) > k$, then $f(x) \in [b_n, \infty) = E_{b_n}^c$ for some $n$. So
$$\{x : f(x) \neq k\} = E_k \cup \{x : f(x) > k\} \subseteq E_k \cup \Big(\bigcup_{n=0}^\infty E_{b_n}^c\Big)$$
is countable and $f$ equals $k$ at all but countably many points.\slug

\noindent Lastly, we show that if $f$ is measurable and takes on the value $k$ at all but countably many points, then $\int_E f\d\mu = k$ for all uncountable $E\in {\cal M}$. (If $E$ is countable then $\int_E f\d\mu = 0$.)

\proof Let $E\in {\cal M}$ be uncountable; so $\mu(E) = 1$. Let $s$ be a simple measurable function such that $0\leq s\leq f$. Suppose that $s$ takes values $\alpha_i$ on the $n$ disjoint sets $A_i$ that cover $X$. We know that one of the $\alpha_i$, call it $\alpha_j$, is equal to a constant $k_s$ and that $A_i$ is at most countable for all $i \neq j$. So for all $i \neq j$, $A_i \cap E$ is countable and $A_j \cap E$ must be uncountable. Then we have
$$\int_E s\d\mu = \sum_{i=1}^n \alpha_i \mu(A_i\cap E) = \alpha_j\mu(A_j\cap E) = k_s\cdot 1 = k_s,$$
where $0\leq k_s \leq k$. So we have
$$\int_E f\d\mu = \sup \Big\{ \int_E s\d\mu : 0 \leq s\leq f\Big\} = \sup \{k_s : 0\leq s \leq f\} = k,$$
which is what we had to show.\slug

\proclaim 7. Does there exist an infinite $\sigma$-algebra which has only countably many members?

\noindent The answer is no.

\proof Let $X$ be a ground set and let ${\cal F}$ be a $\sigma$-algebra (over $X$) with infinitely many members. First we claim that we can always find a set $E\neq \emptyset$ such that the set $\{ F \cap E^c : F\in {\cal F}\}$ is infinite. If this did not hold, then take any $E\neq \emptyset$ in ${\cal F}$. Then by assumption, ${\cal S}_1 = \{ F \cap E^c : F\in {\cal F}\}$ is finite and because $E^c\in {\cal F}$, ${\cal S}_2 = \{ F \cap E : F\in {\cal F}\}$ is finite as well. Since any member of ${\cal F}$ can be expressed as a union of an element of ${\cal S}_1$ with an element of ${\cal S}_2$, the implies that ${\cal F}$ is finite, a contradiction.

Now we may use the claim to construct a countable sequence of pairwise disjoint elements of ${\cal F}$. Let $G_1$ be the set $E$ given by the claim. Now the infinite set ${\cal S}_1$ we constructed before is also a $\sigma$-algebra, so repeat the argument to get a set $G_2$, disjoint from $G_1$. Continuing in this manner, we obtain a sequence $G_1, G_2,\ldots$ where the $G_i$ are pairwise disjoint. Now we see that the map from the power set of the natural numbers to ${\cal F}$ given by
$$ A \mapsto \bigcup_{i\in A} G_i $$
is injective. So the uncountability of ${\cal F}$ follows from the uncountability of ${\cal P}({\bf N})$.\slug

\bye
