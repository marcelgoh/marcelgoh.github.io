\input fontmac
\input mathmac

\def\given{\,|\,}

\maketitle{The Probabilistic Method}{exercise solutions by}{Marcel K. Goh}{14 February 2022}

\bigskip

\section 1. The Basic Method

\proclaim Exercise 1.1. Prove that if there is a real $p$, with $0\le p\le 1$ such that
$${n\choose k}p^{k\choose 2} + {n\choose t}(1-p)^{t\choose 2} < 1,$$
then the Ramsey number $r(k,t)$ satisfies $r(k,t)>n$. Using this, show that
$$r(4,t) \ge \Omega\bigl(t^{3/2}/ (\ln t)^{3/2}\bigr).$$

\proof We follow the proof of Proposition~1.1.1 in the book. We consider a random graph on
$n$ vertices, where each edge is present with probability $p$. Let $K$ be the event that
there is a clique of size $k$ in the graph, and let $I$ be the event that there is an
independent set of size $t$ in the graph. By the union bound,
$$\pr\{ K \cup I\} \le \pr\{K\} + \pr\{I\} \le \sum_{|S| = k} p^{k\choose 2} + \sum_{|S|=t} (1-p)^{t\choose 2}
= {n\choose k}p^{k\choose 2} + {n\choose t}(1-p)^{t\choose 2} < 1.$$
This means that $\pr\{ \neg K \cap \neg I\} > 0$ and since the sample space is finite,
there exists a graph on $n$ vertices with
no clique of size $k$ and no independent set of size $t$ and therefore $r(k,t) > n$.

Next we show that $r(4,t) > \bigl(t/(e\ln t)\bigr)^{3/2}$ for large enough $t$. Note that
$${n\choose 4} p^{k\choose 2} + {n\choose t}(1-p)^{t\choose 2}\le
n^4p^6 + {e^tn^t\over t^t}(1-p)^{t^2/4},$$
by the inequalities
$$ {n^k\over k^k} \le {n\choose k} \le {e^k n^k\over k^k}.$$
Setting $n = t^{3/2} / (e\ln t)^{3/2}$, we have
$$\eqalign{
n^4p^6 + {e^tn^t\over t^t}(1-p)^{t^2/4} &=
\biggl({tp\over e\ln t}\biggr)^6 + {e^t t^{3t/2} \over t^t e^{3t/2} (\ln t)^{3t/2}} (1-p)^{t^2/4} \cr
&=\biggl({tp\over e\ln t}\biggr)^6 + {t^{t/2} \over e^{t/2} (\ln t)^{3t/2}} (1-p)^{t^2/4} \cr
&\le \biggl({tp\over e\ln t}\biggr)^6 + \biggl({t(1-p)^{t/2} \over e(\ln t)^3}\biggr)^{t/2} \cr
&\le \biggl({tp\over e\ln t}\biggr)^6 + \biggl({t \over e^{pt/2+1}(\ln t)^3}\biggr)^{t/2},\cr
}$$
where in the last line we used the inequality $1-p \le e^{-p}$.
Choosing $p = 2\ln t/t$, we simply need $t$ large enough such that
$$ \biggl({t\over e^{\ln t+1} (\ln t)^3}\biggr)^{t/2} = 
\biggl({1\over e(\ln t)^3}\biggr)^{t/2} < 1- \biggl({2\over e}\biggr)^6,$$
which can be done since the left-hand side goes to $0$.\slug

\proclaim Exercise 1.2. Suppose $n\ge 4$ and let $H$ be an $n$-uniform hypergraph with at most
$4^{n-1}/3^n$ edges. Prove that there is a colouring of the vertices of $H$ by $4$ colours so that in
every edge all $4$ colours are represented.

\proof Let each vertex of $H$ be independently given one of the four colours uniformly at random. (If
$H$ is infinite, it does not matter what colour we give to vertices that do not appear in any edge,
so it suffices to consider $H$ finite, which makes the sample space finite.)
Given some edge $e$ of $H$ with $n$ vertices, there are $4^n$ total ways that $e$ may be coloured,
and for each of the four colours, $3^n$ total ways that $e$ may be coloured using only the other three colours.
Let $K(e)$ denote the event that $e$ does not contain all four colours. By the inclusion-exclusion principle,
$$\pr\bigl\{K(e)\bigr\} = 4\cdot 3^n - 6\cdot 2^n + 4$$
Since $6\cdot 2^n \ge 96 > 4$, the probability that a given edge {\it does not} contain all four colours
is (much) less than $3^n/4^{n-1}$. By the union bound,
$$\pr\Bigl\{ \bigcup_{e\in E(H)} K(e)\Bigr\} \le \sum_{e\in E(H)} \pr\bigl\{K(e)\bigr\}
< {4^{n-1}\over 3^n} \cdot {3^n\over 4^{n-1}} = 1.$$
Since the sample space is finite this implies that there is some colouring of the vertices of $H$
in which every edge has all four colours.\slug

\section 4. The Second Moment

\proclaim Exercise 4.1. Let $X$ be a random variable taking integral nonnegative values, let $\ex\{X^2\}$
denote the expectation of its square, and let $\var\{X\}$ denote its variance. Prove that
$$\pr\{ X = 0\} \le {\var\{X\} \over \ex\{X^2\}}.$$

\proof Since $X$ is integer and nonnegative, we have $\pr\{X = 0\} = 1-\pr\{X\ge 1\}$ and since
$\var\{X\} = \ex\{X^2\} - \ex\{X\}^2$, to get our result it suffices to show that
$$\pr\{X\ge 1\} \ge {\ex\{X\}^2 \over \ex\{X^2\}}.$$
We start by noting that
$$\ex\{X\} = \sum_{k=0}^\infty k \pr\{X=k\}
= \pr\{X\ge 1\}\sum_{k=1}^\infty {k\pr\{X=k\}\over \pr\{X\ge 1\}}
= \pr\{X\ge 1\}\ex\{ X \given X\ge 1\}.$$
Since the function $x\mapsto x^2$ is convex, we have, by Jensen's inequality,
$$\ex\{X\}^2 = \pr\{X\ge 1\}^2 \ex\{X\given X\ge 1\}^2 \le \pr\{X\ge 1\}^2 \ex\{X^2 \given X\ge 1\}
= \pr\{X\ge 1\} \ex\{X^2\}.$$
Dividing both sides by $\ex\{X^2\}$ gives us what we want.\slug


\bye
