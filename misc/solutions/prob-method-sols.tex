\input fontmac
\input mathmac

\def\given{\,|\,}
\def\biggiven{\,\big|\,}

\maketitle{The Probabilistic Method}{exercise solutions by}{Marcel K. Goh}{15 February 2022}

\floattext5 \ninebf Note. \ninepoint
A star indicates a starred problem in the text; these are supposed to be harder.
I have decided to put my work here even when I haven't solved the problem yet, so that when I come back
later I can see what I've tried. I won't put the black square
\hbox{\kern1.5pt\vrule width2.5pt height6pt depth1.5pt\kern1.5pt}
until I think the proof
is actually done. Also I wrote $\log$ everywhere to mean $\log_e$
even though the text uses $\ln$.

\bigskip

\section 1. The Basic Method

\proclaim Exercise 1.1. Prove that if there is a real $p$, with $0\le p\le 1$ such that
$${n\choose k}p^{k\choose 2} + {n\choose t}(1-p)^{t\choose 2} < 1,$$
then the Ramsey number $r(k,t)$ satisfies $r(k,t)>n$. Using this, show that
$$r(4,t) \ge \Omega\bigl(t^{3/2}/ (\log t)^{3/2}\bigr).$$

\proof We follow the proof of Proposition~1.1.1 in the book. We consider a random graph on
$n$ vertices, where each edge is present with probability $p$. Let $K$ be the event that
there is a clique of size $k$ in the graph, and let $I$ be the event that there is an
independent set of size $t$ in the graph. By the union bound,
$$\pr\{ K \cup I\} \le \pr\{K\} + \pr\{I\} \le \sum_{|S| = k} p^{k\choose 2} + \sum_{|S|=t} (1-p)^{t\choose 2}
= {n\choose k}p^{k\choose 2} + {n\choose t}(1-p)^{t\choose 2} < 1.$$
This means that $\pr\{ \neg K \cap \neg I\} > 0$ and since the sample space is finite,
there exists a graph on $n$ vertices with
no clique of size $k$ and no independent set of size $t$ and therefore $r(k,t) > n$.

Next we show that $r(4,t) > \bigl(t/(e\log t)\bigr)^{3/2}$ for large enough $t$. Note that
$${n\choose 4} p^{k\choose 2} + {n\choose t}(1-p)^{t\choose 2}\le
n^4p^6 + {e^tn^t\over t^t}(1-p)^{t^2/4},$$
by the inequalities
$$ {n^k\over k^k} \le {n\choose k} \le {e^k n^k\over k^k}.$$
Setting $n = t^{3/2} / (e\log t)^{3/2}$, we have
$$\eqalign{
n^4p^6 + {e^tn^t\over t^t}(1-p)^{t^2/4} &=
\biggl({tp\over e\log t}\biggr)^6 + {e^t t^{3t/2} \over t^t e^{3t/2} (\log t)^{3t/2}} (1-p)^{t^2/4} \cr
&=\biggl({tp\over e\log t}\biggr)^6 + {t^{t/2} \over e^{t/2} (\log t)^{3t/2}} (1-p)^{t^2/4} \cr
&\le \biggl({tp\over e\log t}\biggr)^6 + \biggl({t(1-p)^{t/2} \over e(\log t)^3}\biggr)^{t/2} \cr
&\le \biggl({tp\over e\log t}\biggr)^6 + \biggl({t \over e^{pt/2+1}(\log t)^3}\biggr)^{t/2},\cr
}$$
where in the last line we used the inequality $1-p \le e^{-p}$.
Choosing $p = 2\log t/t$, we simply need $t$ large enough such that
$$ \biggl({t\over e^{\log t+1} (\log t)^3}\biggr)^{t/2} = 
\biggl({1\over e(\log t)^3}\biggr)^{t/2} < 1- \biggl({2\over e}\biggr)^6,$$
which can be done since the left-hand side goes to $0$.\slug

\proclaim Exercise 1.2. Suppose $n\ge 4$ and let $H$ be an $n$-uniform hypergraph with at most
$4^{n-1}/3^n$ edges. Prove that there is a colouring of the vertices of $H$ by $4$ colours so that in
every edge all $4$ colours are represented.

\proof Let each vertex of $H$ be independently given one of the four colours uniformly at random. (If
$H$ is infinite, it does not matter what colour we give to vertices that do not appear in any edge,
so it suffices to consider $H$ finite, which makes the sample space finite.)
Given some edge $e$ of $H$ with $n$ vertices, there are $4^n$ total ways that $e$ may be coloured,
and for each of the four colours, $3^n$ total ways that $e$ may be coloured using only the other three colours.
Let $K(e)$ denote the event that $e$ does not contain all four colours. By the inclusion-exclusion principle,
$$\pr\bigl\{K(e)\bigr\} = 4\cdot 3^n - 6\cdot 2^n + 4$$
Since $6\cdot 2^n \ge 96 > 4$, the probability that a given edge {\it does not} contain all four colours
is (much) less than $3^n/4^{n-1}$. By the union bound,
$$\pr\Bigl\{ \bigcup_{e\in E(H)} K(e)\Bigr\} \le \sum_{e\in E(H)} \pr\bigl\{K(e)\bigr\}
< {4^{n-1}\over 3^n} \cdot {3^n\over 4^{n-1}} = 1.$$
Since the sample space is finite this implies that there is some colouring of the vertices of $H$
in which every edge has all four colours.\slug

\proclaim \llap{*}Exercise 1.3. Prove that for two independent, identically distributed real random
variables $X$ and $Y$,
$$\pr\bigl\{|X-Y|\le 2\bigr\} \le 3\pr\bigl\{|X-Y|\le 1\bigr\}.$$

\proof [Not done. Need to use independence somehow. I think by symmetry
it is enough to show that $\pr\{1 < X-Y\le 2\}
\le \pr\bigl\{|X-Y|\le 1\bigr\}$ or $\pr\bigl\{X-Y > 1 \biggiven |X-Y|\le 2\bigr\}\le 1/3$.
In the second expression,
$X$ and $Y$ are no longer independent.]

\proclaim \llap{*}Exercise 1.4. Let $G=(V,E)$ be a graph with $n$ vertices and minimum degree $\delta>10$.
Prove that there is a partition of $V$ into two disjoint subsets $A$ and $B$ so that
$|A|\le O\bigl((n\log n)/\delta\bigr)$, and each vertex of $B$ has at least one neighbour in $A$ and at least
one neighbour in $B$.

\proof We follow the construction of the dominating set from the proof of Theorem~1.2.2, since the
required $A$ here is a dominating set with some extra conditions. Let $p$ be chosen later and let $X$
be a random set of vertices obtained by independently selecting each $v\in V$ with probability $p$.
Then as in the proof from the textbook, we let $Y$ be the set of all $v\in V\setminus X$ that have no
neighbours in $X$. At this point $X\cup Y$ is a dominating set, but we are not yet done constructing
$A$, since there may still be elements in $V\setminus (X\cup Y)$ all of whose neighbours belong to $X\cup Y$.
Let $Z\subseteq V\setminus (X\cup Y)$ denote all of these elements. For any given $v\in V$, we
have $\pr\{v\in X\} = p$ and
$$\pr\{v\in Y\} = (1-p)^{\deg(v) + 1} \le (1-p)^{\delta+1} \le e^{-p(\delta+1)},$$
since for $v\in Y$ we need $v$
itself as well as all $\deg(v)$ of its neighbours to not be in $X$. Lastly,
$$\pr\{v\in Z\} = (1-p)\prod_{w\in N(v)} \bigl(p+(1-p)^{\deg(w)+1}\bigr)
\le (1-p)\bigl(p+(1-p)^{\delta+1}\bigr)^\delta \le (1-p)(p+e^{-p(\delta+1)})^\delta.$$
Now we compute
$$\eqalign{
\ex\bigl\{|A|\bigr\} &= \sum_{v\in V} v\in \{X\cup Y\cup Z\} \cr
&\le n\bigl( p + e^{-p(\delta+1)} + (1-p)(p+e^{-p(\delta+1)})^\delta \bigr).\cr
}$$
Since $p+e^{-p(\delta+1)} < 1$, we can remove the $\delta$ from its exponent for the looser but
simpler bound
$$\eqalign{
\ex\bigl\{|A|\bigr\} &\le np + ne^{-p(\delta+1)} + np - np^2 + ne^{-p(\delta+1)} - npe^{-p(\delta+1)}\cr
&=(2-p)(np+ne^{-p(\delta+1)}).\cr
}$$
Setting $p = \log(\delta+1)/(\delta+1)$ just as in the dominating set proof, we have
$$e^{-p(\delta+1)} = {1\over \delta+1}$$
and
$$\ex\bigl\{|A|\bigr\} \le \biggl(2-{\log(\delta+1)\over \delta+1}\biggr)\biggl( n{\log(\delta+1)\over \delta+1}
+ {n\over \delta+1}\biggr) = 2n{\log(\delta+1)\over\delta+1} + o\biggl({n\log\delta\over\delta}\biggr),$$
which has the required asymptotics. It remains to choose an $A$ with $|A|$ at least this average.\slug

\section 4. The Second Moment

\proclaim Exercise 4.1. Let $X$ be a random variable taking integral nonnegative values, let $\ex\{X^2\}$
denote the expectation of its square, and let $\var\{X\}$ denote its variance. Prove that
$$\pr\{ X = 0\} \le {\var\{X\} \over \ex\{X^2\}}.$$

\proof Since $X$ is integer and nonnegative, we have $\pr\{X = 0\} = 1-\pr\{X\ge 1\}$ and since
$\var\{X\} = \ex\{X^2\} - \ex\{X\}^2$, to get our result it suffices to show that
$$\pr\{X\ge 1\} \ge {\ex\{X\}^2 \over \ex\{X^2\}}.$$
We start by noting that
$$\ex\{X\} = \sum_{k=0}^\infty k \pr\{X=k\}
= \pr\{X\ge 1\}\sum_{k=1}^\infty {k\pr\{X=k\}\over \pr\{X\ge 1\}}
= \pr\{X\ge 1\}\ex\{ X \given X\ge 1\}.$$
Since the function $x\mapsto x^2$ is convex, we have, by Jensen's inequality,
$$\ex\{X\}^2 = \pr\{X\ge 1\}^2 \ex\{X\given X\ge 1\}^2 \le \pr\{X\ge 1\}^2 \ex\{X^2 \given X\ge 1\}
= \pr\{X\ge 1\} \ex\{X^2\}.$$
Dividing both sides by $\ex\{X^2\}$ gives us what we want.\slug

\proclaim Exercise 4.4. Let $X$ be a random variable with expectation $\ex\{X\} =0$ and variance
$\var\{X\} = \sigma^2$. Prove that for all $\lambda > 0$,
$$\pr\{X\ge \lambda\} \le {\sigma^2\over \sigma^2 + \lambda^2}.$$

\proof Let $\lambda > 0$. Note that if $X\ge \lambda$, then $X$ is also positive so in particular,
for any $a>0$, $X+a$ and $\lambda + a$ are both positive. This implies that $(X+a)^2\ge (\lambda+a)^2$.
Note that since $\var\{X\} = \ex\{X^2\} - \ex\{X\}^2 = \sigma^2$ and $\ex\{X\}=0$, we have
$\ex\{X^2\} = \sigma^2$, which in turn implies that
$$\ex\bigl\{(X+a)^2\bigr\} = \ex\{X^2\} + 2a\ex\{X\} + \ex\{a^2\} = \sigma^2+a^2.$$
Now we take the infimum over all $a>0$ and apply Markov's inequality to get
$$\pr\{X\ge\lambda\} \le \inf_{a > 0} \pr\bigl\{(X+a)^2 \ge (\lambda+a)^2\bigr\} \le
\inf_{a>0} {\sigma^2 + a^2\over (\lambda + a)^2}.$$
We now optimise over $a>0$. We have
$${d\over da} {\sigma^2 + a^2\over (\lambda + a)^2}
  = {(\lambda+a)^2 2a - (\sigma^2+a^2)2(\lambda+a)\over (\lambda + a)^4},$$
which is zero when $a = \sigma^2/\lambda > 0$. Plugging in this value of $a$ gives
$$\pr\{X\ge \lambda\} \le {\sigma^2 + \sigma^4/\lambda^2\over (\lambda + \sigma^2/\lambda)^2}
= {\sigma^2(\lambda + \sigma^2/\lambda) \over \lambda(\lambda + \sigma^2/\lambda)^2}
= {\sigma^2 \over \sigma^2+\lambda^2},$$
exactly what we need.\slug

\bye
