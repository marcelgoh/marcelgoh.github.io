% Macros for typesetting math (assignments)
% Written by Marcel Goh, except for the parts that are not.

% ================== KNUTH ================== %

\def\xskip{\hskip 7pt plus 3pt minus 4pt}

\def\boldbegin#1{\medbreak\noindent {\bf#1}\xskip\ignorespaces}
\def\proof{\medbreak\noindent{\it Proof.}\xskip\ignorespaces}
\def\solution{\medbreak\noindent{\it Solution.}\xskip\ignorespaces}
\def\slug{\quad\hbox{\kern1.5pt\vrule width2.5pt height6pt depth1.5pt\kern1.5pt}\medskip}
\def\noskipslug{\quad\hbox{\kern1.5pt\vrule width2.5pt height6pt depth1.5pt\kern1.5pt}}

% Algorithms
\newdimen\algindent
\newif\ifitempar \itempartrue % normally true unless briefly set false
\def\algindentset#1{\setbox0\hbox{{\bf #1.\kern.25em}}\algindent=\wd0\relax}
\def\algbegin #1 #2{\algindentset{#21}\alg #1 #2} % when steps all have 1 digit
\def\aalgbegin #1 #2{\algindentset{#211}\alg #1 #2} % when 10 or more steps
\def\alg#1(#2). {\medbreak % Usage: \algbegin Algorithm A (algname). This...
  \noindent{\bf#1}({\it#2\/}).\xskip\ignorespaces}
\def\algstep#1.{\ifitempar\smallskip\noindent\else\itempartrue
  \hskip-\parindent\fi
  \hbox to\algindent{\bf\hfil #1.\kern.25em}%
  \hangindent=\algindent\hangafter=1\ignorespaces}
  
% ================ END KNUTH ================ %

% Sets of numbers
\def\N{{\bf N}}
\def\Z{{\bf Z}}
\def\Q{{\bf Q}}
\def\R{{\bf R}}
\def\C{{\bf C}}

% Number an equation oldstyle
\def\oldno#1{\eqno({\oldstyle#1})}

% Theorems and lemmas
\outer\def\thm #1#2\par{\medbreak
  \noindent{\bf Theorem #1.}~\ignorespaces{\sl #2\par}
  \ifdim\lastskip<\medskipamount \removelastskip\penalty55\medskip\fi}
\outer\def\thmparens #1#2#3\par{\medbreak
  \noindent{\bf Theorem #1}\enspace\rm({\it #2\/}).~\ignorespaces{\sl #3\par}
  \ifdim\lastskip<\medskipamount \removelastskip\penalty55\medskip\fi}
\outer\def\lemparens #1#2#3\par{\medbreak
  \noindent{\bf Lemma #1}\enspace\rm({\it #2\/}).~\ignorespaces{\sl #3\par}
  \ifdim\lastskip<\medskipamount \removelastskip\penalty55\medskip\fi}