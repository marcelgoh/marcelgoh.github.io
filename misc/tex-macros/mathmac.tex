% Macros for typesetting math (assignments)
% Written by Marcel Goh, except for the parts that are not.

% ================== KNUTH ================== %

\def\xskip{\hskip 7pt plus 3pt minus 4pt}

\def\boldbegin#1{\medbreak\noindent {\bf#1}\xskip\ignorespaces}
\def\proof{\medbreak\noindent{\it Proof.}\xskip\ignorespaces}
\def\solution{\medbreak\noindent{\it Solution.}\xskip\ignorespaces}
\def\slug{\quad\hbox{\kern1.5pt\vrule width2.5pt height6pt depth1.5pt\kern1.5pt}\medskip}
\def\noskipslug{\quad\hbox{\kern1.5pt\vrule width2.5pt height6pt depth1.5pt\kern1.5pt}}

% Algorithms
\newdimen\algindent
\newif\ifitempar \itempartrue % normally true unless briefly set false
\def\algindentset#1{\setbox0\hbox{{\bf #1.\kern.25em}}\algindent=\wd0\relax}
\def\algbegin #1 #2{\algindentset{#21}\alg #1 #2} % when steps all have 1 digit
\def\aalgbegin #1 #2{\algindentset{#211}\alg #1 #2} % when 10 or more steps
\def\alg#1(#2). {\medbreak % Usage: \algbegin Algorithm A (algname). This...
  \noindent{\bf#1}({\it#2\/}).\xskip\ignorespaces}
\def\algstep#1.{\ifitempar\smallskip\noindent\else\itempartrue
  \hskip-\parindent\fi
  \hbox to\algindent{\bf\hfil #1.\kern.25em}%
  \hangindent=\algindent\hangafter=1\ignorespaces}

% ================ END KNUTH ================ %

% Sets of numbers (not compatible with CWEB)
\def\NN{{\bf N}}
\def\ZZ{{\bf Z}}
\def\QQ{{\bf Q}}
\def\RR{{\bf R}}
\def\CC{{\bf C}}
\def\N{{\bf N}}  % Deprecated
\def\Z{{\bf Z}}
\def\Q{{\bf Q}}
\def\R{{\bf R}}
\def\C{{\bf C}}

\def\op#1{\mathop{\hbox{#1}}\nolimits}
\def\limitop#1{\mathop{\hbox{#1}}}

% Probability
\def\pr{\hbox{\bf P}}
\def\ex{\hbox{\bf E}}
\def\var{\hbox{\bf V}}
\def\one{\hbox{\bf 1}}
\def\indic#1{\one_{\sevenrm[#1]}}

% Graphs
\def\edge{\mathrel-\!\!\mathrel-}

% Analysis
\def\d{\,d}
\def\eps{\epsilon}

% Counters
\newcount\thmcount  % Counts theorems/lemmas/corollaries
\thmcount=1
\newcount\sectcount  % Counts sections
\sectcount=1
\newcount\figcount  % Counts figures
\figcount=1
\newcount\eqcount  % Counts equations
\eqcount=1

% Number an equation oldstyle
\def\oldno#1{\eqno({\oldstyle#1})}
\def\refeq#1{({\oldstyle#1})}
\def\adveq{\oldno{\the\eqcount}\global\advance\eqcount by 1}  % Print and advance equation counter
\def\advthm{\the\thmcount\global\advance \thmcount by 1}

% Numbered section (advances equation counter)
\def\advsect{\section\the\sectcount\global\advance\sectcount by 1. }

% Proclaim with parentheses in italics
\outer\def\parenproclaim #1 (#2).#3\par{\medbreak
  \noindent{\bf #1}\enspace\rm({\it #2\/}).\nobreak\ignorespaces{\sl #3\par}
  \ifdim\lastskip<\medskipamount \removelastskip\penalty55\medskip\fi}

% Exactly what it sounds like
\def\twodots{\mathinner{\ldotp\ldotp}}

% Axioms
\newdimen\axiomindent
\def\axiomindentset#1{\setbox0\hbox{{\bf #1.\kern.25em}}\axiomindent=\wd0\relax}
\def\axiombegin#1{\axiomindentset{#11}} % when all have 1 digit
\def\axiom#1. [#2.]{\ifitempar\par\noindent\else\itempartrue
  \hskip-\parindent\fi%
  \hbox to\axiomindent{\bf\hfil #1.\kern.25em}%
  \hangindent=\axiomindent\hangafter=1[{\it #2.}]}
