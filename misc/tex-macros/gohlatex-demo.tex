% Test document GohLaTeX

\documentclass[10pt]{article}

\usepackage{gohlatex}

\begin{document}

\maketitle{Title}{a demo by}{Marcel K. Goh}{32 February 2143}

\gohsection 1. Introduction

This is a demo of all the features of \GohLaTeX. Shame on you for not learning Plain \TeX. This
is not a perfect imitation; the keen Plain \TeX\ user will notice that some things (e.g., spacing
between paragraphs) still isn't as beautiful as in Plain.

\proclaim Theorem 1. This is how to make a theorem.

\proof Prove the theorem afterwards.\slug

\parenproclaim Lemma 2 (Parentheses). Sometimes you want to name your theorems/lemmas.

\solution Sometimes you want to put the slug in display math mode. We have shown that
$$2+2 = 4.\noskipslug$$

Sometimes you want to typeset an algorithm:

\algbegin Algorithm A (Name). Description of algorithm.

\algstep A1. [Initialise.] Use \GohLaTeX.

\algstep A2. [Fall in love.] The formatting is so exquisite that you want to do unspeakable things to it (like
write your own \TeX\ macros).

\algstep A3. [Convert.] Switch to Plain \TeX.

\algstep A4. [Enlightenment.] Your soul transcends.\slug

I'm not sure why you have to leave a blank space between each algorithm step for it to work. You don't have
to do that in Plain \TeX. When the algorithm has $\geq 10$ steps, you'll want {\tt \char`\\aalgbegin} instead.

\aalgbegin Algorithm B (Math). These are some math macros I added. There are not too many of them and you
should probably use your own macros for other things you like.

\algstep B1. [Sets.] We have the inclusion $\NN\subseteq \ZZ \subseteq \QQ\subseteq  \RR\subseteq  \CC$.

\algstep B2. [Probability.] We find that $\pr\{A\} = 1$, $\ex\{X\} = 2$, and $\var\{X\} = \sigma^2$.

\algstep B3. [Indicators.] The indicator of an event $\one_A$ equals 1 if $A$ is true and 0 if $A$ is false.
You can also spell out the event; for example, if $A$ is the event that $u\edge v$, then you can
write $\indic{u\edge v}$.

\algstep B4. [Dots.] We sometimes want to define $[1\twodots n] = \{1,2,\ldots,n\}$.

\algstep B5. [Equation numbers.] You can number your equations with old-style numerals:
$$[z^n]f(z) = {1\over 2\pi i}\oint {f(z)\over z^{n+1}}\,dz.\oldno 1$$

\algstep B6. [Reference.] You can reference an equation using \refeq{1234567890}.

\algstep B7. [Credit where it's due.] Some of these macros are lifted right out of {\tt plain.tex}, which
was written by Knuth himself.

\algstep B8. [Operators.] You can make your own operators and functions and they can
even have limits, like
$$\limitop{mylim}_{n\to\infty} \op{myfunc}_n(x).$$

\algstep B9. [Stalling.] Can't you tell I'm just trying to get to ten steps?

\algstep B10. [Slug.] Don't forget to end your algorithm with a slug!\slug

This is the end of a subsection.

\medskip
\boldlabel Big bold label. Use this when you don't want to start a whole new section, but you still want
to break up your text.

\end{document}
