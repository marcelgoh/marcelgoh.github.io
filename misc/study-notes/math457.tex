\input fontmac
\input mathmac

\font\ninebf=cmbx9

\def\lcm{\mathop {\rm lcm}}
\def\Im{\mathop {\rm Im}}
\def\Id{\mathop {\rm Id}}
\def\Ev{\mathop {\rm Ev}}
\def\Sym{\mathop {\rm Sym}}
\def\Aut{\mathop {\rm Aut}}
\def\Gal{\mathop {\rm Gal}}
\def\End{\mathop {\rm End}}
\def\Ann{\mathop {\rm Ann}}
\def\Tor{\mathop {\rm Tor}}
\def\Spec{\mathop {\rm Spec}}
\def\Specmax{\Spec_{\max}}
\def\Nil{\mathop {\rm Nil}}
\def\Jac{\mathop {\rm Jac}}
\def\inj{\hookrightarrow}
\def\surj{\rightarrow\!\!\!\!\!\rightarrow}
\def\defn{\noindent {\bf Definition.\enspace}}
\def\bar{\overline}
\def\mod{\!\!\!\!\pmod}
\def\F{{\bf F}}

\maketitle{MATH 457 Honours Algebra 4\footnote{$^*$}{\eightpt Course given by Prof.\ Mika\"el Pichot at McGill University}}{Notes by}{Marcel K. Goh}{23 April 2020}

\floattext5 \noindent {\ninebf Note.} \ninept These notes are rough and may skip over some details. \ninept Some proofs are either omitted or distilled to their main ideas.

\section 1. Rings

A {\it ring} $R$ is a set with operations $+$ and $\cdot$ such that
\medskip
\item{i)} $(R,+)$ is an abelian group;
\smallskip
\item{ii)} $(R,\cdot)$ is a semigroup;
\smallskip
\item{iii)} $\cdot$ distributes over $+$ on both sides:
$$a\cdot (b+c) = a\cdot b + a\cdot c \quad\hbox{and}\quad (a+b)\cdot c = a\cdot c + b\cdot c$$
A {\it semiring is the same as a ring} except that condition (i) above becomes
\medskip
\item{i')} $(R,+)$ is a monoid with absorbing identity 0.
\medskip
A ring is {\it unital} if $(R,\cdot)$ has a unit $1$. We always assume that $1\neq 0$, since if $1=0$ then $R = \{0\}$. Observe that in a unital ring, $(R,+)$ is necessarily abelian. A ring is said to be {\it commutative} if $(R,\cdot)$ is.

Even for commutative rings, there are many possible ring structures for $(R,+) = \Z^2$. For example we can take the {\it Gaussian integers} $\Z[i] = \{a+bi : a,b\in\Z\}$ or the {\it Eisenstein integers} $\Z[\omega] = \{a + b\omega : a,b\in \Z\}$ where
$$\omega = -{1 + i\sqrt{3} \over 2}.$$
In both cases the second binary operation is complex multiplication. Since $i$ and $\omega$ are both solutions to equations of the form $x^2 + Bx + C = 0$, they are called {\it quadratic integers} and $\Z[i]$ and $\Z[\omega]$ are called {\it quadratic rings}.

The definition of a ring is meant to describe a class of $\Z$-like objects, but many rings have properties different from the integers. For example, the ring $\Z[\sqrt{-5}]$ does not have Euclidean division. There are also many non-commutative rings such as the {\it Lipschitz quaternions}
$$\{a + bi + cj + dk : a,b,c,d\in \Z\}$$
or the {\it Hurwitz quaternions}
$$\bigg\{a + bi + cj + dk : a,b,c,d\in \Z\ \hbox{or}\ a,b,c,d\in \Z + {1\over 2}\bigg\}.$$

If $R$ is a ring, then a subgroup of $(R,+)$ that is closed under multiplication is called a {\it subring}. If a ring is unital, then any unital subring will have the same unit. A {\it homomorphism} between two rings $R$ and $S$ is a map $f:R\to S$ that preserves both operations:
$$f(a+b) = f(a) + f(b) \qquad\hbox{and}\qquad f(a\cdot b) = f(a)\cdot f(b)$$
A homomorphism that preserves the units is called {\it unital}.

An {\it ideal} in a ring $R$ is a subgroup $(I,+)$ such that
\medskip
\item{i)} $ab\in I$ for all $a\in R$, $b\in I$;
\smallskip
\item{ii)} $ab\in I$ for all $b\in I$, $a\in R$.
\medskip
If (i) holds, $I$ is called a {\it left ideal} and if (ii) holds, $I$ is called a right ideal. Let $I \subseteq R$ be an ideal. One defines the {\it quotient ring} $R/I$ as follows. Since $(I,+)$ is a normal subgroup of $(R,+)$, $R/I$ is an abelian group. We associate $r\sim r'$ if $r-r'\in I$. Then we can define multiplication in $R/I$ as $(a+I)(b+I) = ab+I$. This is well-defined because $I$ is an ideal and distributivity holds.

The isomorphism theorems for groups extend to rings as well.

\parenproclaim Theorem A (First isomorphism theorem). Let $f:R\surj S$ be a surjective ring homomorphism. Then $f$ descends to a a ring homomorphism $f' : R/I \to S$ that takes $a+I$ to $f(a)$, where $I$ is the kernel of $f$.\slug

\parenproclaim Theorem B (Second isomorphism theorem). Let $S$ be a subring and $I$ and ideal in a ring $R$. Then $S+I$ is a subring of $R$, $I$ is an ideal in $S+I$, and the map $S \surj (S+I)/I$ is a surjective ring homomorphism with kernel $S\cap I$.\slug

\parenproclaim Theorem C (Third isomorphism theorem). Let $R$ be a ring and $I\subseteq J\subseteq R$ be ideals. Then $R/I \surj R/J$ is a surjective ring homomorphism with kernel $J/I$.\slug

\parenproclaim Theorem D (Fourth isomorphism theorem). Let $f:R\surj S$ be a surjective ring homomorphism. There is a bijection between the ideals in $R$ containing $\ker f$ and the set of all ideals in $S$.\slug

Note that the correspondence in Theorem D works with subrings as well, not just ideals.

An element $r$ in a unital ring $R$ is said to be {\it invertible} if there exists $s\in R$ such that $rs = sr = 1$. The set of invertible elements is denoted $R^\times$ and this is a group under $\times$, called the {\it group of units}. A {\it field} is a ring in which every nonzero element is a unit. Non-commutative fields are called {\it division rings} or {\it skew fields} (the quaternions are an example of a skew field).

Let $K$ be a field. The set $K[x]$ of polynomials with coefficients in $K$ is a ring. Then the set
$$K(x) = \{ f / g : f,g\in K[x], g\neq 0\}$$
is a field, called the {\it field of rational functions}. The set $K[[x]]$ is called the {\it ring of formal series}: possibly infinite sums $\sum_{n\geq 0} a_nx^n$. Addition is done pointwise and multiplication is convolution of power series. The map $K[x] \to K[[x]]$ is a homomorphism and some elements become invertible. For example, $1-x$ becomes invertible, since $1/(1-x) = \sum_{n\geq 0} x^n$. Not every element in $K[[x]]$ is invertible, but one can invert the elements to get a new field $K((x))$: the set of sequences $K^\Z$ that are eventually zero when going to the left.

A {\it zero-divisor} is an element $r\in R$, $r\neq 0$ for which there exists $s\in R$ such that $rs = 0$. A ring is {\it cancellative} if $rs = rs'$ implies that $s = s'$. Then we define an {\it integral domain} to be a unital, commutative, cancellative ring. Every integral domain embeds into a field, called the {\it field of fractions}. The construction is analogous to building the rational numbers from the integers.

\proclaim Proposition Z. If $R$ is a ring with unity, there exists a unique unital homomorphism $f:\Z\to R$.\slug

\proof Since $f(1) = 1$, we have $f(n) = 1 + 1 + \cdots + 1 \in R$.\slug

The nonnegative integer $n$ which generates $\ker f$ is called the {\it characteristic} of $R$. The image of $f$ is called the {\it characteristic subring}. For example $\Z/n\Z$ has characteristic $n$.

\proclaim Proposition P. The characteristic of an integral domain $R$ is either $0$ or a prime number.\slug

An {\it algebra} over a commutative ring $R$ is a ring $A$ with a homomorphism $\eta : \R\to A$ whose image lies in the {\it centre} of $A$. Examples of algebras include rings of functions and matrices $M_n(R)$.

For a group $G$ and a ring $R$, we can define the {\it group ring} $G[R]$ as the set of all finitely supported functions from $G$ to $R$. This forms a ring with addition $(f+g)(s) = f(s) = g(s)$ and multiplication $(fg)(s) = \sum_{uv=s} f(u)g(u)$.

\section 2. Ideals

Every element $r$ in a unital ring $R$ generates a {\it principal} ideal $(r)$. More generally any subset $S\subseteq R$ does. The ideal $(S)$ is the intersection of all ideals that contain $S$. If $R$ is commutative, then $(r) = rR = Rr$. In $\Z$, the ideals are the of the form $(n) = n\Z$. Then $(n)\subseteq (m)$ if and only if $m\setminus n$ (this is true in any commutative ring). A ring $R$ in which every ideal is principal is called a {\it principal ring} and if $R$ is also an integral domain, we call it a {\it principal ideal domain} or PID.

Principal ideals determine their generators up to unit. If $(r) = (s)$, then $s = ar$ and $r = bs$ together imply that both $a$ and $b$ are units. Elements $r$ and $s$ of a ring $R$ are called {\it associate} if there exists a unit $a$ such that $r=as$.

We can define three operations on ideals. Let $I,J\subseteq R$ be ideals.
\medskip
\item{i)} $I\cap J$ is an ideal.
\smallskip
\item{ii)} $I+J = \{a + b : a\in I, b\in J\} = (I\cup J)$ is an ideal.
\smallskip
\item{iii)} $IJ = \{ab : a\in I, b\in J\}$ is an ideal.
\medskip

\proclaim Lemma P. Let $R$ be a commutative ring. Let $I = (S)$ and $J = (T)$ be two ideals. Then $IJ = (ST)$. \slug

In the ring of integers $\Z$, we have $(m)(n) = (mn)$, $(m)\cap(n) = \big(\lcm(m,n)\big)$, and $(m)+(n) = \big(\gcd(m,n)\big)$. When $I\subseteq J$ is an inclusion of ideals, one may think of it as a kind of divisibility $J\setminus I$. For example, $\gcd(m,n)\setminus \lcm(m,n)\setminus mn$.

\proclaim Lemma D. If $I,J\subseteq R$ are ideals, then
$$IJ \subseteq I\cap J \subseteq I+J.\noskipslug$$

The set of ideals forms a semiring where the two operations are $I+J$ and $IJ$. The semiring in $\Z$ is $\N$ with the addition $m + n = \gcd(m,n)$ and ordinary multiplication.

For an ideal $I\subseteq R$, we define the {\it radical} of $I$ to be the set
$$\sqrt{I} = \{a \in R : a^n \in I\ \hbox{for some}\ n\in \N\}.$$
This is an ideal and it has the property that $\sqrt{\sqrt{I}} = \sqrt{I}$. Furthermore, if $I\subseteq J$, then $\sqrt{I} \subseteq \sqrt{J}$.

An ideal $I\subseteq R$ is called {\it maximal} if it is proper and whenever $I\subseteq J\subseteq R$, then either $J = I$ or $J=R$.

\proclaim Lemma M. Let $R$ be a unital ring. Then every proper ideal is included in a maximal ideal.

\proof This is an application of Zorn's Lemma. Let $I$ be a proper ideal and let $X$ be the set of all proper ideals containing $I$, ordered by inclusion. Then this set is inductive (increasing union of ideals is an ideal) so there is a maximal element $M$.\slug

\proclaim Lemma F. Let $R$ be unital and commutative. Then an ideal $I\subseteq R$ is maximal if and only if $R/I$ is a field.

\proof This follows from the fourth isomorphism theorem.\slug

Let $R$ be a unital ring. An ideal $I$ of $R$ is {\it prime} if it is proper and for any ideals $A,B$ of $R$, $AB\subseteq I$ implies that $A\subseteq I$ or $B\subseteq I$. The {\it spectrum} of $R$ is the set of all prime ideals and it is denoted $\Spec(R)$. The {\it maximal spectrum} of $R$, denoted $\Specmax(R)$, is the set of all maximal ideals of $R$.

Maximal ideals are always prime (so $\Specmax(R) \subseteq \Spec(R)$), but not all prime ideals are maximal. For example, $(0)$ is prime in $\Z$ but certainly not maximal. A ring is called {\it local} if it has a unique maximal ideal. A ring $R$ is local if and only if $R\setminus R^\times$ is an ideal.

\proclaim Lemma C. Let $R$ be a unital commutative ring. Let $I\subseteq R$ be a proper ideal. Then $I$ is prime if and only if $ab\in I$ implies that $a\in I$ or $b\in $I.\slug

\proclaim Lemma I. Let $R$ be a unital commutative ring. Then $I\subseteq R$ is a prime ideal if and only if $R/I$ is an integral domain.\slug

Since all fields are integral domains, this proves that all maximal ideals are prime. We also have that a commutative ring $R$ is an integral domain if and only if $(0)$ is a prime ideal in $R$ (if $R$ is not commutative, then we say it is a {\it prime ring}). If $R$ is a PID, then every nonzero prime ideal is maximal.

We can view elements in a commutative unital ring $R$ as ``functions'' on the set $\Spec(R)$ of prime ideals. To $r\in R$ we identify a function $f_r$ such that $f_r(P) = r \bmod P\in R/P$. We have a bundle at every $P\in \Spec(R)$ and a fibre $R/P$ which is an integral domain. The {\it total space} $B(R)$ is the union of $R/P$ over all prime ideals $P$. A {\it section} is a map $s: \Spec(R)\to B(R)$ such that $s(P)\in R/P$. $\Gamma(R)$ is the set of all sections and $\Gamma_{\max}(R)$ is its restriction to $\Specmax(R)$. Let $\pi : R\to \Gamma(R)$ map $r\mapsto f_r$ and $\pi_{\max} : R\to \Gamma_{\max}(R)$ take $r$ to $f_r$, restricted to $\Specmax(R)$. We want to know when $\pi$ and $\pi_{\max}$ are faithful.

\proclaim Proposition K. The kernel of $\pi$ is the intersection of all prime ideals and the kernel of $\pi_{\max}$ is the intersection of all maximal ideals.\slug

For a unital commutative ring $R$, we define the {\it nilradical} of $R$ to be the intersection $\Nil(R) = \bigcap P$ of all prime ideals $P$. The {\it Jacobean radical} is the intersection $\Jac(R) = \bigcap M$ of all maximal ideals $M$. Since $\Specmax(R) \subseteq \Spec(R)$, $\Jac(R) \supseteq \Nil(R)$. An element $r\neq 0$ in a ring $R$ is called {\it nilpotent} if $r^n = 0$ for some $n$. It turns out that there is a connection between nilpotency and prime ideals.

\proclaim Proposition N. Let $R$ be unital and commutative. Then $\Nil(R)$ is the set of all nilpotent elements, i.e.
$$\sqrt{(0)} = \{r\in R : r^n = 0\ \hbox{for some}\ n\in \N\} = \bigcap_{P\in \Spec(R)} P.$$

\proof To show that a nilpotent element $r$ belongs to every prime ideal $P$, note that $r^n \in P$, so $r\cdot r^{n-1}\in P$ and we can iterate this until we get that $r\in P$. Conversely, if $r$ is not nilpotent, we can let $X$ be the set of ideals $I$ such that $r^n$ is not in $I$ for any $n$. $X$ is nonempty and inductive, so by Zorn's Lemma there is a maximal element and it can be shown that this ideal is prime.\slug

Let $R$ be a commutative ring and let $p\in R$ be a nonzero non-unit. Then $p$ is said to be
\medskip
\item{i)} {\it prime} if $p\setminus ab$ implies that $p\setminus a$ or $p\setminus b$;
\smallskip
\item{ii)} {\it irreducible} if $p=ab$ implies $a$ is a unit or $b$ is a unit.
\medskip

To find irreducible elements in a ring, may attempt the ``bisection process''. Let $r\in R$. If $r$ is irreducible, we stop. If $r$ is not irreducible, then $r = r_1 r_2$. If neither is irreducible, we continue by splitting $r_1$ and $r_2$ in the same way. This process may not terminate.

\proclaim Proposition I. Let $R$ be an integral domain. If an element $p\in R$ is prime, then it is irreducible.

\proof. Let $p\in R$ be a prime element. Assume that $p=ab$. This implies that $p\setminus a$ or $p\setminus b$. Say $a = pc$ for some $c\in R$. Then $p=ab=pcb$ and $cb = 1$. So $b$ is a unit.\slug

Note that the converse does not hold. For example, in the ring $\Z[\sqrt{-3}]$, we have $4=(1+\sqrt {-3})(1-\sqrt {-3})$. The element 2 is irreducible, but it is not prime because $2$ divides $4$ but does not divide either of $(1+\sqrt{-3})$ and $(1-\sqrt{-3})$.

\proclaim Proposition A. Let $R$ be an integral domain. Let $p$ be a nonzero element in $R$. Then $p$ is prime if and only if $(p)$ is prime and $p$ is irreducible if and only if $(p)$ is maximal among principal ideals.\slug

This proposition implies that in a PID, irreducible elements are prime.

A ring $R$ is a {\it unique factorisation domain} if every $r\in R$ can be expressed as a product $r = p_1\cdots p_n$ of irreducible elements, which is unique up to the order of the $p_i$. The rings $\Z$, $K[x]$, and $K[x,y]$ are all examples of UFDs. Every PID is a UFD and in a UFD, all irreducible elements are prime.

\proclaim Lemma S. In a PID, every chain of ideals stabilises.

\proof $I = \bigcup_{n\geq 1} I_n$ is an ideal. Since $R$ is a PID, $I = (x)$ for some $x$ and $x\in I_n$ for some $n$. This implies that $I = I_n$.\slug

\proclaim Lemma N. Let $R$ be a unital ring. Then every increasing chain of ideals stabilises if and only if every ideal is finitely generated.

\proof If $I = (x_1,x_2,\ldots)$ is not finitely generated, then $I_n = (x_1,\ldots,x_n)$ is an increasing chain of ideals that does not stabilise. Conversely, if every ideal is finitely generated, then let $I_1\subseteq I_2\subseteq \cdots$ be a chain of ideals and let $I = \bigcup_{n\geq 1} I_n$. There exist $(x_1,\ldots,x_n)$ that generate $I$, so there exists a $k$ such $x_i\in I_k$ for all $i$ and we find that $I = I_k$.\slug

A ring is called {\it Noetherian} if it the equivalent conditions from Lemma N hold.

For elements $r$ and $s$ of a ring, a {\it greatest common divisor} or gcd is an element $d$ dividing both $r$ and $s$ such that if any $d'$ divides both $r$ and $s$, then $d'$ divides $d$. An integral domain $R$ is called a {\it B\'ezout domain} if $(r)+(s)$ is principal for every $r,s\in R$ (of course, every PID is a B\'ezout domain) and it is called a {\it GCD domain} if any two $r,s\in R$ have a gcd. Every UFD is a GCD domain.

\proclaim Lemma B. The following statements regarding B\'ezout domains are true.
\medskip
\item{i)} A ring $R$ is B\'ezout if and only if every finitely generated ideal is principal.
\smallskip
\item{ii)} A B\'ezout domain is a GCD.
\smallskip
\item{iii)} If a ring is both Noetherian and a B\'ezout domain, then it is a PID.\noskipslug
\medskip

\section 3. Gaussian Integers

Recall from Section 1 that the Gaussian integers are the ring
$$\Z[i] = \{a + bi : a,b\in \Z\}.$$
We write $N$ for the complex modulus, squared. So $N(z) = z\bar z = a^2 + b^2$. This is called the {\it norm} and it is a group homomorphism $\C^\times\to \R^\times$, since $N(zz') = N(z)N(z')$. $N(z) = 0$ implies that $z = 0$. The norm $n$ takes $\Z[i]$ to $\N$. The kernel of $N$ on $\C^\times$ is the unit circle $\{z\in \C : |z| = 1\}$. Let $\ker N$ denote the kernel of $N$ restricted to $\Z[i]$, i.e.\ $\{\pm 1, \pm i\}$. These are the units of $\Z[i]$.

The image of $N$ is
$$\Im(N) = \{n\in \N : n = a^2 + b^2\ \hbox{for some}\ a,b\in \Z\}.$$
This set is stable under product, since if $n = N(z)$ and $n' = N(z')$, then $nn' = N(zz')$. Gauss was interested in studying the number of integer numbers less than a given $n$ that can be expressed as a sum of two squares. We will return to this point later.

We say that a prime number {\it splits} if it is no longer prime in $\Z[i]$ and we say that it is {\it inert} otherwise.

\proclaim Lemma S. Let $p$ be a prime. Then $p$ is a sum of two squares if and only if it splits in $\Z[i]$.

\proof If $p = a^2 + b^2$ then $p = (a + ib)(a-ib)$ and $N(a+ib)N(a-ib) = p^2$ implies that neither of these factors are units. So $p$ is not prime in $\Z[i]$. Conversely, if $p = \alpha\beta$ in $\Z[i]$, then $N(\alpha) = N(\beta) = p$ means that $p$ is the sum of two squares.\slug

\proclaim Lemma I. A prime $p$ splits if and only if $p\equiv 1 \mod 4$.

\proof If $p$ splits, then by the previous lemma, $p = a^2 + b^2$ and the sum of two squares is never $3$ modulo 4. So if $p$ is an odd prime it is congruent to 1 modulo 4. Conversely, assume that $p\equiv 1\mod 4$. Then $p = 1 + 4n$ for some $n$ and there exists $x\in \Z$ such that $x^2 \equiv 1\mod p$. (In fact, $x = (2n)!$ works.) Then $p$ divides $x^2 + 1 = (x+i)(x-i)$. So $p$ divides $(x+1)$ or $(x-i)$, so $p$ divides $i$ and is not inert.\slug

All this talk of divisibility leads nicely into a discussion of Euclidean division. In $\Z$, the goal of Euclidean division for integers $a$ and $b$ is to find a $q\in \Z$ such that $a-bq$ is small, in some sense. The following proves a similar result in $\Z[i]$.

\proclaim Proposition E. There is a Euclidean division in $\Z[i]$.

\proof. Let $a,b\in \Z[i]$, $b\neq 0$. We can divide them in $\C$ to get $z = a/b$. Then there is a (not necessarily unique) $q\in \Z[i]$ that is of minimal distance to $z$. We have $|z-q|<1$; in fact $|z-q| \leq \sqrt{2}/2 < 1$. So $|a-bq| < |b|$.\slug

Let us now define this generally. An integral domain $R$ is a {\it Euclidean domain} if there exists a function $N : R\to \N$ called the {\it norm} such that $N(0) = 0$ and for all $a,b\in R$, $b\neq 0$, either $b$ divides $a$ or there exists $q\in R$ such that $N(a-bq) < N(b)$. Proposition E showed that $Z[i]$ is a Euclidean domain with the complex norm, and other familiar examples include $\Z$ with the absolute value function and $K[x]$ with the degree of a polynomial as its norm. In general, the Euclidean division algorithm does not give a unique answer. Even in $\Z$, we can end up with $q$ or $-q$ as a quotient.

\proclaim Proposition T. $\Z[\sqrt{-2}]$ is a Euclidean domain.

\proof We repeat the same proof as for $\Z[i]$ except for the computation of $|z-q|$, which is now $\leq \sqrt{3}/2$.\slug

Recall that $\Z[\sqrt{-3}]$ is not a Euclidean domain. It is not even a UFD, since $4 = 2\cdot 2 = (1+\sqrt{-3})(1-\sqrt{-3})$. But $\Z[\sqrt{-3}]\subseteq \Z[\omega]$ and this is a Euclidean domain, with norm $N(a+b\omega) = a^2 - ab + b^2$. The units in $\Z[\omega]$ are the elements of norm 1: $\{\pm 1, \pm \omega, \pm \omega^2\}$ and we have unique factorisation up to units.

\proclaim Proposition P. Every Euclidean domain is a PID (and consequently a UFD).

\proof Let $R$ be a Euclidean domain and $I\subseteq R$ an ideal. Let $b\neq 0$ be an element of $I$ of minimal norm. If $a\in I$ then $b$ divides $a$. Otherwise, there exists $q\in R$ such that $N(a-bq) < N(b)$, contradicting the minimality of $b$'s norm. So $I$ is principal.\slug

A corollary of this fact is that every ideal in $\Z[i]$ is principal.

\section 4. Modules

For any set $X$, the set of symmetries $\Sym(X)$ is a group and an action of a group $G$ on $X$ is a group homomorphism $G\to \Sym(X)$. If $X$ is a group, we can define the {\it ring of endomorphisms} $\End(X)$ as the set of group homomorphisms from $X$ to $X$.

\proclaim Lemma M. Let $M$ be an abelian group. Then $\End(M)$ is a ring.

\proof Addition is pointwise addition from $M$ and multiplication is composition of maps.\slug

Let $R$ be a a unital commutative ring. A {\it module} $M$ over $R$ is a ring homomorphism $R\to \End(M)$. Explicitly, the list of axioms of a module are very similar to those of a vector space (in fact, if $R$ is a field, then a module is a vector space). For $r,s\in R$ and $m,n\in M$, we have
\medskip
\item{i)} $r(m+n) = rm + rn$;
\smallskip
\item{ii)} $(r+s)m) = rm + sm$;
\smallskip
\item{iii)} $(rs)m = r(sm)$;
\smallskip
\item{iv)} $1m = m$.
\medskip
These axioms also work if $R$ is not commutative; in this case, we call $M$ a {\it left $R$-module}. The kernel of $R\to \End(M)$ is called the {\it annihilator} of $M$:
$$\Ann(M) = \{r\in R : rm = 0\ \hbox{for all}\ m\in M\}$$
A module is said to be {\it faithful} if $\Ann(M)$ is trivial. If $M$ is an $R$-module, then $M$ is a faithful $S$-module where $S = R/\Ann(M)$.

\proclaim Proposition I. Any ideal $I$ in a ring $R$ is a module over $R$.

\proof For $a\in I$ and $r\in R$, we have $ra\in I$. The rest of the axioms follow.\slug

Quotients $R/I$ are also modules. When $R = \Z$, the action is determined by the group structure in $M$. For example,
$$2m = (1 + 1)m = 1m + 1m = m + m.$$
When $R = K[x]$ for some field $K$, we have the following interesting lemma.

\proclaim Lemma V. $K[x]$-modules are operators on vector spaces and vice versa.

\proof Let $M$ be a $K[x]$-module. The restriction of the $K[x]$ action to $K$ gives a $K$-module structure on $M$. This is a vector space. Furthermore, the indeterminate $x$ also acts on $M$ by taking $m\mapsto xm$. This gives a map $x: M \to M$ such that $x(m + n) = x(m)$ and $x(rm) = (xr)m = (rx)m = r\cdot x(m)$. So $x$ is a linear map.

Conversely, if $V$ is a $K$-vector space, and $T:V\to V$ is a linear map, then $V$ is a $K[x]$-module, because for any $p\in K[x]$, $p(T)$ is a linear map on $V$.\slug

Note that the module is not faithful, because $K[x]$ has infinite dimension but $\End(V)$ has finite dimension when $V$ has finite dimension. If $G$ is a group, $K[G]$ is the group ring and a $K[G]$-module is a linear representation of $G$.

A {\it submodule} $M'$ of $M$ is a subgroup that is stable under the action of the ring, i.e.\ for all $m, n\in M'$ and $r\in R$, $m+rn\in M'$. For example, ideals are submodules of $R$ and if $M$' is a submodule, we can define the {\it quotient module} $M/M'$ with the action of $R$:
$$ r(m+M') = rm + M'$$
If $M$ and $M'$ are modules, then $M\times M'$ is a module. If a module has no proper nontrivial submodules, then it is called {\it simple}.

An $R$-module map is a group homomorphism $f:M\to M'$ such that $f(rm) = rf(m)$ for all $r\in R$ and $m\in M$. The kernel $\ker f$ is a submodule of $M$ and the image of $f$ is a submodule of $M'$. The isomorphism theorems for modules are exactly analogous to the ones given for rings in Section 1.

\parenproclaim Lemma S (Schur's lemma). Let $M$ be a simple module. Then $\End_R(M)$ is a skew field.

\proof Let $f:M\to M'$ be a module map that is not identically zero. The kernel of $f$ is a submodule of $M$, so since $f\neq 0$, $\ker f = \{0\}$. Then the image of $f$ is a submodule of $M'$ since $f\neq 0$, $\Im f = M'$. Hence $f$ is an isomorphism.\slug

If $M$ is an $R$-module and $I\subseteq R$ is an ideal, then
$$IM = \Big\{\sum r_im_i : r_i\in I, m_i\in M\Big\} \subseteq M$$
is a submodule.

\parenproclaim Theorem C (Chinese remainder theorem). Let $I,J$ be ideals in a ring $R$. Let $M$ be an $R$-module. Then the map
$$M\to M/IM \times M/JM$$
has kernel $IM\cap JM$.\slug

If $I+J = R$ then the map is surjective and $(I\cap J)M = IJM$. With $n$ ideals such that $I_k + I_l = R$ for $k\neq l$, we have
$$M/(I_1\cdots I_n)M \cong M/I_1M \times \cdots \times M/I_nM.$$

Let $M$ be an $R$-module. If $A\subseteq M$, then
$$(A) = \Big\{\sum r_ia_i : r_i\in R, a_i \in A\Big\}$$
is the submodule of $M$ {\it generated} by $A$. A module is {\it finitely generated} if it admits a finite generating set and {\it cyclic} (or {\it singly generated}) if it is generated by one element. If $M = (a)$ is cyclic, then the map $R\to M$ that sends $r\mapsto ra$ is surjective with kernel $\Ann(M)$.

\proclaim Lemma P. Let $R$ be an integral domain. Then the nonzero principal ideals are isomorphic to $R$.

\proof Let $I = (a)$ be an ideal (so it is an $R$-module). If $r\in \Ann(I)$ then $r$ is a zero divisor. So $R\to I$ is an isomorphism.\slug

A finitely generated $R$-module $M$ is called {\it free} if it is isomorphic to $R^n$ for some $n$. For example, if $R$ is a field, every module (finite-dimensional vector space) is free. Equivalently, an $R$-module is free if there exists a basis, that is, a generating set $A$ such that any $m\in M$ can be written in a unique way as a finite sum
$$m = \sum_{a\in A} r_a a.$$
The set $A$ is called a {\it free generating set} and the cardinality of $A$ is called the {\it rank} of $M$.

In a PID, every ideal is a free module (isomorphic to the ring itself). For any set $A$ and ring $R$, we can let $F_A$ be the set of all functions from $A$ to $R$ with finite support. This is a group under pointwise addition and $r$ acts on $F_A$: $(rf)(a) = r\cdot f(a)$. A basis for $F_A$ is the set $(\delta_a)_{a\in A}$ of delta functions, where $\delta_a(b) = 1$ if $b=a$ and $0$ otherwise.

\parenproclaim Proposition U (Universal property of free modules). Let $\phi$ be a map from a set $A$ to an $R$-module $M$. Then there is a unique extension of $\phi$ to a module map $\bar\phi : F_A\to M$.

\proof Take any element $f\in F_A$ and express it as
$$f = \sum_{a\in A} r_a\delta_a$$
for some $r_a\in R$. Then let $\bar\phi$ be given by
$$\bar\phi(f) = \sum_{a\in A} r_a\phi(a).\noskipslug$$

\proclaim Proposition S. Let $N\inj M \surj F$ be a short exact sequence of modules (so $F \cong N/M$), where $F$ is a free module. Then the sequence splits, i.e.\ $M\cong N\oplus F$.

\proof We need to construct the section $s$ of $\pi : M\surj F$. Let $A$ be a basis of $F$. Since $\pi$ is surjective, for any $a\in A$ we can find $m_a\in M$ such that $\pi(m_a) = m$. This gives a map $s_* : A \to M$ and by the universal property there is a unique extension $s : F_A\to M$. We have $\pi \circ s = \Id$ on the basis and therefore everywhere on $F$. So $s$ is a section of $\pi$. Let $F' = \Im(s) \subseteq M$. So $F'\cong F$ as $R$-modules. We claim that $M = N\oplus F'$ (viewing $N$ as a submodule of $M$).

Firstly, $N\cap F' = \{0\}$, since if $m\in N\cap F'$, then $\pi(m) = 0$ and there exists $f\in F$ such that $s(f) = m$. But this implies that $f = \pi\big(s(f)\big) = \pi(m) = 0$, so $m=0$. And $M = N + F'$ because any $m\in M$ can be expressed as the sum of $\big(m-s\circ \pi(m)\big) + s\circ\pi (m)$.\slug

The following theorem shows that the rank of a free module is well-defined.

\proclaim Theorem R. If $R^n\cong R^m$ as $R$-modules, then $n = m$.

\proof Since $R$ is unital and commutative, it contains a maximal ideal $M$. Let $K = R/M$ and consider the submodule $MR^n = \big\{s(x_1,\cdots,x_n) \in R^n: s\in M, x_i\in R\big\}$. The quotient module is $K^n = (R/M)^n$ and a module isomorphism $R^n \cong R^m$ descends to a $R$-module isomorphism $K^n \cong K^m$. This map has kernel $M$ and is a $K$-vector space isomorphism. So the dimension of the two vector spaces are the same and thus $n=m$.\slug

Let $M$ be a module over an integral domain $R$. The set of {\it torsion elements}
$$\Tor(M) = \{m\in M : rm = 0\ \hbox{for some}\ r\neq 0\}$$
is a submodule of $M$. A module is called {\it torsion} if $\Tor(M) = M$ and {\it torsion-free} if $\Tor(M) = \{0\}$. Note that $R^n$ is torsion-free, since it has a basis $\{e_n\}$ and if $am = ra_1e_1 + \cdots + ra_ne_n = 0$, then $ra_i = 0$ for all $i$ and $m = 0$.

\proclaim Proposition T. For any module $M$ over an integral domain $R$, $M/Tor(M)$ is torsion-free.

\proof Let $N = M/\Tor(M)$ and let $\bar m\in N$. Suppose there exists $r\neq 0$ such that $r\bar m = 0$. So $\bar{rm} = 0$ and $rm \in \Tor(M)$. Thus there exists $s\neq 0$ such that $(rs)\bar m = 0$. But $\bar{rs}\neq 0$ so $m$ must be 0. Hence $\Tor(N) = \{0\}$.\slug

\proclaim Lemma G. A module $M$ over an integral domain $R$ is torsion if and only if it is generated by torsion elements.

\proof The forward direction is clear. Conversely, suppose $M = (A)$ and every element in $A$ is torsion. Let $m = s_1a_1 + \cdots + s_na_n \in M$ for some $s_i\in R$ and $a_i\in A$. Each $a_i$ is a torsion element so there is $r_i$ such that $r_ia_i = 0$. Let $r = r_1\cdots r_n$. Then $rm = 0$.\slug

\proclaim Proposition F. Let $M$ be a finitely generated module over an integral domain $R$. There exists a free module $F\subseteq M$ such that $M/F$ is torsion.

\proof Let $M = (A)$ where $A$ is finite. Let $B\subseteq A$ be a maximal basis which generates a free module $F$ of rank $n = |B|$. Let $N$ be the quotient $M/F$. For every $a\in A\setminus B$, the module $(B\cup \{a\})$ is not free. So there exists $r\in R, r_b\in R$, not all zero, such that
$$ra + \sum_{b\in B} r_b b = 0.$$
Note that $r\neq 0$, otherwise $B$ would not be a basis. But $ra = 0 \bmod F$, so $N$ is generated by torsion elements and by the previous lemma, $N$ is torsion.\slug

\section 5. Modules Over PIDs

An $R$-module has properties very much like a vector space when $R$ is a PID.

\proclaim Proposition F. Let $R$ be a PID. Then every submodule of a free module $R^n$ is free of rank $k\leq n$.

\proof We proceed by induction. When $n=1$, every ideal $I\subseteq R$ is free, isomorphic to $R$. Now assume the proposition is true for $R^n$. Let $M$ be a submodule of $R^{n+1}$. Let $\pi : R^{n+1} \surj R^n$ be the projection map on to the first $n$ coordinates. So we have a short exact sequence
$$\ker(\pi_{|M}) \inj M \surj \pi(M).$$
But $\pi(M)$ is a submodule of $R^n$ so, by the induction hypothesis, it is free and the sequence splits. Thus $M\cong \ker(\pi_{|M}) \oplus \pi(M)$ is free.\slug

A module $M$ over a unital ring $R$ is called a {\it Noetherian module} if every submodule is finitely generated.

\proclaim Proposition N. Let $M$ be a left $R$-module. The following are equivalent:
\medskip
\item{i)} $M$ is Noetherian.
\smallskip
\item{ii)} $M$ satisfies the ascending chain condition on left modules.
\smallskip
\item{iii)} If $\frak F$ is a nonempty family of submodules, there exists a maximal element in $\frak F$ with respect to inclusion.
\medskip

\proof To show that (i) implies (ii), we let $N_1\subseteq N_2 \subseteq N_3 \subseteq \cdots$ be an increasing sequence of modules. We need to know there is an upper bound. Let $N = \bigcup_{i\geq 1} N_i$. Since $N$ is finitely generated, there will be a first index $i$ such that all the generators of $N$ belong to $N_i$. Thus $N = N_i$ for some $i$.

We show that (ii) implies (iii) by contraposition. If (iii) fails, then there exists a family $\frak F$ of submodules for which there is no maximal element. Pick $N_1\in \frak F$. We can find $N_2$ such that $N_1$ is properly contained in $N_2$. Continuing in this way, we are left with an increasing chain that does not stabilise.

Lastly, assume that (i) does not hold; i.e.\ there is a submodule $N$ that is not finitely generated. Let $\{n_1,n_2,\ldots\}$ be an infinite countable subset of $N$ such that, for every $k$, $N_k = (n_1,\ldots,n_k)$ is properly contained in $N_{k+1} = (n_1,\ldots,n_{k+1})$. Now $\frak F = \{N_k\}$ is a family of submodules without a maximal element, so (iii) fails.\slug

\proclaim Proposition S. Let $N \inj P\surj Q$ be an exact sequence of modules. Then $N$ and $Q$ are Noetherian if and only if $P$ is Noetherian.

\proof Clearly $N$ is Noetherian if $P$ is. To show that $Q$ is Noetherian, let $M\subseteq Q$ be a submodule. Then, by the fourth isomorphism theorem, $M$ is the image of a submodule $M'$ of $P$. Since $M'$ is finitely generated, $M$ is as well.

Conversely, assume that $N$ and $Q$ are Noetherian and let $M\subseteq P$ be a submodule. Let $\pi:P\surj Q$ be the quotient map and consider the exact sequence $\ker(\pi_{|M})\inj M\surj \pi(M)$. Let $X$ be a finite generating set for $\ker(\pi_{|M})$ and $Y$ be a finite set in $M$ such that $\bar Y = \pi(Y)$ is a finite generating set of $\pi(M)$. Then for any $m\in M$, then there exist some $r_x$ and $r_y$ in $R$ such that
$$m = \sum_{x\in X} r_x x + \sum_{y\in Y} r_y y$$
and $X\cup Y$ generates $M$.\slug

\proclaim Theorem R. The following are equivalent:
\medskip
\item{i)} $R$ is a Noetherian ring.
\smallskip
\item{ii)} The free module $R^n$ is Noetherian for every $n$.
\smallskip
\item{iii)} Every finitely generated $R$-module is Noetherian.\noskipslug
\medskip

A corollary of Theorem R is that every finitely generated module over a PID is Noetherian. For example, $R = K[x_1,\ldots,x_n]$ is Noetherian.

\proclaim Lemma N. If $R$ is a PID and $M$ a torsion-free $R$-module, then $M$ is free.

\proof In general, we showed that there exists $F$ free such that $F\inj M \surj T$, where $T$ is torsion. Since $M$ is Noetherian, we can choose a maximal $F$ satisfying this property. We claim that $M=F$. Let $\pi:M\surj T$ denote the quotient map and let $m\in M$. Since $\pi(m)\in T$ is a torsion element, there exists $r\in R$ such that $r\pi(m) = 0$. So $rm\in \ker\pi$ and $rm\in F$. Let $f_r : M \to M$ be the map that sends $m\mapsto rm$. This map is injective because $M$ is torsion-free. Since $f_r(F) \subseteq F$ and $f_r(M) \subseteq F$, so the submodule $f_r(F,M)$ is contained in $F$. By Proposition F, $f_r(F,M)$ is free, so $M$ is free.\slug

\proclaim Theorem T. Let $M$ be a finitely generated module over a PID $R$. Then $M\cong R^n \oplus \Tor(M)$.

\proof Consider the exact sequence $\Tor(M)\inj M\surj N$ where $N$ is torsion-free. Since $R$ is a PID, $N$ is free and the sequence splits, giving us the desired direct sum decomposition.\slug

The integer $n$ given by Theorem T is called the {\it free rank} of a module. If two modules $M$ and $N$ are isomorphic, then their free ranks are equal and $\Tor(M)\cong \Tor(N)$. The following theorem is called the structure theorem for finitely generated modules over a PID.

\proclaim Theorem S. Let $R$ be a PID and let $F \cong R^n$ be a finitely generated free module. Let $M$ be a finitely generated submodule of $F$. Then there exists a basis $(e_1,\ldots, e_n)$ of $F$ and elements $r_1,\ldots,r_m\in R$ such that $(r_1e_1,\ldots,r_me_m)$ forms a basis of $M$:
$$F/M \cong R^{n-m} \oplus R/(r_1) \oplus \cdots \oplus R/(r_m)$$
The elements $r_i$ are unique up to multiplication by a unit if we assume that $r_i$ divides $r_{i+1}$.\slug

The elements $r_i$ in Theorem S are called the {\it invariant factors} of the module.

\section 6. Fields and Polynomials

Because the kernel of a homomorphism is an ideal, then any nontrivial homomorphism $f:K\to R$ is injective when $K$ is a field. If $R = L$ is a field, then $K\subseteq L$ is a subfield and we call $L$ an {\it extension} of $K$. We will  often denote this by $L/K$. If $L/K$ is a field extension then $L$ is a vector space over $K$ and the dimension $[L:k] = \dim_K L$ is called the {\it degree} of the extension. Every there is a basis $(\alpha_i)$ of $L$ such that any element $l\in L$ can be expressed as $\lambda_1\alpha_1 + \cdots \lambda_n\alpha_n$ where $\lambda_i\in K$. If $K\subseteq L\subseteq M$ is a chain of extensions and $(\beta_j)$ is a basis of $M$ over $L$, then it can shown that $(\alpha_i\beta_j)$ is a basis of $M$ over $K$. So $[M:K] = [M:L][L:K]$.

A field is {\it prime} if it contains no proper nontrivial subfields. Any field $K$ is an extension of a prime field that contains $1$, $1+1$, etc. as well as their inverses. If the characteristic of $K$ is 0, then the prime field is $\Q$, and if the characteristic is a prime $p$, then the prime field is $\F_p$.

\proclaim Lemma F. Let $K$ be a field of characteristic $p$. The Frobenius map $x\mapsto x^p$ is a field homomorphism (so it is injective).

\proof For any $x,y\in K$, we have $(x+y)^p = x^p + y^p$ (by the binomial theorem) and $(xy)^p = x^py^p$.\slug

Let $L/K$ be an extension and $S\subseteq L$ be a set. Then $K(S)$ is the subfield of $L$ generated by $S$. The extension field is finitely generated if $S$ is finite. If $S$ consists of a single element $\alpha$, then $L = K(\alpha)$ is called a {\it simple} extension and $\alpha$ is a {\it primitive element}. For $\alpha_1,\ldots,\alpha_n$, the extensions $K(\alpha_1,\ldots,\alpha_n)$ and $K(\alpha_1)\cdots(\alpha_n)$ are the same and the order in which the elements are adjoined does not matter.

If $K$ is a field then $K[x]$ is a PID. So for any irreducible polynomial $f\in K[x]$, $(f)$ is maximal and $L = K[x]/(f)$ is a field extension. This is called the {\it Kronecker construction}.

\proclaim Lemma R. Every $f\in K[x]$ admits a root in a finite-degree extension.

\proof We may assume that $f$ is irreducible. It is of finite degree so $L = K[x]/(f)$ is a finite degree extension and if $\alpha = x \bmod f$, then $f(\alpha) = 0$ in $L$.\slug

Kronecker's construction is universal in the following sense. Let $L/K$ be an arbitrary extension and let $\alpha\in L$. Consider the map $\Ev_\alpha : K[\alpha] \to L$ that takes a polynomial $f$ to $f(\alpha)$. Since $K[x]$ is a PID, the $\ker(\Ev_\alpha)$ is principal and equals $(f_\alpha)$ for some polynomial $f_a\in K[x]$. Because $L$ is an integral domain, one of two things may happen. The first is that $f_\alpha$ is an irreducible polynomial, in which case we say that $\alpha$ is {\it algebraic}. The second is that the kernel is trivial and in this case we call $\alpha$ {\it transcendental}.

When $\alpha$ is algebraic, the unique monic irreducible polynomial $f_\alpha$ such that $f_\alpha(\alpha) = 0$ is called the {\it minimal polynomial} of $\alpha$ and $K(\alpha) \subseteq L$ is obtained by the Kronecker construction
$$K(\alpha)\cong K[x]/(f_\alpha).$$
If $\alpha$ is transcendental, $\Ev_\alpha : K[x] \inj L$ is injective and it extends to the fraction field $K(x) \inj L$ by taking $f/g \mapsto f(\alpha)/g(\alpha)$ (since $g(\alpha) \neq 0$ whenever $g\neq 0)$. Liouville established the existence of transcendental numbers in 1844 by proving that
$$L = \sum_{n\geq 0} {1\over 10^{n!}}$$
is transcendental. We have $\Q(L)\cong \Q(x) \subseteq \R$. Other famous transcendental numbers are $\pi$ and $e$.

An extension $L/K$ is {\it algebraic} if every $\alpha\in L$ is algebraic over $K$ and the following lemma proves some properties of algebraic extensions.

\proclaim Lemma A. Assume that an extension $L$ is generated by $\alpha_1,\ldots,\alpha_n$ over $K$. The following are equivalent:
\medskip
\item{i)} The elements $\alpha_1,\ldots,\alpha_n$ are algebraic over $K$.
\smallskip
\item{ii)} The degree $[L:K]$ is finite.
\smallskip
\item{iii)} Every $\alpha\in L$ is algebraic over $K$.
\medskip

\proof Suppose (i) holds. We have
$$K \subseteq K(\alpha_1)\subseteq K(\alpha_1, \alpha_2) \subseteq \cdots \subseteq L,$$
and since each $\alpha_i$ is algebraic over $K$, it is algebraic over $K(\alpha_1,\ldots,\alpha_{i-1}$, whence
$$\big[ K(\alpha_1,\ldots,\alpha_i) : K(\alpha,\ldots,\alpha_{i-1})\big] < \infty.$$
By the multiplicativity of the degree, $[L:K] \leq \infty$.

Suppose (iii) fails, i.e.\ some $\alpha\in L$ is not algebraic. Then $K(x) \cong K(\alpha) \inj L$ is an injection into $L$, contradicting the fact that $[K(x) : K] < \infty$. Thus (ii) implies (iii).

That (iii) implies (i) is obvious, so we are done.\slug

To construct extensions of a field, we need to find irreducible polynomials. Over $\Q$, we can consider $x^n - p$ where $p$ is a prime. Then $\Q(\root n\of p)$ is an extension of degree $n$ over $Q$. A general check for irreducibility is given by the following criterion.

\parenproclaim Theorem E (Eisenstein's criterion). Let $R$ be an integral domain and let $f\in R[x]$ be a monic polynomial of degree $n$. If there exists a prime ideal $\frak p$ such that $f = x^n \bmod \frak p$ and $f(0) \notin \frak p^2$, then $f$ is irreducible.

\proof Suppose, towards a contradiction, that $f = ab$ is reducible. Then, we have $x^n = \bar a \bar b$, where the bar indicates the polynomials modulo $\frak p$. In particular, $\bar a\bar b$ has zero constant term. The ideal $\frak p$ is prime, so $R/\frak p$ is an integral domain, so both $\bar a$ and $\bar b$ have zero constant term modulo $\frak p$, meaning that the constant terms of $a$ and $b$ both belong to $\frak p$. This is a contradiction, since it is clear that the constant term of $f$ belongs to $\frak p^2$.\slug

We can use Eisenstein's criterion to show that cyclotomic polynomials of the form
$$\Phi_p(x) = {x^p -1\over x-1} = x^{p-1} + x^{p-2} + \cdots + x + 1$$
for $p$ prime are irreducible. The criterion does not immediately apply, but if we consider
$$\Phi_p(x+1) = x^{p-1} + px^{p-2} + {p(p-1)\over 2}x + p,$$
we find that Eisenstein's criterion applies, so $\Phi_p(x+1)$ is irreducible and this implies that $\Phi_p(x)$ is irreducible, since any factorisation for $\Phi_p(x)$ would give a factorisation for $\Phi_p(x+1)$ (replacing $x$ with $x+1$).

There are other many other criteria for reducibility/irreducibility; we give two more famous ones.

\parenproclaim Theorem C (Cohn's criterion). Suppose that $p$ is prime and in some base $b$,
$$p = a_nb^n + \cdots + a_1b + a_0.$$
Then $f = a_nx^n + \cdots + a_1x + a_0$ in $\Z[x]$.\slug

\parenproclaim Lemma G (Gauss' lemma). Let $R$ be a UFD with fraction field $K$ and let $f\in R[x]$ be a polynomial of degree $n$ such that $\gcd(a_n, \ldots, a_0) = 1$. If $f$ is reducible in $K[x]$, then $f$ is reducible in $R[x]$.\slug

Gauss' lemma is often applied with $R = \Z$ and $K = \Q$.

\section 7. Splitting Fields

We begin with a lemma regarding the interchangeability of the roots of a polynomial.

\proclaim Lemma I. Let $f\in K[x]$ be a monic, irreducible polynomial and let $L/K$ be an extension of the field $K$. If $\alpha,\beta$ are two roots of $f$ in $L$, then there is a field isomorphism $K(\alpha) \cong K(\beta)$.

\proof This follows from the universality of the Kronecker construction: $K(\alpha) \cong K[x]/(f) \cong K(\beta)$.\slug

More generally, any field isomorphism $\phi : K\to L$ extends uniquely to a ring isomorphism $\bar\phi : K[x] \to L[x]$ defined by applying $\phi$ on the coefficients. Then $f\in K[x]$ is irreducible if and only if $\bar\phi(f)$ is irreducible. Let $\alpha$ be an arbitrary root of an irreducible polynomial $f\in K[x]$ and let $\beta$ be an arbitrary root of $\bar\phi(f)$. Then there exists a unique field isomorphism $\phi^* : K(\alpha) \to L(\beta)$ that takes $\alpha$ to $\beta$ and whose restriction to $K$ is $\phi$. A corollary of this fact is that if $f\in K[x]$ is irreducible, then all roots of $f$ have the same multiplicity in an algebraic closure (this will be expanded on later).

A {\it splitting field} for a polynomial $f\in K[x]$ is an extension $L/K$ such that
$$f = \prod_{i=1}^n (x - \alpha_i)$$
for $\alpha_i\in L$ and $L = K(\alpha_1,\ldots,\alpha_n)$.

\proclaim Proposition S. Every polynomial $f\in K[x]$ of degree $n$ admits a splitting field of degree at most $n!$.

\proof Let $\alpha_1$ be an abstract root of $f$ in $K(\alpha_1)$ obtained by the Kronecker construction. Then $f = (x-\alpha_1)f_1$ for some $f_1\in K(\alpha_1)$. Let $\alpha_2$ be a root of an irreducible factor of $f_1$ and extend the field to $K(\alpha_1,\alpha_2)$. This process happens at most $n$ times, by which time we will have found a splitting field $L$ of $f$. We have
$$K \subseteq K(\alpha_1) \subseteq K(\alpha_1,\alpha_2) \subseteq \cdots \subseteq L,$$
and the degree of each $f_i$ is $n-i$. So by multiplicity of the degrees we have $[L:K] \leq n!$.\slug

For example, the polynomial $f = x^4 - 1$ has roots $\pm 1, \pm i$. When $K = \Q$, the abstract bound for the degree of the splitting field is $4! = 24$, but in fact $\Q(i)$ is a splitting field for $f$, of degree 2. More generally, when $f = x^n - 1$, the roots are the $n$th roots of unity $1, \omega, \ldots, \omega^{n-1} \in\C$, where $\omega = e^{2ki\pi/n}$ for $i = 0,\ldots,n-1$. The roots form a group isomorphic to $\Z/n\Z = \langle\omega\rangle$ and $K(\omega)$ is the splitting field of $x^n - 1$, since if one root is added, all of its powers come along for the ride. If $n$ is prime, then the degree of this splitting field is $n-1$; in general, the degree is equal to $\varphi(n)$ where $\varphi$ denotes Euler's totient function.

The following theorem shows that splitting fields are unique up to $K$-isomorphism.

\parenproclaim Theorem K (Kronecker{\rm, 1887}). Let $f\in K[x]$ be an irreducible polynomial. Let $\alpha$ and $\alpha'$ be two roots of $f$ in two splitting fields $L/K$ and $L'/K$ respectively. Assume the existence of a map $\theta_0 \in \Aut(K)$ that fixes coefficients of $f$. Then there exists an isomorphism $\theta : L \to L'$ such that $\theta_{|K} = \theta_0$ and $\theta(\alpha) = \alpha'$.

\proof The proof is by strong induction on $n = \deg f$. If $f$ splits in $K[x]$, then $L = K = L'$. We can take $\theta = \theta_0$, finishing the case when $n=1$ and when every irreducible factor of $f$ has degree 1.

Now we assume that the theorem is proved for any field $K$, automorphism $\theta_0$ and polynomial $f$ of degree less than $n$. Now let $p\in K[x]$ be an irreducible factor of $f$ of degree at least 2. If $\alpha\in L$ and $\alpha'\in L'$ are roots of $p$, then we can extend $\theta_0$ to an isomorphism $\theta : K(\alpha) \to K(\alpha')$. Let $K_1 = K(\alpha)$ and ${K_1}' = K(\alpha')$ for short. We have $f = (x-\alpha)f_1$ in $K_1$ and $f = (x-\alpha'){f_1}'$ in ${K_1}'$ where $f_1$ and ${f_1}'$ have degree $n-1$. Now $L$ is a splitting field for $f_1$ over $K_1$ and $L'$ is a splitting field for ${f_1}'$ over ${K_1}'$. Since the degrees of $f_1$ and ${f_1}'$ are less than $n$, by the induction hypothesis there is an isomorphism $\theta^* : L \to L$ that extends the isomorphism $\theta : K_1 \to {K_1}'$. The restriction of $\theta^*$ to $K_1$ is $\theta$, and the restriction of that onto $K$ is $\theta_0$.\slug

Let $L/K$ and $L'/K$ be field extensions. A {\it $K$-embedding} $L \inj L$ is an injective homomorphism that fixes $K$. If $\theta : L\to L'$ is a bijection, we call it a {\it $K$-automorphism}. The {\it Galois group} of a polynomial $f\in K[x]$ is the group of $K$-automorphisms of a splitting field of $K$. If $L$ is a splitting field, we denote this group by $\Gal(L/K)$ or $\Aut(L/K)$. For short, we may use the notation $\Gal(f)$ for a polynomial $f$, but this is only well-defined up to conjugacy. If $L/K$ and $L'/K$ are two splitting fields, then by Theorem K there exists a $K$-isomorphism $\theta:L\to L'$ and $\Aut(L/K) \cong \Aut(L'/K)$.

\proclaim Lemma A. Let $R$ be the roots of a polynomial $f$. Then $\Gal(f)$ acts on $R$.

\proof If $\theta\in \Gal(f)$ and $f(\alpha) = 0$, we have
$$\alpha^n + \lambda_{n-1}\alpha^{n-1} + \cdots + \lambda_1\alpha + \lambda_0 = 0$$
and
$$\theta(\alpha)^n + \lambda_{n-1}\theta(\alpha)^{n-1} + \cdots + \lambda_1\theta(\alpha) + \lambda_0 = 0$$
so $\theta(\alpha)\in R$. This defines an action $\Gal(f) \to \Sym(R)$.\slug

\proclaim Proposition F. The action of $\Gal(f)$ on the set of roots $R$ is faithful.

\proof Let $L = K(R) = K(\alpha_1,\ldots,\alpha_n)$ be a splitting field. Suppose that $\theta\in \Gal(f)$ acts trivially on $R$, i.e.\ $\theta(\alpha_i) = \alpha_i$ for all $i$. Since $\theta_{|K} = \Id$, $\theta = \Id$ as an automorphism of $L$.\slug

We have established that $\Gal(f) \inj \Sym(R)$ is a group of permutations of the roots of $f$.

\proclaim Proposition O. Let $\Gal(f)$ act on the set $R$ of roots of $f$. Then the orbit $\Gal(f)\alpha$ of $\alpha\in R$ is the set $R_1$ of roots of $f_1$ where $f_1$ is the irreducible factor of $f$ such that $f_1(\alpha) = 0$.

\proof If $f_1(\alpha) = 0$, since $f_1\in K[x]$. we have $f_1\big(\theta(\alpha)\big) = 0$ so $\theta(\alpha)\in R_1$ (the set of roots of $f_1$). Thus $\theta(R) \subseteq R_1$. Furthermore, since $f_1$ is irreducible, Kronecker's uniqueness theorem shows that for any two roots $\alpha$ and $\beta$ of $f_1$, there exists some $\theta_i : L\to L$ such that $\theta_i(\alpha) = \beta$. So $\Gal(f)\alpha = R_1$.\slug

As a corollary, if $f$ is already irreducible, then $\Gal(f)$ is transitive. In general, it is not tractable to classify all the transitive subgroups (up to conjugacy) of $S_n$. This has been done for small $n$, however; for example, when $n = 6$ there are 16 different possible subgroups.

Let us look at an example. Consider $f = (x^2 - 2)(x^2 - 3)$ over the field $K = \Q$. Then $R = \{\pm \alpha = \sqrt 2, \pm \beta = \sqrt 3\}$ and $L = \Q(\alpha, \beta)$. The Galois group cannot take $\sqrt 2$ to $\sqrt 3$ because such a permutation does not preserve the relations between the roots: If $\theta(\alpha) = \beta$, then $2 = \theta(\alpha^2) = \theta(\alpha)^2 = 3$, a contradiction. In this case, $\Gal(f)$ is the Klein four-group $V$. Let $\theta_2$ be the permutation that switches $\pm \sqrt 2$ and let $\theta_3$ transpose $\pm \sqrt 3$. Then the two commute and generate a group isomorphic to $V$.

As another example, consider $f = x^4 - 2$ over $K = \Q$. Eisenstein's criterion with $p=2$ tells us that $f$ is irreducible, and the set of roots turns out to be $R = \{\pm \alpha = \root 4 \of 2, \pm \beta = i \root 4 \of 2\}$. The splitting field is $L = \Q(\alpha, \beta)$. We can employ a useful trick, namely that {\sl if the coefficients of $f$ are real, then complex conjugation permutes the roots.} This gives us an element in $\Gal(f)$ of order 2: the permutation that fixes $\alpha$ and takes $\beta\mapsto -\beta$. In any case, we will employ a more general method to compute $\Gal(f)$. Let $\theta\in \Gal(f)$. Since $\alpha^2 + \beta^2 = 0$, we have
$$\theta(\alpha)^2 + \theta(\beta)^2 = 0.$$
Suppose that $\theta(\alpha) = \beta$. Then $\theta(\beta)^2 = -\beta^2$ and $\theta(\beta)$ is $\pm i\beta$. If $\theta(\beta) = -i\beta = \alpha$, then we have the order 2 automorphism
$$s = (\alpha \leftrightarrow \beta, -\alpha\leftrightarrow-\beta).$$
If, instead, we have $\theta(\beta) = i\beta = -\alpha$, then we have the order 4 automorphism
$$r = (\alpha \mapsto \beta \mapsto -\alpha \mapsto -\beta \mapsto \alpha).$$
We conclude that $\Gal(f) = \langle r,s \rangle = D_4$.

A field is {\it algebraically closed} if every $f\in K[x]$ admits a root in $K$. For example, $\C$ is algebraically closed. Every algebraically closed field is infinite. The Kronecker construction tells us that for any finite set $F\subseteq K[x]$, there is a finite extension $L$ of $K$ such that every polynomial $f\in F$ splits in $L$. We can extend our definition of a splitting field to (not necessarily finite) subsets of $K[x]$. Then an {\it algebraic closure} of $F$ is a splitting field for $K[x]$.

The following theorem gives the existence and uniqueness (up to $K$-isomorphism) of the algebraic closure $\bar K$ for every field $K$. The construction is ``universal'' in the sense that if $L/K$ is an algebraic extension, then there exists a $K$-embedding $L\inj \bar K$.

\parenproclaim Theorem S (Steinitz{\rm, 1910}). Let $K$ be a field. There exists an algebraic closure $\bar K$ of $K$ and this extension is unique up to $K$-isomorphism.

\proof First we show existence. Let $\frak A$ be the set of all algebraic extensions of $K$, ordered by set inclusion. Observe that $\frak A$ is inductive; indeed, if $L_1\subseteq L_2\subseteq \cdots$ is a chain, then $\bigcup_{i=1}^\infty L_i$ is in $\frak A$. By Zorn's Lemma, there exists a maximal element of $\frak A$, call it $L$. We claim that $L$ is algebraically closed. Let $f\in L[x]$ be a polynomial. By the Kronecker construction, there is an extension $L(\alpha)$ of $L$. Then $L(\alpha)$ is algebraic and $L\subseteq L(\alpha)$. Since $L$ is a maximal element in $\frak A$, $L = L(\alpha)$ and $f$ admits a root in $L$.

Next we show uniqueness of the algebraic closure. It is enough to show that if $L$ is algebraic over $K$, then $L\inj \bar K$ (if there were two algebraic closures, this would make them isomorphic). So fix an algebraic extension $L/K$. Consider the set of all intermediate extensions $K\subseteq L_\phi\subseteq L$ and for each $L_\phi$ there exists an embedding $\phi : L_\phi \to \bar K$. Let $\frak B$ be the set of all such $\phi$, partially ordered in the following manner: $\phi \leq \phi'$ if $L_\phi \subseteq L_{\phi'}$, i.e.\ $\phi'$ extends $\phi$.

The poset $\frak B$ is inductive, since if $\phi_1 \leq \phi_2 \leq \cdots$, then we can let
$$L_\phi = \bigcup_{i=1}^\infty L_{\phi_i}.$$
For any $\alpha\in L_\phi$, we have $\alpha\in L_{\phi_i}$ for some $i$ and we can simply set $\phi(\alpha) = \phi_i(\alpha)$. (This does not depend on the choice of $i$ since the functions extend one another.) By Zorn's Lemma, $\frak B$ admits a maximal element $\phi : L_\phi \to \bar K$. We want to show that $L_\phi = L$. If not, then there exists $\alpha\in L\setminus L_\phi$. By the uniqueness of Kronecker, $\phi: L_\phi \to \bar K$ admits an extension $\bar\phi : L_\phi(\alpha) \to \bar K$. But since $\bar \phi$ is an extension of $\phi$, by the maximality of $\phi$ we have $\phi= \bar\phi$ so $\alpha\in L_\phi$.\slug

As corollaries of Theorem S we have $\bar K = \bar{\bar K}$ and if $L/\bar K$ is algebraic, then $L = \bar K$.



\bye
