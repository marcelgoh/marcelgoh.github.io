% Ergodic Group Theory Notes
% Class given by Prof. Mikael Pichot. Notes by Marcel Goh

\input fontmac
\input mathmac

% Useful functions

\def\Sym{\mathop{\rm Sym}\nolimits}
\def\Stab{\mathop{\rm Stab}\nolimits}
\def\Aut{\mathop{\rm Aut}\nolimits}
\def\Out{\mathop{\rm Out}\nolimits}
\def\Inn{\mathop{\rm Inn}\nolimits}
\def\Im{\mathop{\rm Im}\nolimits}
\def\Id{\mathop{\rm Id}\nolimits}
\def\graph{\mathop{\rm graph}\nolimits}

\leftrighttop{NOTES ON ERGODIC GROUP THEORY}{MARCEL GOH}

\maketitlenodate{Notes on Ergodic Group Theory}{by}{Marcel K.\ Goh}

\floattext6 Disclaimer. These notes are based on a course given by Prof. Mika\"el Pichot at McGill University in Winter 2020. Where I found it necessary, I have supplied extra material (e.g.\ definitions). My notes are not officially endorsed in any way and I may have introduced errors. Please contact me if you find any such error.

\section 1. PRELIMINARY NOTIONS

Let $X$ be a topological space. The smallest $\sigma$-algebra containing all the open sets in $X$ is called the {\it Borel $\sigma$-algebra} and members of this collection are called {\it Borel sets}. A map $f$ between topological spaces $X$ and $Y$ is called {\it Borel} provided that for every open set $V$ in $Y$, $f^{-1}(V)$ is a Borel set in $X$. The {\it graph} of a Borel map $f$ is the set
$$\graph(f) = \{(x, f(x)) : x\in X\}.$$
[Borelness and other things.]
A topological space $X$ is called a {\it Polish space} if it is homeomorphic to a separable complete metric space. In other words, it is a separable completely metrisable topological space. Examples of Polish spaces are $[0,1]$, $(0,1)$, manifolds, and Hilbert spaces.

Let $G$ be a group and let $X$ be a Polish space. We define a {\it Borel action} to be a Borel map from $G\times X\rightarrow X$ (where pairs $(s,x)$ are usually denoted $sx$) such that $ex = x$ for all $x\in X$ and $s(tx) = (st)x$ for all $s,t\in G, x\in X$. The set
$$Gx = \{ y \in X : y = sx\ \hbox{for some}\ s\in G\}$$
is called the {\it orbit} of $x\in X$. The partition of $X$ into orbits is called the {\it orbit equivalence relation} of the action.

\section REFERENCES

The contents of this document are heavily based on MATH 596 lectures given by Mika\"el Pichot at McGill University in Winter 2020.

\end
