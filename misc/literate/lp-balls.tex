\input cwebmac
\def\cwebtitle#1{\gdef\title{\expandafter\uppercase\expandafter{#1}}}
\def\namedatethis{
  \def\startsection{
    \leftline{\sc Written by Marcel K. Goh. Last updated \today\ at \hours}%
\bigskip
    \let\startsection=\stsec\stsec
  }
}
\cwebtitle{lp-balls}

\input fontmac
\input epsf

\def\bb{{\bf b}}
\def\setZ{{\bf Z}}
\def\RR{{\bf R}}
\def\norm#1{\vert\!\vert#1\vert\!\vert}

\namedatethis


\N{1}{1}Introduction. This literate program contains various functions for
experimenting with points
in the unit ball of $L^p$-space over $\RR^d$. We will maintain a set of points
in this space, and provide
functionality for randomly generating new ones, as well as printing the current
state of the point set.

\fi

\M{2}This is the main outline of the program. We have a couple of global
variables storing $p\in (0,\infty]$,
the dimension $d$, the set \PB{\\{points}} of points we are working with (and
the maximum size \PB{\\{max\_points}}
of this array). The three parameters will be supplied by the user via
command-line arguments. One specifies
the case $p=\infty$ by supplying a negative number for $p$.
We represent vectors in $\RR^d$ as \CEE/ \PB{\&{double}} arrays of length
$d+1$. While perhaps a bit wasteful,
this allows us to index from $1$ to $d$ rather than from $0$ to $d-1$.

\Y\B\8\#\&{include} \.{<float.h>}\6
\8\#\&{include} \.{<math.h>}\6
\8\#\&{include} \.{<stdio.h>}\6
\8\#\&{include} \.{<stdlib.h>}\6
\8\#\&{include} \.{<time.h>}\7
\&{double} \|p;\6
\&{double} \\{p\_inv};\6
\&{int} \|d;\6
\&{int} \\{max\_points};\6
\&{int} \\{num\_points};\6
\&{double} ${}{*}{*}\\{points}{}$;\7
\X12:Print points to PostScript file\X;\6
\X3:Point set functions\X;\6
\X4:Random variate generation\X;\6
\X15:Committee-generating algorithm\X;\6
\X19:Inner product computation\X;\6
\X20:Draw loci when $p=\infty$\X;\7
\&{int} \\{main}(\&{int} \\{argc}${},\39{}$\&{char} ${}{*}\\{argv}[\,]){}$\1\1%
\2\2\6
${}\{{}$\1\6
\X24:Gather input and initialise global variables\X;\6
\X5:Seed the pseudorandom number generator\X;\6
\&{if} ${}(\|d\E\T{2}\W\|p<\T{0}){}$\1\5
${}\\{random\_locus}(\T{5},\39\.{"locus.ps"},\39\T{150},\39\T{30});{}$\2\6
\4${}\}{}$\2\par
\fi

\N{1}{3}The point set.
This section contains functions on points in $L^p$-space.
First, we provide facilities to add new points, delete all points, and to list
the current point set on
the console.
We use the current size of the point set, stored in the variable \PB{\\{num%
\_points}} to help index into
the \PB{\\{points}} array. Clearing the point set is done by setting \PB{\\{num%
\_points}} to $0$, so everything
in the array at and after the index \PB{\\{num\_points}} possibly contains
garbage values.

\Y\B\4\X3:Point set functions\X${}\E{}$\6
\&{void} \\{add\_point}(\&{double} ${}{*}\\{point}){}$\1\1\2\2\6
${}\{{}$\1\6
\&{if} ${}(\\{num\_points}\G\\{max\_points}){}$\5
${}\{{}$\1\6
\\{printf}(\.{"I\ failed\ to\ add\ poi}\)\.{nt\ because\ array\ is\ }\)\.{full.%
\\n"});\6
\&{return};\6
\4${}\}{}$\2\6
\&{for} (\&{int} \|i${}\K\T{1};{}$ ${}\|i\Z\|d;{}$ ${}\PP\|i){}$\1\5
${}\\{points}[\\{num\_points}][\|i]\K\\{point}[\|i];{}$\2\6
${}\PP\\{num\_points};{}$\6
\4${}\}{}$\2\7
\&{void} \\{clear\_points}(\,)\1\1\2\2\6
${}\{{}$\1\6
${}\\{num\_points}\K\T{0};{}$\6
\4${}\}{}$\2\7
\&{void} \\{list\_points}(\,)\1\1\2\2\6
${}\{{}$\1\6
\&{for} (\&{int} \|i${}\K\T{0};{}$ ${}\|i<\\{num\_points};{}$ ${}\PP\|i){}$\5
${}\{{}$\1\6
\\{printf}(\.{"("});\6
\&{for} (\&{int} \|j${}\K\T{1};{}$ ${}\|j\Z\|d;{}$ ${}\PP\|j){}$\5
${}\{{}$\1\6
${}\\{printf}(\.{"\%.2f"},\39\\{points}[\|i][\|j]);{}$\6
\&{if} ${}(\|j\I\|d){}$\1\5
\\{printf}(\.{",\ "});\2\6
\4${}\}{}$\2\6
\\{printf}(\.{")\\n"});\6
\4${}\}{}$\2\6
\4${}\}{}$\2\par
\U2.\fi

\N{1}{4}Random variate generation.
The main goal is to generate a point uniformly at random from the interior of
the unit ball in $L^p$-space
over $\RR^d$.
To do this, we
will require an exponential random variable with mean $1$.

\Y\B\4\X4:Random variate generation\X${}\E{}$\6
\X6:Uniform and exponential variates\X;\6
\X11:Normal random variate\X;\6
\X7:Gamma random variate\X;\6
\X8:Uniform random point in unit ball\X;\par
\U2.\fi

\M{5}First, in the \PB{\\{main}} function, we initialise
a pseudorandom number generator with the current time.

\Y\B\4\X5:Seed the pseudorandom number generator\X${}\E{}$\6
\&{time\_t} \|t;\7
\\{srand}((\&{unsigned}) \\{time}${}({\AND}\|t));{}$\6
\&{for} (\&{int} \|i${}\K\T{0};{}$ ${}\|i<\T{10};{}$ ${}\PP\|i){}$\5
${}\{{}$\1\6
\\{rand}(\,);\6
\4${}\}{}$\2\par
\U2.\fi

\M{6}To generate our exponential random variable, we use von Neumann's
algorithm,
as described by L.~Devroye on p.~126 of {\sl Non-Uniform Random Variate
Generation} (New York: Springer, 1986).

\Y\B\4\X6:Uniform and exponential variates\X${}\E{}$\6
\&{double} \\{uniform01}(\,)\1\1\2\2\6
${}\{{}$\1\6
\&{return} ((\&{double}) \\{rand}(\,))${}/\.{RAND\_MAX};{}$\6
\4${}\}{}$\2\7
\&{double} \\{exponential1}(\,)\1\1\2\2\6
${}\{{}$\1\6
\&{int} \|Z${}\K\T{0};{}$\6
\&{double} \|Y;\6
\&{int} \|k;\7
\&{do}\5
${}\{{}$\1\6
${}\|Y\K\\{uniform01}(\,);{}$\6
${}\|k\K\T{1};{}$\7
\&{double} \|W${}\K\|Y;{}$\6
\&{int} \\{stop}${}\K\T{0};{}$\7
\&{do}\5
${}\{{}$\1\6
\&{double} \|U${}\K\\{uniform01}(\,);{}$\7
\&{if} ${}(\|U>\|W){}$\5
${}\{{}$\1\6
${}\\{stop}\K\T{1};{}$\6
\4${}\}{}$\2\6
\&{else}\5
${}\{{}$\1\6
${}\|W\K\|U;{}$\6
${}\PP\|k;{}$\6
\4${}\}{}$\2\6
\4${}\}{}$\2\5
\&{while} ${}(\R\\{stop});{}$\6
${}\PP\|Z;{}$\6
\4${}\}{}$\2\5
\&{while} ${}(\|k\MOD\T{2}\E\T{0});{}$\6
\&{return} ${}((\&{double})(\|Z-\T{1}))+\|Y;{}$\6
\4${}\}{}$\2\par
\U4.\fi

\M{7}We will also make use of the gamma distribution, which, for a parameter
$a>0$, has density
$$f(x) = {1\over \Gamma(a)}x^{a-1}e^{-x}.$$
We use an algorithm of R.~C.~H.~Cheng [{\sl Applied Statistics} {\bf 26}
(1977), 71--75], described on
p.~413 of {\sl Non-Uniform Random Variate Generation}, which works for $a\ge
1$. (Note that there is
a mistake in the printed algorithm, but the corrigenda on Devroye's website
contain the required amendments.)

\Y\B\4\X7:Gamma random variate\X${}\E{}$\6
\&{double} \\{gamma\_dist}(\&{double} \|a)\1\1\2\2\6
${}\{{}$\1\6
\&{double} \|b${}\K\|a-\\{log}(\T{4});{}$\6
\&{double} \\{lambda}${}\K\\{sqrt}(\T{2}*\|a-\T{1});{}$\6
\&{double} \|c${}\K\|a+\\{lambda};{}$\6
\&{int} \\{accept}${}\K\T{0};{}$\6
\&{double} \|U${},{}$ \|V${},{}$ \|Y${},{}$ \|X${},{}$ \|Z${},{}$ \|R;\6
\&{double} \|S${}\K\T{4.5}*\|Z-(\T{1}+\\{log}(\T{4.5}));{}$\7
\&{do}\5
${}\{{}$\1\6
${}\|U\K\\{uniform01}(\,);{}$\6
${}\|V\K\\{uniform01}(\,);{}$\6
${}\|Y\K(\T{1.0}/\\{lambda})*\\{log}(\|V/(\T{1}-\|V));{}$\6
${}\|X\K\|a*\\{exp}(\|Y);{}$\6
${}\|Z\K\|U*\|V*\|V;{}$\6
${}\|R\K\|b+\|c*\|Y-\|X;{}$\6
${}\\{accept}\K(\|R\G\|S);{}$\6
\&{if} ${}(\R\\{accept}){}$\1\5
${}\\{accept}\K(\|R\G\\{log}(\|Z));{}$\2\6
\4${}\}{}$\2\5
\&{while} ${}(\R\\{accept});{}$\6
\&{return} \|X;\6
\4${}\}{}$\2\par
\U4.\fi

\M{8}To get a uniform random point from the unit ball in $L^p(\RR^d)$, we
sample $d$ independent
random variables, call them $X_1,\ldots, X_d$ from the density
$$f(x) = {1\over 2\Gamma(1+1/p)} e^{-\vert x\vert ^p}.$$
This is called an exponential power distribution, and Devroye notes in {\sl
Non-Uniform Random Variate Generation}
that if $X = VY^{1/p}$ with $V$ uniformly distributed on $[-1,1]$ and $Y$ is
gamma-distributed with parameter
$1+1/p$, then $X$ has the density $f(x)$.
We also sample an exponential random variable $Y$ with mean $1$, and then
output the random vector
$${(X_1, \ldots, X_d) \over (Y+ \sum_{i=1}^d \vert X_i\vert^p)^{1/p}}.$$
This method and extensions thereof are described by F.~Barthe, O.~Guedon,
S.~Mendelson, and A.~Naor
[{\sl Annals of Probability} {\bf 33} (2005), 480--513].
The function we write returns a point by modifying an
array that is passed as an argument.

\Y\B\4\X8:Uniform random point in unit ball\X${}\E{}$\6
\&{void} \\{random\_point}(\&{double} ${}{*}\\{point}){}$\1\1\2\2\6
${}\{{}$\1\6
\&{double} ${}{*}\|X\K\\{malloc}((\|d+\T{1})*\&{sizeof}(\&{double}));{}$\7
\&{for} (\&{int} \|i${}\K\T{1};{}$ ${}\|i\Z\|d;{}$ ${}\PP\|i){}$\1\5
${}\|X[\|i]\K\T{0.0};{}$\2\6
\&{if} ${}(\|p<\T{0}){}$\5
${}\{{}$\1\6
\X9:The case $p=\infty$\X;\6
\4${}\}{}$\2\6
\&{else}\5
${}\{{}$\1\6
\X10:The case $p\ne\infty$\X;\6
\4${}\}{}$\2\6
\\{free}(\|X);\6
\4${}\}{}$\2\par
\U4.\fi

\M{9}When $p=\infty$, we simply return a point uniformly from the box
$[-1,1]^d$.

\Y\B\4\X9:The case $p=\infty$\X${}\E{}$\6
\&{for} (\&{int} \|i${}\K\T{1};{}$ ${}\|i\Z\|d;{}$ ${}\PP\|i){}$\1\5
${}\\{point}[\|i]\K\T{2}*\\{uniform01}(\,)-\T{1}{}$;\2\par
\U8.\fi

\M{10}In all the other cases, we sample using the method described above.

\Y\B\4\X10:The case $p\ne\infty$\X${}\E{}$\6
\&{for} (\&{int} \|i${}\K\T{1};{}$ ${}\|i\Z\|d;{}$ ${}\PP\|i){}$\5
${}\{{}$\1\6
\&{double} \|V${}\K\T{2}*\\{uniform01}(\,)-\T{1};{}$\7
${}\|X[\|i]\K\|V*\\{pow}(\\{gamma\_dist}(\T{1}+\\{p\_inv}),\39\\{p\_inv});{}$\6
\4${}\}{}$\2\7
\&{double} \|Y${}\K\\{exponential1}(\,);{}$\6
\&{double} \\{pow\_sum}${}\K\|Y;{}$\7
\&{for} (\&{int} \|i${}\K\T{1};{}$ ${}\|i\Z\|d;{}$ ${}\PP\|i){}$\1\5
${}\\{pow\_sum}\MRL{+{\K}}\\{pow}(\\{fabs}(\|X[\|i]),\39\|p);{}$\2\7
\&{double} \\{scale\_factor}${}\K\T{1.0}/(\\{pow}(\\{pow\_sum},\39\\{p%
\_inv}));{}$\7
\&{for} (\&{int} \|i${}\K\T{1};{}$ ${}\|i\Z\|d;{}$ ${}\PP\|i){}$\1\5
${}\\{point}[\|i]\K\\{scale\_factor}*\|X[\|i]{}$;\2\par
\U8.\fi

\M{11}I originally needed a normal random deviate to handle the case $p=2$, so
I added this function,
but it is no longer necessary since we now have a general function. I left it
here just for kicks.
We generate the normal using the Box-Muller transform,
due to G.~E.~P.~Box and M.~E.~Muller [{\sl Annals of Mathematical Statistics} {%
\bf 29} (1958), 610--611].
If $U_1$ and $U_2$ are two independent random variables uniformly distributed
in $[0,1]$, then
$$V_1 = \sqrt{-2\ln U_1}\cos(2\pi U_2)\qquad\hbox{and}\qquad V_2 = \sqrt{-2\ln
U_1}\sin(2\pi U_2)$$
are independent normal random variables with mean $0$ and variance $1$. To make
the return type
of our function simpler, we
simply output one of these values (so we will end up calling this function
twice as often as is actually
necessary).

\Y\B\4\X11:Normal random variate\X${}\E{}$\6
\&{double} \\{normal01}(\,)\1\1\2\2\6
${}\{{}$\1\6
\&{double} \.{U1}${}\K\\{uniform01}(\,);{}$\6
\&{double} \.{U2}${}\K\\{uniform01}(\,);{}$\6
\&{double} \\{scale}${}\K\\{sqrt}({-}\T{2.0}*\\{log}(\.{U1}));{}$\7
\&{return} \\{scale}${}*\\{cos}(\T{2}*\.{M\_PI}*\.{U2});{}$\6
\4${}\}{}$\2\par
\U4.\fi

\N{1}{12}File output.
When $d=2$, it is easy to plot our set of points graphically. Our program does
this by generating a PostScript
file.

\Y\B\4\X12:Print points to PostScript file\X${}\E{}$\6
\&{void} \\{plot\_single\_point}(\&{FILE} ${}{*}\\{file},\39{}$\&{double} %
\\{red}${},\39{}$\&{double} \\{green}${},\39{}$\&{double} \\{blue}${},\39{}$%
\&{double} \|x${},\39{}$\&{double} \|y)\1\1\2\2\6
${}\{{}$\1\6
${}\\{fprintf}(\\{file},\39\.{"\%f\ \%f\ \%f\ setrgbcolo}\)\.{r\ \%f\ \%f\ dot%
\\n"},\39\\{red},\39\\{green},\39\\{blue},\39\|x,\39\|y);{}$\6
\4${}\}{}$\2\7
\&{void} \\{to\_postscript}(\&{const} \&{char} ${}{*}\\{filename},\39{}$\&{int}
\\{radius})\1\1\2\2\6
${}\{{}$\1\6
\&{if} ${}(\|d\I\T{2}){}$\5
${}\{{}$\1\6
\\{printf}(\.{"I\ cannot\ output\ Pos}\)\.{tScript\ unless\ d\ equ}\)\.{als\ 2!%
\\n"});\6
\&{return};\6
\4${}\}{}$\2\7
\&{FILE} ${}{*}\\{file}\K\\{fopen}(\\{filename},\39\.{"w"});{}$\7
\X13:Preamble\X;\6
\X14:Plot the points array\X;\6
${}\\{fprintf}(\\{file},\39\.{"showpage\\n"});{}$\6
\\{fclose}(\\{file});\6
\4${}\}{}$\2\par
\U2.\fi

\M{13}We first add a bare-bones preamble to the file to draw the axes and
declare various PostScript functions.
The axes are drawn with center at $(\PB{\\{radius}}, \PB{\\{radius}})$

\Y\B\4\X13:Preamble\X${}\E{}$\6
\&{char} ${}{*}\\{preamble}\K\.{"\%!PS\\n/dot\ \{\ 1.5\ 0\ }\)\.{360\ arc\
closepath\ fi}\)\.{ll\ \}\ def\\n/square\ \{\ }\)\.{/r\ exch\ def\ /y\ exch\ }%
\)\.{def\ /x\ exch\ def\\nnew}\)\.{path\ x\ r\ sub\ y\ r\ sub}\)\.{\ moveto\ 0\
r\ 2\ mul\ rl}\)\.{ineto\\nr\ 2\ mul\ 0\ rli}\)\.{neto\ 0\ r\ 2\ mul\ neg\ r}\)%
\.{lineto\\nr\ 2\ mul\ neg\ }\)\.{0\ rlineto\ closepath\ }\)\.{fill\}\ def%
\\n0.5\ setli}\)\.{newidth\\n"};{}$\7
${}\\{fprintf}(\\{file},\39\.{"\%s"},\39\\{preamble});{}$\6
${}\\{fprintf}(\\{file},\39\.{"\%d\ \%d\ translate\\n"},\39\\{radius},\39%
\\{radius});{}$\6
${}\\{fprintf}(\\{file},\39\.{"newpath\ 0\ \%d\ moveto}\)\.{\ 0\ \%d\ lineto\
"},\39{-}\\{radius},\39\\{radius});{}$\6
${}\\{fprintf}(\\{file},\39\.{"\%d\ 0\ moveto\ \%d\ 0\ li}\)\.{neto\ stroke%
\\n"},\39{-}\\{radius},\39\\{radius}){}$;\par
\Us12\ET23.\fi

\M{14}Next, we plot each of the elements of the \PB{\\{points}} array, with a
colour gradient to indicate the relative
orderings of points. The first point in the array is drawn in \PB{\\{colour1}},
the final point is drawn in
\PB{\\{colour2}}, and points in between have their colours interpolated
accordingly.

\Y\B\4\X14:Plot the points array\X${}\E{}$\6
\&{double} \\{colour1}[\,]${}\K\{\T{1.0},\39\T{0.0},\39\T{0.0}\}{}$;\C{ red }\6
\&{double} \\{colour2}[\,]${}\K\{\T{0.0},\39\T{0.0},\39\T{1.0}\}{}$;\C{ blue }\7
\&{for} (\&{int} \|i${}\K\T{0};{}$ ${}\|i<\\{num\_points};{}$ ${}\PP\|i){}$\5
${}\{{}$\1\6
\&{double} \|t${}\K{}$((\&{double}) \|i)${}/((\&{double})(\\{num\_points}-%
\T{1}));{}$\6
\&{double} \|r${}\K(\T{1.0}-\|t)*\\{colour1}[\T{0}]+\|t*\\{colour2}[\T{0}];{}$\6
\&{double} \|g${}\K(\T{1.0}-\|t)*\\{colour1}[\T{1}]+\|t*\\{colour2}[\T{1}];{}$\6
\&{double} \|b${}\K(\T{1.0}-\|t)*\\{colour1}[\T{2}]+\|t*\\{colour2}[\T{2}];{}$\7
${}\\{plot\_single\_point}(\\{file},\39\|r,\39\|g,\39\|b,\39\\{points}[\|i][%
\T{1}]*\\{radius},\39\\{points}[\|i][\T{2}]*\\{radius});{}$\6
\4${}\}{}$\2\par
\U12.\fi

\N{1}{15}Growing a committee by consensus voting.
Consider the point set as representing a group of people (each person is a
vector of real numbers
each describing a different trait). The committee grows in discrete time steps.
At each step, two
candidates are presented to the committee, and each committee member votes for
the candidate closest
to itself in $L^p$-distance. A candidate can only win by {\it consensus}; that
is, if any two committee
members vote for different candidates, then the election is inconclusive and no
new member is added.

\Y\B\4\X15:Committee-generating algorithm\X${}\E{}$\6
\X16:Distance computation\X;\6
\X17:Consensus algorithm\X;\par
\U2.\fi

\M{16}First we need to compute distances between two points $x = (x_1,\ldots,
x_d)$
and $y = (y_1,\ldots, y_d)$ in $L^p(\RR^d)$. When $p\neq\infty$, this is given
by the formula
$$\norm{x-y}_p = \Bigl(\sum_{i=1}^d \vert x_i-y_i\vert ^p\Bigr)^{1/p},$$
and when $p=\infty$, we simply take the maximum of the coordinate-wise
distances:
$$\norm{x-y}_\infty = \max_{1\leq i\leq d} \vert x_i - y_i\vert$$

\Y\B\4\X16:Distance computation\X${}\E{}$\6
\&{double} \\{p\_dist}(\&{double} ${}{*}\|x,\39{}$\&{double} ${}{*}\|y){}$\1\1%
\2\2\6
${}\{{}$\1\6
\&{if} ${}(\|p<\T{0}){}$\5
${}\{{}$\1\6
\&{double} \\{max}${}\K\.{DBL\_MIN};{}$\7
\&{for} (\&{int} \|i${}\K\T{1};{}$ ${}\|i\Z\|d;{}$ ${}\PP\|i){}$\5
${}\{{}$\1\6
\&{double} \\{diff}${}\K\\{fabs}(\|x[\|i]-\|y[\|i]);{}$\7
\&{if} ${}(\\{diff}>\\{max}){}$\1\5
${}\\{max}\K\\{diff};{}$\2\6
\4${}\}{}$\2\6
\&{return} \\{max};\6
\4${}\}{}$\2\6
\&{else}\5
${}\{{}$\1\6
\&{double} \\{sum}${}\K\T{0};{}$\7
\&{for} (\&{int} \|i${}\K\T{1};{}$ ${}\|i\Z\|d;{}$ ${}\PP\|i){}$\1\5
${}\\{sum}\MRL{+{\K}}\\{pow}(\\{fabs}(\|x[\|i]-\|y[\|i]),\39\|p);{}$\2\6
\&{return} \\{pow}${}(\\{sum},\39\T{1.0}/\|p);{}$\6
\4${}\}{}$\2\6
\4${}\}{}$\2\par
\U15.\fi

\M{17}Now we present the algorithm for growing the committee, which takes a
parameter indicating how many
rounds of voting should be undertaken. We start by clearing the point set and
initialising the committee with a point chosen uniformly at random from the
unit ball (this counts
as the first round of voting). In each round,
we compute distances from each point to the two candidates and keep track of
whether each candidate has
votes. If at any point, both candidates have votes, we can continue to the next
election, since that round
is inconclusive. The return value is the size of the committee after all rounds
have elapsed.

\Y\B\4\X17:Consensus algorithm\X${}\E{}$\6
\&{int} \\{consensus}(\&{int} \\{max\_t}${},\39\&{double}({*}\\{dist}){}$(%
\&{double} ${}{*},\39{}$\&{double} ${}{*})){}$\1\1\2\2\6
${}\{{}$\1\6
\\{clear\_points}(\,);\7
\&{double} ${}{*}\\{cand1}\K\\{malloc}((\|d+\T{1})*\&{sizeof}(\&{double}));{}$\6
\&{double} ${}{*}\\{cand2}\K\\{malloc}((\|d+\T{1})*\&{sizeof}(\&{double}));{}$\7
\\{random\_point}(\\{cand1});\6
\\{add\_point}(\\{cand1});\6
\&{for} (\&{int} \|t${}\K\T{0};{}$ ${}\|t<\\{max\_t}-\T{1};{}$ ${}\PP\|t){}$\5
${}\{{}$\1\6
\\{random\_point}(\\{cand1});\6
\\{random\_point}(\\{cand2});\7
\&{int} \\{voted1}${}\K\T{0};{}$\6
\&{int} \\{voted2}${}\K\T{0};{}$\7
\&{for} (\&{int} \|i${}\K\T{0};{}$ ${}\|i<\\{num\_points};{}$ ${}\PP\|i){}$\5
${}\{{}$\1\6
\&{double} \\{dist1}${}\K\\{dist}(\\{points}[\|i],\39\\{cand1});{}$\6
\&{double} \\{dist2}${}\K\\{dist}(\\{points}[\|i],\39\\{cand2});{}$\7
\&{if} ${}(\\{dist1}>\\{dist2}){}$\5
${}\{{}$\1\6
${}\\{voted1}\K\T{1};{}$\6
\4${}\}{}$\2\6
\&{else}\5
${}\{{}$\1\6
${}\\{voted2}\K\T{1};{}$\6
\4${}\}{}$\2\6
\&{if} ${}(\\{voted1}\W\\{voted2}){}$\1\5
\&{break};\2\6
\4${}\}{}$\2\6
\&{if} ${}(\\{voted1}\W\R\\{voted2}){}$\1\5
\\{add\_point}(\\{cand1});\2\6
\&{if} ${}(\\{voted2}\W\R\\{voted1}){}$\1\5
\\{add\_point}(\\{cand2});\2\6
\4${}\}{}$\2\6
\\{free}(\\{cand1});\6
\\{free}(\\{cand2});\6
\&{return} \\{num\_points};\6
\4${}\}{}$\2\par
\U15.\fi

\N{1}{18}Illustrations and examples.
Here are some examples of the output of the committee-generating algorithm. In
each case, $1\,000,000$
rounds of voting were conducted. In each figure, redder points were added
earlier and bluer points were
added later.

For $p=\infty$, a committee of 20 members was formed:
$$\epsfbox{infinity.ps}$$
For $p=1$, a committee of $16$ members was formed:
$$\epsfbox{one.ps}$$
And for $p=2$, a committee of $33$ members was formed:
$$\epsfbox{two.ps}$$
These committee sizes seemed representative of typical behaviour, in that
running the program multiple times
with the same parameters did not produce wildly different values in any of the
cases.

\fi

\N{1}{19}Orthogonal committees.
We now restrict ourselves to the case $p = 2$, when $L^p(\RR^d)$ is an
inner-product space. We can perform
the same experiment, but instead of measuring distance with the $L^p$-norm, we
can measure distance
between two vectors as the absolute value of their inner product.

\Y\B\4\X19:Inner product computation\X${}\E{}$\6
\&{double} \\{inner\_product}(\&{double} ${}{*}\|x,\39{}$\&{double} ${}{*}%
\|y){}$\1\1\2\2\6
${}\{{}$\1\6
\&{if} ${}(\|p\I\T{2}){}$\5
${}\{{}$\1\6
\\{printf}(\.{"I\ can\ only\ take\ inn}\)\.{er\ products\ when\ p\ =}\)\.{\ 2!%
\\n"});\6
\&{return} \T{0.0};\6
\4${}\}{}$\2\7
\&{double} \\{sum}${}\K\T{0.0};{}$\7
\&{for} (\&{int} \|i${}\K\T{1};{}$ ${}\|i\Z\|d;{}$ ${}\PP\|i){}$\1\5
${}\\{sum}\MRL{+{\K}}\|x[\|i]*\|y[\|i];{}$\2\6
\&{return} \\{sum};\6
\4${}\}{}$\2\7
\&{double} \\{abs\_inner\_product}(\&{double} ${}{*}\|x,\39{}$\&{double} ${}{*}%
\|y){}$\1\1\2\2\6
${}\{{}$\1\6
\&{return} \\{fabs}${}(\\{inner\_product}(\|x,\39\|y));{}$\6
\4${}\}{}$\2\7
\&{double} \\{one\_minus\_inner\_product}(\&{double} ${}{*}\|x,\39{}$\&{double}
${}{*}\|y){}$\1\1\2\2\6
${}\{{}$\1\6
\&{return} \T{1.0}${}-\\{inner\_product}(\|x,\39\|y);{}$\6
\4${}\}{}$\2\par
\U2.\fi

\N{1}{20}Locus of points further to committee than candidate.
In this section, we consider the case when $p=\infty$ and $d=2$. Given a
committee $G$ of points and a candidate
$c$, we will draw the locus of all points $c'$ such that the committee $G$ will
reach a consensus and elect
$c$ over $c'$.

\Y\B\4\X20:Draw loci when $p=\infty$\X${}\E{}$\6
\X21:Draw squares in PostScript\X;\6
\X22:Locus computation\X;\6
\X23:Generate a drawing with random points\X;\par
\U2.\fi

\M{21}First, we supply a facility to draw $L^\infty$ balls, which look like
squares, in PostScript.

\Y\B\4\X21:Draw squares in PostScript\X${}\E{}$\6
\&{void} \\{draw\_square}(\&{FILE} ${}{*}\\{file},\39{}$\&{double} \\{red}${},%
\39{}$\&{double} \\{green}${},\39{}$\&{double} \\{blue}${},\39{}$\&{double} %
\|x${},\39{}$\&{double} \|y${},\39{}$\&{double} \|r)\1\1\2\2\6
${}\{{}$\1\6
${}\\{fprintf}(\\{file},\39\.{"\%f\ \%f\ \%f\ setrgbcolo}\)\.{r\ \%f\ \%f\ \%f\
square\\n"},\39\\{red},\39\\{green},\39\\{blue},\39\|x,\39\|y,\39\|r);{}$\6
\4${}\}{}$\2\par
\U20.\fi

\M{22}Next, given a set of points $G = \{g_1, \ldots, g_n\}$
representing a committee as well as a candidate point $c$, the locus we seek is
$$[0,1]^2\setminus \bigcup_{i=1}^n B_\infty\bigl(g_i, d_\infty(g_i, c)\bigr),$$
where $d_\infty(x,y) = \norm{x-y}_\infty$ and $B_\infty(x,r)$ is the set of all
points $y$ with $d_\infty(x,y) <r$.
The way we draw this is to first fill in $[0,1]^2$ in red, then superimpose
various balls in blue. The remaining
red area is the locus of all points that is further from each committee member
than the candidate is.

\Y\B\4\X22:Locus computation\X${}\E{}$\6
\&{void} \\{draw\_locus}(\&{FILE} ${}{*}\\{file},\39{}$\&{int} \\{committee%
\_size}${},\39{}$\&{double} ${}{*}{*}\\{committee},\39{}$\&{double} ${}{*}%
\\{candidate},\39{}$\&{int} \\{radius})\1\1\2\2\6
${}\{{}$\1\6
${}\\{draw\_square}(\\{file},\39\T{1},\39\T{0},\39\T{0},\39\T{0},\39\T{0},\39%
\\{radius}){}$;\C{ background drawn in red }\6
\&{for} (\&{int} \|i${}\K\T{0};{}$ ${}\|i<\\{committee\_size};{}$ ${}\PP\|i){}$%
\5
${}\{{}$\C{ complement of locus drawn in blue }\1\6
${}\\{draw\_square}(\\{file},\39\T{0.2},\39\T{0},\39\T{1},\39\\{committee}[%
\|i][\T{1}]*\\{radius},\39\\{committee}[\|i][\T{2}]*\\{radius},\39\\{p\_dist}(%
\\{committee}[\|i],\39\\{candidate})*\\{radius});{}$\6
\4${}\}{}$\2\6
\&{for} (\&{int} \|i${}\K\T{0};{}$ ${}\|i<\\{committee\_size};{}$ ${}\PP\|i){}$%
\5
${}\{{}$\C{ draw committee in black }\1\6
${}\\{plot\_single\_point}(\\{file},\39\T{0},\39\T{0},\39\T{0},\39%
\\{committee}[\|i][\T{1}]*\\{radius},\39\\{committee}[\|i][\T{2}]*%
\\{radius});{}$\6
\4${}\}{}$\2\6
${}\\{plot\_single\_point}(\\{file},\39\T{0},\39\T{1},\39\T{0},\39%
\\{candidate}[\T{1}]*\\{radius},\39\\{candidate}[\T{2}]*\\{radius}){}$;\C{
candidate in green }\6
${}\\{draw\_square}(\\{file},\39\T{1},\39\T{1},\39\T{1},\39\T{2}*\\{radius},\39%
\T{0},\39\\{radius}){}$;\C{ clean up with white squares }\6
${}\\{draw\_square}(\\{file},\39\T{1},\39\T{1},\39\T{1},\39\T{0},\39\T{2}*%
\\{radius},\39\\{radius});{}$\6
${}\\{draw\_square}(\\{file},\39\T{1},\39\T{1},\39\T{1},\39\T{2}*\\{radius},\39%
\T{2}*\\{radius},\39\\{radius});{}$\6
\4${}\}{}$\2\par
\U20.\fi

\M{23}Lastly, we sample random points and generate the corresponding drawing.

\Y\B\4\X23:Generate a drawing with random points\X${}\E{}$\6
\&{void} \\{random\_locus}(\&{int} \\{committee\_size}${},\39{}$\&{const} %
\&{char} ${}{*}\\{filename},\39{}$\&{int} \\{radius}${},\39{}$\&{int} \\{num%
\_pages})\1\1\2\2\6
${}\{{}$\1\6
\&{double} ${}{*}\\{candidate}\K\\{malloc}((\|d+\T{1})*\&{sizeof}(%
\&{double}));{}$\6
\&{double} ${}{*}{*}\\{committee}\K\\{malloc}(\\{committee\_size}*{}$%
\&{sizeof}(\&{double} ${}{*}));{}$\7
\&{for} (\&{int} \|i${}\K\T{0};{}$ ${}\|i<\\{committee\_size};{}$ ${}\PP\|i){}$%
\1\5
${}\\{committee}[\|i]\K\\{malloc}((\|d+\T{1})*\&{sizeof}(\&{double}));{}$\2\6
\&{if} ${}(\|d\I\T{2}){}$\5
${}\{{}$\1\6
\\{printf}(\.{"I\ cannot\ output\ Pos}\)\.{tScript\ unless\ d\ equ}\)\.{als\ 2!%
\\n"});\6
\&{return};\6
\4${}\}{}$\2\7
\&{FILE} ${}{*}\\{file}\K\\{fopen}(\\{filename},\39\.{"w"});{}$\7
\X13:Preamble\X;\6
\&{for} (\&{int} \\{page}${}\K\T{0};{}$ ${}\\{page}<\\{num\_pages};{}$ ${}\PP%
\\{page}){}$\5
${}\{{}$\1\6
\&{for} (\&{int} \|i${}\K\T{0};{}$ ${}\|i<\\{committee\_size};{}$ ${}\PP\|i){}$%
\1\5
\\{random\_point}(\\{committee}[\|i]);\2\6
\\{random\_point}(\\{candidate});\6
\&{if} ${}(\\{page}\I\T{0}){}$\1\5
${}\\{fprintf}(\\{file},\39\.{"\%d\ \%d\ translate\\n"},\39\\{radius},\39%
\\{radius});{}$\2\6
${}\\{draw\_locus}(\\{file},\39\\{committee\_size},\39\\{committee},\39%
\\{candidate},\39\\{radius});{}$\6
${}\\{fprintf}(\\{file},\39\.{"showpage\\n"});{}$\6
\4${}\}{}$\2\6
\\{fclose}(\\{file});\6
\\{free}(\\{candidate});\6
\&{for} (\&{int} \|i${}\K\T{0};{}$ ${}\|i<\\{committee\_size};{}$ ${}\PP\|i){}$%
\1\5
\\{free}(\\{committee}[\|i]);\2\6
\4${}\}{}$\2\par
\U20.\fi

\N{1}{24}Handling user input.
These components of the \PB{\\{main}} function deal with command-line input.
The program takes up to three command-line
arguments in the following order: \PB{\|p}, \PB{\|d}, and \PB{\\{max\_points}}.
If some or all of
these arguments are missing, we default to \PB{$\|p\K\T{2}$}, \PB{$\|d\K%
\T{2}$}, and \PB{$\\{max\_points}\K\T{300}$}.

\Y\B\4\X24:Gather input and initialise global variables\X${}\E{}$\6
$\|p\K\T{2.0};{}$\6
${}\|d\K\T{2};{}$\6
${}\\{max\_points}\K\T{300};{}$\6
${}\\{num\_points}\K\T{0};{}$\6
\&{if} ${}(\\{argc}\G\T{2}){}$\5
${}\{{}$\1\6
${}\|p\K\\{atof}(\\{argv}[\T{1}]);{}$\6
${}\\{p\_inv}\K\T{1.0}/\|p;{}$\6
\4${}\}{}$\2\6
\&{if} ${}(\\{argc}\G\T{3}){}$\5
${}\{{}$\1\6
${}\|d\K\\{atoi}(\\{argv}[\T{2}]);{}$\6
\&{if} ${}(\|d<\T{0}){}$\5
${}\{{}$\1\6
\\{printf}(\.{"I\ expect\ d\ to\ be\ >=}\)\.{\ 0.\\n"});\6
\&{return} \T{1};\6
\4${}\}{}$\2\6
\4${}\}{}$\2\6
\&{if} ${}(\\{argc}\G\T{4}){}$\5
${}\{{}$\1\6
${}\\{max\_points}\K\\{atoi}(\\{argv}[\T{3}]);{}$\6
\&{if} ${}(\\{max\_points}<\T{10}){}$\5
${}\{{}$\1\6
\\{printf}(\.{"I\ expect\ max\_points}\)\.{\ to\ be\ >=\ 0.\\n"});\6
\&{return} \T{1};\6
\4${}\}{}$\2\6
\4${}\}{}$\2\6
${}\\{points}\K\\{malloc}(\\{max\_points}*{}$\&{sizeof}(\&{double} ${}{*}));{}$%
\6
\&{for} (\&{int} \|i${}\K\T{0};{}$ ${}\|i<\\{max\_points};{}$ ${}\PP\|i){}$\1\5
${}\\{points}[\|i]\K\\{malloc}((\|d+\T{1})*\&{sizeof}(\&{double}));{}$\2\6
${}\\{printf}(\.{"Program\ started\ wit}\)\.{h\ p\ =\ \%f,\ d\ =\ \%d,\ an}\)%
\.{d\ max\_points\ =\ \%d\\n"},\39\|p,\39\|d,\39\\{max\_points}){}$;\par
\U2.\fi

\N{1}{25}Miscellaneous tests.
Here are collected some snippets that were used during testing phases.

\Y\B\4\X25:Sample mean test\X${}\E{}$\6
\&{double} \\{sum}${}\K\T{0.0};{}$\6
\&{int} \\{num\_samples}${}\K\T{0};{}$\7
\&{for} (\&{int} \|i${}\K\T{0};{}$ ${}\|i<\\{num\_samples};{}$ ${}\PP\|i){}$\5
${}\{{}$\1\6
\&{double} \\{sample}${}\K\\{exponential1}(\,);{}$\7
${}\\{sum}\MRL{+{\K}}\\{sample};{}$\6
${}\\{printf}(\.{"\%f\\n"},\39\\{sample});{}$\6
\4${}\}{}$\2\6
${}\\{printf}(\.{"Sample\ mean:\ \%f\\n"},\39\\{sum}/\\{num\_samples}){}$;\par
\fi

\M{26}\B\X26:Generate random point cloud\X${}\E{}$\6
\&{double} ${}{*}\\{point}\K\\{malloc}((\|d+\T{1})*\&{sizeof}(\&{double}));{}$\7
\&{for} (\&{int} \|i${}\K\T{0};{}$ ${}\|i<\\{max\_points};{}$ ${}\PP\|i){}$\5
${}\{{}$\1\6
\\{random\_point}(\\{point});\6
\\{add\_point}(\\{point});\6
\4${}\}{}$\2\6
\\{free}(\\{point});\par
\fi

\M{27}\B\X27:Test $L^p$\X${}\E{}$\6
\&{int} \\{rounds}${}\K\\{max\_points}*\\{max\_points};{}$\7
${}\\{printf}(\.{"L\^p-distance\ voting}\)\.{\ for\ \%d\ rounds\ produ}\)\.{ced%
\ a\ committee\ with}\)\.{\ \%d\ members.\\n"},\39\\{rounds},\39\\{consensus}(%
\\{rounds},\39\\{p\_dist})){}$;\par
\fi

\M{28}\B\X28:Test orthogonal\X${}\E{}$\6
\&{int} \\{rounds}${}\K\\{max\_points};{}$\7
${}\\{printf}(\.{"Orthogonal\ voting\ f}\)\.{or\ \%d\ rounds\ produce}\)\.{d\ a%
\ committee\ with\ \%}\)\.{d\ members.\\n"},\39\\{rounds},\39\\{consensus}(%
\\{rounds},\39\\{abs\_inner\_product})){}$;\par
\fi

\M{29}\B\X29:Test close inner product\X${}\E{}$\6
\&{int} \\{rounds}${}\K\\{max\_points};{}$\7
${}\\{printf}(\.{"Close\ inner\ product}\)\.{\ voting\ for\ \%d\ round}\)\.{s\
produced\ a\ committ}\)\.{ee\ with\ \%d\ members.\\}\)\.{n"},\39\\{rounds},\39%
\\{consensus}(\\{rounds},\39\\{one\_minus\_inner\_product})){}$;\par
\fi

\N{1}{30}Index.
\fi

\inx
\fin
\con
