\input fontmac
\input mathmac

\def\down{\downarrow\!}
\def\up{\uparrow\!}
\maketitle{Posets and incidence algebras}{by}{Marcel K. Goh}{23 May 2021}

\advsect Preliminaries

A {\it partially ordered set} or {\it poset}
is a pair $(P,\leq)$ where $P$ is a ground set and $\leq$ is a binary relation on $P$
that satisfies the following three properties.
\medskip\axiombegin P
\axiom P1. [Reflexivity.] For all $x\in P$, $x\leq x$.
\smallskip
\axiom P2. [Antisymmetry.] For all $x,y\in P$, $x\leq y$ and $y\leq x$ implies that $x=y$.
\smallskip
\axiom P3. [Transitivity.] For all $x,y,z\in P$, if $x\leq y$ and $y\leq z$, then $x\leq z$.
\medskip
\noindent
When $x\leq y$, we write $y\geq x$; when $x\leq y$ and $x\neq y$, we may write $x<y$ and $y>x$.
Since it is cumbersome to always to refer to a poset as a pair,
will also allow ourselves to denote the entire poset by its ground set $P$ when no confusion can arise.

For our purposes, a {\it subposet} of a poset $(P,\leq)$ is obtained by taking a subset $S\subseteq P$
and declaring that $x\leq y$ in $(S,\leq)$ if and only if $x\leq y$ in $(P,\leq)$ (this is sometimes
called an {\it induced} subposet). An {\it order ideal} is a subset $I$ of $P$ such that if $x\in I$ and
$y<x$, then $y\in I$. Dually, a {\it filter} is a
subset $F$ of $P$ such that if $x\in I$ and $y>x$, then $y\in I$. The {\it principal order ideal} $\down x$
of an element $x\in P$ is the set of all $y\in P$ such that $y\leq x$; similarly, the {\it order ideal generated
by $X = \{x_1,x_2,\ldots,x_k\}\subseteq P$} is the set $\down X$
of all $y\in P$ such that $y\leq x_i$ for some $1\leq i\leq k$.
(One defines the dual notion of a {\it principal filter} $\up x$ and {\it filter $\up X$
generated by $X$} in the obvious way.)

Whenever $x\leq y$, we can define the {\it interval} $[x,y]$ to be the subposet $\{z\in P: x\leq z\leq y\}$.
If the interval $[x,y]$ is the two-element set $\{x,y\}$, then we say that $y$ {\it covers} $x$.
Given two elements $x,y\in P$, we define an {\it upper bound} to be an element $z\in P$ such that
$z\geq x$ and $z\geq y$. Similarly, we can define a {\it lower bound} of two elements, and by induction,
we obtain a notion of upper and lower bounds of any finite subset of $P$. If an upper bound $z$ of $x$ and $y$
is such that for any upper bound $w$ of $x$ and $y$, we have $z\leq w$, then $z$ is said to be the {\it least
upper bound} or {\it join}
of $x$ and $y$ and we may denote $z=x\vee y$. Dually, we define the {\it greatest lower bound} or {\it meet}
$x\wedge y$ of two elements $x$ and $y$. In either case, the existence of such an element
implies its uniqueness.



\bye
