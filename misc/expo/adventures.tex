\input fontmac
\input mathmac
\classicmode

\def\A{{\cal A}}
\def\C{{\cal C}}
\def\F{{\cal F}}

\def\edge{\mathrel-\!\!\!\mathrel-}

\maketitle{Adventures in additive combinatorics}{by}{Marcel K. Goh}{18 December 2020}

\advsect Introduction

This set of notes will contain neat theorems and problems that are (at least a little bit)
related to additive combinatorics. The individual sections are intended to be largely self-contained, and
will not delve too much into theory, but focus on famous problems and their solutions, especially if
the proofs are rather elegant (particular preference has been given to probabilistic proofs). None of these
proofs are new in any way; they all appear in one of the references listed at the end of the
document.

\medskip
\boldlabel Notation and terminology. The notation $[a,b]$ always denotes the {\it discrete interval}
$\{n\in \ZZ : a\leq n\leq b\}$. For a set $X$, we let $2^X$ denote the set of all subsets of $X$ (the {\it
power set}) and the notation $X^{(r)}$ indicates the collection of all subsets
of $X$ with cardinality $r$.

A {\it graph} $G$ is a pair $(V,E)$ where $E\subseteq V^{(2)}$. If, for $u,v\in
V$, we have $\{u,v\}\in E$, then we write $u \edge v$ and say that $u$ and $v$ are {\it adjacent}. The
{\it neighbourhood} of a node $u\in V$ is the set $N(u)=\{v\in V: u\edge v\}$, that is, the set of
nodes adjacent to $u$, and the {\it degree} of $u$, denoted $d(u)$, is the cardinality of this set.

\advsect The Littlewood-Offord problem

We begin with a simple problem, which J. E. Littlewood and A. C. Offord came across in 1943 while studying the
zeroes of random polynomials. Let $z_1, z_2, \ldots, z_n$ be complex numbers of modulus at least $1$ and form
all $2^n$ possible sums. How many of these sums can differ by less than $1$? To attack this problem, we will
need a couple of definitions and inequalities. Given a ground set $X$,
an {\it antichain} or {\it Sperner system} is a collection
of subsets $\F$ such that for any $A\subseteq B\in 2^X$, either $A\notin \F$ or $B\notin F$. In other words,
if all the elements of $2^X$ are ordered by inclusion, then no two elements of $\F$ are comparable. On the
other hand, a {\it chain} is a collection $\C = \{C_1, C_2, \ldots, C_k\}$ of subsets of $X$ such that
$C_1\subset C_2 \subset \cdots \subset C_k$.

The following inequality was proved independently by D.~Lubell, K.~Yamamoto, and L.~D.~Meshalkin,
and has thus become known as the LYM inequality. However, it also follows from an earlier result of B.~Bollob\'as,
and in fact, the probabilistic proof we present is due to Bollob\'as.

\parenproclaim Lemma B (LYM inequality). Let $X$ be a finite set of size $n$;
for simplicity, we assume that $X = [1,n]$. Let $\F\subseteq 2^X$ be an antichain; then
$$\sum_{A\in \F} {n\choose |A|}^{-1} \leq 1.$$

\proof Let $\sigma : X\to X$ be a permutation chosen uniformly from all $n!$ such permutations. For
$A\subseteq X$, $\sigma(A)$ denote the set $\{\sigma(x) : x\in A\}$ and consider the probability that
$\sigma(A) = [1,|A|]$. Note that $\sigma(A)$ is uniformly distributed over the ${n\choose |A|}$ members of
$X^{(|A|)}$, so the probability that it equals $[1,|A|]$ is ${n\choose |A|}^{-1}$. Note that for two sets
$A,B\in 2^X$, if $\sigma(A) = [1,|A|]$ and $\sigma(B) = [1,|B|]$ for the same fixed $\sigma$,
then either $A\subseteq B$ or $B\subseteq
A$. Since $\F$ is an antichain, the events $\sigma(A) = [1,|A|]$ must then be disjoint and the sum of their
probabilities is at most $1$. This proves the inequality.\slug

Because ${n \choose k}$ is maximal when $k = \lfloor n/2\rfloor$, we can derive the following theorem of
E. Sperner as a corollary.

\parenproclaim Corollary S (Sperner, {\rm 1928}).
Let $X$ be a finite set and let $\F\subseteq 2^X$ be an antichain. Then $\F$ has size at most ${n\choose \lfloor n/2\rfloor}$.

\proof We have, by Lemma B,
$${|A|\over {n\choose \lfloor n/2\rfloor}} \leq \sum_{A\in \F} {n\choose |A|}^{-1} \leq 1.$$
Multiplying on both sides by ${n\choose \lfloor n/2\rfloor}$ yields the result.\slug

With this theorem in hand, we can already solve the Littlewood-Offord problem in the special case
when all the $z_i$ are real. The
following construction is due to P. Erd\H os (1945). Let $x_1, x_2, \ldots, x_n$ be real numbers with absolute
value at least 1. For $A\subseteq [1,n]$, let $x_A = \sum_{i\in A}x_i$. We may assume all of the $x_i$ are
positive, because replacing $x_i$ with $-x_i$ and $A$ with $A \cup \{i\} \setminus \big(A\cap \{i\}\big)$ does not
change the relative differences between the $2^n$ sums.
This operation simply causes all the $x_A$ to permute amongst themselves and shift by $-x_i$. Now let $\F$
be a family of subsets of $[1,n]$ such that $|x_A-x_B|<1$ for every distinct pair of $A,B\in \F$. It is an
antichain because if $A$ is a proper subset of $B$ then $x_B>x_A$ and we have
$$|x_B - x_A| = \sum_{x\in B}x_i - \sum_{x\in A}x_i = \sum_{x\in B\setminus A}x_i = x_{B\setminus A},$$
and this is at least 1 since $B\setminus A$ is nonempty.
Hence at most one of $A$ and $B$ belongs to $\F$. By Corollary~S, $|\F|\leq {n\choose n/2}$.

\advsect Independent sets

For a graph $G=(V,E)$, an {\it independent set} is a subset $S\subseteq V$ of vertices such that no two
elements of $S$ are adjacent. The largest independent set in $G$ is denoted $\alpha(G)$.
In this section, we present two beautiful probabilistic proofs that give lower bounds for $\alpha(G)$.
The first is due to Tur\'an; sometimes this theorem is called Tur\'an's theorem, but more often, that name is
reserved for a somewhat dual statement about the maximum number of edges in a graph that has no
complete graph on $r+1$ vertices as a subgraph.

\proclaim Theorem T. Let $G=(V,E)$ be a graph. The size $\alpha(G)$ of the largest independent set in $G$
satisfies the inequality
$$\alpha(G) \geq \sum_{u\in V} {1\over d(u) + 1}.$$

\proof Suppose $V$ has size $n$ and let $\sigma:V\to [1,n]$ be a bijection, chosen uniformly at random from
all $n!$ such bijections. Regarding this as an ordering of the vertices, consider the set
$$S = \{ u \in V : \sigma(u) < \sigma(v)\ \hbox{for all}\ v\in N(u)\},$$
which consist of vertices that come before all of their neighbours in the ordering. For any ordering,
the set $S$ that it induces must be an independent set, for if two vertices $u,v\in V$ are adjacent
then the vertex which was assigned a larger value in the ordering cannot be in $S$. Fixing a node $u\in V$,
the probability that it is the first among all of its neighbours is $1/\big(d(u) + 1\big)$, and by linearity
of expectation, we have
$$\ex\{|S|\} = \ex\Big\{\sum_{u\in V} \indic{u\in S}\Big\} = \sum_{u\in V}{1\over d(u) + 1}.$$
So there exists an independent set $S$ with size at least the value of this sum, which was what we
wanted.\slug

In particular, if maximal degree over all vertices in $G$ is $d$, then there is an independent set of size
at least $n/(d+1)$.

\section References

\bye
