\input fontmac
\input mathmac

\input epsf
\input color
\input soul.sty

\def\norm#1{|\!|#1|\!|}
\def\threenorm#1{|\!|\!|#1|\!|\!|}
\def\bignorm#1{\big|\!\big|#1\big|\!\big|}
\def\Norm#1{\Big|\!\Big|#1\Big|\!\Big|}
\def\normm#1{\bigg|\!\bigg|#1\bigg|\!\bigg|}
\def\B{{\cal B}}
\def\supp{\op{supp}}

\widemargins
\bookheader{TSIRELSON'S SPACE}{MARCEL K. GOH}

\maketitle{Tsirelson's space and other exotic constructions}{}{Marcel K. Goh}{16 December 2021}

\floattext4.5 \ninebf Abstract.
\ninepoint
This set of expository notes was
written as a final report for the class MATH 567 Introduction to Functional Analysis,
taught by Prof.~Gantumur Tsogtgerel at McGill University in the Fall 2021 semester.
It builds up the necessary theory to describe Tsirelson's example of an infinite-dimensional
Banach space with no subspace isomorphic to $c_0$ or $l_p$ for $1\le p<\infty$, and
modifications thereof.

\advsect Preliminary notions

In this section we state definitions and well-known results that we shall require later on.
Proofs are not given for many of the facts given here; they can be found in introductory textbooks on
linear analysis (see, e.g.,~\ref{bollobas},~\ref{lindenstrauss},~\ref{schechter}).
The reader comfortable with the terminology of Banach space theory
may wish to skip this section, returning only upon finding an unfamiliar concept or definition.

Let $K$ denote either the real or complex field and let $V$ be a vector space over $K$. A
{\it norm} is a function $\norm{\cdot}:V\to\RR$ satisfying
\medskip
\item{i)} $\norm v \ge 0$ for all $v\in V$ and $\norm v = 0$ if and only if $v = 0$;
\smallskip
\item{ii)} $\norm{\alpha \cdot v} = |\alpha|\cdot\norm v$ for all $\alpha\in K$ and $v\in V$; and
\smallskip
\item{iii)} $\norm{v+w}\le \norm v + \norm w$ for all $v,w\in V$.
\medskip
A vector space equipped with such a function is called a {\it normed vector space}, and one can
define a metric $d$ on the space by setting $d(x,y) = \norm{x-y}$. Thus one has a notion of
Cauchy sequences in these spaces, and if a normed vector space $X$ is complete
(every Cauchy sequence in $X$ has a limit in $X$), we say that $X$ is a {\it Banach space}.
Any two norms on a finite-dimensional normed vector space are equivalent, in that they induce the
same topology. This fact can be used to show that any finite-dimensional normed vector space over
$\RR$ or $\CC$ is a Banach space, since these fields are complete.

The field $\RR$ or $\CC$ under the norm given by the absolute value $|\cdot|$ is of course the
simplest Banach space. Other basic examples include the following.
\medskip
\item{i)} Consider an $n$-dimensional vector space over $K$. Writing an element $x$ of this space
as $x = (x_1, \ldots, x_n)$, this vector space can be endowed with the {\it $p$-norm}
$$\norm{x}_p = \Bigl(\sum_{i=1}^n |x_i|^p\Bigr)^{1/p}.$$
for $1\le p<\infty$. This defines a Banach space which we shall denote by $l_p^n$. As $p\to\infty$,
this approaches the {\it maximum norm}
$$\norm{x}_\infty = \max_{1\le i\le n} |x_i|,$$
and we denote this Banach space by $l_\infty^n$.
\smallskip
\item{ii)} For $1\le p< \infty$, the $p$-norm for infinite sequences $x = (x_n)_{n=1}^\infty$ is given by
$$\norm{x}_{l_p} = \Bigl(\sum_{i=1}^\infty |x_i|^p \Bigr)^{1/p}.$$
As $p\to\infty$, this approaches the {\it supremum norm}
$$\norm{x}_{l_\infty} = \sup_{i\ge 1} |x_i|.$$
For $1\le p\le \infty$, the set $l_p$ of all infinite sequences $x$ with $\norm{x}_{l_p}$ finite is a Banach space;
when $p=\infty$, these are the bounded sequences.
\smallskip
\item{iii)} Let $c_0$ be the normed vector subspace of $l_\infty$ consisting of sequences that tend to zero.
It can be shown that this subspace is closed and thus itself a Banach space.
\medskip
The latter two examples are infinite-dimensional, meaning that in these cases one can exhibit an
infinite sequence of vectors that are all linearly independent.

\medskip\boldlabel Hamel bases.
Assuming the axiom of choice, one can
find a basis for any vector space. This is a set of elements $\{e_i\}_{i\in I}$ such that any
$x\in X$ can be expressed uniquely as $\sum_{i\in F} a_ie_i$ for a finite set $F\subseteq I$
and scalars $a_i$, both depending on $x$.
When dealing with infinite-dimensional spaces, such a basis
is called a {\it Hamel basis} and are not awfully useful. To see why, we first need to establish (a variant of)
the Baire category theorem, as well as simple lemma.

\parenproclaim Theorem B (Baire category theorem). Let $X$ be a metric space that equals the union of
countably many closed sets. If $X$ is complete, then at least one of the closed sets has nonempty interior.\slug

\proclaim Lemma \advthm. Let $X$ be a normed vector space and let $Y$ be a subspace of $X$. If $Y$ has nonempty
interior, then $Y=X$.

\proof Suppose that for some $r>0$ and $y\in Y$, the ball $B_r(y) = \{x\in X : \norm{x-y} < r\}$ is contained
in $Y$. Now let $x\in X$ and note that the vector
$$z = y + {r\over 2}\cdot{x\over\norm x}$$
is at distance $r/2$ from $y$, meaning that $z\in B_r(y)\subseteq Y$. Since $Y$ is a subspace, we find that
$$x = {2\norm x\over r}(z-y)$$
is in $Y$, completing the proof.\slug

We now show that Hamel bases of Banach spaces cannot be countably infinite.

\proclaim Lemma \advthm. Let $X$ be an infinite-dimensional normed vector space. If $X$ is also complete
(and thus a Banach space), then any Hamel basis of $X$ is uncountable.

\proof Suppose that $\{e_1\}_{i=1}^\infty$ is a countable Hamel basis for $X$. For integers $n\ge 1$,
let $X_n$ denote the linear span of $\{e_1,\ldots,e_n\}$. Each normed vector space $X_n$ is finite-dimensional
and thus complete, which implies that each $X_n$ is closed in $X$. But each of these subspaces is proper,
so by the lemma above, so all of these subspaces have empty interior, which contradicts the Baire category
theorem.\slug

Since many infinite-dimensional
Banach spaces have the cardinality of the continuum to begin with,
the above lemma tells us that Hamel bases can essentially be as complicated as the underlying space.
(In fact, it has been shown that
the cardinality of the Hamel basis of a Banach space always equals the cardinality of the space
itself~\ref{halbeisen2000}.)

\medskip\boldlabel Completions.
Consider now the infinite-dimensional vector space $c_{00}$ of sequences that are eventually zero. For all
integers $i\ge 1$, let $e_i$ denote the sequence that is $1$ at index $i$ and $0$ elsewhere. The set
$\{e_i\}_{i=1}^\infty$ is countable and clearly
a Hamel basis for $c_{00}$, so we conclude that $c_{00}$, under any norm, cannot be a Banach space. However,
for every normed vector space $V$, we can always find a Banach space $X$ such that $V$ is dense in $X$.
We do this by letting $X$ be the completion of $V$ as a metric space and then defining scalar multiplication
and addition as follows. If $x,y\in X$ are such that $x_n\to y$ and $y_n\to y$, then $\lambda x+\mu y$
shall simply be defined as $\lim_{n\to \infty} (\lambda x_n + \mu y_n)$. We leave it as an easy exercise
for the reader to check
that this makes $X$ a normed vector space under the norm $\norm x = \lim_{n\to\infty}\norm{x_n}$, and that
$X$ is complete with respect to this norm.

\medskip\boldlabel Dual spaces and the weak topology.
Let $X$ be a normed vector space.
A linear operator from $X$ to its base field is called a {\it linear functional},
and is continuous if and only if it is bounded.
The set of bounded linear functionals on a normed vector space $X$ is a Banach space under the operator norm
$$\norm{T} = \sup_{x\in X} {\norm {Tx}\over \norm{x}}.$$
This space is called the {\it dual space
of $X$} and is denoted $X^*$. If $(x_n)$ is a sequence of vectors in $X$ such that for some $x\in X$,
$x^*(x_n)\to x^*(x)$ for all $x^*\in X^*$, then we say that $(x_n)$ {\it converges weakly to $x$}.
If $(x_n)$ converges weakly to $0$, then we call $(x_n)$ {\it weakly null}. The topology that
this definition of convergence
induces on $X$ is called the {\it weak topology}; it is the coarsest topology such that every element
of $X^*$ is still a continuous function.
% When distinguishing the topology given by the norm from the
% weak one, we shall call the norm topology the {\it strong topology}.
% Since $x_n\to x$ implies that $x_n$ converges weakly
% to $x$, the weak topology is, in general, strictly coarser than the strong topology. However, there are
% many ``weak-to-strong'' principles in the theory of Banach spaces, the most celebrated of which is the following.
% 
% \parenproclaim Theorem U (Uniform boundedness principle). Let $X$ be a Banach space, $Y$ a normed vector space,
% and ${\cal T}$ a collection of linear operators from $X$ to $Y$. If
% $$\sup_{T\in{\cal T}} \bignorm{T(x)}_Y < \infty,$$
% then
% $$\sup_{T\in {\cal T}} \sup_{x\in X} {\bignorm{T(x)}_Y\over \norm{x}_X} <\infty.\noskipslug$$
% 
% The uniform boundedness principle can be used to show that any set of vectors
% that is bounded in the weak topology is also bounded in the strong topology.
% This fact will come in handy in a proof later on.

\advsect Schauder bases

We saw in the previous section that Hamel bases are not very useful for performing analysis on infinite-dimensional
normed vector spaces.
A better notion in the infinite-dimensional setting is that of a {\it Schauder basis}.
This is a countable sequence of vectors $(e_i)_{i=1}^\infty$
such that every $x\in X$ has a unique representation
$$x = \sum_{i=1}^\infty a_ie_i$$
for some sequence $(a_i)_{i=1}^\infty$ of scalars, where convergence of the infinite sum is defined in
terms of the metric induced by the norm. From here on out, we shall sometimes write ``basis'' to mean ``Schauder
basis'', for brevity's sake.
A basis $(e_i)$ for which there exists $C$ such that
$1/C \le \norm{e_i} \le C$ for all $i$ is called {\it seminormalised} and a seminormalised basis
whose corresponding constant $C$ equals $1$ is said to be {\it normalised}.
A sequence of vectors $(x_i)$
that is a basis for the closure of its linear span is called a {\it basic sequence}.

\medskip\boldlabel Equivalent bases.
A basis $(x_i)$ for a Banach space $X$ and a basis $(y_i)$ for a Banach space $Y$ are
said to be {\it equivalent} if the map $T$ sending $x_i$ to $y_i$ extends to a linear homeomorphism
between $X$ and $Y$. Of course, this relation is reflexive, symmetric, and transitive.
One can also show this to be the same as saying that there are constants $0<m\le M$ such that
for any sequence $(a_i)$ of scalars,
$$m\cdot\Norm{\sum_{i=1}^\infty a_ix_i}\le\Norm{\sum_{i=1}^\infty a_iy_i}\le
 M\cdot\Norm{\sum_{i=1}^\infty a_ix_i}.$$
To every Banach space $X$ with a basis $(e_i)_{i=1}^\infty$,
we can associate a sequence $(P_n)_{n=1}^\infty$ of bounded linear operators given by
$$P_n\Bigl(\sum_{i=1}^\infty a_ie_i\Bigr) = \sum_{i=1}^n a_ie_i.$$
The quantity $\sup_{n\ge 1} \norm{P_n}$ is finite and called the {\it basis constant} of $(e_i)$.
A basis $(e_i)$ with basis constant equal to $1$ is said to be {\it monotone}. Every Schauder basis
is monotone with respect to the norm $\threenorm{x} = \sup_n \norm{P_n x}$, and this norm happens to be
equivalent to the original one. Hence, given a basis, we can always pass to a norm under which the basis
is monotone.

\medskip\boldlabel Block basic sequences.
Let $(e_i)$ be a basic sequence in a Banach space. A sequence of vectors $(x_j)$, each of the form
$$x_j = \sum_{i=n_j+1}^{n_{j+1}} a_ie_i$$
where the $a_i$ are scalars and $n_1<n_2<\cdots$ is an increasing sequence of positive integers, is
called a {\it block basic sequence} or {\it block basis} of $(e_i)$.
Defining the {\it support} $\supp x$ of a vector
$ x = \sum_{i=1}^\infty a_i e_i$ with respect to a basis $(e_i)$
to be the set of indices $i$ for which $a_i$ is nonzero, a block basis is a sequence of vectors $x_j$ for
which $\supp x_j$ is finite for all $j$ and $\max \supp x_j < \min \supp x_{j+1}$ for all $j\ge 1$.

\medskip\boldlabel Unconditional bases.
Note that the definition of a Schauder basis may rely very much on the ordering
imposed on its elements. There are some bases for which this is not the case. The following conditions are
equivalent for a basis $(e_i)$ of a Banach space $X$.
\medskip
\item{i)} For every permutation $\pi$ of the positive integers, $\bigl(e_{\pi(i)}\bigr)_{i=1}^\infty$ is
a basis of $X$.
\smallskip
\item{ii)} If a sum of the form $\sum_{i=1}^\infty a_i e_i$ converges, then any reordering of its terms
produces a sum which converges to the same value.
\smallskip
\item{iii)} There exists a constant $C$ such that for all pairs of scalar sequences $(a_i)$ and $(b_i)$
with $|a_i|\le |b_i|$ for all $i$,
$$\Norm{\sum_{i=1}^\infty a_ie_i} \le C\Norm{\sum_{i=1}^\infty b_ie_i}.$$
\medskip
The proof of this equivalence can be found in Section~1.c of~\ref{lindenstrauss}. A basis satisfying any
(and thus all) of these properties is called an {\it unconditional basis}, and a basis is said to be
{\it $C$-unconditional} if it satisfies (iii) for some specific constant $C$.

\medskip\boldlabel Symmetric and subsymmetric bases. In the sequence spaces $c_0$ or $l_p$ for $1\le p<\infty$,
the canonical basis consists of vectors that have a $1$ at some index and zeroes everywhere else. Since
each coordinate is weighted equally in $l_\infty$ or $l_p$ norm, this basis is equivalent to any of its
permutations. Such a basic sequence in a Banach space is said to be {\it symmetric}. Note that
this is a bit stronger than the condition we saw above for
a basic sequence $(e_i)$ to be unconditional. In that case, we only needed $\bigl(e_{\pi(i)}\bigr)$ to be a
basic sequence for all permutations $\pi$.

An unconditional basis $(e_i)$ of a Banach space $X$ is called {\it subsymmetric} if for all $i_1<i_2<\cdots$,
the subsequence $\bigl(e_{i_k}\bigr)$ is equivalent to $(e_i)$. Note that this time, the requirement that
the basis be unconditional is not redundant. It can be shown, however, that every symmetric basis
is subsymmetric (see Prop.~3.a.3 of~\ref{lindenstrauss}); we shall use this fact in the next section.

Another fact that we'll need later regards a certain dichotomy in the family of subsymmetric basic sequences.

\newcount\dichotomy
\dichotomy=\thmcount
\proclaim Lemma \advthm. Let $(x_i)$ be a bounded subsymmetric basic sequence. Then $(x_i)$ is either equivalent
to the standard unit vector basis for $l_1$ or else $(x_i)$ is weakly null.

\proof If $(x_i)$ is weakly null, then we are done. If not, then we know that there exists some
$f\in X^*$ and $\delta>0$ such that $f(x_i) \ge \delta$ for all $i$. (If not, then $(x_i)$ would have a subsequence
that is weakly null, and thus by subsymmetry the sequence itself is weakly null.)
By scaling $f$ we can assume that it has
operator norm equal to $1$. For any $n$, and scalars $(b_1,\ldots,b_n)$ we have
$$\Norm{\sum_{i=1}^n |b_i| x_i} =\norm f\cdot \Norm{\sum_{i=1}^n |b_i| x_i} \ge f\Bigl(\sum_{i=1}^n |b_i|x_i\Bigr)
= \sum_{i=1}^n |b_i|f(x_i)\ge \delta \sum_{i=1}^n |b_i|.$$
Now let $C$ be the constant given by the unconditionality of $(x_i)$ and setting $a_i = |b_i|$, we have
$$\Norm{\sum_{i=1}^n |b_i|x_i} \le C \cdot\Norm{\sum_{i=1}^n b_ix_i}.$$
Putting everything together establishes
$$\Norm{\sum_{i=1}^n b_ix_i} \ge {1\over C} \Norm{\sum_{i=1}^n |b_i|x_i}\ge
{\delta\over C} \sum_{i=1}^n |b_i|$$
for all $n$, and taking $n$ to infinity gives a corresponding lower bound on all infinite sequences $(b_i)$.

The upper bound is a simple consequence of the triangle inequality. Let $M$ be the constant bounding $(x_i)$
from above. Then since $\norm{x_i}\le M$ for all $i$, for an arbitrary infinite sequence $(b_i)$ of scalars we have
$$\Norm{\sum_{i=1}^\infty b_ix_i}\le \sum_{i=1}^\infty |b_i|\cdot\norm{x_i}\le M \sum_{i=1}^\infty |b_i|,$$
which combines with the lower bound to give
$${\delta\over C} \sum_{i=1}^\infty |b_i| \le \Norm{\sum_{i=1}^\infty b_ix_i} \le M \sum_{i=1}^\infty |b_i|.$$
Hence $(x_i)$ is equivalent to the unit vector basis of $l_1$.
\slug

\advsect Tsirelson's space and its dual

For many decades, the infinite-dimensional
Banach spaces known to functional analysts all contained a subspace linearly homeomorphic
to $c_0$ or $l_p$ for $1\le p<\infty$. The first infinite-dimensional space without such a subspace was
constructed by B.~S.~Tsirelson in 1974~\ref{tsirelson},
and that same year T.~Figiel and W.~B.~Johnson gave a more explicit
characterisation of its dual~\ref{figieljohnson}.
In time, the dual of the original space has come to be known as $T$, with
Tsirelson's original space denoted $T^*$. Properties of Tsirelson's space and variations thereof are collected
in a book by P.~G.~Casazza and T.~J.~Shura~\ref{casazzashura}. We shall describe the space $T$ of
Figiel and Johnson in this section, following Casazza and Shura's exposition but supplying some of the proofs
that are omitted there.

Let $E$ and $E'$ be subsets of the positive integers. We write $E<F$ to mean that the maximum element of $E$
is less than the minimum element of $F$ and likewise we write $E\le F$ to mean the same thing, with
``less than'' replaced by ``at most''. A sequence of finitely many sets $E_1,E_2,\ldots,E_k$ of positive integers
is said to be {\it admissible} if
$$ \{k\} \le E_1 < E_2 < \cdots < E_k.$$
For $x\in c_{00}$ and a subset $E$ of positive integers, we write $Ex$ to denote the sequence that equals
$x_i$ at index $i$ for all $i\in E$ and $0$ elsewhere. For a fixed $x\in c_{00}$ there are only
finitely many $E$ such that $Ex$ is nonzero, since $x$ is finitely supported. Fixing a norm $\norm\cdot$
on $c_{00}$ now,
we say that
$$\sum_{i=1}^k \norm{E_i x}$$
is an {\it admissible sum for $x$} if $E_1,\ldots,E_k$ is admissible.

The following proposition establishes a sequence of norms on $c_{00}$ in terms of admissible sequences
of sets.

\newcount\inductive
\inductive=\thmcount
\proclaim Proposition \advthm.
For $x\in c_{00}$, we let $\norm{x}_0 = \norm{x}_{l_\infty}$
and for $n\ge 1$, we inductively set
$$\norm{x}_n = \max \biggl\{\norm{x}_{n-1}, {1\over 2}\max \sum_{i=1}^k \norm{E_i x}_{n-1}\biggr\},$$
where the inner maximum is taken over the finitely many admissible sequences $E_1, E_2, \ldots, E_k$ for which
the admissible sum for $x$ is nonzero. Then the following statements hold.
\medskip
\item{i)} For all $n\ge 0$, $\norm{\cdot}_n$ is a norm.
\smallskip
\item{ii)} For all $n\ge 0$ and $x\in c_{00}$, $\norm{x}_n\le \norm{x}_{l_1}$.
\smallskip
\item{iii)} The limit $\lim_{n\to \infty} \norm{x}_n$ exists and defines a norm on $c_{00}$.
\medskip

\proof The proof of (i) is inductive; the base case $n=0$ corresponds to $\norm\cdot _{l_\infty}$, which
we already know to be a norm. Now assume the statement proven for $n-1$. The only property that isn't quite
immediate is the triangle inequality. Note that for $x,y\in c_{00}$ nonzero,
if $E(x+y)$ is nonzero and $Ex=0$, then $E(x+y) = Ey$. So
$$\max^{(x+y)} \sum_{i=1}^k \norm{E_i (x+y)}_{n-1} \le
   \max^{(x)}\sum_{i=1}^k \norm{E_ix}_{n-1}+ \max^{(y)} \sum_{i=1}^k \norm{E_iy}_{n-1},$$
where $\max^{(x)}$ ranges over all admissible sequences whose corresponding admissible sum for $x$ is nonzero.
Then we have
$$\eqalign{
\norm{x+y}_n &= \max\biggl\{\norm{x+y}_{n-1}, {1\over 2}\max^{(x+y)}\sum_{i=1}^k \norm{E_i (x+y)}_{n-1}\biggr\}\cr
&\le \max\biggl\{\norm{x}_{n-1} +\norm{y}_{n-1},
{1\over 2}\max^{(x+y)} \sum_{i=1}^k \bigl(\norm{E_i x}_{n-1} + \norm{E_i y}_{n-1}\bigr)\biggr\} \cr
&\le \max\biggl\{\norm{x}_{n-1}, {1\over 2}\max^{(x)}\sum_{i=1}^k \norm{E_i x}_{n-1}\biggr\} \cr
&\qquad\qquad+\max\biggl\{\norm{y}_{n-1}, {1\over 2}\max^{(y)}\sum_{i=1}^k \norm{E_i y}_{n-1}\biggr\}\cr
&= \norm{x}_n + \norm{y}_n,\cr
}$$
by linearity of the maximum function.

The claim (ii) is rather obvious, and again the proof is by induction, with the base case $\norm{x}_{l_\infty}
\le \norm{x}_{l_1}$. Now let $n\ge 1$ and suppose the statement proven for $n-1$. If $\norm{x}_n = \norm{x}_{n-1}$
we are done, and if not, then
$$\norm{x}_n = {1\over 2}\max^{(x+y)}\sum_{i=1}^k \norm{E_i (x+y)}_{n-1}
\le {1\over 2}\max^{(x+y)}\sum_{i=1}^k \norm{E_i (x+y)}_{l_1}$$
and since any admissible sequence consists of disjoint sets, each coordinate $x_i$ in $x$ contributes $|x_i|$
to the sum on the right-hand side at most once. Hence in this case $\norm{x}_n \le \norm{x}_{l_1}$ as well.

Since the norms $\norm{x}_n$ are nondecreasing in $n$ for any fixed $x$ and bounded above by $\norm{x}_{l_1}$,
this value must converge to a limit as $n\to\infty$. The fact that $\norm{x} = \lim_{n\to\infty} \norm{x}_n$
is a norm is easy to see (the triangle inequality follows from linearity of the limit). We have thus proven
(iii).\slug

The above proposition asserts that $c_{00}$ endowed with the norm $\norm\cdot$ is a normed vector space. We
then define Tsirelson's space $T$ to be the completion of $c_{00}$ with respect to this norm. For each index
$i$, let $t_i$ be the vector whose $i$th coordinate is $1$ and whose other coordinates are all $0$. The
sequence $(t_i)$ forms a normalised $1$-unconditional Schauder basis for $T$. This can be proven by inductively
verifying that it holds for the completion of $c_{00}$ with respect to $\norm\cdot _n$ for all $n$.

We also have the identity
$$\norm{x} = \max\biggl\{ \norm{x}_{l_\infty}, {1\over 2}\sup \sum_{i=1}^k \norm{E_i x}\biggr\},$$
in which the supremum is taken over all $k$ and all possible admissible sequences $E_1,\ldots,E_k$.
This could actually have been taken as the {\it definition} of the Tsirelson norm (it is the
unique norm with this property), which
might appear to be nonsensical, because the norm appears on both sides of the definition.
However, it is actually perfectly reasonable, since the $E_i x$ on the right hand side have
$|\supp E_i x| < |\supp x|$ and we can assume by induction that the norm is already defined for these vectors.
The base cases are the vectors that have only one nonzero coordinate, for which we must take the $l_\infty$-norm.
In any case, let us prove that this definition is equivalent to the norm that we constructed above.

\newcount\recursive
\recursive=\thmcount
\proclaim Proposition \advthm. For any $x\in T$, we have
$$\norm{x} = \max\biggl\{ \norm{x}_{l_\infty}, {1\over 2}\sup \sum_{i=1}^k \norm{E_i x}\biggr\},$$
where the supremum is taken over all $k$ and all admissible sequences $E_1,\ldots,E_k$.

\proof Since $c_{00}$ is dense in $T$, it suffices to prove the proposition for $x\in c_{00}$. If
$\norm x = \norm{x}_{l_\infty}$, then for every $n\in \NN$ and admissible sequence $E_1,\ldots,E_k$, we
have
$$ {1\over 2}\sum_{i=1}^k \norm{E_i x}_n \le \norm{x}_n.$$
Then by linearity of the supremum, we have
$${1\over 2}\sum_{i=1}^k \norm{E_i x} \le {1\over 2}\sum_{i=1}^k \sup_{n\ge 0} \norm{E_i x}_n
= \sup_{n\ge 0}{1\over 2}\sum_{i=1}^k \norm{E_i x}_n \le \sup_{n\ge 0} \norm{x}_n = \norm{x},$$
which shows that the identity holds in this case. If $\norm x\ne \norm_{l_\infty}$, then for all $\eps >0$
there is $n\ge 1$ such that $\norm{x}< \norm{x}_n + \eps$ and $\norm{x}_n > \norm{x}_{n-1}$. This means
there is some $k'$ and $E_1,\ldots,E_{k'}$ such that
$$ \norm {x}_n = {1\over 2} \sum_{i=1}^{k'} \norm{E_i x}_{n-1}.$$
It follows that
$$\eqalign{
\norm x - \eps &< \norm{x}_n = {1\over 2} \sum_{i=1}^{k'} \norm{E_i x}_{n-1}
\le {1\over 2} \sum_{i=1}^{k'} \norm{E_i x}
\le \sup{1\over 2} \sum_{i=1}^{k} \norm{E_i x}.
}$$
For an upper bound on the supremum, we pick $k''$ and an admissible sequence $E_1,\ldots, E_{k''}$ such that
$$\sup{1\over 2} \sum_{i=1}^{k} \norm{E_i x} < {1\over 2} \sum_{i=1}^{k''} \norm{E_i x} + \eps.$$
We then pick $n'$ such that
$$\sup{n\ge 1}{1\over 2} \sum_{i=1}^{k''} \norm{E_i x}_n <
{1\over 2} \sum_{i=1}^{k''} \norm{E_i x}_{n'} + \eps.$$
This yields the upper bound
$$\eqalign{
{1\over 2} \sum_{i=1}^{k} \norm{E_i x} &< {1\over 2} \sum_{i=1}^{k''} \norm{E_i x} + \eps \cr
&= {1\over 2} \sum_{i=1}^{k''} \sup_{n\ge 1} \norm{E_i x}_n + \eps \cr
&= \sup_{n\ge 1} {1\over 2} \sum_{i=1}^{k''} \norm{E_i x}_n + \eps \cr
&< {1\over 2} \sum_{i=1}^{k''} \norm{E_i x}_{n'} + 2\eps \cr
&\le \norm{x}_{n'+1} + 2\eps\cr
&\le \norm{x} + 2\eps.\cr
}$$
We have shown that
$$\norm{x}-\eps < \sup{1\over 2} \sum_{i=1}^{k} \norm{E_i x} < \norm{x} + 2\eps,$$
which, since $\eps$ was arbitrary, gives us equality.\slug

This recursive identity is the only definition one really uses when performing any serious computations
with Tsirelson's norm. In particular, it asserts that we have,
for any $x\in T$ and admissible sequence $E_1,\ldots,E_k$,
$$\sum_{i=1}^k \norm{E_i x} \le 2 \norm x.$$
As a demonstration of the utility of the identity, consider the proof of the
following proposition, which gives us an idea of
how Tsirelson's norm behaves on subsets of a seminormalised block basis.

\newcount\tsirelsonblock
\tsirelsonblock=\thmcount
\proclaim Proposition \advthm. Let $(t_i)$ denote the usual unit vector basis of Tsirelson's space
and let $k$ be a positive integer and let $M\ge 1$. For any $k$ blocks $(y_j)_{j=1}^k$,
where each $y_j$ satisfies $1/M \le \norm{y_j}\le M$ and is of the form
$$y_j = \sum_{i=n_j+1}^{n_{j+1}} a_i t_i$$
for some scalars $a_i$ and $k-1\le n_1<n_2 <\cdots < n_{k+1}$, we have
$${1\over 2M} \sum_{j=1}^k |b_j| \le \Norm{\sum_{j=1}^k b_jy_j} \le M\sum_{j=1}^k |b_j|$$
for all $k$-tuples of scalars $(b_1,\ldots, b_k)$.

\proof By the triangle inequality and the fact that $\norm{y_j} \le M$ for all $j$, we have
$$\Norm{\sum_{j=1}^k b_jy_j} \le \sum_{j=1}^k \norm{b_jy_j} \le  M\sum_{j=1}^k |b_j|.$$
For the lower bound, we let $x = \sum_{j=1}^k b_jy_j$ and let
$$E_j = \{n_j+1, n_j+2, \ldots, n_{j+1}-1\}$$
for $1\le j\le k$. Then
$$\norm x \ge {1\over 2}\sum_{j=1}^k \norm{E_j x} = {1\over 2}\sum_{j=1}^k \norm{b_j y_j}
  \ge {1\over 2M}\sum_{j=1}^k |b_j|.\noskipslug$$

We are almost ready to prove that $T$ does not contain a copy of $l_1$. We just need one more lemma,
which is well-known and due to R.~C.~James~\ref{james1964}.

\parenproclaim Lemma J (James, {\rm 1964}). If a normed linear space contains a subspace isomorphic
to $l_1$, then for any $\delta>0$, there is a normalised block basic sequence $(u_j)_{j=0}^\infty$ such that
$$(1-\delta)\sum_{j=0} |a_j| \le \Norm{\sum_{j=0}^\infty a_ju_j} \le \sum_{j=0}^\infty |a_j|$$
for every sequence $(a_j)_{j=0}^\infty$ of scalars such that at least one $a_j$ is nonzero.

\proof The statement holds trivially when $\delta\ge 1$, so let $0<\delta<1$.
The subspace $B$ isomorphic to $l_1$ contains a sequence $(x_i)_{i=0}^\infty$ for which there
exist constants $m$ and $M$ such that
$$ m\sum_{i=0}^\infty |a_i| \le \Norm{\sum_{i=0}^\infty a_i x_i} \le M\sum_{i=0}^\infty |a_i|$$
for all sequences $(a_i)_{i=0}^\infty$ of scalars. For $n\ge 1$, let
$$ K_n = \inf \biggl\{ \Norm{ \sum_{i=n}^\infty a_ix_i } : \sum_{i=0}^\infty |a_i| = 1\biggr\} $$
and set $K$ to $\lim_{n\to \infty} K_n$. We of course have $m\le K\le M$. Let $\phi >1$ and $0<\theta<1$
be such that $1-\delta \le \theta/\phi$. Choose $n_0<n_1<n_2$ such that $K_{n_0+1} \ge \theta K$ and for which
there is a block
basic sequence $(y_j)_{j=0}^\infty$ with each $y_j$ of the form
$$ y_j = \sum_{j=n_j+1}^{n_{j+1}} a_i x_i $$
for some scalars $a_i$ (depending on $j$) and with $\norm{y_j} \le \phi K$ for all $j$.
Let $(a_j)_{j=0}^\infty$ be an arbitrary sequence of scalars.
The normalised
block basis consisting of $u_j = y_j/\norm{y_j}$ satisfies
$$\Norm{\sum_{j=0}^\infty a_ju_j} \le \sum_{j=0}^\infty |a_j|$$
by the triangle inequality. By the construction of the block basis,
each $y_j$ has $\supp y_j > \{n_0\}$, so by the definition of $K_{n_0+1}$,
$$\Norm{\sum_{j=0}^\infty a_jy_j} = \Norm{\sum_{j=n_0+1}^\infty a_jy_j} \ge K_{n_0+1} \sum_{j=0}^\infty |a_j|
\ge \theta K \sum_{j=0}^\infty |a_j|.$$
Then since $\norm{y_j} \le \phi K$ for all $K$ and $\theta/\phi \ge 1-\delta$, we have
$$\Norm{\sum_{j=0}^\infty a_j u_j} = \normm{\sum_{j=0}^\infty a_j {y_j\over \norm{y_j}}}
\ge {1\over \phi K} \Norm{\sum_{j=0}^\infty a_j y_j} \ge (1-\delta)\sum_{j=0}^\infty |a_j|,$$
which proves the lower bound.\slug

Note that in James' original statement of the lemma, the $u_j$ are just vectors contained in the closed unit
ball. However, the construction in his proof actually produces a block basic sequence, and we
normalised it in our own version of the proof above.
We are now set to show that $l_1$ does not embed into $T$.

\newcount\nolone
\nolone=\thmcount
\proclaim Theorem \advthm. Tsirelson's space $T$ does not contain a subspace isomorphic to $l_1$.

\proof Suppose, towards a contradiction, that $T$ did contain a subspace isomorphic to $l_1$. Then, applying
Lemma~J with $\delta = 1/9$, there exists a normalised block basic sequence $(y_j)_{j=0}^\infty$
such that for all sequences $(a_j)_{j=0}^\infty$ of scalars,
$${8\over 9} \sum_{j=0}^\infty |a_j| \le \Norm{\sum_{j=0}^\infty a_jy_j} \le\sum_{j=0}^\infty |a_j|.$$
In particular, if we let $r\ge 1$ be an integer and let $(a_j)_{j=0}^\infty$ be the sequence with
$$a_j = \cases{ 1,& if $j=0$;\cr 1/r,& if $1\le j\le r$;\cr 0,& if $j>r$,}$$
then we have $\sum_{j=0}^\infty |a_j| = 2$. This yields
$$\Norm{ y_0 + {1\over r}\sum_{j=1}^r y_j} = \Norm{\sum_{j=0}^\infty a_jy_j} \ge
{8\over 9}\sum_{j=0}^\infty |a_j| = {16\over 9}$$
for all integers $r\ge 1$.

Now let $\{k\}\le E_1 < E_2 < \cdots < E_k$ be an arbitrary admissible sequence of sets, and let
$r_0 =\max\supp y_0$. If $k > r_0$, then
$$\sum_{i=1}^k \normm{ E_i \Bigl( y_0 + {1\over r}\sum_{j=1}^r y_j\Bigr)}
=\sum_{i=1}^k \normm{ E_i \Bigl({1\over r}\sum_{j=1}^r y_j\Bigr)} \le 2\cdot\Norm{{1\over r}\sum_{j=1}^r y_j}
\le 2.$$
If $k\le r_0$, then we set
$$S = \bigl\{1\le j\le r : \norm{E_i y_j}\ne 0\ \hbox{for at least two values of}\ i\bigr\}$$
and
$$T = \bigl\{1\le j\le r: \norm{E_i y_j}\ne 0\ \hbox{for at most one value of}\ i\bigr\}.$$
Note that if $j\in S$, then $y_j$ straddles the border between $E_i$ and $E_{i+1}$ for some $i$, and
no other $y_{j'}$ can do so, since the supports of the $y_j$ are disjoint. So the cardinality of $S$ is
at most $k-1$. It is also clear that $S$ and $T$ are disjoint and $|S|+|T| = r$, so
$$2|S| + |T| \le 2(k-1) + r- k+1 = r+k-1.$$
Since the $y_j$ are unit vectors, we have
$$\eqalign{
\sum_{i=1}^k \normm{E_i \Bigl({1\over r}\sum_{j=1}^\infty y_i\Bigr)} &=
\sum_{j\in S} \sum_{i=1}^k \norm{E_iy_j} + \sum_{j\in T} \sum_{i=1}^k \norm{E_iy_j}\cr
&\le  \sum_{j\in S} 2\cdot \norm{y_j} + \sum_{j\in T} \norm{y_j},\cr
&\le r+k-1\cr
}$$
by rearranging sums.
Dividing by $r$ and adding the unit vector $y_0$ into the mix, we have
$$\eqalign{
\sum_{i=1}^k \normm{ E_i \Bigl( y_0 + {1\over r} \sum_{j=1}^r y_j\Bigr)}
  &\le\sum_{i=1}^k \norm{E_i y_0} + {1\over r}\sum_{i=1}^k\normm{E_i\Bigl({1\over r}\sum_{j=1}^\infty y_i\Bigr)}\cr
&\le 2 + {r+k-1\over r} \cr
&\le 3 + {n_0\over r}.\cr
}$$
Selecting some $r\ge 2n_0$, the right-hand side is at most $7/2$, meaning that
$$\Norm{y_0 + {1\over r} \sum_{j=1}^r y_j} \le 7/4,$$
and we have $7/4 \ge 16/9$, the contradiction we sought.\slug

In the proof that $c_0$ and the other $l_p$ spaces do not embed into $T$ either, we shall make use of
the following theorem, a version of what is called the Bessaga--Pe\l czy\'nski Selection Principle
(see~\ref{bessagapelczynski}, Theorem 3).

\parenproclaim Theorem S (Bessaga--Pe\l czy\'nski, {\rm 1958}).
Let $X$ be a Banach space with basis $(e_i)$ and let $(e_i^*)$ be the coefficient
functionals given by
$$ e_j^*\Big(\sum_{i=1}^\infty a_i e_i\Bigr) = a_j.$$
If $(y_i)_{i=1}^\infty$ is a sequence of vectors in $X$ such that
\medskip
\item{i)} $\lim_{i\to\infty} \norm{y_i} > 0$; and
\smallskip
\item{ii)} $\lim_{i\to\infty} e_j^*(y_i) = 0$ for all indices $j\ge 1$,
\medskip\noindent
then there exists a subsequence of $(y_i)$ which is equivalent to a block basis with respect to the original
basis $(e_i)$.\slug

To prove that $c_0$ and $l_p$ do not embed into $T$, we shall actually prove the following stronger result.

\proclaim Theorem \advthm. Tsirelson's space $T$ does not contain a seminormalised subsymmetric basic
sequence.

\proof Towards a contradiction, suppose that $(y_i)$ is a seminormalised subsymmetric basic sequence in $T$.
Since the sequence is seminormalised, it is bounded from above, so by Lemma~{\the\dichotomy}, either $(y_i)$
is equivalent to the canonical basis for $l_1$ or $(y_i)$ is weakly null. We already proved Theorem~{\the\nolone}
above, which deems the first scenario impossible, so $(y_i)$ must be weakly null and thus in particular
satisfies condition (ii) of Theorem S. In addition, since $(y_i)$ is seminormalised, it is uniformly bounded
away from zero, so we are in the position to conclude from Theorem S that there is a subsequence of $(y_i)$
which is equivalent to a block basic sequence $(x_i)$ against the unit vector basis $(t_i)$ of $T$.
We shall now pass to subsequences both in $(x_i)$
and in $(y_i)$. First, pass to a subsequence and reindex $(x_i)$ to ensure that $\supp x_i > \{i\}$ for all $i$.
Now we pass to a subsequence of $(y_i)$ that is equivalent to this new $x_i$ and reindex; this is equivalent
to the original sequence by subsymmetry.

Since $y_i$ is seminormalised and there is a linear homeomorphism mapping each $y_i\mapsto x_i$,
$(x_i)$ is seminormalised as well; that is, there is
$M\ge 1$ such that $1/M \le \norm{x_i}\le M$.
Now since $\supp x_i > \{i\}$ for all $i$,
we can apply Proposition~{\the\tsirelsonblock} with the constant $M$ on any $n$ blocks
$x_{i_1}, x_{i_2}, \ldots, x_{i_n}$ of $(x_i)$
to find that for all $n$-tuples $(b_1,\ldots, b_n)$ of scalars,
$${1\over 2M}\sum_{k=1}^n |b_k| \le \Norm{\sum_{k=1}^n b_kx_{i_k}} \le M\sum_{k=1}^n |b_k|.$$
We thus find that for any $n$, $(x_i)$ (and hence $(y_i)$)
contains a subsequence of length $n$ which is equivalent to the standard basis of $l_1^n$. Complete
each of these finite sequences to infinite subsequences. Recall that $(y_i)$ is equivalent to all of them,
so for all $n$, the first $n$ elements of $(y_i)$ are equivalent to the standard basis of $l_1^n$. Thus
$(y_i)$ is equivalent to the standard basis of $l_1$ and we know this cannot happen.\slug

\proclaim Corollary \advthm. Tsirelson's space $T$ does not contain $c_0$ or $l_p$ for $1<p<\infty$.

\proof The unit vector bases of these spaces are normalised (hence seminormalised) subsymmetric
basic sequences.\slug

\advsect Banach spaces inspired by Tsirelson's space

Tsirelson's example marked a bit of a turning point in the theory of Banach spaces. It is important not
only because it settled a long-standing open problem, but because the construction that underlied it turned
out to be modifiable in various ways to tackle other problems. In this section we shall give a brief survey
of some ``Tsirelson-inspired'' Banach spaces. The aim of this section is breadth rather than rigour,
so we shall not give proofs as we did in the previous section for Tsirelson's space. Instead, we are
more interested in the constructions themselves, how they build upon each other, and the properties
they satisfy.

\medskip\boldlabel Schlumprecht's arbitrarily distortable space. Let $X$ be an infinite-dimen\-sional Banach
space under the norm $\norm\cdot$
For $\lambda > 1$, we say that $X$ is {\it $\lambda$-distortable} if there exists an equivalent norm
$\threenorm\cdot$ on $X$ such that for each infinite-dimensional subspace $Y$ of $X$, we have
$$\sup\biggl\{{\threenorm{y_1}\over\threenorm{y_2}}:y_1,y_2\in Y,\,\norm{y_1} = \norm{y_2} = 1\biggr\}\ge\lambda.$$
The norm $\threenorm\cdot$ is called a {\it $\lambda$-distortion} if this is true.
The space $X$ is called {\it distortable} if it is $\lambda$-distortable for some $\lambda>1$, and
{\it arbitrarily distortable} if it is distortable for every $\lambda>1$. In 1991, T.~Schlumprecht
proved the existence of an arbitrarily distortable Banach space~\ref{schlumprecht1991}.

Schlumprecht's construction relies on a function $f : [1,\infty)\to [1,\infty)$ satisfying
\medskip
\item{i)} $f(1) = 1$ and $f(x)<x$ for all $x>1$;
\smallskip
\item{ii)} $f$ is strictly increasing to infinity;
\smallskip
\item{iii)} $lim_{x\to\infty} f(x)/x^q = 0$ for all $q>0$;
\smallskip
\item{iv)} the function $g(x) = x/f(x)$ is concave for $x\ge 1$; and
\smallskip
\item{v)} $f(x)f(y) \ge f(xy)$ for all $x,y\ge 1$.
\medskip
Of course, we need such a function to actually exist, but it is easily seen that $f(x) = \log_2(x+1)$ does.
We then define a sequence $\threenorm{\cdot}_n$ of norms on $c_{00}$ by setting
$\threenorm{x}_0 = \norm{x}_{l_\infty}$
and for $n\ge 1$ we put
$$\threenorm{x}_{n} = \max_{k\ge 2} \max {1\over f(k)} \sum_{i=1}^k \threenorm{E_i x}_{n-1},$$
where the inner maximum is over all subsets $E_1<E_2<\cdots <E_k$ of positive integers (unlike in
Tsirelson's example, we do not need ${k}\le E_1$ this time) for which $E_i x$ is nonzero for some $1\le i\le k$.
Schlumprecht's space $X$ is then defined to be the completion of $c_{00}$ with respect to the limit
$\norm\cdot$ of these norms.

Let $(e_i)$ denote the sequence of vectors in $c_{00}$, where $e_i$ has a $1$ at the $i$th coordinate and
zeroes elsewhere. It is an unconditional basis for $X$ just as it was for $T$. But whereas $T$ contained no
seminormalised subsymmetric basic sequences, $(e_i)$ is actually a normalised subsymmetric basis of $X$.
Just as in Tsirelson's example, one can show that the norm on $X$ satisfies the
recursive identity
$$\norm x = \max\Bigl\{\norm{x}_{l_\infty}, \sup_{k\ge 2} \sup {1\over f(k)} \sum_{i=1}^k \norm{E_i x}\Bigr\},$$
where the inner supremum runs over all subsets $E_1 < E_2<\cdots <E_k$ of positive integers.
Note that $X$ also does not contain $c_0$ or $l_p$ for $1\le p<\infty$,
and the fact that $X$ is arbitrarily
distortable is established by the following theorem.

\proclaim Theorem D. Let $X$ denote Schlumprecht's space and $\norm\cdot$ its norm. For any positive integer
$k$ and $x\in X$, let
$$\norm{x}_k = \sup {1\over f(k)} \sum_{i=1}^k \norm{E_i x}.$$
For every $k$, every $\eps>0$, and every infinite-dimensional subspace $Y$ of $X$,
there are unit vectors $y_1, y_2\in Y$ such that
$$\norm{y_1}_k \ge 1-\eps\qquad\hbox{\rm and}\qquad \norm{y_2}_k \le {1+\eps\over f(k)}.$$
In particular, this means that $\norm\cdot _k$ is an $f(k)$-distortion on $X$.\slug

\medbreak
\boldlabel The unconditional basic sequence problem.
In Section~2 we defined what it means for a basis or a basic sequence to be unconditional.
Both Tsirelson's space and Schlumprecht's space have unconditional bases, and so do all classical Banach spaces.
It was unknown for many years whether every infinite-dimensional Banach space contains an unconditional basic
sequence. Soon after Schlumprecht presented his example in the summer of 1991,
W.~T.~Gowers and B.~Maurey independently constructed
Banach spaces without an unconditional basic sequences. They subsequently discovered that both examples
were fundamentally the same, and ended up publishing jointly~\ref{gowersmaurey1993}.

Recall that we say that $X$ is the {\it direct sum} of $Y$ and $Z$ and write $X = Y\oplus Z$ if
the map $Y\times Z\to X$ sending $(y,z)\mapsto y+z$ is a linear homeomorphism. An infinite-dimensional
Banach space $X$ is said to be {\it decomposable} if $X$ can be written as $X = Y\oplus Z$ where
both $Y$ and $Z$ are also infinite-diimensional. An infinite-dimensional
Banach space $X$ without any decomposable subspaces is called
{\it hereditarily indecomposable}. If a space $X$ has an unconditional basic sequence $(x_i)_{i=1}^\infty$, then
letting $Y$ be the closed linear span of $(x_{2i-1})_{i=1}^\infty$ and $Z$ be the closed linear span
of $(x_{2i})_{i=1}^\infty$, we see that $X$ is not hereditarily indecomposable. So to produce an
infinite-dimensional space without an unconditional basic sequence, it suffices to produce a hereditarily
indecomposable one, and that is what Gowers and Maurey did.

Before describing their construction, we need to establish a fair bit of terminology and notation.
Let $J$ be
a set of positive integers such that if $m,n\in J$ with $m<n$, then $\log\log\log n \ge 4m^2$. We also
assume that the smallest element of $J$ is greater than $256$. We write
$J$ in increasing order as $\{j_1, j_2,\ldots\}$, let $K$ be the subset of elements with odd index, and let $L$
be the subset of elements with even index. We recall here that a sequence $y=(y_i)\in l_1$
defines a linear functional $f_y$ on $c_{00}$ with $\norm{f_y} = \norm{y}_{l_1}$ given by
$$f_y\bigl((x_i)\bigr) = \sum_{i=1}^k x_i y_i.$$
Any bounded linear functional $f$ on $c_{00}$ can be associated to an element of $l_1$ by evaluating $f$
at each of the standard basis elements of $c_{00}$, so the topological dual of $c_{00}$ is $l_1$.

We say that a sequence $x_1,x_2,\ldots,x_n$ of scalar sequences is {\it successive} and write
$x_1<x_2<\ldots<x_n$ if $\supp x_i < \supp x_{i+1}$ for all $1\le i<n$.
Let $f = \log_2(1+x)$, as in Schlumprecht's example. Let $Q\subseteq$ denote
the set of finitely-supported sequences whose coordinates are all rational numbers in the range $[-1,1]$.
Let $\sigma$ be an injection from the set of finite successive sequences of elements of $Q$ to $L$ such
that for all such sequences $x_1, \ldots, x_s$ is such a sequence, $S = \sigma(x_1,\ldots,x_s)$, and
$x = x_1 + \cdots + x_s$, then
$$\left|\supp z\right| \le {\sqrt{f\bigl(S^{1/40}\bigr)}\over 20}.$$
This injection is used to define functionals on any normed space with $c_{00}$ as the underlying set.
Let $X=(c_{00}, \norm\cdot)$ be such a normed space. For $m\in \NN$, we let $A^*_m(X)$ be the set of functionals
of the form
$${1\over f(m)} \sum_{i=1}^m x_i^*,$$
where the $x_i^*$ are functionals with $\norm{x_i^*}\le 1$ for all $i$ and $x_1^* < \cdots < x_m^*$
(as sequences). Note that members in $A^*_m$ have operator norm at most $1$ for any $m$. For $k\in \NN$,
let $\Gamma_k^X$ be the set of sequences $(g_1,\ldots, g_k)$ such that $g_i\in Q$ for all $i$,
$g_1\in A^*_{j_{2k}}(X)$, and $g_{i+1}\in A^*_{\sigma(g_1,\ldots,g_i}$ for all $1\le i<k$. We call the
elements of $\Gamma_k^X$ {\it special sequences}. Now let $B_k^*(X)$ be the set of functionals of the form
$$ {1\over \sqrt{f(k)}} \sum_{j=1}^k g_j $$
such that $(g_1, \ldots, g_k)$ is a special sequence. An element of $B_k^*(X)$ is called a {\it special
functional}.

The space without an unconditional basic sequence that Gowers and Maurey exhibit is also constructed via
a sequence of norms, as Tsirelson's and Schlumprecht's spaces were. Let $\norm{\cdot}_0 = \norm{\cdot}_{l_\infty}$
and
set $X_0 = (c_{00}, \norm\cdot_0)$. For $n\ge 1$, we let $X_n = (c_{00}, \norm\cdot _n)$,
where $\norm{x}_n$ be the maximum of
$$\sup_{m\in \NN} \sup {1\over f(m)} \sum_{i=1}^m \norm{E_i x}_{n-1},$$
where the second supremum is over all sequences of subsets $E_1 < \cdots < E_m$, and
$$\sup_{k\in K} \sup_{g\in B^*_k(X_{N-1})} \sup_{E\subseteq \NN} \bigl|g(Ex)\bigr|.$$
This is a nondecreasing sequence of norms that is bounded above by the $l_1$-norm, so it has a limit, which
we call $\norm \cdot$. The completion $X$ of $(c_{00}, \norm\cdot)$ is the space we're after, though the proof
that $X$ is hereditarily indecomposable would run us aground.

An operator $T:X\to Y$ is {\it strictly singular} if there is some $c>0$ such that $\norm{Tx} \ge c\norm x$ for
all $x\in X$. Gowers and Maurey show that every operator from
$X$ to $X$ can be written as a scalar multiple of the identity, plus a strictly singular operator. This
implies that there are, in some sense, very few operators on $X$. A subsequent paper of Gowers and Maurey,
published in 1997,
describes a more general method of producing
Banach spaces whose spaces of operators are small~\ref{gowersmaurey1997}. Recall that an operator
is said to be {\it compact} if it sends bounded sets in the domain to sets with compact closure in the image.
Every compact operator is strictly singular.
In their original
1993 paper, the authors ask if there exists a Banach space $X$ such that every operator from $X$ to $X$
is a scalar multiple of the identity, plus a compact operator. This question was answered positively
S.~A.~Argyros and R.~G.~Haydon in 2009; the paper was published in 2011~\ref{argyroshaydon}.

\medskip\boldlabel The Banach hyperplane problem. In a book published in 1932,
S.~Banach~\ref{banach} asked whether an infinite-dimensional Banach space $X$ over the real
numbers is always isomorphic to $X\oplus \RR$. This amounts to asking whether every infinite-dimensional
Banach space $X$ is such that
that every subspace of codimension $1$ is isomorphic to $X$ itself. Since a subspace of codimension $1$ is
sometimes called a {\it hyperplane}, this question came to be known as the hyperplane problem.
It turns out that the answer to the question is no, and there is an infinite-dimensional
Banach space that is not isomorphic
to any of its hyperplanes.

This problem was solved by W.~T.~Gowers soon after he solved the unconditional basic sequence problem.
His construction, which appeared in print in 1994, is extremely similar~\ref{gowers1994}.
We shall thus retain the definitions from the previous
subsection and immediately define this space. As the reader probably suspects at this point, the construction
is inductive.
Set $X_0 = (c_{00}, \norm\cdot_0)$. For $n\ge 1$, we let $X_n = (c_{00}, \norm\cdot _n)$,
where $\norm{x}_n$ be the maximum of
$$\sup_{m\in \NN} \sup {1\over f(m)} \sum_{i=1}^m \norm{E_i x}_{n-1},$$
where the inner supremum is over all sequences of subsets $E_1 < \cdots < E_m$, and
$$\sup_{k\in K} \sup_{g\in B^*_k(X_{N-1})} \bigl|g(x)\bigr|.$$
The notation we use here is not exactly the same as in the original paper, but using the same letters
as before makes it clear that the only change from the space without an unconditional basic sequence
is the replacement of $\sup_{E\subseteq \NN} \bigl|g(Ex)\bigr|$ has changed to the simpler $\bigl|g(x)\bigr|$.
The space $X$ is the completion of $c_{00}$ with respect to the limit of the norms $\norm{x}_n$ above.
Perhaps surprisingly, this new space $X$ has an unconditional basis. It is noted in Gowers' paper
that the Gowers--Maurey
space of the previous section is also a counterexample to the hyperplane problem, though is is
more difficult to prove.
The proof that $X$ is not isomorphic to any of its hyperplanes is done by showing that $X$
satisfies the hypothesis of the following lemma, which is attributed to P.~G.~Casazza.

\proclaim Lemma C. If $X$ is a Banach space in which no two equivalent sequences $(y_i)$ and $(z_i)$ satisfy
$y_i < z_i < y_{i+1}$ for all $i$, then $X$ is not isomorphic to any proper subspace.\slug


\section References

\bye
