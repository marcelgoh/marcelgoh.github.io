\input fontmac

\def\quo#1.{\par\textindent{\it #1.}}

\widemargins
\bookheader{MEDITATIONS OF A MATHEMATICS STUDENT}{MARCEL K. GOH}

\maketitle{Meditations of a mathematics graduate student}{}{Marcel K. Goh}{\sl after Marcus Aurelius}


\vskip24pt

\centerline{\bf I}
\medskip

\quo 1. From my grandfather I learned the multiples of five and how they may be used to tell the time.
\quo 2. From my mother, how my fingers can be used to recall the multiples of nine.
\quo 3. From Luc, how to do research: how to formulate problems and seek out their solutions.
From him too I learned that, weather permitting, mathematics should be done and taught outdoors; and that
incorporating exercise and fresh air into one's day often sharpens one's productivity.
\quo 4. With my friends Rosie, Anna, Jad, and Jonah (among others), that the joy of
discovery is something easily shared. I also learned through their example
the value of hard work, and how diligence
and devotion to one's studies can be translated into erudition and academic success.
\quo 5. From the books and writings of Knuth, not to chase scholarly trends, and how indeed the popularity
of a given subfield is often in direct correlation with the shallowness of the research therein. Knuth also
impressed on me the importance of striking a balance between theory and practice in all my academic endeavours.

\bigskip\vskip\parskip

\centerline{\bf II}
\medskip

\quo 1. Recall how early in your university days you were discouraged by the ease with which
your peers seemed to grasp course concepts, and later discovered this impression to be largely baseless, that
their skills were not drastically better than your own. Take care, even in times of greatest insecurity,
not to overrepresent your
knowledge when speaking to a acquaintance,
for such a fa\c{c}ade will only impede your own learning, and by boasting you are liable to discourage others.

\bigskip
\rightline{\sl Last updated 12.IX.2022, Edmonton, Alberta.}

\bye
