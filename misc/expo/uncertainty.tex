\input fontmac
\input mathmac

\def\FF{{\bf F}}
\def\TT{{\bf T}}
\def\bar{\overline}
\def\hat{\widehat}
\def\norm#1{|\!|#1|\!|}
\def\bignorm#1{\big|\!\big|#1\big|\!\big|}
\def\Norm#1{\Big|\!\Big|#1\Big|\!\Big|}
\def\normm#1{\bigg|\!\bigg|#1\bigg|\!\bigg|}
\def\supp{\op{{\rm supp}}}
\def\smallsupp{\op{{\sevenrm supp}}}
\def\divides{\backslash}

\widemargins
\bookheader{THE DISCRETE FOURIER UNCERTAINTY PRINCIPLE}{MARCEL K. GOH}

\maketitle{The discrete Fourier uncertainty principle}{by}{Marcel K. Goh}{6 October 2022}

\bigskip

\advsect Introduction

Let $Z$ be a finite abelian group. A {\it character} on $Z$ is a homomorphism from $Z$ to the multiplicative
group $\CC\setminus \{0\}$. It is easily seen that $|\chi(x)|$ must equal $1$ for all $x\in Z$.
The set of characters forms a group, which we shall call $\hat Z$.
Now if $Z = \ZZ_{n_1}\times \ZZ_{n_2} \times\cdots\times \ZZ_{n_r}$, then for every $u = (u_1,\ldots,u_r)\in Z$
the function $\chi_u : Z\to \CC$ given by
$$\chi_u(x_1,\ldots,x_r) = \prod_{i=1}^r \exp\biggl( {2\pi i u_i x_i\over n_i} \biggr)$$
is a character, and in fact the map $u\mapsto \chi_u$ gives an isomorphism of groups from $Z$ to $\hat Z$.

The space of functions from $Z$ to $\CC$ can be made into an inner product space by setting
$$\langle f,g\rangle = \ex_{x\in Z} f(x)\bar{g(x)},$$
where $\ex_{x\in Z} F(x) = |Z|^{-1} \sum_{x\in Z} F(x)$, and likewise we define an inner product on
the space of functions from $\hat Z$ to $\CC$ by putting
$$\langle \hat f, \hat g\rangle = \sum_{\chi \in \hat Z} \hat f(\chi) \bar{\hat g(\chi)}.$$
For $f:Z\to \CC$, the {\it Fourier transform} of $f$ is the function $\hat f : \hat Z\to \CC$ given by
$$ \hat f(\chi) = \langle f,\chi\rangle
= \ex_{x\in Z} f(x) \bar{\chi(x)}.$$
Of course, we can associate to any $\alpha\in Z$ the character $\chi_\alpha\in \hat Z$, so we may write
$\hat f(\alpha)$ to mean $\hat f(\chi_\alpha)$, and this is called the {\it Fourier coefficient
of $f$ at $\alpha$}.

We have the following important formulas, whose proofs
can be found in any book on Fourier analysis.

\parenproclaim Theorem P (Parseval--Plancherel identity). Let $Z$ be a finite abelian group and
let $f,g : Z\to \CC$. If $\hat f$ and $\hat g$ are the Fourier transforms of $f$ and $g$ respectively,
then $\langle f,g \rangle = \langle \hat f, \hat g\rangle$.\slug

\parenproclaim Theorem I (Fourier inversion formula). Let $Z$ be a finite abelian group and let $f:Z\to \CC$.
Then
$$f(x) = \sum_{\chi\in \hat Z} \hat f(\chi) \chi(x).\noskipslug$$

Recall also the {\it Cauchy--Schwarz inequality}, which wears many disguises but in our context says that
$$ \Bigl( \sum_{x\in Z} |f(x)|\cdot |g(x)| \Bigr)^2
\le \Bigl(\sum_{x\in Z} |f(x)|^2\Bigr)\Bigl( \sum_{x\in Z} |g(x)|^2\Bigr)$$
for all $f,g : Z\to \CC$.

\advsect The uncertainty principle

The {\it support} of a function $f:Z\to \CC$ is the set $\{x\in Z : f(x)\ne 0\}$. We will write
$\norm{f}_0$ for the size $|\supp(f)|$ of the support, and it is also convenient to write
$\norm{f}_\infty$ for the quantity $\max_{x\in Z} |f(x)|$. (These are defined analogously for
functions on $\hat Z$.) The uncertainty principle states that
the support of $f:Z\to \CC$ and the support of its Fourier transform $\hat f:\hat Z\to \CC$ cannot both
be small. We will make this fact quantitative very soon. First off, let us prove a lemma.

\proclaim Lemma \advthm. Let $f$ be a function from an abelian group $Z$ to $\CC$ and let $\hat f$
be its Fourier transform. Then
$$\norm{\hat f}_\infty \le \ex_{x\in Z} |f(x)|.$$

\proof Let $\chi\in \hat Z$ be given. We have, by the definition of Fourier transform
and the triangle inequality,
$$|\hat f(\chi)| = \Bigl| \ex_{x\in Z} f(x) \bar{\chi(x)} \Bigr|
\le \ex_{x\in Z} \bigl| f(x) \bar{\chi(x)} \bigr|,$$
but since $|\chi(x)| = 1$ for all $x$, this is exactly the right-hand side of the lemma statement
and we are done since $\chi$ was arbitrary.\slug

We now state and prove the Fourier uncertainty principle.

\parenproclaim Theorem~{\advthm} (Fourier uncertainty principle).
Let $Z$ be a finite abelian group and $\hat Z$ be its dual. If $f : Z\to \CC$ is not identically zero
and $\hat f: \hat Z\to \CC$ is its Fourier transform, then
$$\norm{f}_0 \cdot \norm{\hat f}_0 \ge |Z|.$$

\proof By the previous lemma and the definition of the support,
$$\norm{\hat f}_\infty \le \ex_{x\in Z} |f(x)| = {1\over |Z|} \sum_{x\in Z} |f(x)|
= {1\over |Z|} \sum_{x\in \smallsupp(f)} |f(x)|.$$
We then use the Cauchy--Schwarz inequality to obtain
$$\sum_{x\in \smallsupp(f)} |f(x)| \le \sqrt{\sum_{x\in \smallsupp(f)} |f(x)|}\sqrt{\sum_{x\in \smallsupp(f)} 1^2}
= \sqrt{\norm{f}_0 \sum_{x\in Z} |f(x)|^2},$$
and so far we have shown that
$$\norm{\hat f}_\infty \le {1\over |Z|} \sqrt{\norm{f}_0 \sum_{x\in Z} |f(x)|^2}.$$
But by the Parseval--Plancherel identity, we have
$$\sum_{x\in Z} |f(x)|^2 = |Z| \sum_{\chi\in \hat Z} |\hat f(\chi)|^2
\le |Z|\cdot \norm{\hat f}_0 \cdot \norm{\hat f}_\infty^2, $$
and plugging this in above, we have
$$\norm{\hat f}_\infty \le \norm{\hat f}_\infty \sqrt{ \norm{f}_0\cdot \norm{\hat f}_0 \over |Z|}.$$
Since $f$ is not the zero function, we can divide both sides by $\norm{\hat f}_\infty$, square the
inequality, then rearrange to get the theorem statement.\slug

It can be shown that we have equality above if and only if $f$ is (some multiple of) the characteristic
function of a coset of a subgroup of $Z$.

So far so good, but for $Z = \ZZ_p$ a much stronger uncertainty principle holds, and the rest of these notes
will be dedicated to establishing the algebraic machinery needed to prove it.

\advsect Cyclotomic polynomials

Let $n$ be a positve integer. An {\it $n$th root of unity} is any complex number $\omega$ such
that $\omega^n = 1$. Note that if $d$ divides $n$, then any $\omega$ with $\omega^d = 1$
also satisfies $\omega^n = 1$, so in some sense this number should be associated to $d$ and not
$n$. An $n$th root
of unity is called {\it primitive} if it is not an $m$th root of unity for any $1\le m<n$. (Thus
any $n$th root of unity is a primitive $d$th root of unity for exactly one $d$ dividing $n$.) The {\it $n$th
cyclotomic polynomial}, which we shall denote by $\Phi_n$, is given by
$$\Phi_n(z) = \prod_{\omega} (z-\omega),$$
where in the product, $\omega$ runs over the primitive $n$th roots of unity.
As some small examples, we have $\Phi_1(z) = z-1$, $\Phi_2(z) = z+1$, $\Phi_3(z) = z^2 + z + 1$,
and $\Phi_4(z) = z^2 + 1$. Observe that so far, all the coefficients have been polynomial, a fact which
is not obvious from the definition but can be shown by induction (and indeed we shall).

In the proof of the next lemma we will also require the {\it von Mangoldt function} $\Lambda(n)$,
which is defined on positive integers by the rule
$$\Lambda(n) = \cases{ \log p, & if $n=p^k$ for some prime $p$ and some integer $k\ge 1$;\cr 0, & otherwise.}$$
By the fundamental theorem of arithmetic, any integer $n$ can be factored into
$n = {p_1}^{e_1} {p_2}^{e_2} \cdots {p_s}^{e_s}$, and taking logarithms of both sides we see that
$$ \log n = \sum_{i=1}^s e_i \log p_i = \sum_{d\divides n} \Lambda(d).$$

\proclaim Lemma \advthm. Let $n\ge 1$. The $n$th cyclotomic polynomial $\Phi_n$ is monic
with integer coefficients and
we have
$$\Phi_n(1) = \cases{ 0, & if $n=1$;\cr p,
& if $n=p^k$ for some integer $k\ge 1$;\cr 1, & otherwise.}$$

\proof Let $\Omega_n$ be the set of all $n$th roots of unity, primitive or not. Then the polynomial
$z^n-1$ factors as
$$z^n-1 = \prod_{\omega\in \Omega_n} (z-\omega).$$
Now since every $n$th root of unity is a primitive
$d$th root of unity for exactly one $d$ dividing $n$, we can group
roots together and write
$$z^n - 1 = \prod_{d\divides n,} \Phi_d(z).$$

Let us prove the formula for $\Phi_n(1)$ first. Of course, $\Phi_1(1) = 1-1 = 0$. Then for $n>1$,
$${z^n-1\over \Phi_1(z)} = \lim_{z\to 1} {z^n-1 \over z-1} = \lim_{z\to 1} {nz^{n-1} \over 1} = n,$$
giving us the formula
$$ n = \prod_{d\divides n,\,d>1} \Phi_d(1) .$$
Taking logarithms of both sides, we have
$$\log n = \sum_{d\divides n,\,d>1} \log \Phi_d(1),$$
and by the formula above for the von Mangoldt function $\Lambda$, as well as the fact that $\Lambda(1) = 0$,
we have
$$\sum_{d\divides n,\,d>1} \Lambda(d) = \sum_{d\divides n,\,d>1} \log \Phi_d(1).$$
The claim is that these two sums are actually equal term-by-term. When $n$ is prime, the statement above
already shows that $\log \Phi_p(1) = \Lambda(p) = \log p$,
and supposing the claim proven for all $m<n$, we cancel all smaller terms in the formula
to conclude that $\Lambda(n) = \log\Phi_n(1)$, which is what we needed to show.

Now we prove that $\Phi_n$ has integer coefficients. Again, the proof starts with the decomposition
of $z^n-1$ into linear factors, which this time we write as
$$z^n - 1 = \Phi_n(z) \prod_{d\divides n,\, d<n} \Phi_d(z).$$
With the base case $\Phi_1(z) = z-1$, strong induction would prove the claim if we can show that in a factorisation
$$z^n-1 = (a_0 + a_1z + \cdots + a_rz^r) (b_0 + b_1z + \cdots + b_sz^s),$$
the hypotheses $b_s = 1$ and $b_j$ being integer for all $1\le j< s$ implies that the coefficients $a_i$
are all integer for $1\le i\le r$ and that this polynomial is monic as well. The fact that $a_r = 1$ is
obvious. Then since $b_0$ is an integer and $a_0b_0 = -1$, both $a_0$ and $b_0$
must be $\pm 1$. Now assume that for some $t\ge 0$, $a_i$ is integral for all $1\le i\le t$, and consider
the coefficient of $z^{t+1}$ of the left-hand side. Call this coefficient $c_{t+1}$ and note that it is
an integer (in fact, it is either $0$ or $1$, but that is unimportant). We expand
$$c_{t+1} = a_{t+1} b_{0} + a_t b_1 + \cdots + a_0 b_{t+1},$$
and rearrange to obtain
$$a_{t+1} = {c_{t+1} - a_t b_1 - a_{t-1} b_2 - \cdots - a_0 b_{t+1}\over b_0},$$
from which we conclude by induction on $t$ that
$$a_{t+1} = \pm (c_{t+1} - a_t b_1 - a_{t-1} b_2 - \cdots - a_0 b_{t+1})$$
is an integer. This also completes the induction on $n$, so we have shown that $\Phi_n$ is a monic polynomial
with integer coefficients for all $n$.

\advsect Irreducibility of cyclotomic polynomials

A polynomial $p(z)$ with integer coefficients is said to be {\it irreducible over $\ZZ$} if it cannot be expressed
as a product of two nonconstant polynomials in $\ZZ[z]$. This section will be devoted to proving that
the cyclotomic polynomials $\Phi_n$ are irreducible over $\ZZ$.

\proclaim Theorem \advthm. The $n$th cyclotomic polynomial is irreducible over $\ZZ$.

\proof Suppose, towards a contradiction, that $\Phi_n = fg$ for nonconstant $f$ and $g$ in $\ZZ[z]$. Then
we can partition the primitive $n$ roots of unity into two disjoint nonempty classes $A$ and $B$ such that
$$f(z) = \prod_{\omega\in A} (z-\omega)\qquad\hbox{and}\qquad g(z) = \prod_{\omega\in B} (z-\omega).$$
Since any two primitive roots are powers of one another, there exists $\omega\in A$ and an integer $m>1$ such
that $\omega^m \in B$. Factor $m$ into primes $m = p_1p_1\cdots p_k$. Let $\omega_0 = \omega$ and
for $1\le i\le k$ let $\omega_i = \omega^{p_1p_2\cdots p_i}$. Let $j$ be the
smallest integer such that $\omega_j\in B$. (Since $\omega_0\in A$ and $\omega_k = \omega^m\in B$, such
a $j$ must exist.) Now formally replacing $\omega$ by $\omega^{p_1\cdots p_{j-1}}$ and setting $p=p_j$,
we have found some $\omega\in A$ and some prime $p$ such that $\omega^p\in B$.

This means that $\omega$ is a root of both $f(z)$ and $g(z^p)$. Let $h(z)$ be the greatest common divisor
of $f(z)$ and $g(z^p)$. By the Euclidean algorithm there exist polynomials $r(z)$ and $s(z)$
such that
$$ h(z) = f(z)r(z) + g(z^p)s(z),$$
showing that $h(z)$ has $\omega$ as a root and, in particular, is not constant.
Now we work modulo $p$.
By Fermat's little theorem, $a^p = a$ for all $a\in \FF_p$, so we have
$h(z^p) = h(z) = h(z)^p$ and $z^{np}-1 = z^n-1 = (z^n-1)^p$.
Now since $\Phi_n(z^p) = f(z^p)g(z^p) = f(z)^pg(z^p)$, we find that in $\FF_p$,
the polynomial $h(z)^{p+1}$ divides $\Phi_n(z^p)$, and because $\Phi_n(z^p)$ divides $z^{np}-1 = (z^n -1)^p$,
we see that $h(z)^{p+1}$ divides $(z^n-1)^p$ as well. This means that $h(z)^2$ divides $z^n-1$.
Putting $p(z) = z^n-1$, this means that there is some polynomial $q$ such that $p = h^2q$. Then we
find that $nz^{n-1} = p' = 2h h'q + h^2 q'$ is divisible by $h$, and thus $z^n-1$ and $nz^{n-1}$
have a nonconstant common factor.

On the other hand, letting $n^{-1}$ be the multiplicative inverse of $n$ in $\FF_p$, we can
run the Euclidean algorithm on $z^n-1$ and $nz^{n-1}$:
$$\eqalign{
z^n - 1 &= (n^{-1} z) (nz^{n-1}) + (-1) \cr
nz^{n-1} &= (-1)(-nz^{n-1}) + 0,\cr
}$$
discovering that the greatest common divisor of these two polynomials is $1$.
This contradiction shows that $\Phi_n(z)$ is irreducible over $\ZZ$.\slug

\section References

\bye

