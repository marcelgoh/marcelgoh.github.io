\input fontmac
\input mathmac

\def\eps{\epsilon}
\def\FF{\hbox{\bf F}}
\def\bar#1{\overline{#1}}
\def\hat#1{\widehat{#1}}
\def\norm#1{|\!|#1|\!|}
\def\bignorm#1{\big|\!\big|#1\big|\!\big|}
\def\Norm#1{\Big|\!\Big|#1\Big|\!\Big|}
\def\normm#1{\bigg|\!\bigg|#1\bigg|\!\bigg|}
\def\argmax{\limitop{\rm arg$\,$max}}
\def\dTV{d_{\rm TV}}

\def\advthm{\the\sectcount.\the\thmcount\global\advance \thmcount by 1}
\def\advsect{\global\advance\sectcount by 1\section\the\sectcount\global\thmcount=1. }
\sectcount=0

\widemargins
\bookheader{IDEMPOTENT THEOREMS AND MATRICES}{MARCEL K. GOH}

\maketitle{Idempotent theorems and matrices}{}{Marcel K. Goh}{\sl Department of Mathematics and Statistics,
McGill University}

\floattext4.5 \ninebf Abstract.
\ninepoint These expository notes move at a leisurely pace though background material needed to
appreciate the Cohen idempotent theorem and its quantitative version, due to Green and Sanders.
We introduce Schur multipliers and investigate a conjectured analogue in this setting.

\bigskip

\advsect The structure of locally compact abelian groups

{\sc A topological abelian group} is a Hausdorff space that is also an abelian group, such that
the map from $G\times G$ to $G$ sending $(x,y) \mapsto x-y$ is continuous (where the topology
on $G\times G$ is the product topology).
If, in addition, every point in $G$ is contained in some compact neighbourhood of $G$, then
$G$ is said to be a {\it locally compact abelian} group.

The following proposition shows that by passing to closures, we can always assume subgroups to be closed.

\proclaim Proposition \advthm. Let $G$ be a locally compact abelian group and $H$ a subgroup of $G$.
Then the closure $\bar H$ of $H$ is also a subgroup of $G$.

\proof Let $x,y\in \bar H$ and suppose, for a contradiction, that $xy\notin \bar H$. :

A topological space is called {\it discrete} if for every point $p$ in it, the singleton set $\{p\}$
is open.

Let $G$ be a locally compact abelian group.
The set of all measures $\mu$ on $G$ forms a topological semigroup under the convolution operation
$$\mu * \nu = \int_G 

\section References

\bye

