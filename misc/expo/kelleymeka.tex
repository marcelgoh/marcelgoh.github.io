\input fontmac
\input mathmac

\def\eps{\epsilon}
\def\FF{{\bf F}}
\def\bar#1{\overline{#1}}
\def\hat#1{\widehat{#1}}
\def\norm#1{|\!|#1|\!|}
\def\bignorm#1{\big|\!\big|#1\big|\!\big|}
\def\Norm#1{\Big|\!\Big|#1\Big|\!\Big|}
\def\normm#1{\bigg|\!\bigg|#1\bigg|\!\bigg|}

\widemargins
\bookheader{KELLEY AND MEKA'S PROOF OF ROTH'S THEOREM}{MARCEL K. GOH}

\maketitle{Kelley and Meka's proof of Roth's theorem}{by}{Marcel K. Goh}{5 August 2023}

\bigskip

\advsect Introduction

Let $A$ be a subset of $\ZZ$. We want to know how dense $A$ can be before it must contain a subset
$\{x,y,z\}\subseteq A$ with $x+z = 2y$, that is, an arithmetic progression of length $3$.

\medskip
\boldlabel Basic definitions and elementary lemmas.
We will use $G$ primarily to refer to a finite abelian group, on which we have the normalised counting measure.
For functions $f,g:G\to \CC$ we have the inner product
$$\langle f,g\rangle = \ex_{x\in G} f(x)\bar{g(x)}$$
and the $L_p$ norm
$$ \norm{f}_p = \Bigl( \ex_{x\in G} \bigl| f(x) \bigr|^p \Bigr)^{1/p}.$$
In $L_p$ spaces we have the useful H\"older inequality
$$ \bigl| \langle f,g\rangle \bigr| \le \norm{f}_p  \cdot \norm{g}_{1-p},$$
for $p,q\in [1,\infty]$ with $1/p + 1/q = 1$.
Assuming now that $f$ and $g$ are $\RR$-valued, we also have the convolution
$$ (f*g)(x) = \ex_{y\in G} f(y) g(x-y)$$
and the difference convolution
$$ (f\circ g)(x) = \ex_{y\in G} f(y) g(x+y)$$
that are related by the following adjoint property.

\parenproclaim Lemma~{\advthm} (Adjoint property).
Let $G$ be a finite abelian group and let $f,g,h:G\to \RR$. Then
$$\langle f,g*h\rangle = \langle h\circ f, g\rangle.$$

\proof First expand
$$\eqalign{
\langle f, g*h \rangle &= \ex_{x\in G} f(x) (g*h)(x) \cr
&= \ex_{x\in G} f(x) \ex_{y\in G} g(y) h(x-y) \cr
&= \ex_{y\in G} g(y) \ex_{x\in G} f(x) h(x-y). \cr
}$$
Then substituting $z = x-y$ so that $x = z+y$ yields
$$\eqalign{
\langle f, g*h\rangle &= \ex_{y\in G} g(y) \ex_{z\in G} f(z+y) h(z) \cr
&= \ex_{z\in G} (h\circ f)(z) g(z) \cr
&= \langle h\circ f, g\rangle .\noskipslug\cr
}$$

For a group $G$ the dual group $\hat G$ is the set of all homomorphisms from $G\to \CC^\times$.
The Fourier transform of $f:G\to \RR$ is the function $\hat f:\hat G\to \CC$ given by
$$ \hat f(\chi) = \ex_{x\in G} f(x)\chi(-x).$$
The following lemma describes how the convolution and difference convolution behave under the Fourier
transform.

\parenproclaim Lemma~{\advthm} (Convolution laws).
Let $G$ be a finite abelian group and let $f,g:G\to \RR$. Then the following identities hold:
\medskip
\item{i)} $\hat{f*g} = \hat f\cdot \hat g$
\smallskip
\item{ii)} $\hat{f\circ g} = \bar{\hat f} \cdot \hat g$
\medskip\noindent
In particular, $\hat {f\circ f} = |\hat f|^2$.

\proof Expand
$$\hat{f*g}(\chi) = \ex_{x\in G} (f*g)(\chi)\chi(-x)$$
and multiply the right-hand side by $1 = \chi(-y)\chi(y)$ to get
$$\hat{f*g}(\chi) = \ex_{x\in G} \ex_{y\in G} f(y)g(x-y) \chi(-y) \chi(y-x).$$
Then we may interchange the order of summation and substitute $z = x-y$ to arrive at
$$\hat{f*g}(\chi) = \ex_{y\in G} \ex_{z\in G} f(y)g(z) \chi(-y) \chi(-z) = \hat f(\chi)\hat g(\chi),$$
which proves (i). For part (ii), we expand and multiply by the same $1$ to get
$$\widehat{f\circ g}(\chi) = \ex_{x\in G} (f\circ g)(\chi)\chi(-x)
= \ex_{x\in G} \ex_{y\in G} f(y)g(x+y) \chi(y) \chi(-x-y).$$
We again interchange the order of summation; this time substituting $z = x+y$ gives us
$$\eqalign{
\hat{f\circ g}(\chi) &= \ex_{y\in G} \ex_{z\in G} f(y)g(z) \chi(y)\chi(-z) \cr
&= \bar{\ex_{y\in G} f(y)\chi(-y) } \ex_{z\in G} g(z)\chi(-z) \cr
&=  \bar{\hat f(\chi)}\hat g(\chi),\cr
}$$
which is what we wanted.\slug


For sets $A$ and $X$, let $\mu_X(A) = |A\cap X|/|X|$ denote the
relative density of $A$ in $X$, and if $X$ is understood to be a subset of a larger set $G$, then we
use $\mu_X$ also to denote the normalised indicator function given by
$$\mu_X(x) = \cases{ 1/\mu_G(X), & if $x\in X$;\cr 0, & otherwise.}.$$
The scaling is done so that $\norm{\mu_X}_1 = 1$ for any $X\subseteq G$, as can easily be checked.

\advsect H\"older lifting and unbalancing

Here we state and prove two lemmas that are general enough to apply in both the integer and finite-field
settings.

\parenproclaim Lemma~{\advthm} (H\"older lifting). Let $\eps\ge 0$ and let $A$ and $C$ be subsets of
a finite abelian group $G$, where $C$ has relative density $\gamma$. Then at least one of
the following two statements holds.
\medskip
\item{i)} $\bigl|\langle \mu_A *\mu_A, \mu_C\rangle\bigr| \le \eps$
\smallskip
\item{ii)} $\norm{\mu_A\circ \mu_A - 1}_p \ge \eps/2$ for some $p = O\bigl(\log(1/\gamma)\bigr)$.

\proof Bilinearity of the inner product gives
$$\langle \mu_A *\mu_A -1, \mu_C \rangle = \langle \mu_A * \mu_A ,\mu_C\rangle + \langle -1,\mu_C\rangle
= \langle\mu_A*\mu_A,\mu_C\rangle -1,$$
so if the first statement does not hold, then for $q = 1/(1-1/p)$, we have, by H\"older's inequality,
$$\eqalign{
\eps &< \bigl|\langle \mu_A *\mu_A, \mu_C\rangle\bigr|
\le \norm{\mu_A*\mu_A-1}_p \Bigl( \ex_{x\in G} \bigl| \mu_C(x)\bigr|^q\Bigr)^{1/q} \cr
&\le \norm{\mu_A*\mu_A-1}_p \gamma^{1/q-1} \le \norm{\mu_A*\mu_A-1}_p \gamma^{-1/p}.\cr
}$$
Letting $p$ be an even integer greater than $\log_2(1/\gamma)$, we have $\log\gamma \ge p\log (1/2)$,
whence $\gamma^{1/p} \ge 1/2$, which gives the inequality
$$ \norm{\mu_A*\mu_A - 1}_p \ge {\epsilon\over 2}.$$
Lastly, observe that since $p$ is even,

\advsect The finite-field case

First, we give bounds for the size of a $3$-AP-free subset of $\FF_q^n$. As is common with problems of this
sort, the finite-field case is a simpler prototype of the integer one.







\section References

\bye

