\input fontmac
\input mathmac

\def\eps{\epsilon}
\def\FF{{\bf F}}
\def\bar#1{\overline{#1}}
\def\hat#1{\widehat{#1}}
\def\norm#1{|\!|#1|\!|}
\def\bignorm#1{\big|\!\big|#1\big|\!\big|}
\def\Norm#1{\Big|\!\Big|#1\Big|\!\Big|}
\def\normm#1{\bigg|\!\bigg|#1\bigg|\!\bigg|}
\def\Bohr{\op{\rm Bohr}}

\widemargins
\bookheader{KELLEY AND MEKA'S PROOF OF ROTH'S THEOREM}{MARCEL K. GOH}

\maketitle{Kelley and Meka's proof of Roth's theorem}{by}{Marcel K. Goh}{12 October 2023}

\floattext4.5 \ninebf Note.
\ninepoint This exposition of Kelley and Meka's proof closely follows that of Bloom and Sisask,
but with many extra details supplied. I hope that it will be useful to other students at the Ph.D.-student
level.

\bigskip

\advsect Definitions and elementary facts

We will use $G$ primarily to refer to a finite abelian group.
For functions $f,g:G\to \CC$ we have the inner product
$$\langle f,g\rangle = \ex_{x\in G} f(x)\bar{g(x)}$$
and the $L_p$ norm
$$ \norm{f}_p = \Bigl( \ex_{x\in G} \bigl| f(x) \bigr|^p \Bigr)^{1/p}.$$
In $L_p$ spaces we have the useful H\"older inequality
$$ \bigl| \langle f,g\rangle \bigr| \le \norm{f}_p  \cdot \norm{g}_q,$$
for $p,q\in [1,\infty)$ with $1/p + 1/q = 1$.
Assuming now that $f$ and $g$ are $\RR$-valued, we also have the convolution
$$ (f*g)(x) = \ex_{y\in G} f(y) g(x-y)$$
and the difference convolution
$$ (f\circ g)(x) = \ex_{y\in G} f(y) g(x+y).$$
It is easy to check that for all $x\in G$, $(f*g)(x) = (g*f)(x)$, but with the difference convolution
we have $(f\circ g)(x) = (g\circ f)(-x)$.
We also have the following adjoint property.

\parenproclaim Proposition~{\advthm} (Adjoint property).
Let $G$ be a finite abelian group and let $f,g,h:G\to \RR$. Then
$$\langle f,g*h\rangle = \langle h\circ f, g\rangle.$$

\proof First expand
$$\eqalign{
\langle f, g*h \rangle &= \ex_{x\in G} f(x) (g*h)(x) \cr
&= \ex_{x\in G} f(x) \ex_{y\in G} g(y) h(x-y) \cr
&= \ex_{y\in G} g(y) \ex_{x\in G} f(x) h(x-y). \cr
}$$
Then substituting $z = x-y$ so that $x = z+y$ yields
$$\eqalign{
\langle f, g*h\rangle &= \ex_{y\in G} g(y) \ex_{z\in G} f(z+y) h(z) \cr
&= \ex_{z\in G} (h\circ f)(z) g(z) \cr
&= \langle h\circ f, g\rangle .\noskipslug\cr
}$$

For a group $G$ the dual group $\hat G$ is the set of all homomorphisms from $G$ to $\CC^\times$.
The Fourier transform of $f:G\to \RR$ is the function $\hat f:\hat G\to \CC$ given by
$$ \hat f(\chi) = \ex_{x\in G} f(x)\chi(-x).$$
The following proposition describes how the convolution and difference convolution behave under the Fourier
transform.

\edef\convolutions{\the\thmcount}
\parenproclaim Proposition~{\advthm} (Convolution laws).
Let $G$ be a finite abelian group and let $f,g:G\to \RR$. Then the following identities hold:
\medskip
\item{i)} $\hat{f*g} = \hat f\cdot \hat g$
\smallskip
\item{ii)} $\hat{f\circ g} = \bar{\hat f} \cdot \hat g$
\medskip\noindent
In particular, $\hat {f\circ f} = |\hat f|^2$.

\proof Expand
$$\hat{f*g}(\chi) = \ex_{x\in G} (f*g)(\chi)\chi(-x)$$
and multiply the right-hand side by $1 = \chi(-y)\chi(y)$ to get
$$\hat{f*g}(\chi) = \ex_{x\in G} \ex_{y\in G} f(y)g(x-y) \chi(-y) \chi(y-x).$$
Then we may interchange the order of summation and substitute $z = x-y$ to arrive at
$$\hat{f*g}(\chi) = \ex_{y\in G} \ex_{z\in G} f(y)g(z) \chi(-y) \chi(-z) = \hat f(\chi)\hat g(\chi),$$
which proves (i). For part (ii), we expand and multiply by the same $1$ to get
$$\widehat{f\circ g}(\chi) = \ex_{x\in G} (f\circ g)(x)\chi(-x)
= \ex_{x\in G} \ex_{y\in G} f(y)g(x+y) \chi(y) \chi(-x-y).$$
We again interchange the order of summation; this time substituting $z = x+y$ gives us
$$\eqalign{
\hat{f\circ g}(\chi) &= \ex_{y\in G} \ex_{z\in G} f(y)g(z) \chi(y)\chi(-z) \cr
&= \bar{\ex_{y\in G} f(y)\chi(-y) } \ex_{z\in G} g(z)\chi(-z) \cr
&=  \bar{\hat f(\chi)}\hat g(\chi),\cr
}$$
which is what we wanted.\slug

When we convolve two functions $\hat f$ and $\hat g$ on the dual group, we
take a sum instead of an expectation:
$$(\hat f\circ \hat g)(\chi) = \sum_{\psi\in G} \hat f(\psi)\hat g(\chi\psi^{-1}).$$
The same goes in the definition of the inner product $\langle \hat f, \hat g\rangle$.

Let $f^{*k}$ denote the $k$-fold
convolution of a function $f$.
The next proposition interprets $k$-norms in terms of $k$-fold convolutions of the Fourier transform.

\edef\knorms{\the\thmcount}
\proclaim Proposition \advthm. Let $G$ be a finite abelian group, let $k\ge 1$ be an integer, and let
$\chi_0$ denote the identity element of the dual group $\hat G$ of $G$. We have the identity
$$\ex_{x\in G} f(x)^k = {\hat f} ^{*k} (\chi_0).$$

\proof
Expand by the Fourier inversion formula to get
$$\eqalign{
\ex_{x\in G} f(x)^k &= \ex_{x\in G} \Bigl(\sum_{\chi\in \hat G} \hat f(\chi) \chi(x)\Bigr)^k \cr
&= \ex_{x\in G} \sum_{\chi_1 \in \hat G} \cdots \sum_{\chi_k \in \hat G} \hat f(\chi_1)\cdots \hat f(\chi_k)
\chi_1(x)\cdots\chi_k(x) \cr
&= \sum_{\chi_1 \in \hat G} \cdots \sum_{\chi_k \in \hat G} \hat f(\chi_1)\cdots \hat f(\chi_k)
\ex_{x\in G} \chi_1\cdots\chi_k(x). \cr
}$$
By orthogonality of characters, the inner expectation is zero when $\chi_1\cdots\chi_k \ne \chi_0$, so we have
$$\ex_{x\in G} f(x)^k = \sum_{\chi_1\cdots\chi_k = \chi_0} \hat f(\chi_1)\cdots \hat f(\chi_k) =
{\hat f}^{*p}(\chi_0).\noskipslug$$

For sets $A$ and $X$, let $\mu_X(A) = |A\cap X|/|X|$ denote the
relative density of $A$ in $X$, and if $X$ is understood to be a subset of a larger set $G$, then we
use $\mu_X$ also to denote the normalised indicator function given by
$$\mu_X(x) = \cases{ 1/\mu_G(X), & if $x\in X$;\cr 0, & otherwise.}.$$
The scaling is done so that $\norm{\mu_X}_1 = 1$ for any $X\subseteq G$, as can easily be checked.
We denote the ordinary indicator function by $\one_X = \mu_G(X)\mu_X$, and sometimes
write $\one_x$ for the indicator function $\one_{\{x\}}$ of a singleton set. Lastly, we also sometimes use the
same symbol to denote the indicator function of a statement; i.e., $\indic{P}$ is $1$ if the statement $P$
is true and $0$ if it is false.

It is easy to check that if $\mu$ has $\norm{\mu}_1 = 1$, then so does $\mu*\mu$ and $\mu\circ\mu$.
The following proposition concerns functions with this $1$-norm.

\edef\normonefourier{\the\thmcount}
\proclaim Proposition \advthm. Let $G$ be a finite abelian group.
If $\mu:G\to \RR_{\ge 0}$ has $\norm{\mu}_1 = 1$, then
$$ \hat{\mu-1} = \hat \mu (1-\one_{\chi_0}).$$

\proof We expand
$$\eqalign{
\hat{\mu-1}(\chi) &=  \ex_{x\in G} \bigl(\mu(x)-1\bigr) \chi(-x) \cr
&= \ex_{x\in G} \mu(x)\chi(-x) - \ex_{x\in G} \chi(-x).\cr
&= \hat\mu(\chi) - \ex_{x\in G} \chi(-x).\cr
}$$
If $\chi\ne\chi_0$, then the expectation vanishes, and
if $\chi = \chi_0$, then the expectation
clearly equals $1$, and $\mu(\chi_0) = \norm{\mu}_1 = 1$, so the whole
expression is zero.\slug

For $\mu:G\to \RR_{\ge 0}$ with $\norm{\mu}_1 = 1$ and arbitrary $f,g:G\to \CC$ we write
$$\langle f,g\rangle_\mu = \ex_{x\in G} \mu(x)f(x)\bar{g(x)}$$
for the inner product relative to $\mu$, and for $1\le p<\infty$ we write
$$\norm{f}_{p(\mu)} = \Bigl( \ex_{x\in G} \mu(x) \bigl| f(x)\bigr|^p \Bigr)^{1/p}$$
for the $L_p$ norm relative to $\mu$. The following basic proposition establishes the monotonicity of
$L_p$ norms with respect to $p$.

\edef\monotonenorm{\the\thmcount}
\parenproclaim Proposition~{\advthm} (Monotonicity of $L_p$ norms). Let $G$ be a finite abelian group.
Let $\mu:G\to \RR_{\ge 0}$ have $\norm{\mu}_1 = 1$ and let $f:G\to \CC$.
For $1\le p<q<\infty$, we have
$$\norm{f}_{p(\mu)} \le \norm{f}_{q(\mu)}.$$

\proof Let $r = q/p > 1$ and let $s = r/(r-1)$ so that $1/r+1/s = 1$.
We have
$$\ex_{x\in G} \mu(x) \bigl| f(x)\bigr|^p = \ex_{x\in G} \mu(x) \bigl| f(x)\bigr|^{q/r} \cdot 1$$
Now by H\"older's inequality, we have
$$\eqalign{
\ex_{x\in G} \mu(x) \bigl| f(x)\bigr|^p &\le
\Bigl(\ex_{x\in G} \mu(x) \bigl| f(x)\bigr|^q \Bigr)^{1/r} \Bigl( \ex_{x\in G} 1^s\Bigr)^{1/s} \cr
&= \Bigl(\ex_{x\in G}\mu(x)\bigl| f(x)\bigr|^q \Bigr)^{p/q}. \cr
}$$
Taking $p$th roots of both sides now produces the inequality we wanted.\slug

For convenience, when $\mu:G\to \RR_{\ge 0}$ has $\norm{\mu}_1 = 1$ and $X\subseteq G$, we write
$$\mu(X) = \norm{\one_X}_{1(\mu)} = \ex_{x\in G} \mu(x) \one_X(x),$$
and refer to this quantity as the {\it density of $X$ relative to $\mu$}.

\advsect H\"older lifting and unbalancing for finite groups

With preliminaries out of the way, we begin the proof of Kelley and Meka~\ref{kelleymeka2023}, as described
and reworked by Bloom and Sisask~\ref{bloomsisask2023}.
In this section we perform the first two steps of the proof, in the general setting of finite groups.

\edef\holderlifting{\the\thmcount}
\parenproclaim Lemma~{\advthm} (H\"older lifting). Let $\eps\ge 0$ and let $A$ and $C$ be subsets of
a finite abelian group $G$, where $C$ has relative density at least $\gamma$. Then at least one of
the following two statements holds.
\medskip
\item{i)} $\bigl|\langle \mu_A *\mu_A, \mu_C\rangle - 1\bigr| \le \eps$
\smallskip
\item{ii)} $\norm{\mu_A\circ \mu_A - 1}_p \ge \eps/2$ for some $p = O\bigl(\log(1/\gamma)\bigr)$.

\proof Linearity of the inner product in the first argument gives
$$\langle \mu_A *\mu_A -1, \mu_C \rangle = \langle \mu_A * \mu_A ,\mu_C\rangle + \langle -1,\mu_C\rangle
= \langle\mu_A*\mu_A,\mu_C\rangle -1,$$
so if the first statement does not hold, then for $q = 1/(1-1/p)$, we have, by H\"older's inequality,
$$\eqalign{
\eps &< \bigl|\langle \mu_A *\mu_A-1, \mu_C\rangle\bigr|
\le \norm{\mu_A*\mu_A-1}_p \Bigl( \ex_{x\in G} \bigl| \mu_C(x)\bigr|^q\Bigr)^{1/q} \cr
&\le \norm{\mu_A*\mu_A-1}_p \gamma^{1/q-1} \le \norm{\mu_A*\mu_A-1}_p \gamma^{-1/p}.\cr
}$$
Letting $p$ be an even integer greater than $\log_2(1/\gamma)$, we have $\log\gamma \ge p\log (1/2)$,
whence $\gamma^{1/p} \ge 1/2$. This gives the inequality
$$ \norm{\mu_A*\mu_A - 1}_p \ge {\epsilon\over 2}.$$
Since $p$ is even,
$$ \norm{\mu_A*\mu_A-1}_p^p = \ex_{x\in G} \bigl|(\mu_A*\mu_A-1)(x)\bigr|^p
= \ex_{x\in G} (\mu_A*\mu_A-1)(x)^p,$$
and we can apply Proposition~{\knorms} to get
$$\norm{g}_p^p = \hat g^{*p} (\chi_0),$$
where we have put $g = \mu_A*\mu_A-1$.
It was noted earlier that $\mu_A*\mu_A$ has $1$-norm equal to $1$, so we can apply
Propositions~{\normonefourier} and~{\convolutions} in that order to get
$$\norm{\mu_A*\mu_A-1}_p^p
= \bigl(\hat{\mu_A*\mu_A}(1-\one_{\chi_0})\bigr)^{*p} (\chi_0)
= \bigl(\hat{\mu_A}^2(1-\one_{\chi_0})\bigr)^{*p} (\chi_0).$$
Repeating this whole process with $\mu_A\circ\mu_A$ in place of $\mu_A*\mu_A$ produces the very similar identity
$$\norm{\mu_A\circ\mu_A-1}_p^p = \bigl(|\hat{\mu_A}|^2(1-\one_{\chi_0})\bigr)^{*p} (\chi_0),$$
from which we conclude that
$$\norm{\mu_A\circ\mu_A-1}_p^p \ge \norm{\mu_A*\mu_A-1}_p^p \ge {\epsilon\over 2}.\noskipslug$$

This lemma tells us that if $\langle \mu_A*\mu_A,\mu_C\rangle \ge 1/2$, then
$\norm{\mu_A\circ \mu_A-1}_p \ge 1/4$
for some $p = O\bigl(\log(1/\gamma)\bigr)$. This information can then be fed to the following general lemma.

\edef\unbalancing{\the\thmcount}
\parenproclaim Lemma~{\advthm} (Unbalancing of spectrally nonnegative functions). Let $\eps\in (0,1)$
and let $\nu : G\to \RR_{\ge 0}$ have $\norm{\mu}_1 = 1$ and $\hat \nu\ge 0$. If $f : G\to \RR$ has
$\hat f\ge 0$ and $\norm{f}_{p(\nu)} \ge \eps$ for some $p\ge 1$, then
$$ \norm{f+1}_{p'(\nu)} \ge 1 + {\eps\over 2}$$
for some $p' = O\bigl(\eps^{-1} \log(\eps^{-1}) p\bigr)$.

\proof Proposition~{\monotonenorm} tells us that $\norm{f}_{p(\mu)}$ is monotonically increasing in $p$, so
without loss of generality we can pick $p$ odd and at least $5$. As usual, we denote the identity in $\hat G$
by $\chi_0$. Using the Fourier inversion formula
and orthogonality of characters as we did in the proof of Proposition~{\knorms}, we observe that
$$\eqalign{
\norm{f}_{p(\nu)}^p &= \ex_{x\in G}
\Bigl(\sum_{\chi\in \hat G} \hat\nu(\chi)\chi(x)\Bigr)\Bigl(\sum_{\chi\in \hat G} \hat f(\chi)\chi(x)\Bigr)^p \cr
&= \sum_{\chi_1\in \hat G} \cdots \sum_{\chi_{p+1}\in \hat G}
\hat f(\chi_1)\cdots\hat f(\chi_p) \hat \nu(\chi_{p+1} )\ex_{x\in G} \chi_1(x)\cdots\chi_{p+1}(x) \cr
&= \sum_{\chi_1\cdots\chi_{p+1} =\chi_0} \hat f(\chi_1) \cdots \hat f(\chi_p) \hat\nu(\chi_{p+1} )\cr
&= \hat\nu * {\hat f}^{*p}(\chi_0). \cr
}$$
Let $P = \{x\in G : f(x)\ge 0\}$ and let $g(x) = \max\{f(x), 0\}$. It is easy to see that
$2g(x) = f(x) + \bigl|f(x)\bigr|$, so
$$\eqalign{
2\langle \one_P, f^p\rangle_\nu &= 2 \ex_{x\in G} \nu(x)\one_P(x)f(x)^p \cr
&= 2\ex_{x\in G} \nu(x) g(x) f(x)^{p-1}  \cr
&= \langle 2g,f^{p-1}\rangle_\nu \cr
&= \ex_{x\in G} \nu(x) f(x)^p + \bigl\langle |f|, f^{p-1}\bigr\rangle_\nu \cr
&= \hat\nu * {\hat f}^{*p}(\chi_0)+ \bigl\langle |f|, |f|^{p-1}\bigr\rangle_\nu, \cr
}$$
where in the last line we used the fact that $f$ is real-valued as well as evenness of $p-1$.
Since the Fourier transforms of $f$ and $\nu$ are both nonnegative, the first term is nonnegative, so
$$\langle \one_P, f^p\rangle_\nu \ge {\bigl\langle |f|, |f|^{p-1}\bigr\rangle_\nu \over 2}
= {\norm{f}_{p(\nu)}^p\over 2} \ge {\eps^p\over 2} .$$
Now let $T = \{x\in P : f(x) \ge 3\eps / 4\}$. Then
$$\eqalign{
\langle \one_T, f^p \rangle_\nu &= \langle \one_P - \one_{P\setminus T}, f^p \rangle_\nu \cr
&\ge \langle\one_P, f^p\rangle_\nu - \langle \one_{P\setminus T}, f^p\rangle_\nu \cr
&\ge {\eps^p\over 2}- \ex_{x\in G} \one_{P\setminus T} f(x)^p\nu(x)\cr
&> {\eps^p\over 2}- \ex_{x\in G} (3\eps/4)^p \nu(x)\cr
&\ge {\eps^p\over 4}, \cr
}$$
where in the last line we used the fact that $(3/4)^p \le (3/4)^5 < 243/1024 < 4$.
From this we deduce
$$\eqalign{
{\eps^p\over 4} &\le \langle \one_T,f^p\rangle_\nu \cr
&= \ex_{x\in G} \bigl( \nu(x)^{1/2}\one_T(x)\bigr) \bigl(\nu(x)^{1/2} f(x)^p\bigr) \cr
&\le \Bigl( \ex_{x\in G} \nu(x)\one_T(x)^2\Bigr)^{1/2} \Bigl(\ex_{x\in G}\nu(x) f(x)^{2p}\Bigr)^{1/2} \cr
&= \Bigl( \ex_{x\in G} \nu(x)\one_T(x)\Bigr)^{1/2} \Bigl(\ex_{x\in G} \nu(x) |f(x)|^{2p}\Bigr)^{p/(2p)} \cr
&= \nu(T)^{1/2} \norm{f}_{2p(\nu)}^p\cr
}$$
by the Cauchy--Schwarz inequality.

Now if $\norm{f+1}_{2p(\nu)} > 2$, then we could take $p' = 2p$, so assume that this norm is at most $2$.
By the triangle inequality, we have
$$\norm{f}_{2p(\nu)} \le \norm{-1}_{2p(\nu)} + \norm{f+1}_{2p(\nu)} \le 3,$$
hence
$$ \nu(T)^{1/2} 3^p \ge {\eps^p\over 4}.$$
Once again using the fact that $4 < 1024/243$, we have $4^{1/p} < 4/3$ and thus
$$ \nu(T) \ge {\eps^{2p} \over 16 \cdot 3^{2p}} = \biggl( {\eps \over 4^{1/p} \cdot 3} \biggr)^{2p}
> \biggl({\eps\over 4}\biggr)^{2p}.$$
This allows us to bound
$$\eqalign{
\norm{f+1}_{p'(\nu)}
&= \Bigl( \ex_{x\in G} \nu(x) \bigl| f(x)+1\bigr|^{p'}\Bigr)^{1/p'} \cr
&\ge \Bigl( \ex_{x\in G} \nu(x) \one_T(x) \bigl| f(x)+1\bigr|^{p'}\Bigr)^{1/p'} \cr
&\ge \bigl( \nu(T) (1+3\eps/4)^{p'} \bigr)^{1/p'} \cr
&> \biggl( {\eps \over 4} \biggr)^{2p/p'} \biggl(1+{3\over 4}\eps\biggr). \cr
}$$
Now if $p' \ge (8p/\eps)\log(4/\eps) = O\bigl(\eps^{-1} \log(\eps^{-1}) p\bigr)$, then
$\eps/4 \ge (2p/p')\log(4/\eps)$ and thus
$$ -{2p\over p'} \log\biggl({4\over \eps}\biggr) \ge -{\eps\over 4}.$$
Taking $e$ to the power of both sides gives us
$$\biggl({4\over \eps}\biggr) ^{-2p/p'} \ge e^{-\eps/4} \ge 1-{\eps\over 4},$$
and plugging this in above gives the bound
$$\norm{f+1}_{p'(\nu)} > \biggl(1-{\eps\over 4}\biggr)\biggl(1+{3\over 4}\eps\biggr)
= 1+{\eps \over 2} -{3\eps^2\over 16} \ge 1+{\eps\over 2},$$
which is what we needed.\slug

\advsect Dependent random choice

The next lemma uses a dependent random choice argument to pass the information from the previous
step down to high density subsets, which allows us to iterate the argument.

\parenproclaim Lemma~{\advthm} (Dependent random choice). Let $G$ be a finite abelian group and let $A$
be a subset of $G$ with density $\alpha$. Let $B_1,B_2\subseteq G$ and $\mu = \mu_{B_1}\circ \mu_{B_2}$.
For any function $f:G\to \RR_{\ge 0}$ there exist sets $A_1\subseteq B_1$ and $A_2\subseteq B_2$ with
densities satisfying
$$\min\bigl\{\mu_{B_1}(A_1), \mu_{B_2}(A_2)\bigr\}
\ge {1\over 4} \alpha^{2p} \norm{\mu_A\circ\mu_A}_{p(\mu)}^{2p}.$$
and such that
$$ \langle \mu_{A_1}\circ \mu_{A_2}, f\rangle \le 2 {\bigl\langle (\mu_A\circ\mu_A)^p, f\bigr\rangle_\mu
\over \norm{\mu_A\circ\mu_A}_{p(\mu)}^p}.$$

\proof For $s = (s_1,\ldots,s_p)\in G^p$ let
$A_1(s) = B_1\cap (A+s_1) \cap \cdots \cap (A+s_p)$, and define $A_2(s)$ analogously.
First we expand
$$\eqalign{
\bigl \langle (\mu_A\circ \mu_A)^p, f\bigr\rangle_{\mu}
&= \ex_{x\in G} \mu(x) (\mu_A\circ \mu_A)(x)^p f(x) \cr
&=  \ex_{x\in G} \ex_{y\in G} \mu_{B_1}(y) \mu_{B_2}(x+y)  (\mu_A\circ \mu_A)(x)^p f(x)\cr
&= {1\over |B_1|\cdot |B_2|} \sum_{x\in G} \sum_{y\in G}
\one_{B_1}(y) \one_{B_2}(x+y) (\mu_A\circ\mu_A)(x)^p f(x).\cr
}$$
Renaming $b_1 = y$ and performing the change of variable $b_2 = x+b_1 = x+y$, we have
$$\eqalign{
\bigl \langle (\mu_A\circ \mu_A)^p, f\bigr\rangle_{\mu}
&= {1\over |B_1||B_2|} \displaystyle\sum_{\scriptstyle b_1\in G\atop \scriptstyle b_2\in G}
\one_{B_1}(b_1) \one_{B_2}(b_2) (\mu_A\circ\mu_A)(b_2-b_1)^p f(b_2-b_1)\cr
&= \ex_{b_1\in B_1, b_2\in B_2} (\mu_A\circ\mu_A)(b_2-b_1)^p f(b_2-b_1)\cr
&= \ex_{b_1\in B_1, b_2\in B_2} \Bigl( \alpha^{-2}\ex_{y\in G} \one_A(y) \one_A(b_2-b_1+y)\Bigr)^p f(b_2-b_1).\cr
}$$
Now since $y\in A$ if and only if $b_1\in A+b_1-y$ and $b_2-b_1+y \in A$ if and only if $b_2 \in A+b_1-y$, so
writing $t = b_1-y$ and changing variables, we have
$$\eqalign{
\bigl \langle (\mu_A\circ \mu_A)^p, f\bigr\rangle_{\mu}
&= \ex_{b_1\in B_1, b_2\in B_2}
 \Bigl( \alpha^{-2}\ex_{t\in G} \one_{A+t}(b_1) \one_{A+t}(b_2)\Bigr)^p f(b_2-b_1)\cr
&= \alpha^{-2p}\ex_{b_1\in B_1, b_2\in B_2} \ex_{s\in G^p} \one_{A_1(s)}(b_1) \one_{A_2(s)}(b_2) f(b_2-b_1).\cr
}$$
Putting $y=b_2-b_1$ so that $b_2 = y+b_1$, we have
$$\eqalign{
\bigl \langle (\mu_A\circ \mu_A)^p, f\bigr\rangle_{\mu}
&= \alpha^{-2p} \ex_{s\in G^p} \ex_{b_1\in B_1} {|G|\over|B_2|} \ex_{y\in G}
 \one_{A_1(s)}(b_1) \one_{A_2(s)}(y+b_1) f(y)\cr
&= {|G|\over \alpha^{2p}|B_2|} \ex_{s\in G^p} \ex_{b_1\in B_1}
\bigl\langle \one_{A_1(s)}\circ \one_{A_2(s)}, f\bigr\rangle\cr
&= {|G|^2\over \alpha^{2p}|B_1|\cdot|B_2|} \ex_{s\in G^p} 
\bigl\langle \one_{A_1(s)}\circ \one_{A_2(s)}, f\bigr\rangle.\cr
}$$
Thus we let $\beta_i = |B_i|/|G|$ and $\alpha_i(s) = {|A_i(s)|\over|G|}$
for $i\in\{1,2\}$, and then apply the above in the case where $f$ is the constant function $1$ to obtain
$$\eqalign{
\bignorm{\mu_A\circ \mu_A}_{p(\mu)}^p
&= {|G|^2\over \alpha^{2p}|B_1|\cdot|B_2|} \ex_{s\in G^p} 
\ex_{x\in G} \ex_{y\in G} \one_{A_1(s)}(y) \one_{A_2(s)}(x+y)\cr
&= {1\over \alpha^{2p}\beta_1\beta_2} \ex_{s\in G^p} 
\ex_{y\in G} \one_{A_1(s)}(y) \ex_{x\in G}\one_{A_2(s)}(x+y)\cr
&= {1\over \alpha^{2p}\beta_1\beta_2} \ex_{s\in G^p} \alpha_1(s)\alpha_2(s)\cr
}$$
The constants out front do not depend on $f$, so we see that
$$ {\bigl\langle (\mu_A\circ\mu_A)^p, f\bigr\rangle_{\mu}\over \norm{\mu_A\circ\mu_A}_{p(\mu)}^p}
= {\ex_{s\in G^p} \bigl\langle \one_{A_1(s)}\circ \one_{A_2(s)}, f\bigr\rangle
\over \ex_{s\in G^p} \alpha_1(s)\alpha_2(s)};$$
call this quotient $\eta$ for brevity.
Now we consider the quantity
$$\ex_{s\in G^p} \ex_{x\in G} \one_{A_1(s)}(x).$$
For a given $s = (s_1,\ldots,s_p)\in G^p$ and $x\in G$, the corresponding $\one_{A_1(s)}(x)$ term is $1$ if and
only if $x\in B_1$ and $x-s_i\in A$ for all $1\le i\le p$. Hence we have
$$\ex_{s\in G^p} \ex_{x\in G} \one_{A_1(s)}(x) = {|B_1|\cdot |A|^p\over |G|^{p+1}} = \alpha^p\beta_1.$$
The analogous identity holds for $A_2(s)$. So, letting
$$M = {1\over 2} \alpha^p (\beta_1\beta_2)^{1/2} \norm{\mu_A\circ\mu_A}^p_{p(\mu)},$$
we have
$$\eqalign{
\ex_{s\in G^p} \indic{\alpha_1(s)\alpha_2(s)<M^2} \alpha_1(s)\alpha_2(s)
 &< \ex_{s\in G^p} M \sqrt{\alpha_1(s)\alpha_2(s)} \cr
 &\le \Bigl( \ex_{s\in G^p} M \alpha_1(s) \Bigr)^{1/2} \Bigl( \ex_{s\in G^p} M \alpha_2(s) \Bigr)^{1/2} \cr
 &= M \Bigl( \ex_{s\in G^p} \ex_{x\in G} \one_{A_1(s)}(x) \Bigr)^{1/2} \cr
 &\hskip70pt\Bigl( \ex_{s\in G^p} \ex_{x\in G} \one_{A_2(s)}(x) \Bigr)^{1/2} \cr
 &= M\alpha^p \sqrt{\beta_1\beta_2} \cr
 &= {1\over 2} \alpha^{2p}\beta_1\beta_2 \norm{\mu_A\circ\mu_A}^p_{p(\mu)} \cr
&= {1\over 2} \ex_{s\in G^p} \alpha_1(s)\alpha_2(s) \cr
}$$
and consequently
$$\ex_{s\in G^p} \indic{\alpha_1(s)\alpha_2\ge M^2} \alpha_1(s)\alpha_2(s)
> {1\over 2} \ex_{s\in G^p} \alpha_1(s)\alpha_2(s).$$
So we have
$$\eqalign{
\ex_{s\in G^p} \bigl\langle \one_{A_1(s)}\circ\one_{A_2(s)}, f\bigr\rangle
&= \eta \ex_{s\in G^p} \alpha_1(s)\alpha_2(s) \cr
  &< 2\eta \ex_{s\in G^p} \alpha_1(s)\alpha_2(s)\indic{\alpha_1(s)\alpha_2(s)\ge M^2}, \cr
}$$
and thus there must be some $s$ such that
$$\bigl\langle \one_{A_1(s)}\circ\one_{A_2(s)}, f\bigr\rangle
< 2\eta \alpha_1(s)\alpha_2(s)\indic{\alpha_1(s)\alpha_2(s)\ge M^2}.$$
Since $f(x)\ge 0$ for all $x$, the left-hand side is nonnegative, meaning that the right-hand side cannot be $0$.
Thus such an $s$ must satisfy $\alpha_1(s)\alpha_2(s)\ge M^2$. Letting $A_1 = A_1(s)$ and
$A_2 = A_2(s)$ for this particular $s$, we have
$${|A_1|\cdot|A_2|\over |G|^2} \ge
{1\over 4} \alpha^{2p} {|B_1|\cdot|B_2|\over |G|^2}\norm{\mu_A\circ\mu_A}_{p(\mu)}^{2p},$$
whence
$$\mu_{B_1}(A_1)\mu_{B_2}(A_2) \ge {1\over 4} \alpha^{2p} \norm{\mu_A\circ\mu_A}_{p(\mu)}^{2p},$$
so neither $\mu_{B_1}(A_1)$ nor $\mu_{B_2}(A_2)$ can be less than the right-hand side.

On the other hand, letting $\alpha_1 = \alpha_1(s)$ and $\alpha_2 = \alpha_2(s)$, we also have
$$\eqalign{
\langle\mu_{A_1}\circ\mu_{A_2}, f\rangle &= \ex_{x\in G} \ex_{y\in G} \mu_{A_1}(y)\mu_{A_2}(x+y) f(x) \cr
&= {\alpha_1}^{-1}{\alpha_2}^{-1} \ex_{x\in G} \ex_{y\in G} \one_{A_1}(y)\one_{A_2}(x+y) f(x) \cr
&= {\alpha_1}^{-1}{\alpha_2}^{-1} \langle \one_{A_1}\circ\one_{A_2},f\rangle  \cr
&< 2\eta \cr
&= 2 {\bigl\langle (\mu_A\circ\mu_A)^p, f\bigr\rangle_\mu \over \norm{\mu_A\circ\mu_A}_{p(\mu)}^p}, \cr
}$$
which proves the lemma.\slug

This lemma is slightly more general than we shall require. The version that suffices for all our applications
is the following.

\edef\dependentrandom{\the\thmcount}
\proclaim Lemma~\advthm. Let $G$ be a finite abelian group, let $p\ge 1$ be an integer, and let
$\eps,\delta>0$. Let $B_1$ and $B_2$ be subsets of $G$ and let $\mu = \mu_{B_1}\circ \mu_{B_2}$.
If $A\subseteq G$ has density $\alpha$ and
$$S = \bigl\{x\in G : (\mu_A\circ\mu_A)(x) > (1-\eps)\norm{\mu_A\circ\mu_A}_{p(\mu)}\bigr\},$$
then there exist $A_1\subseteq B_1$ and $A_2\subseteq B_2$ with densities satisfying
$$\min\bigl\{ \mu_{B_1}(A_1), \mu_{B_2}(A_2)\bigr\}
= \Omega\bigl( (\alpha \norm{\mu_A\circ\mu_A}_{p(\mu)})^{2p+O(\eps^{-1}\log(\delta^{-1}))}\bigr).$$
such that
$$\langle \mu_{A_1}\circ\mu_{A_2}, \one_S\rangle \ge 1-\delta.$$

\proof Let $p'$ be the smallest even integer at least $p+\eps^{-1}\log(\delta^{-1})$.
By the previous lemma applied to the function
$\one_{G\setminus S}$, there exist sets $A_1\subseteq B_1$ and $A_2\subseteq B_2$ with densities
satisfying
$$\eqalign{
\min\bigl\{\mu_{B_1}(A_1), \mu_{B_2}(A_2)\bigr\} &\ge
{1\over 4} \alpha^{2p'} \norm{\mu_A\circ\mu_A}_{p'(\mu)}^{2p'}  \cr
&\ge {1\over 4}\bigl(\alpha\norm{\mu_A\circ\mu_A}_{p(\mu)}\bigr)^{2p+2\eps^{-1}\log(\delta^{-1})+O(1)} \cr
&= \Omega\bigl( (\alpha \norm{\mu_A\circ\mu_A}_{p(\mu)})^{2p+O(\eps^{-1}\log(\delta^{-1}))}\bigr)\cr
}$$
such that
$$\langle \mu_{A_1}\circ\mu_{A_2}, \one_{G\setminus S}\rangle
\le {\bigl\langle (\mu_A\circ\mu_A)^{p'}, \one_{G\setminus S}\bigr\rangle_{\mu} \over
\norm{\mu_A\circ\mu_A}^{p'}_{p'(\mu)}}.$$
Our construction of $S$ ensures that
$${\bigl\langle (\mu_A\circ\mu_A)^{p'}, \one_{G\setminus S}\bigr\rangle_{\mu} \over
\norm{\mu_A\circ\mu_A}^{p'}_{p'(\mu)}} \le (1-\eps)^{p'},$$
and since $p'\ge \eps^{-1}\log(\delta^{-1})$, we have
$$(1-\eps)^{p'}\le e^{-\eps p'} \le \delta.$$
Putting everything together, we have
$$\langle \mu_{A_1}\circ\mu_{A_2}, \one_{G\setminus S}\rangle \le \delta,$$
so that
$$\langle \mu_{A_1}\circ\mu_{A_2}, \one_S\rangle
= 1- \langle \mu_{A_1}\circ\mu_{A_2}, \one_{G\setminus S}\rangle \ge 1-\delta,$$
which completes the proof.\slug


\advsect The finite-field case

We now use the methods of Kelley and Meka to give upper bounds on the size of a subset of $\FF_q^n$
without any three-term arithmetic progressions.
First, we restate the dependent random choice lemma in the special case that applies to this finite field
context.

\edef\deprandcorollary{\the\thmcount}
\proclaim Corollary~{\advthm}. Let $p\ge 1$ be an integer and $\eps\in (0,1/2]$. If $A\subseteq G$ is such that
$\norm{\mu_A\circ\mu_A}_p \ge 1+\eps$ and
$S = \bigl\{x\in G: (\mu_A\circ\mu_A)(x) > 1+\eps/2\bigr\}$, then there are
subsets $A_1$ and $A_2$ of $G$, each of density $\Omega\bigl( \alpha^{2p+O(\eps^{-1}\log(\eps^{-1}))}\bigr)$,
such that
$$\langle \mu_{A_1}\circ\mu_{A_2}, \one_S\rangle \ge 1-\eps/8.$$

\proof Let $S' = \bigl\{x\in G : (\mu_A\circ\mu_A)(x) > (1+\eps)\norm{\mu_A\circ\mu_A}_p\bigr\}$, and apply
Lemma~{\dependentrandom} with the same $p$ and $\eps$, but $\delta$ set to $\eps/8$, $S$ set to $S'$,
and $B_1 = B_2 = G$ so that $\mu = \mu_{B_1} = \mu_{B_2}$ is the constant function $1$ on $G$. Hence the
sets $A_1$ and $A_2$ given by the lemma will each have density
$$\Omega\bigl((\alpha(1+\eps))^{2p+O(\eps^{-1}\log(8/\eps))}\bigr)
=\Omega\bigl(\alpha^{2p+O(\eps^{-1}\log(\eps^{-1}))}\bigr)$$
in $G$, and since
$$(1-\eps)\norm{\mu_A\circ\mu_A}_p \ge (1-\eps)(1+\eps) = 1-\eps^2 \ge 1-\eps/2,$$
we have $S'\subseteq S$ and thus
$$\langle \mu_{A_1}\circ\mu_{A_2}, \one_S\rangle \ge
\langle \mu_{A_1}\circ\mu_{A_2}, \one_{S'}\rangle \ge 1-\eps/8.\noskipslug$$

There is a theorem that we need which, for now, we shall just state without proof. (This is Theorem 3.2
of~\ref{schoensisask2016}.)

\edef\almostperiodicity{\the\thmcount}
\parenproclaim Theorem~{\advthm} (Schoen--Sisask, {\rm 2016}).
Let $\eps\in (0,1)$, let $S\subseteq \FF_q^n$, and
let $A_1,A_2\subseteq \FF_q^n$ be subsets
of relative density at least $\alpha$. There is a subspace $V$ of codimension
$O\bigl(\eps^{-2}\log(\eps^{-1}\alpha^{-1})^2\log(\alpha^{-1})^2\bigr)$
such that
$$\bigl| \langle \mu_V*\mu_{A_1}*\mu_{A_2}, \one_S\rangle -
 \langle \mu_{A_1}*\mu_{A_2}, \one_S\rangle\bigr| \le \eps.$$

The following lemma encapsulates the density increment argument that underlies the proof of Roth's theorem.

\edef\densityincrement{\the\thmcount}
\parenproclaim Lemma~{\advthm} (Density increment). Let $\eps\in (0,1)$ and let $A$ and $C$
be subsets of $G =\FF_q^n$ with relative densities $\alpha$ and $\gamma$, respectively. Then either
\medskip
\item{i)} $\bigl|\langle \mu_A*\mu_A,\mu_C\rangle-1\bigr| \le \eps$; or
\smallskip
\item{ii)} there is a subspace $V$ of codimension
$$O\Bigl(\eps^{-2}\bigl(\log(1/\gamma)+\eps^{-1}\log(\eps^{-1})\bigr)^4\log(1/\alpha)^4\Bigr)$$
such that $\max_{x\in G}\;(\one_A*\mu_V)(x) \ge \bigl(1+\eps/32\bigr)\alpha$.
\medskip

\proof Suppose that (i) fails. Then by Lemma~{\holderlifting} there is some $p = O\bigl(\log(1/\gamma)\bigr)$
such that $\norm{\mu_A\circ\mu_A - 1}_p\ge \eps/ 2$. By the second convolution law (part (ii) of
Proposition~{\convolutions}), $\mu_A\circ\mu_A$ is a nonnegative function, and note also that if
$\nu$ is the constant function $1$ on $\FF_q^n$, then for any $\chi:\hat{\FF_q^n}\to \CC$,
$$\hat\nu(\chi) = \ex_{x\in G} \nu(x)\chi(-x) = \cases{1, & if $\chi$ is the trivial character;
\cr 0, & otherwise.}$$
This implies in particular that $\hat\nu \ge 0$,
so by Lemma~{\unbalancing} applied with $f=\mu_A\circ\mu_A$ and the constant function $1$ for $\nu$, we find
that $\norm{\mu_A\circ\mu_A}_{p'}\ge 1+\eps/4$ for some
$p' = O\bigl((2/\eps)\log(2/\eps) p\bigr)= O\bigl(\eps^{-1}\log(\eps^{-1})\log(1/\gamma)\bigr)$.
Let $C(\eps) = \eps^{-1}\log(\eps^{-1})$. By Corollary~{\deprandcorollary},
there are sets $A_1,A_2\subseteq G$, each of density
$\Omega(\alpha^{2p'+O(C(\eps))})$, such that
$$\langle \mu_{A_1}\circ\mu_{A_2}, \one_S\rangle \ge 1 - \eps/32,$$
where $S = \bigl\{ x \in G:(\mu_A\circ\mu_A)(x)\ge 1+\eps/8\bigr\}$. Feeding $-A_1$, $A_2$, and $S$
into Theorem~{\almostperiodicity} with $\eps/32$ gives us a subspace $V$ of codimension
$$\eqalign{
O\Bigl(\eps^{-2}&\log\bigl(\eps^{-1}\alpha^{-2p'-O(C(\eps))}\bigr)^2
\log\bigl(\alpha^{-2p'-O(C(\eps))}\bigr)^2\Bigr) \cr
&= O\Bigl(\eps^{-2} (2p'+C(\eps))^2\log(1/\alpha)^2
\bigl((2p'+C(\eps))\log(1/\alpha)+\log(\eps^{-1})\bigr)^2
\Bigr)\cr
&=O\Bigl(\eps^{-2}\bigl(\log(1/\gamma)+C(\eps)\bigr)^4\log(1/\alpha)^4\Bigr) \cr
}$$
such that
$$\bigl| \langle \mu_V*\mu_{-A_1}*\mu_{A_2}, \one_S\rangle -
 \langle \mu_{-A_1}*\mu_{A_2}, \one_S\rangle\bigr| \le \eps/32.$$
It is easily checked that $\mu_{-A_1}*\mu_{A_2} = \mu_{A_1}\circ\mu_{A_2}$, so we find that
$$\bigl\langle \mu_V*(\mu_{A_1}\circ\mu_{A_2}), \one_S\bigr\rangle \ge 1-\eps/16.$$
Now we observe that
$$\bignorm{(\mu_{A_1}\circ\mu_{A_2})\circ\mu_{A_2}}_1
= \ex_{z\in G}\mu_{A_1}(z)\ex_{y\in G} \mu_{A_2}(y+z) \ex_{x\in G} \mu_A(x+y) = 1,$$
so that,
$$\eqalign{
\max_{x\in G}\;(\mu_V*\one_A)(x) &= \alpha\max_{x\in G}\;(\mu_V*\mu_A)(x) \cr
&= \alpha \bignorm{(\mu_{A_1}\circ\mu_{A_2})\circ\mu_A}_1 \max_{x\in G}\;(\mu_V*\mu_A)(x)\cr
&\ge \alpha\bigl\langle \mu_V*\mu_A, (\mu_{A_1}\circ\mu_{A_2})\circ\mu_A\bigr\rangle\cr
&= \alpha\bigl\langle \mu_V*\mu_A * (\mu_{A_1}\circ\mu_{A_2}), \mu_A\bigr\rangle\cr
&= \alpha\bigl\langle \mu_V* (\mu_{A_1}\circ\mu_{A_2}), \mu_A\circ\mu_A\bigr\rangle,\cr
}$$
where in the last two lines we have employed the adjoint property
$\langle f,g*h \rangle = \langle h\circ f,g\rangle$, as well as the commutative properties
$f*g = g*f$ and $\langle f,g\rangle = \langle g,f\rangle$,
all of which hold for functions $f,g,h : G\to \RR$. But by the construction of $S$, we have
$\one_S(x)(\mu_A\circ\mu_A)(x) \ge (1+\eps/8)\one_S$, so
$$\eqalign{
\bigl\langle \mu_V* (\mu_{A_1}\circ\mu_{A_2}), \mu_A\circ\mu_A\bigr\rangle
&\ge \bigl\langle \mu_V* (\mu_{A_1}\circ\mu_{A_2}), \one_S(\mu_A\circ\mu_A)\bigr\rangle \cr
&\ge (1+\eps/8)\bigl\langle \mu_V* (\mu_{A_1}\circ\mu_{A_2}), \one_S\bigr\rangle \cr
&\ge (1+\eps/8)(1-\eps/16) \cr
&\ge 1+\eps/32, \cr
}$$
hence we conclude that
$$\max_{x\in G}\;(\one_A*\mu_V)(x) \ge (1+\eps/32)\alpha\noskipslug$$

\parenproclaim Theorem~{\advthm} (Finite field). Let $q$ be a power of an odd prime and let $A$ be
a subset of $G = \FF_q^n$ of cardinality $\alpha q^n$. The number of (possibly trivial)
three-term arithmetic progressions contained in $A$ is at least
$${\alpha^3\over 2}q^{2n-O(\log(1/\alpha)^9)}.$$
Hence if $A\subseteq \FF_q^n$ contains no nontrivial three-term arithmetic progressions, then
$\alpha \le q^{-\Omega(n^{1/9})}$.

\proof Let $C = 2\cdot A = \{2a : a\in A\}$, so that $\gamma = |C|/q^n = \alpha$. By Lemma~{\densityincrement}
applied to $A$ and $C$ with parameter $\eps = 1/2$, we find that either
$\langle \mu_A*\mu_A,\mu_C\rangle \ge 1/2$ or there is a subspace $V$ of codimension
$O\bigl(\log(1/\alpha)^8\bigr)$ such that $\max_{x\in G}\;(\one_A*\mu_V)(x)\ge (1+\eps/64)\alpha$.

In the second case, there exists some $x\in G$ such that
$$\ex_{y\in G} \one_A(y)\mu_V(x-y) \ge (1+\eps/64)\alpha.$$
But $x-y\in V$ if and only if $y-x\in V$ if and only if $y\in V+x$, so we find that
$$|A\cap V+x| \ge (1+\eps/64)\alpha|V|.$$
Now $A$ has exactly the same number of three-term arithmetic progressions as $A-x$, so we can invoke
Lemma~{\densityincrement} again with $V$ in place of $G$ and $A-x$ in place of $A$, but note that $\alpha$
has been replaced by $(1+\eps/64)\alpha$, so this iteration can only happen
$\log_{1+\eps/64}(1/\alpha) = O\bigl(\log(1/\alpha)\bigr)$ times
before the second case of the lemma becomes impossible, since $\alpha\le 1$. Hence we deduce that
there is some subspace $V$ of codimension $O\bigl(\log(1/\alpha)^9\bigr)$ and some translate $A+x'$ of $A$
such that
$$\ex_{x\in V} \ex_{y\in V} \mu_{A'}(y)\mu_{A'}(x-y)\mu_{2\cdot A'}(x) \ge {1\over 2},$$
where $A' = (A+x)\cap V$ and
the relative densities are taken with respect to the subspace $V$. Expanding further, this implies that
$${|V|^3\over |A'|^3|V|^2} \sum_{x\in V}\sum_{y\in V}
\one_{A'}(y) \one_{A'}(x-y)\one_{2\cdot A'}(x) \ge {1\over 2};$$
that is,
$$\bigl| \{ (x,y)\in (2\cdot A) \times A : x-y\in A'\}\bigr| \ge {|A'|^3|V|^5\over 2|V|^3}
\ge {\alpha^3\over 2}q^{2n-O(\log(1/\alpha)^9)}.$$
Renaming variables, this counts the number of pairs $(x,z) \in A'\times A'$ such that $x+z = 2y$ for some
$y\in A'$. Since this equation implies that $z-y = y-x$, the above expression
counts the number of three-term arithmetic
progressions in $A'$, including the $|A'|$ trivial instances of $x=y=z$. This proves the first part of the
theorem.

For the last part of the theorem statement,
suppose that $A$ does not contain any nontrivial three-term arithmetic progressions.
Then
$${\alpha^3\over 2}q^{2n-O(\log(1/\alpha)^9)} \le \bigl| \{ (x,y)\in (2\cdot A) \times A : x-y\in A'\}\bigr|
\le |A'| \le \alpha q^n,$$
whence
$$q^n \le {2q^{O(\log(1/\alpha)^9)}\over \alpha^2},$$
and taking $q$th logs of both sides yields
$$n \le \log_q\biggl({2q^{O(\log(1/\alpha)^9)}\over \alpha^2}\biggr) = O\bigl(\log(1/\alpha)^9\bigr)
= O\bigl(\log_q(1/\alpha)^9\bigr).$$
Letting $C$ be the constant implicit in the last big-$O$ bound, we invert this to obtain
$$\alpha\le q^{-n^{1/9}/C^{1/9}} = q^{-\Omega(n^{1/9})},$$
which is what we wanted.\slug

\advsect Bohr sets

To transfer the ideas of the finite-field proof over to the integer case, we will need the machinery of Bohr
sets. These are, in some sense, the analogue in general abelian groups to subspaces in $\FF_q^n$.

Let $G$ be a finite abelian group, $\Gamma$ be a nonempty subset of $\hat G$, and let $\nu:\Gamma\to [0,2]$.
The {\it Bohr set} $B = \Bohr_\nu(\Gamma)$ corresponding to this data is the set
$$\Bohr_\nu(\Gamma) = \bigl\{x\in G: |1-\gamma(x)| \le \nu(\gamma)\ \hbox{for all}\ \gamma\in \Gamma\bigr\}.$$
The set $\Gamma$ is called the {\it frequency set} of $B$, and $\nu$ is its {\it width function}.
We shall also say that $B$ has rank $d$ if $|\Gamma| = d$.
Bohr sets contain $0$, since $\gamma(0) = 1$ for all characters $\gamma$, and $x\in B$ if and only if $-x\in B$,
since $|1-\gamma(x)| = |1-\gamma(-x)|$.
Note that the set $\Bohr_\nu(\Gamma)$ does not uniquely determine the pair $(\Gamma,\nu)$.
When we write $B'\subseteq B$ for Bohr sets $B'$ and $B$, we mean an inclusion of sets, and do not intend
to say anything about the corresponding frequency sets or width functions.

Let $B = \Bohr_\nu(\Gamma)$ be a Bohr set and let $\rho>0$. Let $\nu_\rho:\Gamma\to[0,2]$ be the width function
given by $\nu_\rho(\gamma) = \max\bigl\{\rho\cdot\nu(\gamma),2\bigr\}$. We define the {\it dilate} of
$B$ by $\rho$
to be the Bohr set $\Bohr_{\rho\nu}(\Gamma)$. If $\rho\le 1$, $\rho \nu(\gamma)\le \nu$, meaning that
the condition $|1-\gamma(x)|\le \rho\nu(\gamma)$ is now stronger. so $B_\rho\subseteq B$, and similarly
$B\subseteq B_\rho$ if $\rho\ge 1$.

We now introduce a definition that characterises when a Bohr set is approximately closed under
addition. Let $B$ be a Bohr set of rank $d$. We say that $B$ is {\it regular} if for all $\kappa$
with $|\kappa|\le 1/(100d)$,
$$\bigl(1-100d|\kappa|\bigr)|B| \le |B_{1+\kappa}| \le \bigl(1+100d|\kappa|\bigr)|B|.$$

The following proposition will be used frequently in the following sections.

\proclaim Proposition \advthm. Let $G$ be a finite abelian group and let $k$ be an integer with
$\gcd\bigl(k,|G|\bigr) = 1$. If $B = \Bohr_\nu(\Gamma)$ is a regular Bohr set of rank $d$, then
$k\cdot B$ is also a regular Bohr set of rank $d$.

\proof We will construct the frequency set and width function of $k\cdot B$.
Note first that if we have $kx = kx'$, then $k(x-x') = 0$. But since $\gcd\bigl(k,|G|\bigr) = 1$,
$k(x-x') = 0$ implies that $x-x' = 0$, so $x=x'$ and thus the function $x\mapsto kx$ permutes
$G$. We will write its inverse as $x\mapsto x/k$. For all $\gamma\in\Gamma$,
let $\gamma_k$ be given by $\gamma_k(x) = \gamma(kx)$, and let $\gamma_{k^{-1}}(x) = \gamma(x/k)$.
We let
$$\Gamma' = \{\gamma_{k^{-1}} : \gamma\in \Gamma\}$$
and let $\nu':\Gamma'\to [0,2]$ be given by $\nu'(\gamma') = \gamma'_k$.
Then we have a bijection $\gamma \mapsto \gamma'$ such that $\nu(\gamma) = \nu'(\gamma')$,
so we see that
$$\eqalign{
k\cdot B &= \bigl\{ kx\in G : \gamma(x) \le \nu(\gamma)\ \hbox{for all}\ \gamma\in \Gamma\bigr\} \cr
&= \bigl\{ x\in G : \gamma_{k^{-1}}(x) \le \nu(\gamma)\ \hbox{for all}\ \gamma\in \Gamma\bigr\} \cr
&= \bigl\{ x\in G : \gamma'(x) \le \nu'(\gamma')\ \hbox{for all}\ \gamma'\in \Gamma'\bigr\} \cr
&= \Bohr_{\nu'}(\Gamma'), \cr
}$$
so $k\cdot B$ is a Bohr set of rank $d$ as well.

To check regularity, it suffices to show that $\bigl| (k\cdot B)_{1+\kappa}\bigr| = |B_{1+\kappa}|$.
We actually have the stronger fact that
$$\eqalign{
(k\cdot B)_{1+\kappa}&=\bigl\{x\in G:\gamma'(x)\le(1+\kappa)\nu'(\gamma')\hbox{ for all }\gamma'\in \Gamma'\bigr\}\cr
&=\bigl\{x\in G:\gamma_{k^{-1}}(x)\le(1+\kappa)\nu(\gamma)\hbox{ for all }\gamma\in \Gamma\bigr\}\cr
&=\bigl\{kx\in G:\gamma(x)\le(1+\kappa)\nu(\gamma)\hbox{ for all }\gamma\in \Gamma\bigr\}\cr
&= k\cdot B_{1+\kappa} .\noskipslug
}$$

\section References

\bye

