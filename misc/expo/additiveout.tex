\input fontmac
\input mathmac

\def\FF{{\bf F}}
\def\TT{{\bf T}}
\def\bar{\overline}
\def\hat{\widehat}
\def\norm#1{|\!|#1|\!|}
\def\bignorm#1{\big|\!\big|#1\big|\!\big|}
\def\Norm#1{\Big|\!\Big|#1\Big|\!\Big|}
\def\normm#1{\bigg|\!\bigg|#1\bigg|\!\bigg|}
\def\mone{-\!\one}  % aesthetic quibble


\widemargins
\bookheader{NOTES ON ADDITIVE COMBINATORICS}{MARCEL K. GOH}

\maketitle{Notes on additive combinatorics}{by}{Marcel K. Goh}{5 May 2023}

\floattext4.5 \ninebf Note.
\ninepoint
I am compiling this set of expository notes primarily to solidify these fundamental results of
additive combinatorics in
my own mind. Sources that are far more comprehensive exist on the web, so I'm not sure
how useful this will be to anyone else. Currently, the bulk of these notes come from
a series of YouTube lectures of Timothy Gowers (given in early 2022), though I intend
in future to add more relevant topics as I learn about them.

\bigskip

\advsect Pl\"unnecke's theorem

Let $A$ and $B$ be finite subsets of an abelian group. We define the {\it sumset}
$A+B = \{a+b : a\in A,\, b\in B\}$ as well as the {\it difference set}
$A-B = \{a-b : a\in A,\, b\in B\}$. For a nonnegative integer $r$, the {\it $r$-fold iterated sumset}
$rA$ is defined recursively by $0A = \{0\}$ and $rA = A+(r-1)A$. The goal of this first section
is to show that if the cardinality of
a sumset $|A+B|$ is small compared to the size of the set $A$, then the difference
$rB-sB$ of two iterated sumsets must also be small relative to the size of $A$,
in some sense. This result is called Pl\"unnecke's
theorem as it follows from a statement given by H.~Pl\"unnecke in~1969, though this particular
formulation of it was stated and proved by I.~Z.~Ruzsa in 1989.

\edef\thmplunnecke{\the\thmcount}
\parenproclaim Theorem {\advthm} (Pl\"unnecke, {\rm 1969;} Ruzsa, {\rm 1989}). Let $A_0$ and
$B$ be finite subsets of an abelian group and suppose that $|A_0+B|\le K_0|A_0|$ for some constant $K_0$.
Then there exist $A\subseteq A_0$ and $K\le K_0$ such that
for any nonnegative integers $r$ and $s$, not both zero, we have $|rB-sB| \le K^{r+s}|A|$. In particular,
this means that $|rB-sB| \le {K_0}^{r+s}|A_0|$.

Ruzsa used Menger's theorem from graph theory to prove this theorem, but we will use a shorter
argument from a 2012 paper by G.~Petridis. It hinges on the following lemma. In this proof (and the
rest of these notes), we will write $A+x$ instead of $A+\{x\}$ for convenience.

\edef\lempetridis{\the\thmcount}
\parenproclaim Lemma {\advthm} (Petridis, {\rm 2012}). Let $A$ and $B$ be subsets of an abelian
group and let $K_0$ be such that $|A_0+B| = K_0|A_0|$. Then there exist $A\subseteq A_0$ and
$K\le K_0$ such that $|A+B+C| \le K|A+C|$ for every finite subset $C$ of the group.

\proof We pick $A\subseteq A_0$ nonempty such that the ratio $|A+B|/|A|$, which we shall
set as $K$, is minimised. So $|A+B| = K|A|$ and $|A'+B|\ge K|A'|$ for all $A'\subseteq A_0$.
Now we shall show by induction on $C$ that $|A+B+C|\le K|A+C|$ for all $C$. If $C$ is empty we
are done, of course. Now assume the result holds for $C$ and let $x\notin C$ be arbitrary. We have
$$\bigl| A+(C\cup \{x\}) \bigr| = |A+C| + |A+x| - |A'+x|$$
where $A' = \{a\in A : a+x \in A+C\}$. But adding $x$ to everything in a set does not change its
cardinality, so we have
$$|A+C| + |A+x|-|A'+x| = |A+C| + |A| + |A'|.$$
Also, any element of $A'+B+x$ is an element of $A+B+C$, since $a'+x\in A+C$ for all $a'\in A'$.
Applying the induction hypothesis now gives
$$\eqalign{
\bigl| A + B + (C\cup \{x\}) \bigr| &\le |A+B+C| + |A+B+x| - |A'+B+x| \cr
&\le K|A+B| + K|A| = K|A'| \cr
&= K\bigl| A+(C\cup \{x\})\bigr|,
}$$
which completes the proof.\slug

This lemma has an immediate corollary that already has some resemblance to the version of Pl\"unnecke's
theorem that we eventually want to prove.

\proclaim Corollary {\advthm}. If $|A_0 + B|\le K_0|A_0|$, then there is $A\subseteq A_0$ nonempty
and $K\le K_0$ such that $|A+rB| \le K^r |A|$ for all nonnegative integers $r$.

\proof The previous lemma gives $A$ and $K$ such that $|A+B+C|\le K|A+C|$ for all $C$. We now induce
on $r$. The case $r=0$ is obvious. Now for $r>0$ applying the lemma again with $C = (r-1)B$ and
then invoking the induction hypothesis gives
$$|A+rB| = |A+B+(r-1)B| \le K|A+(r-1)B| \le K^r|A|.\noskipslug$$

To get from this corollary to the full theorem, we will use a certain inequality of Rusza (1996), which
is often called the Ruzsa triangle inequality due to its similarity to the triangle inequality in
metric spaces.

\parenproclaim Lemma {\advthm} (Ruzsa triangle inequality). Let $U$, $V$, and $W$ be finite
subsets of an abelian group. Then
$$|U|\cdot |V-W| \le |U-V|\cdot |U-W|.$$

\proof For each $x\in V-W$, pick $v(x)\in V$ and $w(x)\in W$ such that $v(x)-w(x) = x$ (there may
be multiple choices for each of $v(x)$ and $w(x)$, but we arbitrarily pick one).
Define $\phi : U\times (V-W) \to (U-V)\times (U-W)$ by $\phi(u,x) = \bigl(u-v(x), u-w(x)\bigr)$.
If $\bigl(u_1 - v(x_1), u_1 - w(x_2)\bigr) = \bigl(u_2 - v(x_2), u_2 - w(x_2)\bigr)$, then
subtracting the second coordinate of each pair from the first, we find that
$$x_1 = v(x_1) - w(x_1) = v(x_2) - w(x_2) = x_2,$$
and this also implies that $u_1 = u_2$. So $\phi$ gives an injection from $U\times (V-1)$
to $(U-V)\times (U-W)$.\slug

If we define $d(U,V) = |U-V| / \bigl(|U|^{1/2}|V|^{1/2}\bigr)$, then this lemma states
that $d(U,W) \le d(V,U)d(U,W)$. Taking logarithms gives a genuine additive triangle inequality,
but the space of finite subsets of an abelian group does not form a metric space, since
$|U-U|/|U|$ does not equal $1$ in general. In any case, we are now able to prove Ruzsa's
version of Pl\"unnecke's inequality.

\medskip
\noindent{\it Proof of Theorem~{\thmplunnecke}.}\enspace Let $A$ and $K$ be given by
Lemma~{\lempetridis}. By the corollary to that lemma, we have $|A+rB| \le K^r|A|$ and
$|A+sB| \le K^r|A|$. Cardinalities are not changed by taking additive inverses of
everything, so $|{-A}-rB| \le K^r|A|$ and $|{-A}-sB|\le K^r|A|$. Now an application of Ruzsa's triangle
inequality with $U = -A$, $V = rB$ and $W = sB$ yields
$$|A|\cdot|rB-sB| = |{-A}|\cdot|rB-sB| \le |{-A}-rB|\cdot|{-A}-sB| \le K^{r+s}|A|^2,$$
and dividing both sides by $|A|$ proves the theorem.\slug

\advsect Khovanskii's theorem

The goal of this section is to prove the following wonderful theorem of A.~Khovanskii.

\edef\thmkhovanskii{\the\thmcount}
\parenproclaim Theorem {\advthm} (Khovanskii, {\rm 1992}). Let $A$ be a finite
subset of an abelian group and let $f_A(n) = |nA|$ for each $n\in \NN$. Then there exists
$n_A$ and a polynomial $p_A$ such that $f_A(n) = p_A(n)$ for all $n\ge n_A$.

Consider the example when $A = \{0,1,b-1,b\} \subseteq \ZZ$. Then letting $[a,b] = \{x\in \ZZ : a\le x\le b\}$,
we have
$$nA = [0, n] \cup [b-1,\ldots, b+n-1] \cup [2b-2, 2b+n-2] \cup\cdots\cup [n(b-1),nb].$$
This has size $(n+1)^2$ as long as $n< b-1$, but when $n\ge b-1$, $nA$ is simply the set
of integers in the range $[0,bn]$, which has size $bn+1$.

We shall present a combinatorial proof of Khovanskii's theorem,
given by M.~B.~Nathanson and I.~Z.~Ruzsa in 2002.
Let $A = \{a_1,\ldots, a_r\}$ and
for $x = (x_1,\ldots, x_r)\in {\NN_0}^r$, write
$$\sigma(x) = \sum_{i=1}^r x_i\qquad\hbox{and}\qquad
\alpha(x) = \sum_{i=1}^r a_ix_i.$$
Then
$$nA = \{ \alpha(x) : x\in {\NN_0}^r,\, \sigma(x) = n\}.$$
Let $\prec$ be the lexicographic order on ${\NN_0}^r$; that is, $(x_1,\ldots,x_r)\prec (y_1,\ldots,y_r)$
if there is some $i\in [r]$ such that $x_i<y_i$ and $x_j = y_j$ for all $1\le j<i$.
Say that $x$ is {\it useless} if there is $x'\prec x$ such that $\sigma(x') = \sigma(x)$ and
$\alpha(x') = \alpha(x)$; otherwise, say that $x$ is {\it useful}. It is easy to see that
$|nA|$ is exactly the number of useful $x$ with $\sigma(x) = n$. Now let $\le$ denote
the usual partial order on ${\NN_0}^r$, where $(x_1,\ldots,x_r)\le (y_1,\ldots,y_r)$
if $x_i\le y_i$ for all $1\le i\le r$.

\proclaim Lemma {\advthm}. If $x\le y$ and $x$ is useless, then $y$ is useless.

\proof Since $x$ is useless, we can find $x'\prec x$ with $\sigma(x') = \sigma(x)$ and $\alpha(x') = \alpha(x)$.
Let $y' = y+x'-x$. Since $x\le y$, $y-x$ is in ${\NN_0}^r$ and therefore so is $y'$. Then since
$x_i' < x_i$ for the first $i$ at which the two tuples differ, the first coordinate at which $y'$ and
$y$ differ is also $i$ and we have $y_i'<y_i$. Hence $y'\prec y$. Lastly,
$$\sigma(y') = \sigma(y+x'-x) = \sigma(y) + \sigma(x') - \sigma(x) = \sigma(y)$$
and the same is true with $\sigma$ replaced by $\alpha$, so $y$ is useless.\slug

For each useless $x$ we can find a useless $x'$, minimal in the $\le$ order, such that $x'\le x$.
This lemma gives a converse; it says that if $x'$ is a minimal useless element and $x'\le x$, then
$x$ is also useless.

\edef\lemminimalfinite{\the\thmcount}
\proclaim Lemma {\advthm}. For any nonempty $U\subseteq {\NN_0}^r$, the set of minimal elements of
$U$ (with respect to the $\le$ order) is finite.

\proof We perform induction on $r$. If $r=1$, then it is clear that $U$ has exactly one minimal element.
Now let $r>1$ and let $u$ be a minimal element of $u$. Then for every $v\in U$ with $u\not\le v$,
there exists some $i$ and $t< u_i$ such that $v_i = t$. Let
$$V(i,t) = \{v\in U : v_i = t\}$$
for all $1\le i\le r$ and $0\le t<u_i$. Each set $V(i,t)$ is order-isomorphic to a subset of ${\NN_0}^{r-1}$,
so by induction it has finitely many minimal elements. The number of $V(i,t)$ is $r\prod_{i=1}^r u_i$,
and every minimal element of $u$ is either $u$ or a minimal element of some $V(i,t)$, so there are
finitely many minimal elements in $U$.\slug

We will need just one more lemma before we are able to prove Khovanskii's theorem. It shows that
for any fixed $r$-tuple $x'$, the number of $x$ with $x'\le x$ and $\sigma(x) = n$
is given by a polynomial in the variable $n$, with coefficients depending on $x'$ and $r$.

\proclaim Lemma {\advthm}. Let $x'\in {\NN_0}^r$ and for $n\ge \sigma(x')$, let
$$C(x', n) = \{n\in {\NN_0}^r : \sigma(x) = n,\,x'\le x\}.$$
Then
$$\bigl| C(x',n)\bigr| = {n-\sigma(x') + r-1 \choose r-1}.$$

\proof Given $x\in C(x',n)$, map it to $x-x'$. This is a bijection from $C(x',n)$ to $C(0, n-\sigma(x')$.
Now let $m = n-\sigma(x')$. The map $(x_1,\ldots,x_r) \mapsto (x_1,+1,\ldots, x_r+1)$ is a bijection
from $C(0,m)$ to the set of $y\in \NN^r$ such that $y_1+\cdots+y_r = m+r$. Now the map
$$(y_1,\ldots,y_r) \mapsto \{y_1, y_1+y_2, \ldots, y_1+\cdots+y_r\}$$
is a bijection from the previous set to the set of subsets of $[m+r]$ that contain $m+r$. (It is
inverted by the map $\{a_1, \ldots, a_r\} \mapsto (a_1, a_2-a_1, \ldots, a_r-a_{r-1})$.) But this final
set has size ${m+r-1\choose r-1}$, so we are done.\slug

Without further ado, let us now prove Khovanskii's theorem.

\medskip\noindent {\it Proof of Theorem~{\thmkhovanskii}.}\enspace Let $U$ be the set of all
useless $r$-tuples. By Lemma~{\lemminimalfinite}, the set $X\subseteq U$ of all minimal useless
$r$-tuples is finite. Recall that $|nA|$ is the number of useful $x$ such that $\sigma(x) = n$,
so using the notation of the previous lemma, we have
$$|nA| = \bigl| C(0,n)\bigr| - \Bigl| \bigcup_{x'\in X} C(x',n)\Bigr|
=\bigl| C(0,n)\bigr| - \sum_{Y\subseteq X} (-1)^{|Y|} \Bigl| \bigcap_{x'\in Y} C(x',n)\Bigr| $$
But for any $Y\subseteq X$, $\bigcap_{x'\in Y} C(x',n) = C(x_Y, n)$, where $x_Y$ has for its
$i$th coordinate the integer $\max_{y\in Y} y_i$. Hence every term in the inclusion-exclusion
formula depends polynomially on $n$ once
$$n\ge \sum_{1\le i\le r} \max_{x\in X} x_i.\noskipslug$$

\advsect The Balog--Szemer\'edi--Gowers theorem

Let $A$ be a finite nonempty subset of an abelian group. An {\it additive quadruple} in
$A$ is a quadruple $(x,y,z,w)\in A^4$ such that $x+y = z+w$. The {\it additive energy}
$E(A)$ of $A$ is the number of additive quadruples in $A$. If $(x,y,z,w)$ is an additive
quadruple, then $w = x+y-z$, so $E(A)\le |A|^3$. The following lemma roughly states that if
a set has a small sumset, then it has large additive energy.

\proclaim Lemma {\advthm}. Let $A$ be a finite subset of an abelian group and let $C$ be such
that $|A+A|\le C|A|$. Then $E(A)\ge |A|^3/C$.

\proof Define $f : A+A\to \NN$ by letting $f(d)$ equal the number of pairs $(a,b)\in A^2$ such
that $a+b =d$. Then
$$\sum_{d\in A+A} f(d) = |A|^2\qquad\hbox{and}\qquad \sum_{d\in A+A} f(d)^2 = E(A).$$
By the Cauchy--Schwarz inequality,
$$|A|^2 = \sum_{d\ge 1} 1\cdot f(d)
\le \Bigl(\sum_{d\ge 1} 1\Bigr)^{1/2} \Bigl( \sum_{d\ge 1} f(d)^2\Bigr)^{1/2}
\le C^{1/2} |A|^{1/2} E(A)^{1/2}.$$
From here, squaring the whole inequality and then dividing through by $C|A|$ gives the statement
we want to prove.\slug

The converse of this lemma does not hold. For example, we could take the disjoint
union of an arithmetic progression with a geometric progression of the same length.
This set has a large additive energy,
because of the arithmetic progression, but it also has a large sumset, because of the geometric progression.
However, we can pass to a large subset that has a small sumset (namely, the arithmetic progression), and this
is essentially the statement of the Balog--Szemer\'edi--Gowers theorem. We shall state it later on in the
section, after we have given the graph-theoretic definitions and lemmas needed in its proof.

Let $G$ be a bipartite graph with finite vertex sets $X$ and $Y$. For a vertex $v\in X\cup Y$,
let $d(v) = \deg(v)$ denote the {\it degree} of $x$, that is, the number of vertices adjacent to $v$.
For $x_1,x_2\in X$, let $d(x_1,x_2)$ denote the number of paths of length $2$ from $x_1$ to $x_2$. This
is called the {\it codegree} of $x_1$ and $x_2$. The {\it density} of $G$ is the number of edges of $G$
divided by the maximum number of vertices it could have, namely $|X|\cdot|Y|$.

\proclaim Lemma {\advthm}. Let $G$ be a bipartite graph with bipartition $X\cup Y$ and let the density
of $G$ be $\delta$. Then for every $c>0$ there is a subset $X'\subseteq X$ with $|X'|\ge \delta|X|/\sqrt 2$
such that the number of pairs $(x_1,x_2)\in (X')^2$ with $d(x_1,x_2) < c|Y|$ is at most $2c|X|^2/\delta^2$.

\proof Let $y\in Y$ be chosen uniformly at random. Letting $X' = N(y)$, we have
$$\ex\bigl\{ |X'|\bigr\} = \ex_{y\in Y} \bigl\{ d(y)\bigr\} = {1\over |Y|} \sum_{y\in Y} d(y)
= {|E(G)|\over |Y|} = \delta|X|.$$
By convexity of the map $x\mapsto x^2$, we then find that $\ex\bigl\{ |X'|^2\bigr\} \ge \delta^2|X|^2$.
Let
$$B = \bigl\{ (x_1,x_2)\in X^2 : d(x_1,x_2) < c|Y|\bigr\}.$$
The probability that any $(x_1, x_2)\in B$ is also in $(X')^2$ is the same as the probability that
$y$ is connected to both $x_1$ and $x_2$; this is less than $c|Y|$ by the construction of $B$.
So by linearity of expectation, $\ex\bigl\{ |B\cap (X')^2| \bigr\} < c|X|^2$ and in particular, we
have
$$\biggl\{ |X'|^2 - {\delta^2\over 2c} |B\cap(X')^2|\biggr\}
 >\biggl(\delta^2 - {\delta^2\over 2c}\cdot c\biggr)|X|^2 = {\delta^2\over 2}|X|^2.$$
So there is $X'$ such that $|X'| - \bigl(\delta^2/(2c)\bigr) |B\cap (X')^2| \ge \delta^2|X|^2/2$.
This set $X'$ has cardinality at least $\delta |X|/\sqrt 2$ and $|B\cap (X')^2| \le 2c|X'|^2/\delta^2$.\slug

This lemma says that we can find a fairly large subset of $X$ such that almost every pair in this
subset is joined by at least $c|Y|$ paths of length $2$. We would really like to remove the ``almost''
in this statement, but to do so, we will have to pass to paths of length $3$.
First, we need a little lemma about sums of real numbers between $0$ and $1$.

\proclaim Lemma {\advthm}. Let $a_1,\ldots, a_n\in [0,1]$. If $\sum_{i=1}^n a_i \ge an$, then
the number of $i$ such that $a_i\ge b$ is at least
$$ \biggl( {a-b\over 1-b} \biggr) n.$$

\proof Let $I$ be the set of $i$ such that $a_i\ge b$. Then we may write
$$ an \le \sum_{i=1}^n a_i = \sum_{i\in I} a_i + \sum_{i\notin I} a_i
\le |I| + (n-|I|)b.$$
It remains to isolate for $|I|$.\slug

For example, if $b = a/2$, then the number of $i$ with $a_i\ge b$ must be at least $an/2$, and more
generally, if $a= 1-\theta$ and $b=1-\eta$, then the number of such $i$ is at least $(1-\theta/\eta)n$.
If $\eta = 2\theta$, then $a_i\ge 1-2\theta$ at least half the time.
Next we prove the lemma alluded to earlier, concerning paths of length $3$.

\proclaim Lemma {\advthm}. Let $G$ be a bipartite graph with vertex sets $X$ and $Y$ and density $\delta$.
Then there are subsets $X'\subseteq X$ and $Y'\subseteq Y$ such that
$$|X'| \ge {\delta^2\over 8\sqrt 2} |X|\qquad\hbox{and}\qquad
|Y'|\ge {\delta\over 4}|Y|$$
and every $x\in X'$ and $y\in Y'$ are joined by at least $(\delta^6/2^{13}) |X|\cdot |Y|$ paths
of length $3$ in $G$.

\proof By letting $n = |X|$ and setting the $a_i$ equal to the degrees of the vertices in $X$, we have
$a = \delta|Y|$
and we can choose $b=a/2$ in the previous lemma
to deduce that $X$ has a subset $X_1$ such that every vertex in $X_1$ has degree at least
$\delta|Y|/2$ and $|X_1|\ge \delta|X|/2$. Note that the density of the bipartite subgraph with
vertex set $X_1\cup Y$ is $\ge \delta/2$,
so we may invoke the lemma about paths of length $2$ with $\delta$ set to $\delta/2$. It tells us that
for any $c>0$, $X_1$ has a subset $X_2$ of size $\delta|X_1|/2^{3/2}$ such that for all but at most
$8c|X_2|^2/\delta^2$ pairs $(x_1,x_2)\in {X_2}^2$, we have $d(x_1,x_2) \ge c|Y|$.

Now let $\Gamma$ be a graph with vertex set $X_2$ and an edge between $x_1$ and $x_2$ if and only if
$\delta(x_1,x_2)\ge c|Y|$. Note that by this definition, there may be loops in $\Gamma$. In any case,
the average degree in $\Gamma$ is at least $(1-8c/\delta^2)|X_2|$. Invoking the previous lemma again,
at least half the vertices in $X_2$ have degree at least $(1-16c/\delta^2)|X_2|$. Let $X'$ be the set of
all such vertices. Then
$$|X'| \ge {|X_2|\over 2} \ge {\delta|X_1|\over 4\sqrt 2} \ge {\delta^2 |X|\over 8\sqrt 2}.$$
Every vertex in $X_2$ has degree at least $\delta |Y|/2$, so the bipartite subgraph on $X_2\cup Y$
has density at least $\delta/2$ and the average degree of $y\in Y$ is at least $\delta|X_2|/2$. The
previous lemma comes in handy once again, telling us that at least $\delta|Y|/4$ vertices in $Y$ has
degree at least $\delta|X_2|/4$. Let $Y'$ be the set of these vertices.

For any $x\in X'$ and $y\in Y'$, we know that $x$ has at least $(1-16c/\delta^2)|X_2|$ neighbours in $\Gamma$
and $y$ has at least $\delta|X_2|/4$ neighbours in $X_2$. There must be at least
$(\delta/4-16c/\delta^2)|X_2|$ neighbours in commmon. Letting $c=\delta^3/128$, the number of neighbours
in common becomes $\delta|X_2|/8$, and the number of paths of length $3$ from $x$ to $y$ is at least
$$ {\delta |X_2|\over 8} |X_2| \cdot c|Y| = {\delta\over 8}\cdot {\delta^3\over 128}
\cdot {\delta^2\over 4\sqrt 2} |X|\cdot |Y| \ge {\delta^6\over 2^{13}}|X|\cdot |Y|,$$
as foretold.\slug

Let $A$ be a finite subset of an abelian group. We say that an element $d\in A-A$ is {\it $\theta$-popular}
if $\bigl|\bigl\{ (a,b)\in A^2 : b-a=d\bigr\}\bigr|\ge \theta |A|$.

\proclaim Lemma {\advthm}. Let $A$ be a finite subset of an abelian group and let $E(A)$ denote
the additive energy of $A$. If $E(A)\ge c|A|^3$, then there are at least $c|A|/2$ elements of $A-A$
that are $(c/2)$-popular.

\proof Define $f : A-A\to \NN$ by $f(d) = \bigl|\bigl\{ (a,b)\in A^2 : b-a=d\bigr\}\bigr|$.
Note that
$$\sum_{d\in A-A} f(d) = |A|^2\qquad\hbox{and}\qquad \sum_{d\in A-A} f(d)^2 = E(A).$$
Let $P$ be the set of $(c/2)$-popular differences and let $U = (A-A)\setminus P$. Then
$$\eqalign{
c|A|^3 &\le E(A)\cr
&= \sum_{d\in A-A} f(d)^2 \cr
&= \sum_{d\in P} f(d)^2 + \sum_{d\in U} f(d)^2 \cr
&\le |P|\cdot |A|^2 + {c|A|\over 2} \sum_{d\in U} f(d) \cr
&= |P|\cdot |A|^2 + {c|A|^3\over 2}, \cr
}$$
which implies that $|A|\ge c|A|/2$.\slug

We are, at long last, ready to state and prove the Balog--Szemer\'edi--Gowers theorem. It is
named for A.~Balog and E.~Szemer\'edi, who first proved it in 1994, as well as for W.~T.~Gowers,
who gave an alternative proof in 1998 that yielded power-type bounds, a vast improvement to the tower-type
bounds in Balog and Szemer\'edi's proof.

\parenproclaim Theorem {\advthm} (Balog--Szemer\'edi, {\rm 1994;} Gowers, {\rm 1998}).
Let $A$ be a finite subset of an abelian group with $E(A)\ge c|A|^3$. Then $A$ has a subset
$A'$ of size at least $c'|A|$ such that $|A'-A'|\le C|A'|$, where $c'$ and $C$ have a power-type
dependence on $c$. In particular, we may take
$$ c' = {c^4\over 2^8}\qquad\hbox{and}\qquad C = {2^{68}\over c^{36}}.$$

\proof Define a bipartite graph $G$ with both vertex sets being copies of $A$ (though
we'll call one of these copies $B$ to avoid any confusion). Join $a\in A$ to
$b\in B$ if and only if $b-a$ is $(c/2)$-popular. Each $(c/2)$-popular difference leads to at least
$c|A|/2$ edges, so by the lemma, there are at least $c^2|A|^2/4$ edges. Let $\delta = c^2/4$.
By the lemma concerning paths of length $3$, there exist $A'\subseteq A$ and $B'\subseteq B$
with
$$ |A'| \ge {\delta^2\over 8\sqrt 2} |A|\qquad\hbox{and}\qquad
|B'|\ge {\delta \over 4} |A|$$
such that every $a\in A$ and $b\in B$ are joined by at least $\delta^6|A|^2/2^{13}$ paths of length $3$
in $G$.

Given a path $(a,u,v,b)$ of length $3$, we have $b-a = u-a - (u-v) + b-v$, where $u-a$, $u-v$, and $b-v$
are all $(c/2)$-popular. Hence the number of ways of writing $b-a$ as $r_1-s_1 - (r_2-s_2) + r_3 - s_3$
is at least
$${\delta^6\over 2^{13}} |A|^2 \biggl( {c\over 2} |A|\biggr)^3 = {c^{15}\over 2^{28}} |A|^5.$$
But the number of possible $(r_1,s_1,r_2,s_2,r_3,s_3)$ is $|A|^6$. So
$$ |B'-A'|\cdot {c^{15}\over 2^{28}} |A|^5 \le |A|^6 $$
and $|B'-A'|\le 2^{28}|A|\over c^{15}$. Now the Ruzsa triangle inequality with $U = B'$ and
$V = W = A'$ tells us that $|B'|\cdot |A'-A'| \le |B'-A'|^2$, so
$$|A'-A'|\le {|B'-A'|^2\over |B'|} \le {2^{56}|A|^2\over c^{30}}\cdot {2^4\over c^2|A|}={2^{60}\over c^{32}}|A|.$$
But recall that
$$|A'| \ge {\delta^2\over 8\sqrt 2} |A| \ge {c^4\over 2^8}|A|,$$
so $|A'-A'| \le 2^{68}|A'|/c^{36}$.\slug

\advsect Freiman homomorphisms and Ruzsa embedding lemmas

We have been talking a lot about finite subsets of abelian groups, so it's about time we gave
a special name for these sets. An {\it additive set} is a pair $(A,Z)$ where $Z$ is an abelian
group and $A$ is a finite subset of $Z$. In almost all cases we shall suppress mention of $Z$
and speak simply of an additive set $A$.

Let $A$ and $B$ be additive sets (not necessarily in the same group). A function $\phi:A\to B$
is a {\it Freiman homomorphism of order $k$} if for any $2k$ elements $a_1,\ldots,a_k,b_1,\ldots,b_k\in A$
satisfying
$a_1+\cdots +a_k = b_1+\cdots+b_k$, we also have $\phi(a_1)+\cdots+\phi(a_k) =\phi(b_1)+\cdots+\phi(b_k)$.
The existence of such a $\phi$ tells us that the additive relationships among
the elements of $A$ are approximately the same as the additive relationships in $B$. Note that a Freiman
homomorphism of order $k$ is also a Freiman homomorphism of order $k'$ for all $k'>k$.
If we write simply ``Freiman
homomorphism'', then we mean a Freiman homomorphism of order $2$, and we will sometimes write ``$k$-homomorphism''
instead of ``Freiman homomorphism of order $k$''.

A $k$-homomorphism $\phi:A\to B$ induces a map from $kA$ to $kB$. Indeed, if we define $\psi: kA\to kB$
by
$$\psi(a_1+\cdots+a_k) = \phi(a_1) + \cdots + \phi(a_k),$$
then the definition of $k$-homomorphism ensures that $\phi$ is well-defined. More generally, if $\phi$
is a $k(r+s)$-homomorphism, then defining $\psi : rA-sA \to rB-sB$ by
$$\phi(a_1+\cdots+a_r-a_1'-\cdots-a_s') = \phi(a_1)+\cdots+\phi(a_r) -\phi(a_1')-\cdots-\phi(a_s'),$$
it is easy to see that $\psi$ is a $k$-homomorphism. It is also clear that if $\phi$ is the
restriction of a group homomorphism to $A$, then $\phi$ is a $k$-homomorphism for every $k$, and
if $\phi:A\to B$ and $\psi:B\to C$ are $k$-homomorphisms, then $\psi\circ \phi : A\to C$ is a $k$-homomorphism
as well.

If $\phi : A\to B$ is a $k$-homomorphism with an inverse that is also a $k$-homomorphism, then we say that
$\phi$ is a {\it Freiman isomorphism of order $k$} or, more briefly, a {\it $k$-isomorphism}. These
maps preserve a lot of the properties that are of interest in additive combinatorics. Suppose that $A$
is $k$-isomorphic to $B$. Then
\medskip
\item{i)} $|kA| = |kB|$;
\smallskip
\item{ii)} $|rA-sA| = |rB-sB|$ for all $r$ and $s$ with $r+s\le k$;
\smallskip
\item{iii)} $E(A) = E(B)$ if $k=2$; and
\smallskip
\item{iv)} if $k=2$ and $A$ contains an arithmetic progression of length $r$, then so does $B$.
\medskip
Of these statements, the only slightly nontrivial one is (iv). It follows from the fact that an
arithmetic progression is a sequence $(a_1,\ldots,a_r)$ with $a_i+a_i = a_{i-1}+a_{i+1}$ for all $1<i<r$.
Taking $\phi$ of both sides gives
$$\phi(a_i) + \phi(a_i) = \phi(a_i+a_i) = \phi(a_{i-1}+a_{i+1}) = \phi(a_{i-1}) + \phi(a_{i+1}).$$
Let us now state and prove two important ``embedding lemmas''. These lemmas tell us that if
$kA-kA$ is not too large, then the additive set $A$ is $k$-isomorphic to a subset of a group that
is simpler, in some sense.

\parenproclaim Lemma {\advthm} (Ruzsa embedding lemma in $\FF_p^N$). Let $k\ge 1$ and $C>0$.
Let $A\subseteq \FF_p^N$ be such that $|kA-kA|\le C|A|$. Then $A$ is isomorphic to a subset of
$\FF_p^n$, where $p^{n-1} < C|A|$.

\proof Let $X$ be a subspace of $\FF_p^N$ chosen uniformly at random from all subspaces of
codimension $n$, where $n$ will be specified later. Let $q : \FF_p^N \to \FF_p^N/X$ be the
quotient map taking $v$ to $v+X$. Since $q$ is a group homomorphism, its restriction to $A$
is a Freiman homomorphism
of order $k$. The only way that the restriction of $q$ to $A$ can fail to be a {\it $k$-isomorphism}
is if there exist $a_1,\ldots,a_k,b_1,\ldots,b_k\in A$ such that $a_1+\cdots+a_k \ne b_1+\cdots+b_k$
but $q(a_1)+ \cdots + q(a_k) = q(b_1)+\cdots+q(b_k)$. This is equivalent to
$a_1+\cdots+a_k - b_1-\cdots-b_k$ being in $X$. For any nonzero $y\in \FF_p^N$, we have
$$\pr\{y\in X\} = {p^{N-n}-1\over p^N-1} < {1\over p^n},$$
and since $|kA-kA|\le C|A|$,
$$\pr\biggl\{\bigcup_{y\in (kA-kA)\setminus\{0\}} \{y\in X\}\biggr\} \le \sum_{y\in (kA-kA)\setminus 0}
\pr\{y\in X\} < {C|A|\over p^n},$$
the probability that $q$ is a $k$-isomorphism is nonzero so long as $p^n\ge C|A|$. Taking the minimal such $n$,
we have $p^{n-1} < C|A|$ and
can conclude there is a subspace $X$ of dimension $n$ that contains a set $k$-isomorphic to $A$.\slug

Next, we will prove an embedding lemma for subsets of $\ZZ$. In this setting the ``simpler'' group
that we pass to is a cyclic group, but first we need a lemma that essentially says that passing
to a cyclic group preserves the additive structure.

\proclaim Lemma {\advthm}. Let $k$ and $n$ be positive integers and let $A\subseteq \ZZ$ be a set of
diameter less than $n/k$; i.e., $\max A - \min A < n/k$. Then the map $\phi : A\to \ZZ_n$ that
sends $a\mapsto a\bmod n$ is a Freiman isomorphism of order $k$.

\proof The function $\phi$ is the restriction of a group homomorphism to $A$, so it is a Freiman
homomorphism of every order. Now suppose that $\min A = r$ and $\max A = s$, so that $s-r < n/k$.
Then if $a_1,\ldots,a_k,b_1,\ldots,b_k$ are such that $\phi(a_1)+\cdots+\phi(a_k) = \phi(b_1)+\cdots+\phi(b_k)$,
then $a_1+\cdots+a_k -b_1-\cdots-b_k$ is at most $k(s-r) < n$ but also a multiple of $n$.
Hence this difference is zero and $a_1 + \cdots + a_k = b_1+\cdots+b_k$.\slug

As promised, here is the embedding lemma for subsets $A\subseteq \ZZ$. This time, we cannot simply
say anything about the whole set $A$, but we can about a large subset of it.

\parenproclaim Lemma {\advthm} (Ruzsa embedding lemma in $\ZZ$). Let $k\in \NN$ and $C\in \RR$.
If $A\subseteq \ZZ$ with $|kA-kA|\le C|A|$, then there exists $A'\subseteq A$ with $|A'|\ge |A|/k$
such that $A'$ is $k$-isomorphic to a subset of $\ZZ/N\ZZ$, where $N$ is a prime between
$2C|A|$ and $4C|A|$.\slug

\proof It is easily checked that translation is a Freiman isomorphism of every order, so we may assume
that $A\subseteq \NN_0$. Let $p$ be a prime bigger than $k\max A$ and consider the
sequence of functions
$$\ZZ \buildrel{q_1}\over\longrightarrow \ZZ_p \buildrel{\mu_r}\over\longrightarrow \ZZ_p
\buildrel\iota\over\longrightarrow \ZZ \buildrel{q_2}\over\longrightarrow \ZZ_N,$$
where $q_1$ is reduction modulo $p$, $\mu_r$ is multiplication by a random nonzero element $r\in \ZZ_p$,
$\iota$ takes an element of $\ZZ_p$ to its representative in $[0,p)$, and $q_2$ is reduction mod $N$.
The only map here that is not a group homomorphism is $\iota$, so to show that the composition
$\phi = q_2\circ \iota\circ \mu_r\circ q_1$ restricted to $A$ is a $k$-homomorphism, it suffices to worry
about $\iota$. Split the interval $\{0,1,\ldots,p-1\}$ into $k$ subintervals $J_1,\ldots,J_k$,
each of diameter less than $p/k$ and let $I_j = \iota^{-1}(J_j)$ for each $1\le j\le k$. By the previous
lemma, the map $\iota^{-1}$ restricted to any $J_j$ is a $k$-isomorphism, so $\iota$ restricted
to $I_j$ is a $k$-isomorphism as well. Now let
$$A_{j,r} = \bigl\{a\in A : \mu_r\bigl(q_1(a)\bigr) \in I_j\bigr\}$$
and note that $\mu_r\bigl(q_1(a)\bigr) = ra \bmod p$. Thus the composition $\phi$ is a $k$-homomorph\-ism
when restricted to any $A_{j,r}$.

We shall now show that with positive probability, the restriction of $\phi$ to every $A_{j,r}$ is
a $k$-isomorphism. For fixed $r$, the restriction to $A_{j,r}$ is not a $k$-isomorphism
if there exist $j$ and $a_1,\ldots,a_k,b_1,\ldots,b_k\in A_{j,r}$ such that
$a_1+\cdots+a_k \ne b_1+\cdots+b_k$ but $\phi(a_1)+\cdots+\phi(a_k) = \phi(b_1)+\cdots+\phi(b_k)$.

\advsect Rudiments of discrete Fourier analysis

Let $Z$ be a finite abelian group. A {\it character} on $Z$ is a homomorphism from
$Z$ to the multiplicative group $\C\setminus\{0\}$. If $\chi$ is such a homomorphism, then
$\bigl|\chi(x)\bigr| = 1$ for every $x\in Z$, so we can actually regard $\chi$ as being
a function from $Z$ to the unit circle $\TT = \{z\in \C : |z| = 1\}$. The pointwise product
of two characters gives another character, and the function $\chi_0$ that sends every $x\in Z$
to $1$ has the property that if $\chi$ is any character, then $\chi\chi_0 = \chi = \chi_0\chi$.
Lastly, we note that multiplication is commutative and
for any character $\chi$, the product of $\chi$ with $\bar\chi$
gives $\chi_0$, so the set of characters on $Z$ is an abelian group. This is called the
{\it dual group} (or sometimes the {\it Pontryagin dual}) of $Z$ and is denoted $\hat Z$.

We define an inner product on the space $\CC^Z$ of functions from $Z$ to $\CC$ by setting
$$\langle f,g\rangle = \ex_x f(x) \bar{g(x)},$$
where $\ex_{x\in Z} F(x) = |Z|^{-1} \sum_{x\in Z} F(x)$.
We can also make $\hat Z$ into an inner product space by letting
$$\langle \hat f, \hat g\rangle = \sum_{\chi\in \hat Z} \hat f(\chi)\bar{\hat g(\chi)};$$
note that this time we do not normalise by dividing by $|Z|$.
Two functions $f$ and $g$ in an inner product space are said to be {\it orthogonal} if $\langle f,g\rangle = 0$.
The first lemma we'll prove concerns certain orthogonality relations.

\parenproclaim Lemma {\advthm} (Orthogonality relations). Let $Z$ be a finite abelian group.
\medskip
\item{a)} If $\chi_1$ and $\chi_2$ are characters on $Z$, then
$$\langle \chi_1,\chi_2 \rangle = \ex_{x\in Z} \chi_1(x)\bar{\chi_2(x)}
  = \cases{ 1, & if $\chi_1 = \chi_2$;\cr 0, & if $\chi_1 \ne\chi_2$.}.$$
\smallskip
\item{b)} For $x,y\in Z$, we have
$$\sum_{\chi\in \hat Z} \chi(x)\bar{\chi(y)}
  = \cases{ |\hat Z|, & if $x=y$;\cr 0, & if $x\ne y$,}$$
\medskip

\proof Let $\chi_0$ denote the trivial character. We have
$$\ex_{x\in Z} \chi_0(x) = \ex_{x\in Z} 1 = 1.$$
On the other hand, let $\chi$ be a nontrivial character and let $u\in Z$ be such that $\chi(u)\ne 1$. Then from
$$\ex_{x\in Z} \chi(x) = \ex_{x\in Z} \chi(u+x) = \ex_{x\in Z}\chi(u)\chi(x) = \chi(u)\ex_{x\in Z}\chi(x),$$
it follows that $\ex_{x\in Z}\chi(x) = 0$. Now consider $\chi_1\bar{\chi_2}$. If $\chi_1 = \chi_2$, then
this is the trivial character and from our first observation, $\langle \chi_1,\chi_2\rangle = 0$. Otherwise,
we are in the second case and the inner product is zero. This proves part (a).

Part (b) is proven similarly. Note that $\chi(x)\bar{\chi(y)} = \chi(x-y)$. If $x=y$, then we have
$$\sum_{\chi\in\hat Z} \chi(x-y) = \sum_{\chi\in\hat Z} \chi(0) = |\hat Z|.$$
If $x\ne y$, then there is some $\psi\in \hat Z$ such that $\psi(x-y)\ne 1$, and then writing
$$\sum_{\chi\in \hat Z} \chi(x-y) = \sum_{\chi\in\hat Z} \psi(x-y)\chi(x-y)
= \psi(x-y)\sum_{\chi\in\hat Z} \chi(x-y),$$
we see that $\sum_{\chi\in\hat Z} \chi(x-y)$ must be zero.\slug

Since the space of functions from $Z\to \CC$ has dimension $|Z|$, there are at most $|Z|$ characters.
Every finite abelian group can be written as a direct product of cyclic groups, and we shall use this fact
to show that there are exactly $|Z|$ characters; i.e., $|\hat Z| = |Z|$. For brevity, let $\ZZ_n$ stand for $\ZZ/n\ZZ$ for all $n\in\NN$,
and for $\alpha\in \RR$, let $e(\alpha) = e^{2\pi i\alpha}$.

\proclaim Lemma \advthm.
The set of characters on a finite abelian group $Z$ spans the space of functions from $Z$ to $\CC$.

\proof Write $Z \cong \ZZ_{n_1} \times \ZZ_{n_2}\times \cdots\times \ZZ_{n_r}$. For every
$u = (u_1,\ldots, u_r)\in Z$, the function $\chi_u : Z\to \CC$ that maps
$$(x_1,\ldots, x_r) \mapsto \prod_{i=1}^r e\biggl( {u_ix_i\over n_i}\biggr)$$
is a character (this is easy to show, since $e(x_i + x_i') = e(x_i)e(x_i')$). If $u\ne u'$,
then $\chi_u\ne\chi_{u'}$ and $\langle \chi_u,\chi_{u'}\rangle = 0$ by the previous lemma. Hence
the set $\{\chi_u : u\in Z\}$ comprises $|Z|$ linearly independent elements in a space of dimension $|Z|$,
which must be spanning.\slug

Note that the map $u\mapsto \chi_u$ gives an isomorphism from $Z$ to $\hat Z$.
The two main examples we shall consider are when $Z = \ZZ_n$ and when $Z = \FF_p^n$. In the first
case, $\chi_u$ maps $x\mapsto e(ux/N)$ and in the second case $\chi_u$ maps $x\mapsto e(u\cdot x/p)$.
(Recall that if $u = (u_1,\ldots, u_n)$ and $x = (x_1,\ldots, x_n)$ then the dot product $u\cdot x$
is the sum $\sum_{i=1}^n u_ix_i$.)

For a finite set $X$, we shall denote by $L_p(X)$ the normed vector space of all functions $f:X\to \CC$
under the norm
$$\norm{f}_p = \bigl( \ex_{x\in X} \bigl| f(x)\bigr|^p \bigr)^{1/p}.$$
The notation $l_p(X)$ denotes the same set, but with the norm
$$\norm{f}_p = \Bigl( \sum_{x\in X} \bigl| f(x)\bigr|^p \Bigr)^{1/p}$$
instead. Note that $\norm{f}_2^2 = \langle f,f\rangle$ and $\norm{f}_\infty = \max_{x\in X} \bigl|f(x)\bigr|$.

\medskip\boldlabel The Fourier transform.
If $f:Z\to \CC$, the {\it Fourier transform} of $f$ is the function $\hat f : \hat Z\to\CC$ given by
$$\hat f(\chi) = \langle f,\chi\rangle = \ex_{x\in Z} f(x) \bar{\chi(x)}.$$
We shall now state and prove an identity which is extremely useful in Fourier analysis.

\parenproclaim Lemma {\advthm} (Parseval's identity). Let $Z$ be a finite abelian group
and let $f$ and $g$ be functions from $Z$ to $\CC$.
If $\hat f$ and $\hat g$ are the Fourier transforms of $f$ and $g$ respectively, then
$\langle f,g\rangle = \langle \hat f, \hat g\rangle$.

\proof First, we expand all the definitions and rearrange sums to obtain
$$\eqalign{
\langle \hat f, \hat g\rangle &= \sum_{\chi\in \hat Z} \hat f(\chi)\bar{\hat g(\chi)} \cr
&=  \sum_{\chi\in \hat Z} \ex_{x\in Z} f(x)\bar{\chi(x)} \bar{\ex_{y\in Z} g(y)\bar{\chi(y)}} \cr
&=   \ex_{x\in Z} f(x) \bar{\ex_{y\in Z} \bar{g(y)} \sum_{\chi\in \hat Z}\chi(x)\bar{\chi(y)}}.\cr
}$$
By the second orthogonality relation and the fact that $|\hat Z| = |Z|$, the inner sum equals $|Z|$ when
$y = x$ and $0$ otherwise. Thus for any given $x$, we have
$$\ex_{y\in Z} \bar{g(y)} \sum_{\chi\in \hat Z}\chi(x)\bar{\chi(y)} = \bar{g(x)},$$
which is exactly what we need for the whole thing to equal $\langle f,g\rangle$.\slug

We also have the following inversion formula that recovers the original function $f$ from $\hat f$.

\parenproclaim Lemma {\advthm} (Fourier inversion formula). Let $Z$ be a finite abelian group and
let $f:Z\to \CC$. We have
$$f(x) = \sum_{\chi\in \hat Z} \hat f(\chi)\chi(x).$$

\proof We expand
$$\sum_{\chi\in \hat Z} \hat f(\chi)\chi(x) = \sum_{\chi\in \hat Z} \ex_{y\in Z} f(y) \bar{\chi(y)} \chi(x)
= \ex_{y\in Z} f(y) \sum_{\chi\in \hat Z} \chi(x)\bar{\chi(y)},$$
and we noted in the proof of Parseval's theorem that the right-hand side is exactly $f(x)$.\slug

The inversion formula tells us that two functions from $Z\to \CC$ are equal if and only if their Fourier
transforms are equal. For the next lemma, which concerns dilates of a function $f: Z\to \CC$, we will
establish some notation. If $x\in Z$, we define $nx$ recursively for all integers $n\ge 0$ by $0x = 0$
and $nx = x + (n-1)x$. We then set $nz = - (-n)x$ for all negative integers. Since the group operation
on characters is multiplication, we will not write $n\chi$ but instead $\chi^n$ for $\chi$ multiplied
with itself $n$ times.

\parenproclaim Lemma {\advthm} (Dilation rule). Let $Z$ be a finite abelian group and let $a\in \ZZ$
be an integer that is coprime to $|Z|$. Denote the multiplicative inverse of $a$ modulo $|Z|$ by $a^{-1}$.
Letting $f_a : Z\to \CC$ be the function given by $f_a(x) = f(a^{-1} x)$, we have
$\hat{f_a}(\chi) = \hat f(\chi^a)$.

\proof We have
$$\hat{f_a}(\chi) = \ex_{x\in Z} f_a(x) \bar{\chi(x)} = \ex_{x\in Z} f(a^{-1} x)\bar{\chi(x)}.$$
Since $x\mapsto ax$ is a bijection from $Z$ to itself, we can replace $x$ formally by $ax$ in the above
to get
$$\hat{f_a}(\chi) = \ex_{x\in Z} f(x)\bar{\chi(ax)} = \ex_{x\in Z} f(x) \bar{\chi(ax)}^a
= \hat f(\chi^a).\noskipslug$$

\medskip\boldlabel Convolutions.
For functions $f$ and $g$ from a finite abelian group $Z$ to $\CC$, the {\it convolution} $f*g$ is
defined by
$$ (f*g)(x) = \ex_{y+z = x} f(y)g(z).$$
If instead $\hat f$ and $\hat g$ are functions from $\hat Z$ to $\CC$, then
$$ (\hat f * \hat g)(\chi) = \sum_{\chi_1\chi_2 = \chi} \hat f(\chi_1)\hat g(\chi_2).$$

\parenproclaim Lemma {\advthm} (Convolution law). Let $Z$ is a finite abelian group and $f,g : \Z\to \CC$.
For all $\chi\in \hat Z$, we have $\hat{f*g} (\chi) = \hat f(\chi)\hat g(\chi)$.

\proof We start by expanding
$$\hat{f*g}(\chi) = \ex_{x\in Z} (f*g)(x) \bar{\chi(x)} = \ex_{x\in Z} \ex_{y+z=x} f(y)g(z)\bar{\chi(y)\chi(z)}.$$
But note that $x$ does not appear in the summand anymore, meaning that we can simply rewrite this as
an expectation
over {\it all} $y$ and $z$ (their sum will equal $x$ for some $x\in Z$). Thus we can conclude that
$$\hat{f*g}(\chi) = \ex_{y\in Z}\ex_{z\in Z} f(y)g(z)\bar{\chi(y)\chi(z)} = \hat f(\chi)\hat g(\chi),$$
which is what we wanted to show.\slug

If $A$ is a subset of a finite abelian group $Z$, we associate to $A$ the {\it characteristic function}
$\one_A : Z\to \CC$ given by
$$\one_A(x) = \cases{ 1, & if $x\in A$;\cr 0, & otherwise.}$$
When it will not cause confusion, we will abuse notation and write $A(x)$ instead of $\one_A(x)$.
With the definitions and results stated above, we are now in a position to translate properties of $A$ to
Fourier-analytic statements about the function $\one_A$. First off, note that
$$\bignorm{\hat{\one_A}}_2^2 = \norm{\one_A}_2^2 = \ex_{x\in Z} A(x)^2 = \ex_{x\in Z} A(x)
= {|A|\over |Z|}.$$
If $|A|/|Z| = \delta$, then we will say that $A$ has {\it density} $\delta$. Another way of expressing $\delta$
in terms of $\one_A$ is
$$\hat{\one_A}(\chi_0) = \ex_{x\in Z} A(x)\chi_0(x) = \ex_{x\in Z} A(x) = \delta,$$
where $\chi_0$ is the trivial character.

The fact that $\norm{\one_A}_2^2 = \delta$ means that if we sample $x$ and $y$ from $Z$ and
condition that $x=y$, the probability that $x$ (or $y$) is in $A$ is $\delta$. The reason for stating this
obvious fact in such a stilted fashion is that it generalises by way of the convolution operation we defined
earlier; that is, what if we are interested in the probability that a $4$-tuple $(x,y,z,w)$ satisfying $x+y=z+w$
is a member of $A^4$? Well, the convolution law gives
$$\eqalign{
\ex_{x+y=z+w} A(x)A(y)A(z)A(w) &= \ex_{u\in Z} \ex_{x+y=u}\ex_{z+w=u} A(x)A(y)A(z)A(w) \cr
&= \ex_{u\in Z} (\one_A*\one_A)(u)^2 \cr
&= \norm{\one_A*\one_A}_2^2 \cr
&= \bignorm{\hat{\one_A*\one_A}}_2^2 \cr
&= \bignorm{\bigl(\hat{\one_A}\bigr)^2}_2^2 \cr
&= \sum_{\chi\in \hat Z} \bigl| \hat{\one_A}(\chi)\bigr|^4 \cr
&= \norm{\one_A}_4^4, \cr
}$$
and more generally, the probability that a tuple $(a_1,\ldots,a_k, b_1,\ldots,b_k)\in Z^{2k}$
with $a_1+\cdots+a_k=b_1+\cdots+b_k$ is a member of $A^{2k}$ is $\norm{\one_A}_{2k}^{2k}$. Recall that
the number of $(x,y,z,w)\in A^4$ with $x+y = z+w$
is the additive energy $E(A)$ of $A$; we just showed above that
$E(A) = |Z|^3 \norm{\one_A}_4^4$.

\advsect Bohr sets and Bogolyubov's lemma

Let $Z$ be a finite abelian group, let $\chi_1, \ldots, \chi_k$ be characters on $Z$, and let
$\delta>0$. The {\it Bohr set} $B(\chi_1,\ldots,\chi_k;\delta)$ is the set
$$\bigl\{ x\in Z : \chi_i(x)\in e\bigl([-\delta,\delta]\bigr)\ \hbox{for all}\ 1\le i\le k\bigr\}.$$
(As before, we write $e(\alpha)$ for $e^{2\pi i\alpha}$.) We say that such a Bohr set has {\it dimension} $k$
and {\it width} $\delta$.

If $Z= \FF_p^n$, we can write $\chi_i = \chi_{u_i}$ where $\chi_{u_i}(x) = e(u_i\cdot x/p)$. If $\delta<1/p$,
then $\chi_{u_i}(x)\in e\bigl([-\delta,\delta]\bigr)$ if and only if
$(u_i\cdot x) / p \bmod 1\in [-\delta,\delta]$. But $u_i\cdot x$ is an integer, so this is true if and only
if $u_i\cdot x = 0$, so consequently
$$B(\chi_1,\ldots,\chi_k;\delta) = \{ x\in \FF_p^n : u_i\cdot x = 0\ \hbox{for all}\ 1\le i\le k\},$$
which is a subspace cut out by $k$ linear conditions, i.e., of codimension at most $k$.

\parenproclaim Lemma {\advthm} (Bogolyubov). Let $Z$ be a finite abelian group. Let $A\subseteq Z$ be a
subset of density $\alpha$. Then $2A-2A$ contains a Bohr set of dimension at most $1/\alpha^2$ and width
$1/4$.

\proof For $x\in Z$, let
$$\eqalign{
f(x) &= \bigl( \one_A * \one_A * (\mone_A) * (\mone_A)\bigr)(x) \cr
&= \ex_{p+q-r-s=x} \one_A(p)\one_A(q)\one_A(r)\one_A(s).
}$$
Then $x\in 2A-2A$ if and only if $f(x)\ne 0$. We want to find a Bohr set on which $f$ does not vanish.

Note first that for any character $\chi$,
$$\hat{\mone_A}(\chi) = \ex_x \one_A(-x)\bar{\chi(x)} = \ex_x \one_A(x)\bar{\chi(-x)} = \ex_x\one_A(x)\chi(x)
=\bar{\hat{\one_A}(\chi)}.$$
(The last identity relies on the fact that $\one_A(x)$ is real.) This implies that
$\hat{\one_A}(\hat{\mone_A}) = |\one_A|^2$. Now we apply the Fourier inversion
formula and the convolution law to expand
$$\eqalign{
f(x) &= \sum_\chi \hat{\one_A*\one_A*(\mone_A)*(\mone_A)}(\chi)(\chi)(x) \cr
&= \sum_\chi \hat{\one_A}(\chi)^2 \hat{\mone_A}(\chi)^2 \chi(x) \cr
&= \sum_\chi \bigl| \hat{\one_A}(\chi) \bigr|^4 \chi(x). \cr
}$$
Let
$$K = \bigl\{ \chi\in \hat Z : \bigl| \hat{\one_A}(\chi) \bigr| \ge \alpha^{3/2}\bigr\}$$
and let $B$ be the Bohr set $B(K;1/4)$. Then the expression above for $f(x)$ can be split into
$$f(x) =
\bigl| \hat{\one_A}(\chi_0) \bigr|^4
+ \sum_{\chi\in K} \bigl| \hat{\one_A}(\chi) \bigr|^4 \chi(x)
+ \sum_{\chi\notin K\cup \{\chi_0\}} \bigl| \hat{\one_A}(\chi) \bigr|^4 \chi(x).$$
We shall assume that $x\in B$ and deal with each of the three terms separately.

The first term is easy; since $A$ has density $\alpha$, $\hat{\one_A}(\chi_0) = \alpha$, and thus the first
time is $\alpha^4$. For the second term, note that if $\chi\in K$ and $x\in B$, then
$\chi(x)\in e\bigl([-1/4, 1/4]\bigr)$. This means that $\chi(x)$ has argument between $-\pi/2$ and $\pi/2$,
and {\it a fortiori} has nonnegative real part. Now, if $\chi\notin K$ and $\chi\ne \chi_0$,
then $\bigl|\hat{\one_A}\bigr| < \alpha^{3/2}$, so we take the absolute value of the third term
and apply the triangle inequality and Parseval's identity to obtain
$$\Bigl|\sum_{\chi\notin K\cup \{\chi_0\}} \bigl| \hat{\one_A}(\chi) \bigr|^4 \chi(x)\Bigr|
\le \sum_{\chi\notin K\cup\{\chi_0\}} \bigl|\hat{\one_A}(\chi)\bigr|^4
< \sum_{\chi\in \hat Z} \alpha^3 \bigl| \hat{\one_A}(\chi)\bigr|^2
= \alpha^3\langle \hat{\one_A},\hat{\one_A}\rangle.$$
But by Parseval's identity, the right-hand side equals $\alpha^3 \langle \one_A,\one_A\rangle = \alpha^4$,
so we find that the real part of the third term is more than $-\alpha^4$. This means that $\Re f(x) > 0$,
so $f(x)\ne 0$ for all $x\in B$; that is, $B\subseteq 2A-2A$.

It remains to give an upper bound on the size of $K$. We have
$$\alpha = \sum_{\chi\in \hat Z} \bigl| \one_A(\chi)\bigr|^2 \ge
\sum_{\chi\in K} \bigl|\hat{\one_A(\chi)}\bigr|^2 \ge |K|\alpha^3,$$
meaning that $|K|\le 1/\alpha^2$.\slug

When $A$ is a subset of density $\alpha$ in $Z = \FF_p^n$, Bogolyubov's lemma tells us that $2A-2A$ contains
a subspace of codimension at most $1/\alpha^2$.

\advsect Freiman's theorem in vector spaces over fields of prime order

Let $Z$ be an additive group and let $A$ be a finite subset of $Z$. It is possible that $|A+A| = |A|$,
but this happens if and only if $A$ is a coset of a subgroup of $Z$ (which means that $Z$ must be finite).
If $A$ is a finite subset of integers, then without loss of generality we can shift $A$ so that $\min A = 0$.
if $A$ only contains $0$ then $|A+A|$ is $1 = 2|A|-1$, and if $|A| > 1$, then let $m = \max A$ and
note that $A+A$ contains $A$ and $m+A$, and $A\cap (m+A) = \{m\}$, so in this case we also have $|A+A|\ge 2|A|-1$.

It is not hard to see that if $A$ is an arithmetic progression
in the integers, then $A+A$ is exactly $2|A|-1$,
but for instance $A = \{1,\ldots,n\}\subseteq \ZZ$ is not contained in any nontrivial subgroup of $\ZZ$ and
it also has $|A+A| = 2|A|-1$.
So next we might amend our conjecture to say that if $|A+A|$ is small, then $A$ is a large subset of an arithmetic
progression. But this is also false. Consider the set
$$\{0,1,2,\quad 10,11,12,\quad 20,21,22,\quad 30,31,32\}.$$
In some sense this is a ``progression of progressions,'' or a $2$-dimensional arithmetic progression. Such
a set is the projection of some rectangle in $\ZZ^2$, and from this observation it is not hard to see
that $|A+A|\le 4|A|$. If we increase $d$, we can still say that $|A+A|\le 2^d|A|$.

Once we add these $d$-dimensional arithmetic progressions into consideration, it turns out that we can have
an inverse statement. Concretely, if $|A+A|\le C|A|$, then there exists a $d$-dimensional arithmetic
progression $P$ such that $|P|\le K|A|$ and $A\subseteq P$, where $d$ and $K$ depend only on $C$ (otherwise
we could simply embed $A$ into $[\min A, \max A]$).

This statement is called Freiman's theorem, as it was first proved by Freiman over the integers.
In these notes we will prove an analogue of it for subsets of $\FF_p^N$, in which ``$d$-dimensional arithmetic
progression'' is replaced by ``subspace''. First we need a lemma.


\parenproclaim Lemma {\advthm} (Ruzsa covering lemma). Let $Z$ be an abelian group and let $A$ and
$B$ be finite subsets of $Z$. Then there is a set $K\subseteq A$ of size at most $|A+B|/|B|$ such that
$A\subseteq K+B-B$.

\proof Let $K = \{a_1,\ldots,a_k\}$ be a maximal subset of $A$ with the property that the sets $a_i+B$
are disjoint. For all $a\in A$ there is $i$ such that $(a+B) \cap (a_i+B) \ne \emptyset$, otherwise we
could add $a$ to $K$ and contradict maximality. Equivalently, we can find $b$ and $b'$ such that
$a+b = a_i + b'$ so $a\in a_i + B - B$. Hence $A\subseteq K+B-B$. Lastly, $|K|\cdot |B| = |K+B| \le |A+B|$.\slug

Next, we prove two simple lemmas about Freiman isomorphisms, the second concerning
how they behave on subspaces of $\FF_p^n$.

\edef\leminducedfreiman{\the\thmcount}
\proclaim Lemma {\advthm}. A Freiman isomorphism of order $k(r+s)$ from $A$ to $B$ induces a Freiman
isomorphism of order $k$ from $rA-sA$ to $rB-sB$.

\proof This proof is conceptually very easy but notationally a total nightmare.
Let $\phi$ be a $k(r+s)$ isomorphism from $A$ to $B$. Then
$$x_1+\cdots x_{k(r+s)} = y_1+\cdots+y_{k(r+s)}$$
if and only if
$$\phi(x_1)+\cdots \phi(x_{k(r+s)}) = \phi(y_1)+\cdots+\phi(y_{k(r+s)})$$
Now let $g_1,\ldots,g_k, h_1,\ldots,h_k$ be $2k$ elements in $rA-sA$. Then the expression
$$g_1+\cdots g_k = h_1+\cdots+h_k$$
has a sum of $kr$ elements of $A$ and $ks$ elements of $-A$ on each side. We can subtract
off the elements of $-A$ on each side to get $k(r+s)$ elements of $A$ on either side, and the equality
holds if and only if it holds with $\phi$ applied to every term. So letting
$$g_i = g'_{i,1}+\cdots+g'_{i,r}-g''_{i,1}-\cdots-g''_{i,s}$$
and
$$h_i = h'_{i,1}+\cdots+h'_{i,r}-h''_{i,1}-\cdots-h''_{i,s}$$
for all $i$, we find that $g_1+\cdots g_k = h_1+\cdots+h_k$ holds if and only if
$$\eqalign{
\phi(g'_{1,1})+\cdots+&\phi(g'_{k,r}) + \phi(h''_{1,1})+\cdots+\phi(h''_{k,s}) \cr
&= \phi(h'_{1,1})+\cdots+\phi(h'_{k,r}) + \phi(g''_{1,1})+\cdots+\phi(g''_{k,s}).
}$$
Subtracting all the double-prime terms from both sides yields
$$\eqalign{
\phi(g'_{1,1})+\cdots+&\phi(g'_{k,r}) - \phi(g''_{1,1})-\cdots-\phi(g''_{k,s}) \cr
&= \phi(h'_{1,1})+\cdots+\phi(h'_{k,r})  - \phi(h''_{1,1})-\cdots-\phi(h''_{k,s}),
}$$
and letting
$$\psi(g_i) = \phi(g'_{i,1}) + \cdots + \phi(g'_{i,r}) - \phi(g''_{i,1}) -\cdots-\phi(g''_{i,s})$$
and
$$\psi(h_i) = \phi(h'_{i,1}) + \cdots + \phi(h'_{i,r}) - \phi(h''_{i,1}) -\cdots-\phi(h''_{i,s})$$
for all $i$, we see that $g_1+\cdots g_k = h_1+\cdots+h_k$ holds if and only if
$$\psi(g_1)+\cdots \psi(g_k) = \psi(h_1)+\cdots+\psi(h_k).$$
It is invertible, since we can just apply $\phi^{-1}$ termwise to an element of $rB-sB$, so
this is the $k$-isomorphism we need between $rA-sA$ and $rB-sB$.\slug

\edef\lemaffinefreiman{\the\thmcount}
\proclaim Lemma {\advthm}. Let $X$ be a linear subspace of $\FF_p^n$ and let $\phi$ be a $2$-isomorphism
from $X$ to a set $A\subseteq \FF_p^n$. Then $A$ is an affine subspace of $\FF_p^n$ of dimension $\dim X$.

\proof We shall show that $Y = A - \phi(0)$ is a linear subspace of $\FF_p^n$. Clearly it contains $0$,
since $A$ contains $\phi(0)$. Now let $y_1,y_2\in Y$ so that $y_1 + \phi(0)$ and $y_2+\phi(0)$ are both in $A$.
Let $x_1 = \phi^{-1}\bigl(y_1+\phi(0)\bigr)$, $x_2 = \phi^{-1}\bigl(y_1+\phi(0)\bigr)$, and let $x$ denote
their sum in $X$. Then since $x_1 + x_2 = x+0$, we have
$$y_1 + \phi(0) + y_2 + \phi(0) = \phi(x) + \phi(0).$$
Subtracting $\phi(0)$ from both sides, we find that $y_1 + y_2 + \phi(0) = \phi(x)$, meaning
that $y_1 + y_2 \in A - \phi(0) = Y$. This proves that $Y$ is closed under finite sums, and since scalar
multiplication over a finite field can always be re\"expressed as a finite sum, we find that
$A$ is an affine subspace of $\FF_p^n$ in bijection with $X$, that is, of dimension $\dim X$.\slug

We are now able to prove Freiman's theorem for $\FF_p^n$, whose proof amounts to little more than
a concatenation of several previous results.

\parenproclaim Theorem {\advthm} (Freiman's theorem in $\FF_p^n$). Let $A\subseteq \FF_p^n$ and suppose
that $|A+A|\le C|A|$. Then there is a subspace $X$ of $\FF_p^n$ such that $A\subseteq X$ and $|X|\le C'|A|$,
where $C'$ depends only on $C$ and $p$.

\proof By Pl\"unnecke's theorem, $|8A-8A| \le C^{16} |A|$, hence by Ruzsa's embedding lemma, $A$
is $8$-isomorphic to a subset $A_1\subseteq \FF_p^m$ where $p^m \le pC^{16}|A|$. By Bogolyubov's lemma
with $\alpha = p^{-1}C^{-16}$, $2A_1 - 2A_1$ contains a Bohr set $B(K;1/4)$ of dimension at most
$p^2C^{32}$. But letting $\delta < 1/(2p) \le 1/4$, and $X = B(K;\delta)$, we have $X\subseteq B(K;1/4)$
and $X$ is a linear subspace of $\FF_p^m$ of codimension at most $p^2C^{32}$.

Now since $A$ is $8$-isomorphic to $A_1$, we can apply Lemma~{\leminducedfreiman} to deduce that
$2A-2A$ is $2$-isomorphic to $2A_1 - 2A_1$. It follows that
$2A-2A$ contains a set $Y$ that is $2$-isomorphic to the linear subspace $X$. By Lemma~{\lemaffinefreiman},
this set $Y$ is an affine subspace, of dimension at least $m-p^2C^{32}$, and thus of cardinality
at least $p^{m-p^2C^{32}} \ge p^{p^2C^{32}}|A|$ (here we used the fact that $A_1\subseteq \FF_p^m$ and
$A_1$ is Freiman isomorphic to $A$).

Ruzsa's covering lemma tells us that there exists $L$ of size at most $|A+Y|/|Y|$ such that $A\subseteq L+Y-Y$.
But since $Y$ is an affine subspace, $Y-Y$ is a linear subspace; setting $V = Y-Y$, we have
$$|V| = |Y| \le |2A-2A| \le C^4|A|,$$
which follows from another application of Pl\"unnecke's theorem.
Then by possibly adding $|L|$ basis vectors, $L+V$ is contained in a linear subspace of size at most $p^{|L|} |V|
\le p^{|L|} C^4|A|$.
On the other hand, $L$ has size at most
$${|A+2A-2A|\over |V|} \le {C^5|A| \over p^{-p^2C^{32}} |A|} = C^5p^{p^2C^{32}},$$
where yet again we have used Pl\"unnecke's theorem.
Hence we conclude that $A$ is contained in a linear subspace of size at most
$$ p^{C^5 p^{p^2C^{32}}} C^4|A|.\noskipslug$$

\advsect Roth's theorem

In the previous section we showed that any set $A$ with $A+A$ small
must be a dense subset of a structured set. We didn't prove this over $\ZZ$, but we saw briefly that
the notion of ``structured set'' in that case was a $d$-dimensional arithmetic progression. Now we switch
gears and try to find arithmetic progressions in subsets of $\ZZ$. The first major theorem in this vein was
proved by B.~L.~van der Waerden in 1927.

\parenproclaim Theorem W (van der Waerden, {\rm 1927}). For all $k$ and $r$, there is $n$ such
that if we partition $[n]$ into disjoint sets $C_1\cup \cdots \cup C_r$ then some $C_i$ contains an
arithmetic progression of length $k$.\slug

A stronger statement is given by the celebrated 1975 theorem of E.~Szemer\'edi.

\parenproclaim Theorem S (Szemer\'edi, {\rm 1975}). For every $\delta>0$ and $k\in \NN$ there
exists $n$ such that if $A\subseteq [n]$ and $|A|\ge \delta n$, then $A$ contains an arithmetic progression
of length $k$.\slug

By setting $\delta = 1/r$ and invoking the pigeonhole principle, we see that
Szemer\'edi's theorem implies van der Waerden's theorem. We will not prove either of these statements in
these notes. The first has a relatively elementary proof, but not one that is very additive-combinatorial
in nature, while the second is far too complex to prove in full generality here. We will, however,
prove Szemer\'edi's theorem in the case $k=3$. This special case is called Roth's theorem as it was
proved by K.~Roth in 1953.
Before we get there we first establish three lemmas. In their proofs we will write, e.g., $A$ instead
of $\one_A$, and identify
$r$ with its corresponding element $\chi_r\in \hat{\ZZ_N}$, writing $\hat f(r)$ instead of $\hat f(\chi_r)$.

\edef\lemABC{\the\thmcount}
\proclaim Lemma {\advthm}. Let $N$ be odd and let $A$, $B$, and $C$ be subsets of $\ZZ_N$ with densities
$\alpha$, $\beta$, and $\gamma$ respectively. Suppose that $N > 2/(\alpha\beta\gamma)$. Then either
there exists $x$ and $d\ne 0$ such that $(x,x+d,x+2d)\in A\times B\times C$, or there exists $r\ne 0$
such that $|\hat A(r)| \ge \alpha \beta^{1/2} \gamma^{1/2} / 2$.

\proof Consider the quantity
$$\eqalign{
\ex_{x,d} A(x)B(x+d)C(x+2d) &= \ex_{x+z=2y} A(x)C(z)B(y) \cr
&= \ex_u \bigl( \ex_{x+z=u} A(x)C(z)\bigr) B(u/2).
}$$
Using the notation $f_a(x) = f(x/a)$ for a dilation, we see that this expectation is actually the
inner product $\langle A *C, B_2\rangle$, which by Parseval's identity is equal to
$\langle \hat{A*C}, \hat{B_2}\rangle$.
Recall that the characters of $\ZZ_N$ are given by $\chi_a(x) = e(ax/N)$. Applying the dilation rule gives
$$ \hat{B_2}(r) = \hat{B_2}(\chi_r) = \hat B({\chi_r}^2) = \hat B(\chi_{2r})=\hat B(2r)=(\hat B)_{2^{-1}}(r)$$
where $2^{-1}$ denotes the inverse of $2$ modulo $N$.
This, combined with the convolution law, gives
$\langle \hat{A*C}, \hat{B_2}\rangle = \bigl\langle \hat A\hat C, (\hat B)_{2^{-1}}\bigr\rangle$.
But since $B$ is real-valued,
$$\hat B(-r) = \ex_x B(x)\bar{\chi_{-r}(x)} = \ex_x B(x)\chi_r(x) = \bar{\ex_x B(x)\bar{\chi_r(x)}}
= \bar{\hat B(r)},$$
so putting everything together yields
$$\eqalign{
\ex_{x,d} A(x)B(x+d)C(x+2d) &= \bigl\langle \hat A\hat C, (\hat B)_{2^{-1}}\bigr\rangle \cr
&= \sum_r \hat A(r)\hat C(r) \bar{\hat B(2r)} \cr
&= \sum_r \hat A(r)\hat C(r) \hat B(-2r) \cr
&= \alpha\beta\gamma + \sum_{r\ne 0} \hat A(r) \hat C(r) \hat B(-2r). \cr
}$$
Note that by the triangle and Cauchy--Schwarz inequalities,
$$\eqalign{
\Bigl|\sum_{r\ne 0} \hat A(r) \hat C(r) \hat B(-2r)\Bigr|
&\le \max_{r\ne 0} \bigl|\hat A(r)\bigr| \sum_{r\in \ZZ_N} \bigl|\hat B(-2r)\bigr|\cdot\bigl|\hat C(r)\bigr| \cr
&\le \max_{r\ne 0} \bigl|\hat A(r)\bigr|
\Bigl(\sum_{r\in \ZZ_N} \bigl|\hat B(-2r)\bigr|^2\Bigr)^{1/2}
\Bigl(\sum_{r\in \ZZ_N} \bigl|\hat C(r)\bigr|^2\Bigr)^{1/2} \cr
&= \bignorm{\hat B}_2 \bignorm{\hat C}_2 \max_{r\ne 0} \bigl| \hat A(r)\bigr| \cr
&= \beta^{1/2} \gamma^{1/2} \max_{r\ne 0} \bigl| \hat A(r)\bigr|.\cr
}$$
(We used the fact that $N$ is odd in the third line above.) So
$$
\ex_{x,d} A(x)B(x+d)C(x+2d) \ge \alpha\beta\gamma
-\beta^{1/2} \gamma^{1/2} \max_{r\ne 0} \bigl| \hat A(r)\bigr|,$$
meaning that if $\max_{r\ge 0} \bigl| \hat A(r)\bigr| < \alpha \beta^{1/2}\gamma^{1/2} / 2$, then
$\ex_{x,d} A(x)B(x+d)C(x+2d) > \alpha\beta\gamma/2$. The further assumption that $\alpha\beta\gamma/2 > 1/N$
allows us to conclude that there is some $x$ for which $\ex_{d} A(x)B(x+d)C(x+2d) > 1$, and thus some
$d\ge 0$ such that $(x,x+d,x+2d)\in A\times B\times C$.\slug

In the next lemma, we define the {\it diameter} of $X\subseteq \CC$ to be the quantity
$\sup_{z_1,z_2\in X} |z_1-z_2|$.

\proclaim Lemma {\advthm}. For every $\theta\in \RR$, $n\in \NN$, and $\eps > 0$, the
set $[n]$ can be partitioned
into disjoint arithmetic progressions $P_1, \ldots, P_m$, each of size at least $\sqrt{\eps n/(32 \pi)}$,
such that the diameter of $e(\theta P_i)\le \eps$ for all $1\le i\le m$.

\proof Let $t$ be a positive integer to be chosen later. Dividing the unit circle into $t$ segments of
length $2\pi/t$, by the pigeonhole principle there must be some segment that contains more than one of
the $t+1$ points $e(0), e(\theta), \ldots, e(t\theta)$. Thus there are $u$ and $v$ with $0\le u<v\le r$ such that
$\bigl|e(u\theta) - e(v\theta)\bigr|\le 2\pi/t$. Let $d = v-u$. By multiplying both $e(u\theta)$ and
$e(v\theta)$ by $e(-u\theta)$, we find that $\bigl|1-e(d\theta)\bigr| \le 2\pi/t$ as well. Then by telescoping and
applying the triangle inequality, we may expand
$$\bigl|1-e(sd\theta)\bigl| \le \sum_{k=1}^s \bigl|e\bigl((k-1)d\theta-e(kd\theta)\bigr|.$$
But by applying rotations as above, each term in the sum is equal to $\bigl|1-e(d\theta)\bigr|$,
implying that $\bigl|1-e(sd\theta)\bigl| \le 2\pi s/t$.

The reason this is important is that
whenever $P$ is any arithmetic progression with common difference $d$, any $x,y\in P$ will
have $\bigl|e(x\theta) - e(y\theta)\bigr| = \bigl|1-e(sd\theta)\bigr|$ for some $s$ that is at most
the length of $P$. Hence if $P$ is an arithmetic progression with common difference $d$ and length
at most $\eps t/(2\pi)$, then the diameter of $e(\theta P)$ will be at most $\eps$.

Take $[n]$ and partition it into residue classes modulo $d$. Within each residue class, we would like
to further group the elements into ``intervals'' of size roughly $\eps t/(2\pi)$. First we need to make
sure that the residue classes are large enough. If a residue class has size at least $\eps t/(2\pi)$,
then it can be subdivided into intervals of size between $\eps t/(4\pi)$ and $\eps t/(2\pi)$.
To ensure this, we note that the residue classes have size $\lfloor n/d\rfloor \ge \lfloor n/t\rfloor$,
so we shall set $t \le \sqrt{2\pi n / \eps}$. With this choice we find that each $P_i$ has size
$$\biggl\lfloor {\eps t\over 4\pi}\biggr\rfloor \le  \Biggl\lfloor {\eps\over 4\pi} \sqrt{2\pi n\over \eps}
\Biggr\rfloor = \Biggl\lfloor \sqrt{\eps n\over 8\pi} \Biggr\rfloor \le \sqrt{\eps n\over 32\pi}.\noskipslug$$

We use this lemma immediately to prove another one, which is key for the density-increment argument
of Roth's proof.

\edef\lemdensityincrement{\the\thmcount}
\proclaim Lemma {\advthm}. Let $A\subseteq \ZZ_N$ be a set of density $\delta$, let $\eta>0$ and suppose
there exists $r$ such that $\bigl| \hat A(r)\bigr| \ge \eta$. Then there is a progression $P$ of length
at least $\sqrt{\eta N/(64\pi)}$ such that $|A\cap P| \ge (\delta + \eta/4)|P|$.

\proof Define $f(x) = A(x)-\delta$. Then $\sum_{x\in \ZZ_n} f(x) = 0$ and
$$ \hat f(\chi) = \ex_x (A(x) - \delta)\bar{\chi(x)} = \ex_x A(x)\bar{\chi(x)} - \delta \ex_x \bar{\chi(x)} $$
If $\chi = \chi_0$ then this whole expression equals $0$; otherwise the second expectation disappears and
$\hat f(\chi)$ just equals $\hat A(\chi)$. Let $r$ be such that $\bigl| \hat f(r)\bigr| = \bigl| \hat A(r)\bigr|
\ge \eta$. Then
$$ \eta N \le N\bigl| \hat f(r)\bigr| = N \Bigl| \sum_{x\in \ZZ_N} f(x) e(-rx/N)\Bigr|.$$
By the previous lemma with $\theta = -r/N$ and $\eps = \eta/2$,
we can partition $[N]$ into progressions $P_1,\ldots,P_m$, each of size at least $\sqrt{\eta n / (64\pi)}$
such that every set $e(-rP_i/N)$ has diameter at most $\eta/2$. Thus by the triangle inequality,
$$\eta N \le \sum_{i=1}^m \Bigl| \sum_{x\in P_i} f(x) e(-rx/N) \Bigr|.$$
For each $i$, pick some element $x_i\in P_i$. Then
$$\eqalign{
\eta N &\le \sum_{i=1}^m \Bigl| \sum_{x\in P_i} f(x) e(-rx_i/N) \Bigr|
  +\sum_{i=1}^m \Bigl| \sum_{x\in P_i} f(x) \bigl(e(-rx/N) - e(-rx_i/N) \bigr)\Bigr| \cr
&\le \sum_{i=1}^m \bigl|e(-rx_i/N)\bigr|\cdot\Bigl| \sum_{x\in P_i} f(x) \Bigr| + {\eta N\over 2},
}$$
whence
$$ \sum_{i=1}^m \Bigl| \sum_{x\in P_i} f(x) \Bigr| \ge {\eta N\over 2} = {\eta\over 2} \sum_{i=1}^m |P_i|.$$
But $\sum_{i=1}^m \sum_{x\in P_i} f(x) = 0$, so we may add these terms into the sum to obtain
$$ \sum_{i=1}^m \biggl(\sum_{x\in P_i} f(x) + \Bigl| \sum_{x\in P_i} f(x) \Bigr|\biggr)
\sum_{i=1}^m {\eta\over 2}|P_i|,$$
and deduce that there must be some $i$ such that
$$ \sum_{x\in P_i} f(x) + \Bigl| \sum_{x\in P_i} f(x) \Bigr| \ge {\eta\over 2} |P_i|.$$
For this $i$ we have
$$|A\cap P_i| - \delta|P_i| =
\sum_{x\in P_i} A(x) - \delta|P_i| = \sum_{x\in P_i} f(x) \ge {\eta\over 4}|P_i|,$$
which implies that $|A\cap P_i| \ge (\delta + \eta/4)|P_i|$.\slug

We are now equipped to prove Roth's theorem.

\parenproclaim Theorem {\advthm} (Roth, {\rm 1953}). For every $\delta>0$ there exists
$N\le \bigl\lceil\exp\bigl(\exp(300/\delta)\bigr)\bigr\rceil$ such that
every $A\subseteq [N]$ with $|A|\ge \delta N$ contains an arithmetic progression of length $3$.

\proof Let $N$ be an integer to be specified later. Let $A_0$ be a subset of $N$ of density
$\delta = \delta$If $N$ is even, we can express $N = L_0+M_0$
where both $L_0$ and $M_0$ are odd integers at least $N/4$. Either
$$\bigl|A_0 \cap [L_0]\bigr| \ge \delta |L_0|\ \hbox{or}
\ \bigl|A_0\cap [L_0+1, N]\bigr| \ge \delta M_0$$
In the first case let $N_0 = L_0$ and set $A = A_0\cap [L_0]$, and in the second case set $N_0=M_0$
and set $A = \bigl(A_0\cap [L_0+1,N]\bigr) - \{L_0\}$.

Now we reach another crossroads. If $\bigl| A\cap (N_0/3, 2N_0/3]\bigr| < \delta N_0/5$, then one of
$\bigl| A\cap [1,N_0/3]\bigr|$ or $\bigl| A\cap (2N_0/3,N_0]\bigr|$ must be greater than $2\delta N_0 / 5$.
Either way, we get a progression $P$ of length roughly $N_0/3$ such that $|A\cap P| \ge \delta_1|P|$,
where $\delta_1$ is roughly $6\delta/5$. Thus we may iterate the argument with $A = A\cap P$, $N$
roughly equal to $N_0/3$, and $\delta = \delta_1$. Here we can be very sloppy with bounds and say that
the new choice of $N$ is at least $N/24$, and the new choice of $\delta$ is at least $11\delta/10$.

It is the case $\bigl| A\cap (N_0/3, 2N_0/3]\bigr| \ge \delta N_0/5$ in which we must use the lemmas
above. Let $B = C = A\cap [N_0/3, 2N_0/3)$. This choice for $B$ and $C$ ensures that if $d\ne 0$ and
$(x,x+d,x+2d)\in A\times B\times C$ where $A$, $B$, and $C$ are regarded as subsets of $\ZZ_{N_0}$, then
$x,x+d,x+2d$ is still an arithmetic progression in $\ZZ$. This is clear because if
$y,z\in [N_0/3,2N_0/3)\subseteq \ZZ$ and $d$ is the common difference $y-z$, then $|d| < N_0/3$, which
ensures that $x\in [N_0]$.

So if $A\times B\times C$ contains a triple $(x,x+d,x+2d)$ with $d\ne 0$, then we have an arithmetic
progression of length $3$ in $[N_0]$ (and thus in $[N]$), so we are done. If not,
let $\alpha$, $\beta$, and $\gamma$ be the densities of $A$, $B$, and $C$ respectively.
$\alpha \ge \delta$ and $\beta,\gamma\ge \delta/5$. By Lemma~{\lemABC}, so long as
$N_0 > 50/\delta^3 \ge 2/(\alpha\beta\gamma)$, there is some $r\ne 0$ such that
$\hat A(r) \ge \alpha\sqrt{\beta\gamma} \ge \delta^2/10$. By Lemma~{\lemdensityincrement} with
$\eta = \delta^2/10$, there is a progression $P$ in $A$ of size at least $\delta \sqrt{N_0/(640\pi)}$
such that $|A\cap P| \ge (\delta + \delta^2/40)|P|$.

By scaling and translating, we obtain a subset of $\bigl[|P|\bigr]$ of size at least $(\delta+\delta^2/40)|P|$,
so we can repeat the argument with $A$ set to $A = A\cap P$ and $N$ set to $|P|$. It is clear that
we cannot iterate forever, since $\delta \le 1$, so we will eventually find an arithmetic progression
of length $3$. It remains to calculate how many iterations it takes for this to happen, which in turn tells
us an upper bound for $N$.

Well, as long as $\delta \le 4/21$, we have $\delta + \delta^2/40 \le 11\delta/10$, so the
second case has the smaller density increment. In this case, note that if we iterate $40/\delta$ times,
then the density at least doubles. Then to double it again we need to iterate $40/(2\delta) = 20/\delta$
times, and so on. By repeating this argument, we only need to iterate at most
$$ {40\over\delta} \biggl(1 + {1\over 2} + {1\over 4} + \cdots \biggr) = {80\over \delta}$$
times and we will eventually have to find an arithmetic progression of length $3$, otherwise $\delta$ would
become greater than $1$. We can even be extra pedantic here and say that this only gets us to $\delta >4/21$,
and then $20$ more iterations is sufficient to get to $\delta > 1$. (This is because $\log_{11/10}(21/4) < 20$).

In the argument we required $N_0 > 50/\delta^3$, which means that $N$ must always be greater than
$200/\delta^3$. To further simplify our calculations, we note that
$$\delta \sqrt{N_0\over 640\pi } \ge {\delta\sqrt N\over 100} \ge N^{1/3}$$
as long as $N\ge (100/\delta)^6$. Requiring this condition ensures that in any iteration, the next
value of $N$ is at least $N^{1/3}$. So to eventually find a $3$-term arithmetic
progression, we need to start with an $N$ such that $N^{3^{-80/\delta+20}} \ge (100/\delta)^6$.
Taking the logarithm of both sides, we get
$$ 3^{-80/\delta+20} \log N \ge 6\log(100/\delta),$$
and then taking another logarithm yields
$$ \log\log N \ge \biggl({80\over\delta} + 20\biggr)\log 3 +\log 6 + \log\log(100/\delta),$$
We'll be lazy here and set $\log\log N\ge 300/\delta$ as
a sufficient condition, which is the bound claimed in the theorem statement.\slug

We can flip the roles of $\delta$ and $N$ to arrive at the following statement,
equivalent to what we just proved: {\sl Any subset of $[N]$
of size at least $300/\log\log N$ contains a nontrivial arithmetic progression of length $3$.}

\advsect The Furstenberg--S\'ark\"ozy theorem

In this section, we'll prove a bound on the size of {\it square-difference-free sets}. These are sets
$A\subseteq \NN$ such that $A-A$ does not contain any of the squares $\{1,4,9,16,\ldots\}$.
The proof, which is due to B.~Green, T.~Tao, and T.~Ziegler,
makes use of a density increment argument similar to the one used in the proof of
Roth's theorem, but does not use any Fourier analysis. We begin with the following lemma, which allows
us to increment the density of a set containing a square in its difference set.

\proclaim Lemma~\advthm.
Suppose that there is $N\in \NN$ and $A\subseteq [N]$ of density $\delta = |A|/N$ such that
$A-A$ does not contain any positive square numbers. Then there exists an arithmetic progression
$P$ of length $\lfloor N^{1/3}\rfloor$
such that $|A\cap P| \ge (\delta + c\delta^3)|P|$, where $c$ does not depend on $\delta$ or $N$.

The exponents $1/3$ and $1/100$ in this statement are not really important, and the proof
may be modified to replace these with other small constants if desired.

\proof By hypothesis, for any $n\in [N]$ we have
$$ \ex_{r\in [N^{1/3}]} \ex_{h\in [N^{1/100}]} \one_A(n) \one_A\bigl( n+(r+h)^2\bigr) = 0.$$
Now for $n\in [N]$, note that $n+(r+h)^2$ is in $[N]$ for all
$n\le N-(N^{1/3}-N^{1/100})^2 = N-O(N^{2/3})$. This means that
$$ \ex_{n\in [N]} \ex_{r\in [N^{1/3}]} \ex_{h\in [N^{1/100}]}
\one_{[N]}\bigl( n+(r+h)^2\bigr) = 1 - O(N^{-1/3}).$$
By further requiring that $n\in A$ inside the expectation, we have
$$ \ex_{n\in [N]} \ex_{r\in [N^{1/3}]} \ex_{h\in [N^{1/100}]}
\one_A(n)\one_{[N]}\bigl( n+(r+h)^2\bigr) = \delta - O(N^{-1/3}),$$
since the proportion of $n\in [N]$ that are also in $A$ is $\delta$.
Lastly, by a simple change of variables we have
$$\eqalign{
\ex_{n\in [N]} \ex_{r\in [N^{1/3}]} &\ex_{h\in [N^{1/100}]} \one_A\bigl(n+(r+h)^2\bigr) \cr
&=\ex_{m\in A} \ex_{r\in [N^{1/3}]} \ex_{h\in [N^{1/100}]} \one_{[N]}\bigl(m-(r+h)^2\bigr) \cr
&= \delta - O(N^{-1/3}). \cr
}$$
Set $f = \one_A - \delta\one_{[N]}$. From the above computations we deduce that
$$\eqalign{
\ex_{n\in [N]} &\ex_{r\in [N^{1/3}]} \ex_{h\in [N^{1/100}]}
f(n) f\bigl( n+(r+h)^2\bigr) \cr
&=
\ex_{n\in [N]} \ex_{r\in [N^{1/3}]} \ex_{h\in [N^{1/100}]}\cr
&\qquad\qquad\Bigl(\one_A(n)\one_A\bigl(n+(r+h)^2\bigr)
-\delta\one_A(n)\one_{[N]}\bigl(n+(r+h)^2\bigr) \cr
&\qquad\qquad\qquad\qquad\qquad- \delta\one_A\bigl(n+(r+h)^2\bigr)
+ \delta^2\one\bigl(n+(r+h)^2\bigr)\Bigr) \cr
&= 0 - \delta^2 - \delta^2 +\delta^2 + O(N^{-1/3}) \cr
&= -\delta^2 + O(N^{-1/3}) .\cr
}$$
The triangle inequality then gives
$$ \ex_{n\in [N]} \bigl| f(n)\bigr| \ex_{r\in [N^{1/3}]} \Bigl| \ex_{h\in [N^{1/100}]} f\bigl( n+(r+h)^2\bigr)
\Bigr| \ge \delta^2-O(N^{-1/3}) \gg \delta^2,$$
where here and in the rest of the proof, the Vinogradov notation hides an absolute constant that works
for all $N\in \NN$. We also have
$$\eqalign{
\ex_{n\in [N]} \bigl| f(n)\bigr| ^2
&= \ex_{n\in [N]} \Bigl(\one_A(n) - 2\delta \one_A(n)\one_{[N]}(n) + \delta^2 \one_{[N]}(n)\Bigr)\cr
&= \delta-\delta^2, \cr
}$$
so by the Cauchy--Schwarz inequality,
$$\eqalign{
\delta^4-O(N^{1/3}) &\le
\biggl(\ex_{n\in [N]} \bigl| f(n)\bigr| \ex_{r\in [N^{1/3}]} \Bigl| \ex_{h\in [N^{1/100}]} f\bigl( n+(r+h)^2\bigr)
\Bigr|\biggr)^2 \cr
&\le \ex_{n\in [N]} \bigl| f(n) \bigr|^2
\ex_{n\in [N]} \ex_{r\in [N^{1/3}]} \Bigl| \ex_{h\in N^{1/100}} f\bigl(n+(r+h)^2\bigr)\Bigr|^2\cr
&= (\delta-\delta^2)
\ex_{n\in [N]} \ex_{r\in [N^{1/3}]} \Bigl| \ex_{h\in N^{1/100}} f\bigl(n+(r+h)^2\bigr)\Bigr|^2,\cr
}$$
and therefore
$$
\ex_{n\in [N]} \ex_{r\in [N^{1/3}]} \Bigl| \ex_{h\in N^{1/100}} f\bigl(n+(r+h)^2\bigr)\Bigr|^2
\gg {\delta^4\over \delta-\delta^2} \ge \delta^3.$$
We can rewrite this as
$$
\ex_{h,h'\in N^{1/100}}\ex_{n\in [N]}\ex_{r\in [N^{1/3}]}
f\bigl(n+(r+h)^2\bigr) f\bigl(n+(r+h')^2\bigr) \gg \delta^3.
$$
Replacing $n$ by $n-(r+h)^2$ in the innermost expectation introduces an error term of at most $O(N^{-1/3})$
and deleting the diagonal case $h=h'$ introduces an error term of at most $O(N^{-1/100})$,
so we have
$$\eqalign{
\ex_{h,h'\in N^{1/100}}&\ex_{n\in [N]}\ex_{r\in [N^{1/3}]}
f\bigl(n+(r+h)^2\bigr) f\bigl(n+(r+h')^2\bigr) \cr
&\le
\ex_{h,h'\in N^{1/100}}\Bigl|\ex_{n\in [N]}\ex_{r\in [N^{1/3}]}
f\bigl(n+(r+h)^2\bigr) f\bigl(n+(r+h')^2\bigr) \Bigr|\cr
&\le
\ex_{h,h'\in N^{1/100}}\ex_{n\in [N]}\ex_{r\in [N^{1/3}]}  \cr
&\qquad\qquad f(n) f\bigl(n-(r+h)^2+(r+h')^2\bigr) + O(N^{-1/3})\cr
&= \ex_{h,h'\in N^{1/100}}\ex_{n\in [N]}\ex_{r\in [N^{1/3}]}  \cr
&\qquad\qquad f(n) f\bigl(n- (h'-h)(2r+h'-h)\bigr) + O(N^{-1/3})\cr
&= \ex_{h\ne h'\in N^{1/100}}\ex_{n\in [N]}\ex_{r\in [N^{1/3}]} \cr
&\qquad\qquad f(n) f\bigl(n- (h'-h)(2r+h'-h)\bigr) + O(N^{-1/100}),\cr
}$$
whence
$$\ex_{h\ne h'\in N^{1/100}}\Bigl|\ex_{n\in [N]}\ex_{r\in [N^{1/3}]}
f(n) f\bigl(n- (h'-h)(2r+h'-h)\bigr) \Bigr| \gg \delta^3 $$
by the triangle inequality. From this we see that
there are distinct $h$ and $h'$ in $[N^{1/100}]$ with
$$\Bigl|\ex_{n\in [N]}\ex_{r\in [N^{1/3}]}
f(n) f\bigl(n- (h'-h)(2r+h'-h)\bigr)\Bigr| \gg \delta^3$$
Here another application of the triangle inequality and the fact that $\bigl|f(n)\bigr| \le 1$ together
give
$$\ex_{n\in [N]} \Bigl|\ex_{r\in [N^{1/3}]}
f\bigl(n- (h'-h)(2r+h'-h)\bigr)\Bigr|\gg \delta^3.$$
Now we will perform another shift on $n$, this time by $(h'-h)(h'+h) = O(N^{1/50})$. In the expectation
over $n$, this produces an error term of at most $O(N^{-49/50})$, which we can hide behind the Vinogradov notation
as well. Letting $d = 2(h'-h)$, we have
$$\eqalign{
\ex_{n\in [N]} \Bigl| &\ex_{r\in [N^{1/3}]} f(n+dr)\Bigr| \cr
&= \ex_{n\in [N]} \Bigl| \ex_{r\in [N^{1/3}]} f\bigl(n+2(h'-h)r\bigr)\Bigr| \cr
&\ge \ex_{n\in [N]} \Bigl| \ex_{r\in [N^{1/3}]} f\bigl(n+(h'-h)(2r+h+h')\bigr)\Bigr| - O(N^{-49/50}) \cr
&\gg \delta^3 \cr
}$$
But recall that we have
$$\sum_{n\in [N]} f(n) = 0,$$
so since $d \le 2N^{1/100}$ and $r\le N^{1/3}$, we have
$$\sum_{n\in [N]} f(n+dr) = O(N^{1/3+ 1/100})$$
for all $r\in [N^{1/3}]$, and hence
$$\ex_{n\in [N]} \ex_{r\in [N^{1/3}]} f(n+dr) = O(N^{-2/3+ 1/100}).$$
Then since $\bigl(x+|x|\bigr)/2 = \max(x,0)$ for all $x\in \RR$, we have
$$ \eqalign{
\ex_{n\in [N]} \max\Bigl( \ex_{r\in [N^{1/3}]} &f(n+dr), 0\Bigr) \cr
&\gg {\delta^3 + O(N^{-2/3+1/100})\over 2} \cr
&\gg \delta^3.\cr
}$$
By artificially zeroing out all terms corresponding to the largest $2N^{1/3+1/100}$ elements of the interval $[N]$,
we introduce an error
term of $N^{-2/3+1/100}$, which disappears behind the asymptotic notation.
Then by the pigeonhole principle we may find $n\in [N-2N^{1/3+1/100}]$ such that
$$\ex_{r\in [N^{1/3}]} f(n+dr) \gg \delta^3.$$
Let $P\subseteq [N]$ be the arithmetic progression $\bigl\{n + dr : r\in [N^{1/3}]\bigr\}$ and let
$c$ be the absolute constant implied by the Vinogradov notation. We have
$$\ex_{x\in P} f(x) \ge c\delta ^3,$$
and can unravel the definition of $f$ to obtain
$${|A\cap P|\over |P|} \ge \delta + c\delta^3,$$
which is what we wanted.\slug

Recall that at this analogous point in the proof of Roth's theorem, we were able to shift and
rescale our progression to become an interval, and then iterate the argument until the density
increased above $1$. Unfortunately,
now we can't simply scale the progression $P$ to transform it into a {\it bona fide} interval, since then
square differences between elements in $A$ may no longer be square and vice versa. For
this reason, we will need another lemma.

\edef\lemfurstenbergsarkozy{\the\thmcount}
\proclaim Lemma~\advthm.
Suppose that there is $N\in \NN$ and a square-difference-free
subset $A\subseteq [N]$ of density $\delta = |A|/N$. Then there is a square-difference-free
subset $A'\subseteq [N^{1/4}]$
of size $|A'| \ge (\delta + c'\delta^3) N^{1/4}$, where $c$ is an absolute constant.

\proof By the previous lemma, there is a progression $P\subseteq [N]$ of length $\lfloor N^{1/3}\rfloor$
such that $|A\cap P|/|P| \ge \delta + c\delta^3$ for an absolute
constant $c$. We can partition this progression into subprogressions of length $\lfloor N^{1/4}\rfloor$
and step size $d^2$, with a little bit left over at the end that is of size $O(N^{1/4})$. The density
of $A$ on the union of the progressions is at least $\delta + c\delta^3 - O(N^{-1/12})$,
which is at least $\delta + c'\delta^3$ for some new absolute constant $c'$. By the pigeonhole principle,
$A$ has density at least $\delta + c'\delta^3$ on one of the subprogressions, call it
$$P' = \bigl\{ n' + d^2r' : r'\in [N^{1/4}]\bigr\}.$$
Now consider the set $A' = \bigl\{ r'\in [N^{1/4}] : n' + d^2 r'\in A\bigr\}$. We claim that
$A'$ does not contain two elements that differ by a square. For if $A'$ did contain an integer $m$ and $m+s^2$
for some $s\ne 0$, then we would have $n' + d^2m$ and $m'+d^2(m+s^2)$ in
$A$, but these two elements differ by $(ds)^2$, and we know that $A$ does not contain two elements
that differ by a square. This completes the proof.\slug

We are now ready to state and prove a version of the Furstenberg--S\'ark\"ozy theorem.

\parenproclaim Theorem~{\advthm} (Furstenberg, {\rm 1977;} S\'ark\"ozy, {\rm 1978}).
There is an absolute constant $C$ such that the following holds.
For all $\delta > 0$ there exists $N \le \bigl\lceil \exp\bigl(\exp(C/\delta^2)\bigr)\bigr\rceil$ such that
any subset $A\subseteq [N]$
with $|A|\ge \delta N$ contains two elements whose difference is a positive square integer.

\proof If the theorem does not hold, then there is a square-difference-free subset $A\subseteq [N]$ of density
at least $\delta$. Then we can repeatedly apply Lemma~{\lemfurstenbergsarkozy}, at each iteration replacing
$\delta$ by $\delta + c'\delta^3$, where $c'$ is an absolute constant, and replacing $N$ by
$\lfloor N^{1/4}\rfloor$.
After $1/(c\delta^2)$ iterations of the increment $\delta\mapsto \delta + c\delta^3$,
the density at least doubles. Then to double it again we need to iterate $1/\bigl((c(2\delta)^2\bigr)
= 1/(4c\delta^2)$ times, and so on. In general, after iterating at most
$$ {1\over c\delta^2} \biggl( 1 + {1\over 4} + {1\over 16} + \cdots\biggr)
= {4\over 3c\delta^2}$$
times, the density of the set $A$ will exceed $1$, a contradiction.

To avoid any messiness arising from $N$ eventually equalling $1$, we will insist that in the final
round of the iteration,
$N\ge 100$. This means that in the very first step we must have $N^{4^{-4/(3c\delta^2)}} \ge 100$.
Taking the logarithm of both sides, we need
$$ 4^{-4/(3c\delta^2)} \log N \ge \log 100,$$
and then taking the logarithm again makes this equivalent to
$$\log\log N \ge {4\over 3c'\delta^2} \log 4 + \log\log 100.$$
There is an absolute constant $C$ such that the condition $\log\log N \ge C/\delta^2$ implies the one above,
so we see that a sufficient condition on $N$ for this whole argument to work is
$$ N \ge \exp\bigl(\exp(C/\delta^2)\bigr).\noskipslug$$

What we just proved can equivalently be stated as follows: {\sl There exists a constant $C'$
such that for all $N$, any subset of $[N]$ with density at least $C'/\sqrt{\log\log N}$ contains
two elements that differ by a nonzero square.}

\section References



\parskip=0pt
\hyphenpenalty=-1000 \pretolerance=-1 \tolerance=1000
\doublehyphendemerits=-100000 \finalhyphendemerits=-100000
\frenchspacing
\def\bref#1{[#1]}
\def\beginref{\noindent}
\def\endref{\medskip}
\vskip\parskip

\beginref
Antal Balog
and Endre Szemer\'edi,
``A statistical theorem of set addition,''
{\sl Combinatorica}\/
{\bf 14}
(1994),
263--268.
\endref
\beginref
William Timothy Gowers,
``A new proof of Szemer\'edi's theorem for arithmetic progressions of length four,''
{\sl Geometric and Functional Analysis}\/
{\bf 8}
(1998),
529--551.
\endref
\beginref
Askold Khovanskii,
``Newton polyhedron, Hilbert polynomial, and sums of finite sets,''
{\sl Functional Analysis and its Applications}\/
{\bf 26}
(1992),
276--281.
\endref
\beginref
Melvyn Bernard Nathanson
and Imre Zolt\'an Ruzsa,
``Polynomial growth of sumsets in abelian semigroups,''
{\sl Journal de Th\'eorie des Nombres de Bordeaux}\/
{\bf 14}
(2002),
553--560.
\endref
\beginref
Giorgis Petridis,
``New proofs of Pl\"unnecke-type estimates for product sets in groups.,''
{\sl Combinatorica}\/
{\bf 32}
(2012),
721--733.
\endref
\beginref
Helmut Pl\"unnecke,
{\sl Eigenschaften und Absch\"atzungen von Wirkungsfunktionen}
(Bonn:
Berichte der Gesellschaft f\"ur Mathematik und Datenverarbeitung,
1969).
\endref
\beginref
Klaus Friedrich Roth,
``On certain sets of integers,''
{\sl Journal of the London Mathematical Society}\/
{\bf 28}
(1953),
104--109.
\endref
\beginref
Imre Zolt\'an Ruzsa,
``An application of graph theory to additive number theory,''
{\sl Scientia, Series A}\/
{\bf 3}
(1989),
97--109.
\endref
\beginref
Imre Zolt\'an Ruzsa, ``Sums of finite sets,'' {\sl Number Theory: New York Seminar 1991--1995} (1996).
\endref
\beginref
Endre Szemer\'edi,
``On sets of integers containing no $k$ elements in arithmetic progression,''
{\sl Acta Arithmetica}\/
{\bf 27}
(1975),
199--245.
\endref
\beginref
Terence Tao
and Van Ha Vu,
{\sl Additive Combinatorics}
(Cambridge:
Cambridge University Press,
2006).
\endref
\beginref
Bartel Leendert van~der~Waerden,
``Beweis einer Baudetschen Vermutung,''
{\sl Nieuw Archief voor Wiskunde}\/
{\bf 15}
(1927),
212--216.
\endref
\beginref
\goodbreak\bye
