\input fontmac
\input mathmac

\input epsf
\widemargins
\bookheader{\tenpoint\sc marcel goh}{\tenpoint\sc math {\oldstyle 240} fall {\oldstyle 2024}}

\def\datestamp#1#2 {\llap{{\oldstyle #1}{\rm.}{\sc #2}\hskip12pt}}
\def\bar{\overline}
\def\symdiff{\mathrel{\triangle}}
\def\To{\Rightarrow}
\def\Gets{\Leftarrow}

\maketitle{MATH 240 Fall 2024}{notes by}{Marcel Goh}{}

\floattext 4.5 \ninebf A note on these notes.
\ninepoint After each class, this document will be updated with the new material that was
just covered. The datestamps in the left margin indicate when the notes from each day start.
Subsections labelled with a $*$ are optional. This document is heavily based on notes by
Jeremy Macdonald, but any errors are likely my own. Please email me if you find any.

\bigskip

\advsect Set theory

\datestamp{29}{viii}
A {\it set} is a collection of distinct objects, called its {\it elements} or its {\it members}. If
$x$ is a member of set $A$, then we write $x\in A$, and if $x$ is not an element of $A$,
then we write $x\notin A$.
Sets can be written by listing out its elements. For example,
$$\bigl\{1,4,7,10,\sqrt{782}\bigr\}$$
and
$$\bigl\{\{1,2\}, \pi, \{4\}\bigr\}$$
are both sets (the second example shows that sets can themselves contain other sets). The order of the elements
is not important, and duplicate elements are ignored (so $\{1,2\} = \{2,1\} = \{1,1,2\}$).
The notation using $\{$ and $\}$
is useful for defining small, concrete examples, but expressing large sets can become
very cumbersome. The first way one can describe larger sets is to use the $\ldots$ symbol and the power of
suggestion. For instance, anyone faced with the notation
$$A = \{1,3,5,7,9,\ldots\}$$
can quickly guess that this set is supposed to contain all the positive odd integers. We can also use $\ldots$
to define finite sets. Most Canadians will be able to tell you that the set
$$\{\hbox{\rm Alberta}, \hbox{\rm British Columbia}, \ldots, \hbox{\rm Yukon}\}$$
of provinces and territories contains $13$ elements.
But this notation inherently produces some ambiguity. For example, since the sequence of positive palindromic
binary numbers starts $1,3,5,7,9,15,17,21,27,\ldots$,
we are left with some doubt as to whether the set $A$ above should be the set of odd numbers or the set of
palindromic binary numbers.

But there is another, less ambiguous way to define large sets. It is called set-builder notation
and it refers to any construction of the form
$$\bigl\{x \in U : P(x)\bigr\},$$
where $x$ is a variable, $U$ is a set,
and $P$ is a statement about $x$. The resulting set contains {\sl all $x$ such that $P(x)$
holds}. For example, letting $\NN$ denote the set $\{0,1,2,3,\ldots\}$ of counting numbers (more on this later),
to define the set of all odd numbers, we can write
$$A = \{ x\in \NN : \hbox{there exists}\ k\in \NN\ \hbox{such that}\ x = 2k + 1\}.$$
Note that the statement $P(x)$ must contain $x$, but it may also contain other previously defined symbols, as well
as new symbols defined within the statement (such as $k$ in the example above).

\medskip\boldlabel Special sets of numbers.
There are certain infinite sets of numbers that are used so often as to be given special bold notation.
Back in elementary school, you learned to use the counting numbers $0, 1, 2, 3, \ldots$. We already saw this
set in the previous paragraph; in the business,
this set is known as the {\it natural numbers}, because if you go on a nature hike you
can use them to count the number of bluebells, donkeys, etc.~that you see.
(Many mathematicians use the symbol $\NN$ to denote this set without zero. When in doubt,
clarify with the person you're talking to; in this class, $0\in \NN$.)

Sometime towards the start of junior high you were introduced to the concept of natural numbers. The set
$$\ZZ=\{ \ldots, -3, -2, -1, 0, 1, 2, 3, \ldots\}$$
is called the set of {\it integers} or {\it whole numbers}.
(The word {\it integer} just means ``whole'' in Latin; cf.~French {\it entier}. We use the letter $\ZZ$
because of the German word {\it Zahl} meaning ``number.'')

Even before you learned about negative numbers, you probably learned about fractions. They can be defined
in set-builder notation as a collection of ratios of integers, where the denominator is not zero:
$$\QQ = \{ p/q : p, q\in \ZZ, q\ne 0\}.$$
(The nitpicky reader will notice that $p/q$ is not stipulated to be a member of any set here. This is because
a rigorous definition of $\QQ$ involves quotienting out by an equivalence relation, which we don't know how to
do yet.)
This is the set of {\it rational numbers}.
Remember that sets only contain distinct elements, so $2/3$, $4/6$, $6/9$, etc.~are all considered the
same rational number.

Lastly, we have the set $\RR$ of real numbers. Constructing this set using only notions from set theory
and logic is quite the byzantine task and well outside the scope of this course, but you can think of
$\RR$ as the set of decimal numbers with a finite number of digits to the left of the decimal point, and
a possibly infinite number of digits to the right of the decimal point.

A set of numbers near and dear to many mathematicians' hearts is the set of {\it prime} numbers $P$,
defined by
$$P = \bigl\{ p\in \NN : p\ge 2,\ \hbox{and if $p = ab$ then $\{a,b\} = \{1,p\}$}\bigr\}.$$
(You may have met a different definition of prime numbers in the past. Pause a moment and convince
yourself that the statement above defines the set of prime numbers as you know it.)

\medskip\boldlabel Set inclusion.
When we use set builder notation $B = \bigl\{x\in A : P(x)\bigr\}$, every element of $B$ is necessarily
a member of the set $A$ as well, since $B$ is defined to be the set of all $x$ in $A$ satisfying
$P(x)$. This is one way in which we can obtain
a {\it subset} of another set. More generally, we write $B\subseteq A$ if every element of $B$ is also
an element of $A$, and $B\supseteq A$ if every element of $A$ is an element of $B$. Sometimes, if $B\supseteq A$,
we say that $B$ {\it contains} $A$, or $B$ {\it includes} $A$. For example, we have the chain
$$\NN\subseteq \ZZ \subseteq \QQ\subseteq \RR$$
for the special sets of numbers defined earlier.

Symbols like $\subseteq$, $\supseteq$, and $=$ (that are
used to produce statements) can be
negated with a slash; for example, if there is some element of $B$ that is not an element of $A$,
then we write $B\not\subseteq A$.

The concept of set inclusion is important, because
the most common way to prove that two sets $A$ and $B$ are equal is to show that $A$ is a subset of $B$,
then show that $B$ is a subset of $A$. We illustrate this with the following example.

\proclaim Proposition \advthm. The sets
$$A = \{ x\in \ZZ : \hbox{there exists}\ k\in \ZZ\ \hbox{such that}\ x = 2k + 1\}$$
and
$$B = \{ x\in \ZZ : \hbox{there exists}\ l\in \ZZ\ \hbox{such that}\ x = 2l + 5\}$$
are equal. (Both are different ways of expressing the set of all odd integers.)

\proof Let $x\in A$. Then there exists $k\in\ZZ$ such that $x = 2k+1$. Letting $l = k-2$, we find that
$l$ is an integer (since $k$ was). Furthermore,
$$x = 2k+1 = 2k - 4 + 5 = 2(k-2)+5 = 2l+5.$$
We have found $l$ such that $x = 2l+5$, so $x\in B$. This shows that $A\subseteq B$.

On the other hand, let $x\in B$, so that there exists $l\in \ZZ$ such that $x = 2l+5$. Now we let $k = l+2$;
$k\in \ZZ$ since $l\in \ZZ$. We have
$$x = 2l+5 = 2l + 4 + 1 = 2(l+2) + 1 = 2k+1.$$
This shows that $x\in A$, and we have proved that $B\subseteq A$. This combined with the previous paragraph
shows that $A = B$.\slug

The above result is not so important, but pay attention to the structure of this proof. It is called a ``proof
by double inclusion,'' since we have shown that $A$ includes $B$ and $B$ includes $A$.

\medskip\boldlabel Set operations.
Now we describe a number of operations that may be performed on sets to produce other sets. They can all be
built up from the following two operations.
\medskip
\item{$\bullet$} The {\it union} $A\cup B$ of two sets $A$ and $B$ is the set of all elements that are either in
$A$ or in $B$ (or both).
\smallskip
\item{$\bullet$} The {\it intersection} $A\cap B$ of $A$ and $B$ is the set of all elements that are in both $A$
and $B$.
\medskip
As an example, if $A = \{1,2,4\}$ and $B = \{1,3,5\}$, then $A\cup B = \{1,2,3,4,5\}$ and $A\cap B = \{1\}$.

To avoid logical difficulties, we always assume that the sets we're working with are a subset of some
larger ambient set $U$, often called the {\it universe}.
Once we know what $U$ is, we may define the {\it complement}
of a set $A$ to be the set $\bar A$ of all the elements in $U$ except those that are in $A$.
So if $U = \{1,2,3,4,5\}$ in the example above, then $\bar A = \{3,5\}$ and $\bar B = \{2,4\}$.
What about $\bar{A\cup B}$? Well since $A\cup B$ is all of $U$, its complement must be empty, and we can denote
it $\{\}$. This is one valid notation for the {\it empty set}. The other is $\emptyset$.

Now is a good time to define the cardinality $|A|$ of a set $A$.
This is the number of elements in it, so $|A| = |B| = 3$
in our example, and $|A\cup B| = 5$, etc. We have $|\emptyset| = 0$, and it is possible for the cardinality of
a set to be infinity; for example, $|\NN| = \infty$. We also have $|\RR| = \infty$, but this infinity
is, in some sense, larger than $|\NN|$. (More on that later.)

Next, we define the {\it difference} $B\setminus A$ (sometimes $B-A$) of two sets.
This is the set of all elements
in $B$ that are {\it not} in $A$. So, using the complement notation we just learned about,
we can express $B\setminus A = B \cap \bar A$. It is not necessary that $A$ be a subset of $B$. In
the small example above, we have $A\setminus B = \{2,4\}$ and $B\setminus A = \{3,5\}$.

Lastly, we define the {\it symmetric difference} $A \symdiff B$ of two sets $A$ and $B$ to be the set of all
elements that are either in $A$ or in $B$ {\it but not both}.
Invoking the above example one last time, we
have $A\symdiff B = \{2,3,4,5\}$.
We can express it as using unions, intersections,
and complements as
\edef\eqsymdiffdef{\the\eqcount}
$$A \symdiff B = (A\cup B) \cap \bar{A \cap B}.\adveq$$
To practise using all the different operations we just learned,
convince yourself that the following are three more valid ways to express the symmetric difference:
$$A\symdiff B = (A\cup B)\setminus (A\cap B) = (A\setminus B) \cup (B\setminus A) =
\bar{\bar{A\cup B} \cup (A\cap B)}$$
More set identities abound. We state the following proposition without proof; you should try going
though this list and convincing yourself that each identity holds, for all sets $A$, $B$, and $C$.
(This is a great way of practising proofs by double inclusion.)

\edef\propsetlaws{\the\thmcount}
\proclaim Proposition \advthm. Let $A$, $B$, and $C$ be subsets of a universe $U$. Then
\medskip
\item{i)} $A\cap U = A$ and $A\cup \emptyset = A$;
\smallskip
\item{ii)} $A\cup U = U$ and $A\cap \emptyset = \emptyset$;
\smallskip
\item{iii)} $A\cup A = A$ and $A\cap A = A$;
\smallskip
\item{iv)} $\bar{\bar A} = A$;
\smallskip
\item{v)} $A\cup B = B\cup A$ and $A\cap B = B\cap A$;
\smallskip
\item{vi)} $A\cup (B\cup C) = (A\cup B) \cup C$ and $A\cap (B\cap C) = (A\cap B)\cap C)$;
\smallskip
\item{vii)} $A\cup (B\cap C) = (A\cup B) \cap (A\cup C)$ and $A\cap (B\cup C) = (A\cap B)\cup (A\cap C)$;
\smallskip
\item{viii)} $A\cup (A\cap B) = A$ and $A\cap (A\cup B) = A$; and
\smallskip
\item{ix)} $A\cup \bar A = U$ and $A\cap \bar A = \emptyset$.\slug
\medskip

These laws have names, some of which we will use more often than others:
(i) is called the identity law, (ii) the domination law, (iii) the idempotent law,
(iv) the law of double negation, (v) the commutative law, (vi) the associative law,
(vii) the distributive law, (viii) the absorption law, and (ix) the complement law.

\medskip\boldlabel \llap{*}Analogy with addition and multiplication.
Some of these laws bear some resemblance to laws about numbers that you already know. As an exercise,
replace $\cup$ with $+$ (addition), $\cap$ with $\cdot$ (multiplication), $\bar A$ with $-A$ (negation),
$U$ with $1$, and $\emptyset$ with $0$ in all the formulas
above, and now assume that $A$, $B$, and $C$ are arbitrary real numbers. Which identities still hold in
the number setting, and which ones don't? As a more advanced exercise, try replacing $\cup$
with $\symdiff$ in the identities above (some statements will have to be tweaked a bit so that they're
actually true, some won't). Now do the same
replacement as before, except replace $\symdiff$ with $+$. You will find that many more identities
carry over.

\medskip\boldlabel\datestamp{03}{ix} De Morgan's laws.
There are two important laws relating complements with union and intersection.
We shall state them as a proposition, this time giving a proof (of one of them).

\parenproclaim Proposition {\advthm} (De Morgan's laws). Let $A$ and $B$ be sets. Then
\medskip
\item{i)} $\bar{A\cup B} = \bar A \cap \bar B$; and
\smallskip
\item{ii)} $\bar{A\cap B} = \bar A \cup \bar B$.
\medskip

\proof
Let $x\in \bar{A\cup B}$. This means that $x$ does not belong to the union of $A$ and $B$,
$x$ cannot be in $A$, nor can it be in $B$. Since $x\notin A$, $x\in \bar A$, and since $x\notin B$,
$x\in \bar B$. Therefore, $x\in \bar A\cap \bar B$. This shows that $\bar{A\cup B}\subseteq \bar A\cap \bar B$.

Now assume that $x\in \bar A\cap \bar B$. So $x\in \bar A$ and $x\in \bar B$, meaning that $x\notin A$
and $x\notin B$. Since $x$ is in neither $A$ nor $B$, it is also not a member of the union $A\cup B$.
We conclude that $x\in \bar{A\cup B}$. We have shown that $\bar A\cap \bar B \subseteq \bar{A\cup B}$,
which fact, combined with the previous paragraph, completes the proof of (i).

The proof of (ii) is similar and left to the reader as an exercise.\slug

Armed with all of these laws, we are able to perform lots of mechanical set manipulations to simplify
expressions. For example, consider the expression
$$\bigl( (A\setminus B)\cup A\bigr) \cap \bar{A\cap \bar B}.$$
Since $A\setminus B = A\cap \bar B$ and invoking the second De Morgan law on the right of the intersection
yields
$$\bigl( (A\cap \bar B)\cup A\bigr) \cap (\bar A \cup B).$$
Now, we can use absorption on the left-hand side to obtain
$$A \cap (\bar A \cup B),$$
and then distributing gives us
$$(A\cap \bar A) \cup (A\cap B) = \emptyset \cup (A\cap B) = A\cap B.$$
We thus see that the nasty expression $\bigl( (A\setminus B)\cup A\bigr) \cap \bar{A\cap \bar B}$ is
simply another way of writing $A\cap B$.

\medskip\boldlabel The Cartesian product and power set. From the real line $\RR$, we can geometrically construct
the Cartesian plane $\RR^2$ by lining up parallel copies of $\RR$, one for each element of the original line
and all parallel to the original line.
Notationally, $\RR^2$ is the set of all ordered pairs $(a,b)$, where $a,b\in \RR$. Generalising this,
for any sets $A$ and $B$ we can define the {\it Cartesian product} $A\times B$
to be the set
$$A \times B = \bigl\{ (a,b) : a\in A, b\in B\bigr\}.$$
We sometimes write $A^2$ for $A\times A$, and more generally $A^n$ for the $n$-fold Cartesian product
of $A$ with itself. (This explains the notation $\RR^2$ for the Cartesian plane, and $\RR^n$ for the
$n$-dimensional vector space over $\RR$.)
Note that $A\times B$ is not equal to $B\times A$ in general.

\proclaim Proposition \advthm. If $A$ and $B$ are finite sets, then $|A\times B| = |A|\cdot |B|$.

\proof The set $A\times B$ consists of all ordered pairs $(a,b)$ where $a\in A$ and $b\in B$. There are
$|A|$ choices for $a$, and for $a$, there are $|B|$ ways to pair it with a $b$ from $B$. So there
are $|A|\cdot |B|$ pairs in total.\slug

Now we define the {\it power set}. For a set $A$, this is the set of all subsets of $A$, and is
commonly denoted by ${\cal P}(A)$ or $2^A$. (We will use the latter notation in these notes.)
In set-builder notation, we have
$$2^A = \{X\subseteq U : X\subseteq A\}.$$
As an example, if $A = \{1,2,3\}$, then
$$2^A = \bigl\{\emptyset, \{1\}, \{2\}, \{3\}, \{1,2\}, \{1,3\}, \{2,3\}, \{1,2,3\}\bigr\}.$$
Note that even though, say, $1\in A$, we do not have $1\in 2^A$. We do, however, have $\{1\}\in 2^A$.

Another example is $2^\ZZ$, the set of all subsets of integers. If $P$ is the set of primes, then
$P\in 2^\ZZ$. Also in $2^\ZZ$ is the set $S = \{n^2 : n\in \ZZ\}$ of square numbers.

There is a way of encoding subsets with strings of $0$s and $1$s. Suppose we have a set
$$\{-3, 1, 7, 19, 23\}.$$
Fixing this order of the elements, for any arbitrary subset of this set, we can associate to it a binary string.
Consider the subset $\{1, 19, 23\}$. This corresponds to the binary string $01011$: since the first element,
$3$, is not in the subset, we write a $0$. Then $1$ is in the subset, so we write a $1$, and so on.
This is a reversible process. Given a binary string of length $5$, say, $00101$, we can reconstruct the
subset that corresponds to it. The first two $0$s mean that $-3$ and $1$ do not belong to the set,
but $7$ does, $19$ doesn't, and $23$ does. So the subset is $\{7, 23\}$. In this way we see that there
is a one-to-one correspondence between the elements of $2^A$ and binary strings of length $|A|$.
We'll use this fact in the proof of the following proposition.

\edef\propcardpowerset{\the\thmcount}
\proclaim Proposition \advthm. If $A$ is finite, then
$|2^A| = 2^{|A|}$.

\proof We just saw that there is a one-to-one correspondence between elements of $2^A$ and binary strings
of length $|A|$. So it suffices to count the number of binary strings of length $|A|$. Well, each digit
can be either $0$ or $1$, and there are $|A|$ digits, so the number of strings is
$$\underbrace{2\cdot 2\cdot \cdots \cdot 2}_{|A|\ \hbox{\sevenrm times}} = 2^{|A|}.\noskipslug$$

\medskip\boldlabel Counterexamples in proofs.
We finish this subsection with a little example problem. {\sl Is it true that $2^A \cup 2^B = 2^{A\cup B}$
for all sets $A$ and $B$?}

Let's start by trying to prove the statement is true. As usual, we will attempt a double-inclusion proof.
Let $2^A\cup 2^B$. This means $X$ is either a subset of $A$ or it is a subset of $B$. Either way,
$X$ is a subset of $A\cup B$, so $X\in 2^{A\cup B}$. So far so good; we have proved that
$2^A\cup 2^B \subseteq 2^{A\cup B}$.

Now we try the other direction. Let $X\in 2^{A\cup B}$. So $X$ is a subset of $A\cup B$. From here
we want to say that $X$ must be a subset of $A$ or it must be a subset of $B$, but is that necessarily true?
It is possible that $X$ is contained slightly in $A$ and slightly in $B$. So we have failed
to prove that $2^{A\cup B} \subseteq 2^A \cup 2^B$ in general. But just because we have failed to prove
that something is true doesn't mean we have proved it is false!

To actually prove that $2^{A\cup B} \subseteq 2^A \cup 2^B$ doesn't hold in general, we need to find
a {\it counterexample}. That is, we need to construct sets $A$ and $B$ such that the statement is false.
In this case, we can let $A = \{1,2\}$, $B = \{3,4\}$, so that $A\cup B = \{1,2,3,4\}$.
Then the set $\{1,3\}$ is a subset of $A\cup B$ but is not a subset of $A$ and it is not a subset of $B$.
In other words, $\{1,3\}\in 2^{A\cup B}$ but $\{1,3\} \notin 2^A\cup 2^B$, proving that
$2^{A\cup B} \not\subseteq 2^A \cup 2^B$.
\bye

\medskip\boldlabel\datestamp{3}{} Russell's paradox. Earlier, we said that the sets we are working with
need to be a subset of a universe $U$, which has already been proven to be a set. We gave lots of
ways to make sets out of new sets, such as the union and intersection operations, etc. Starting with
the empty set $\emptyset$ is a set, it is possible to define the set of natural numbers as follows.
We can define $0 = \emptyset$, $1 = \{0\}$, and $2 = \{0,1\}$, and so on.
Now we take the set of all of these, and call this $\NN$.
(We can also define addition and multiplication on these set-theoretic ``numbers''
so that they behave like addition and multiplication do on $\NN$.) From here we can do more funky stuff
to define $\ZZ$, $\QQ$, and $\RR$, and prove that these are all sets (you can see this in a higher-level
course on set theory, if you're interested). So there isn't much of a problem
with all the sets we have played with so far; they are all subsets of things that are already known to be sets.

Ungodly things can happen if we don't stick by these rules. An example, due to Bertrand Russell, is the ``set''
$$R = \{\hbox{sets}\ X : X \notin X \}.$$
In plain English, $R$ is defined to be the set of all sets that contain themselves.
This is not a subset of any known thing, so by our criterion above we would not consider it a set.
But supposing it is, let us ask ourselves the following question. Does $R$ contain itself?
If it does not, then $R\notin R$, so $R$ would be a set that satisfied the condition of $R$, so $R\in R$.
But on the other hand, if $R\in R$, then $R$ violates the condition defining $R$, so $R\notin R$.
Round and round we go in a circle of contradiction.

Such is the price of meddling with ``sets'' that aren't subsets of known sets. This also shows that there
is no such thing as ``the set of all sets.''

\advsect Propositional logic

A {\it proposition} is a statement that is true or false. For example ``$8$ is even'' is a statement
we know to be true, and ``$8$ is prime'' is a statement we know to be false. The statement
``$n$ is prime'' is not a proposition because its truth or falsity depends on what $n$ is.
The statement
``$2^{2^{240}} - 1$ is prime'' is a proposition, because it is either true or false (even though you or I
might not know which one it is).

A {\it propositional variable} or a {\it boolean variable} is a variable which can take either the
value $0$ or $1$, where $0$ means ``false'' and $1$ means ``true.'' Usually we use letters $p$, $q$,
and $r$ to denote propositional variables. The simplest logical operator is negation, defined
by the table
$$
\vbox{
\tabskip=1em plus.2em minus .2em
\halign{
\hfil$#$\hfil & \hfil$#$\hfil \cr
p & \neg p \cr
\noalign{\medskip}
\noalign{\hrule}
\noalign{\medskip}
0 & 1 \cr
1 & 0 \cr
}}.
$$
This is also called the {\mc NOT} operator, since if $p$ is true, then $\neg p$
is false, and vice versa.
Next is {\it conjunction}, which has the table
$$
\vbox{
\tabskip=1em plus.2em minus .2em
\halign{
\hfil$#$\hfil & \hfil$#$\hfil & \hfil$#$\hfil \cr
p & q & p\wedge q \cr
\noalign{\medskip}
\noalign{\hrule}
\noalign{\medskip}
0 & 0 & 0 \cr
0 & 1 & 0 \cr
1 & 0 & 0 \cr
1 & 1 & 1 \cr
}}.
$$
This is also called the {\mc AND} operator, because $p\wedge q$ is true if and only if $p$ and
$q$ are both true. The {\mc OR} operator, also called {\it disjunction}, has the table
$$
\vbox{
\tabskip=1em plus.2em minus .2em
\halign{
\hfil$#$\hfil & \hfil$#$\hfil & \hfil$#$\hfil \cr
p & q & p\vee q \cr
\noalign{\medskip}
\noalign{\hrule}
\noalign{\medskip}
0 & 0 & 0 \cr
0 & 1 & 1 \cr
1 & 0 & 1 \cr
1 & 1 & 1 \cr
}}.
$$
We see that $p\vee q$ is true if $p$ or $q$ is true (or both). The symbol $\vee$ is meant to recall
the Latin word {\it vel}, meaning ``or.'' (One of the most important early treatises on mathematical
logic and set theory was {\sl Arithmetices principia, nova methodo exposita}, published in Latin in 1889
by Giuseppe Peano. It established the now-standard axiomatisation of the natural numbers.)

On the other hand, in English, we often use the word ``or''
to mean an {\it exclusive} or; that is, either $p$ and $q$ are true but not both. In mathematics, on the
other hand, ``or'' is usually {\it inclusive}, so both $p$ and $q$ are allowed to hold at the same time.
It is possible to express the exclusive disjunction (often called {\mc XOR})
by a table, however. We will use the symbol $\oplus$ for this operator, and its table looks like this:
$$
\vbox{
\tabskip=1em plus.2em minus .2em
\halign{
\hfil$#$\hfil & \hfil$#$\hfil & \hfil$#$\hfil \cr
p & q & p\oplus q \cr
\noalign{\medskip}
\noalign{\hrule}
\noalign{\medskip}
0 & 0 & 0 \cr
0 & 1 & 1 \cr
1 & 0 & 1 \cr
1 & 1 & 0 \cr
}}
$$
So $p\oplus q$ is true if $p$ is true or $q$ is true, but not both.

A {\it formula} is an expression containing propositional variables, $0$, $1$, logical operators, and
parentheses. The formula must syntactically make sense; for instance, $0 (\vee \wedge q \neg $ is
not a formula. Just as in ordinary mathematical notation,
parentheses are used to clarify which operators should be evaluated first. We will assume that negation
applies first, but an expression such as $p\vee q \wedge r$ is ambiguous. (In many programming languages,
conjunction has higher priority than disjunction, but in this class, just add parentheses to clarify.)

Above we have illustrated the basic logical operators
by writing out their {\it truth tables}. These are tables that give the value of
a formula for all possible values of its variables. We can write truth tables for more complex formulas as
well:
$$
\vbox{
\tabskip=1em plus.2em minus .2em
\halign{
\hfil$#$\hfil & \hfil$#$\hfil & \hfil$#$\hfil & \hfil$#$\hfil & \hfil$#$\hfil \cr
p & q & p\wedge q & \neg(p \wedge q) & \neg(p\wedge q) \oplus q \cr
\noalign{\medskip}
\noalign{\hrule}
\noalign{\medskip}
0 & 0 & 0 & 1 & 1 \cr
0 & 1 & 0 & 1 & 0 \cr
1 & 0 & 0 & 1 & 1 \cr
1 & 1 & 1 & 0 & 1 \cr
}}
$$
Strictly speaking, the third and fourth columns are not necessary, but these intermediary columns help us
verify the accuracy of the following ones.

Two formulas $f_1$ and $f_2$ are said to be {\it logically equivalent} if they have the same truth table;
that is, they produce the same output if given the same input. In this case we write $f_1 \equiv f_2$.
For example, let $f_1 = \neg(p\wedge q) \oplus q$, the formula whose truth table is illustrated above.
Now let $f_2 = p \vee \neg q$. Its truth table is
$$
\vbox{
\tabskip=1em plus.2em minus .2em
\halign{
\hfil$#$\hfil & \hfil$#$\hfil & \hfil$#$\hfil & \hfil$#$\hfil \cr
p & q & \neg q & p\vee \neg q \cr
\noalign{\medskip}
\noalign{\hrule}
\noalign{\medskip}
0 & 0 & 1 & 1 \cr
0 & 1 & 0 & 0 \cr
1 & 0 & 1 & 1 \cr
1 & 1 & 0 & 1 \cr
}},
$$
so we conclude that $f_1 \equiv f_2$. As a larger example, suppose we want to find all
values of $p$, $q$, and $r$ such that
$$ f = (p\vee q)\wedge (\neg q \vee \neg r)$$
evaluates to $1$. The truth table
$$
\vbox{
\tabskip=1em plus.2em minus .2em
\halign{
\hfil$#$\hfil & \hfil$#$\hfil & \hfil$#$\hfil & \hfil$#$\hfil &
\hfil$#$\hfil & \hfil$#$\hfil  \cr
p & q & r & p\vee q & \neg q \vee \neg r & f \cr
\noalign{\medskip}
\noalign{\hrule}
\noalign{\medskip}
0 & 0 & 0 & 0 & 1 & 0 \cr
0 & 0 & 1 & 0 & 1 & 0 \cr
0 & 1 & 0 & 1 & 1 & 1 \cr
0 & 1 & 1 & 1 & 0 & 0 \cr
1 & 0 & 0 & 1 & 1 & 1 \cr
1 & 0 & 1 & 1 & 1 & 1 \cr
1 & 1 & 0 & 1 & 1 & 1 \cr
1 & 1 & 1 & 1 & 0 & 0 \cr
}}
$$
shows that $f$ evaluates to $1$ precisely when
$$(p,q,r) \in \bigl\{(0,1,0), (1,0,0), (1,0,1), (1,1,0)\bigr\}.$$
To write out the truth table for a formula with $n$ variables, we need $2^n$ rows, so this
method is unsuitable for formulas with more than three or four variables.

\medskip\boldlabel Simplifying logical formulas. Just as we have rules for simplifying set expressions,
there are ways to turn complicated logical formulas into simpler ones that are logically equivalent.
What might surprise you is that the rules turn out to be exactly the same! To see the basis for this
correspondence, consider the definition of a union of sets $A$ and $B$. In set-builder notation,
this is
$$A \cup B = \{ x\in U : x\in A\ \hbox{or}\ x\in B\}.$$
The ``or'' in the definition suggests that $\cup$ is intimately related to the $\vee$ operation in
propositional logic. Repeating this process, we have the ``dictionary''
$$
\vbox{
\tabskip=3em plus.005em minus .005em
\halign{
\hfil{\ninepoint #}\hfil & \hfil{\ninepoint #}\hfil \cr
Set theory & Propositional logic\cr
\noalign{\medskip}
\noalign{\hrule}
\noalign{\medskip}
sets $A$, $B$ & variables $p$, $q$ \cr
unions $A\cup B$ & disjunctions $p\vee q$ \cr
intersections $A\cap B$ & conjunctions $p\wedge q$ \cr
complements $\bar A$ & negations $\neg p$ \cr
symmetric differences $A\symdiff B$ & exclusive disjunctions $p\oplus q$ \cr
the empty set $\emptyset$ & $0$ \cr
the universe $U$ & $1$ \cr
}}
$$
Exploiting this connection, we have the following analogue of Proposition~{\propsetlaws}.

\proclaim Proposition 2. Let $p$, $q$, and $r$ be propositional variables. Then
\medskip
\item{i)} $p\wedge 1 \equiv p$ and $p\vee 0 \equiv p$;
\smallskip
\item{ii)} $p \vee 1 \equiv 1$ and $p \wedge 0 \equiv 0$;
\smallskip
\item{iii)} $p \vee p \equiv p$ and $p\wedge p \equiv p$;
\smallskip
\item{iv)} $\neg\neg p \equiv p$;
\smallskip
\item{v)} $p \vee q \equiv q \vee p$ and $p\wedge q \equiv q\wedge p $;
\smallskip
\item{vi)} $p\vee (q\vee r) \equiv (p\vee q) \vee r$ and $p\wedge (q\wedge r) \equiv (p\wedge q)\wedge r)$;
\smallskip
\item{vii)} $p\vee (q\vee r) \equiv (p\vee q) \wedge (p\vee r)$ and
$p\wedge (q\vee r) \equiv (p\wedge q)\vee (p\wedge r)$;
\smallskip
\item{viii)} $p\vee (p\wedge q) \equiv p$ and $p\wedge (p\vee q) \equiv p$; and
\smallskip
\item{ix)} $p\cup \bar p \equiv 1$ and $p\wedge \bar p \equiv 0$.\slug
\medskip

Since they are essentially the same as their set equivalents, the names of these laws are the same as in the
realm of sets.
We also have the propositional equivalent of De Morgan's law, which states that
\medskip\begingroup\sl
\item{x)} $\neg(p\vee q) \equiv \neg p \wedge \neg q$ and $\neg(p\wedge q) \equiv \neg p \vee \neg q$.\endgroup%sl
\medskip

Using these rules, we can now show that $\neg(p\wedge q) \oplus q \equiv p\vee\neg q$, which we
already saw earlier from their truth tables. First we observe that
\edef\eqxordef{\the\eqcount}
$$p\oplus q \equiv (p\vee q)\wedge \neg(p\wedge q),\adveq$$
which is analogous to the identity~\refeq{\eqsymdiffdef} for symmetric differences. So
$$\eqalign{
\neg(p\wedge q)\oplus q&\equiv\bigl(\neg(p\wedge q)\vee q\bigr)\wedge\neg\bigl(\neg(p\wedge q)\wedge q\bigr)\cr
&\equiv (\neg p \vee\neg q \vee q) \wedge \bigl((p\wedge q) \vee \neg q\bigr) \cr
&\equiv (\neg p \vee 1) \wedge \bigl((p\vee \neg q) \wedge (q\vee \neg q)\bigr) \cr
&\equiv 1 \wedge \bigl((p\vee \neg q) \wedge 1\bigr) \cr
&\equiv p\vee \neg q, \cr
}$$
where in the first line we use~\refeq{\eqxordef}, in the second line we use De Morgan's law twice, in
the third line we use the complement and distributive laws, the fourth line we use the domination and
complement laws, and in the last line we use the identity law (twice). (It is not super important
to remember the names of all these laws, as long as you remember their statements, but you may actually find
it is easier to remember the names along with the statements, just as it might be easier to remember faces of
people you've met if you also know their names.)

\medskip\boldlabel\datestamp{4}{} Conditional and biconditional. We now examine the {\it conditional}
logical relation {\mc IF} $p$ {\mc THEN} $q$. It is denoted by $p\To q$ and its truth table is given by
$$
\vbox{
\tabskip=1em plus.2em minus .2em
\halign{
\hfil$#$\hfil & \hfil$#$\hfil & \hfil$#$\hfil \cr
p & q & p\To q \cr
\noalign{\medskip}
\noalign{\hrule}
\noalign{\medskip}
0 & 0 & 1 \cr
0 & 1 & 1 \cr
1 & 0 & 0 \cr
1 & 1 & 1 \cr
}}.
$$
Within the conditional statement, $p$ is called the {\it antecedent}; this is the assumption. The
statement that is asserted, conditional that the antecendent holds, is called the {\it consequent} $q$.
You can quickly check that the relation $p\To q$ is equivalent to $\neg p \vee q$. This fact is useful
when performing mechanical simplifications.
In English, the statement ``if $p$ then $q$'' asserts a causal relation between $p$ and $q$.
Take a moment and reconcile this idea with the truth table above.
It is normal to get a little tripped up if it's your first time seeing this table.
In the first row, $q$ doesn't even happen, so it might feel weird to say that $p$
has ``caused'' $q$ to happen in this case, and
in the second row, $p$ is not true
and $q$ is true, so it might seem odd that we have set $p\To q$ to true, because there seems to be no relation
between $p$ and $q$.
But it should make sense if you think
of $p$ as a precondition to a promise $q$, and then considering whether the promise is broken. As an example,
suppose your friend says: ``If it snows tomorrow I'll work in Trottier with you.'' If it doesn't snow
and she doesn't pull up, she hasn't technically broken her promise. If it doesn't
snow and she shows up, then she still hasn't broken her promise. The only way she can break her promise is if
it snows and she doesn't come to Trottier; this situation corresponds to the only $0$-row in the truth table.

The conditional is a very important logical operator to understand, because most theorem statements assume
some hypothesis and claim some conclusion. You will be asked to prove statements of this form, so it is
important to understand the logical nature of the statements to begin with.

The last relation is called the {\it biconditional}, and it asserts that variables $p$ and $q$ are logically
equivalent. That is, $p$ happens if and only if $q$ happens. It's truth table
$$
\vbox{
\tabskip=1em plus.2em minus .2em
\halign{
\hfil$#$\hfil & \hfil$#$\hfil & \hfil$#$\hfil \cr
p & q & p\Leftrightarrow q \cr
\noalign{\medskip}
\noalign{\hrule}
\noalign{\medskip}
0 & 0 & 1 \cr
0 & 1 & 0 \cr
1 & 0 & 0 \cr
1 & 1 & 1 \cr
}}
$$
is pretty self-explanatory; in it, $p\Leftrightarrow q$ is true whenever $p$ and $q$ have the same truth value.
The name ``biconditional'' is suggested by (part of) the following proposition.

\proclaim Proposition {\advthm}. Let $p$ and $q$ be propositional variables. Then
$$\eqalign{
p \Leftrightarrow q &\equiv (p\To q) \wedge (q\To p) \cr
&\equiv \bigl( (\neg p)\vee q\bigr) \wedge \bigl( (\neg q)\vee p\bigr) \cr
&\equiv (p\wedge q) \vee \bigl( (\neg p) \wedge (\neg q)\bigr).\cr
}$$

\proof We leave the first equivalence to the reader (writing out the truth table is one way of proving it).
The second equivalence follows from our earlier observation that $p\To q \equiv \neg p \vee q$, and
the third equivalence follows from the distributive, complement, and identity laws.\slug

A formula $f$ is called a
\medskip
\item{i)} {\it tautology} if $f\equiv 1$, i.e., $f$ always evaluates to true;
\smallskip
\item{ii)} {\it contradiction} if $f\equiv 0$, i.e., $f$ always evaluates to false;
\smallskip
\item{iii)} {\it contingency} if $f$ can evaluate to both $1$ and $0$, depending on the
values of its variables;
\smallskip
\item{iv)} {\it satisfiable} if $f$ evaluates to $1$ for at least one input; and
\smallskip
\item{v)} {\it falsifiable} if $f$ evaluates to $0$ for at least one input.
\medskip
An example of a tautology is $p\vee\neg p$ and an example of a contradiction is $p\wedge \neg p$.
This follows from the complement laws. To say that something is satisfiable is precisely to
say that it is not a contradiction, and to say that something is falsifiable is equivalent to saying
that it is not a tautology. Contingencies are those formulas that are both satisfiable
and falsifiable (in other words, formulas that are neither tautologies nor contradictions).

Suppose we are asked which of the above definitions the formula
$$f \equiv \bigl( p\wedge (p\To q)\bigr) \To q$$
satisfies. This formula only has two variables, so it is easy enough to use a truth table
for this purpose, but we will take the opportunity to practise simplifying the expression
symbolically. First of all, let's change all conditionals of the form $r \To s$,
to disjunctions of the form $\neg r \vee s$. This gives us
$$f\equiv \neg\bigl( p\wedge (\neg p \vee q)\bigr) \vee q.$$
Now we use the distributive law to distribute the conjunction over the innermost disjunction,
obtaining
$$f\equiv \neg\bigl((p\wedge \neg p) \vee (p\wedge q)\bigr) \vee q;$$
by the complement and identity laws in that order, this simplifies to
$$f\equiv \neg(p\wedge q) \vee q.$$
Now De Morgan's law and associativity give
$$f\equiv (\neg p \vee \neg q) \vee q \equiv \neg p \vee (\neg q \vee q),$$
and thus
$$f\equiv \neg p \vee 1 \equiv 1,$$
by the complement and domination laws in that order. We conclude that $f$ is a tautology,
which also means that it is satisfiable.

The fact that $\bigl( p\wedge (p\To q)\bigr) \To q$ is a tautology symbolically justifies
the argument that of $p$ is true and $p\To q$ is true, then we should be able to conclude $q$.
This form of argument is called {\it modus ponens}, and it dates back to ancient times. You probably
use {\it modus ponens} all the time in everyday life without knowing it, and we will
certainly use it in this class a lot.

\medskip\boldlabel Encoding problems in propositional logic.
Many algorithmic and logical problems can be encoded in propositional logic (and then later
solved by a computer program). For example, suppose we want to play $4\times 4$ Sudoku. In this
game, we have a $4\times 4$ grid and we want to fill it with the numbers $1$ through $4$ such that
\medskip
\item{i)} every row contains $1$ through $4$;
\smallskip
\item{ii)} every column contains $1$ through $4$; and
\smallskip
\item{iii)} the four subsquares each contain $1$ through $4$.
\medskip
In a given instance of the game, some cells are already filled in. The puzzle is: {\sl Is there a
solution and if so, what is it?}
\midinsert
$$\epsfbox{sudoku.ps}$$
\vskip5pt
\caption{An example $4\times 4$ Sudoku game.}
\endinsert
\goodbreak
To represent a Sudoku game in propositional logic, we can define boolean variables
$p_{i,j,k}$, where $i$, $j$, and $k$ range over $\{1,2,3,4\}$. (So there are $4^3$ variables
in total.) We shall set
$$ p_{i,j,k} = \cases{1, & if the number $k$ is in row $i$ and column $j$;\cr 0, & otherwise.}$$
We'll  number the rows increasing from the top and the columns increasing
from left to right.
For example, in Fig.~1 there is a $2$ in row $1$ and column $3$, so $p_{1,3,2} = 1$.

Now we set to work encoding the conditions of a Sudoku grid in propositional logic:
\medskip
\item{i)} To stipulate that every row contain $1$ through $4$, we first define auxiliary variables
$$r_{i,k} = p_{i,1,k} \vee p_{i,2,k} \vee p_{i,3,k} \vee p_{i,3,k},$$
for $i,k\in \{1,2,3,4\}$.
With these helper variables, we now see that
$$ r_{1,1} \wedge r_{1,2} \wedge r_{1,3} \wedge r_{1,4} $$
encodes the requirement that row $1$ contains one of each number.
We do the same for rows $2$, $3$, and $4$ as well, and then combine with {\mc AND}.
\smallskip
\item{ii)} We do a similar thing as in part (i) for each of the four columns.
\smallskip
\item{iii)} Ditto for subsquares.
\smallskip
\item{iv)} We need to set the initial values of the grid. For the grid in Fig.~1, we have
the formula
$$p_{1,3,2} \wedge p_{3,1,1} \wedge p_{3,2,3}.$$
\smallskip
\item{v)} Lastly, we need to make sure that there is not more than one number per cell. To do
this, for each cell $(i,j)\in \{1,2,3,4\}^2$ we write
$$ p_{i,j,1} \To (\neg p_{i,j,2} \wedge \neg p_{i,j,3} \wedge \neg p_{i,j,4}),$$
and so on (four conditionals in total). Of course we'll need to {\mc AND} all these together.
\medskip
Now we combine the formulas from each of these five steps into one long formula $f$ such that
$f$ is satisfiable if and only if the grid has a solution, and the values for $p_{i,j,k}$ give
a solution.

Defining all of these variables was a rather arduous and cumbersome process, and not entirely worth
it for a $4\times 4$ game of Sudoku (which can just be solved by eyeballing the grid).
But one could imagine writing a general computer program to encode larger and larger grids.
In fact, there are lots of problems that can be {\it reduced} to the problem of
determining if a boolean formula is satisfiable. This means that if we have a program
capable of taking a formula $f$ as input and spitting out whether or not it is satisfiable (in a
reasonable amount of time), then there are lots of real-world problems that this program
could be applied to.

This problem is called the {\it boolean satisfiability problem}, often abbreviated {\mc SAT}. One
way of solving it for any given $f$ is to just compute its truth table. We already know the
downside of this approach: if $f$ has $n$ variables, then its truth table will have $2^n$ rows.
Given a few minutes, you are certainly capable of writing down a formula $f$ that has $300$
variables, call them $p_1, \ldots, p_{300}$. The truth table of $f$ will have $2^{300}$ rows,
which is more than the number of atoms in the observable universe. You can learn a lot more
about {\mc SAT} in a higher-level class on computational complexity (e.g., {\mc COMP 360/362}).

\advsect Predicate logic

\datestamp{5}{}
Propositional logic allows us to work with simple declarations, but this isn't powerful enough
to express some deeper mathematical concepts. For this purpose, we now introduce the notion of
a {\it predicate}. This is a statement involving some number of variables, each of which may take
values coming from a universe $U$, such that the statement evaluates to either true or false {\sl once
all variables are assigned values}. The statement $P(n)$ given by ``$n$ is prime''
is an example of this, where $n$
can take any value in the universe $\ZZ$. An example with two variables is the
predicate $L(x,y)$ defined by ``$x$ is less than $y$.''
(A more commonly-used notation for this predicate is ``$x<y$.'')

Predicates contain variables, but at the moment we don't have any way of introducing new variables into
a statement. This is done using two different {\it quantifiers}.
The first is the
{\it universal quantifier}, denoted $\forall$ and meaning ``for all.'' The statement $\forall n:P(n)$
is true if and only if $P(n)$ is true for every possible value that $n$ can take. The second quantifier
is the {\it existential quantifier}, written ``$\exists$'' and with the meaning ``there exists.''
The statement $\exists n:P(n)$ is true if and only if there is (at least) one value that $n$ can take
such that $P(n)$ is true. The colon doesn't really have any mathematical meaning in these formulas;
they just visually set the quantifiers apart from the predicates that follow.

For instance, taking $P(n)$ to be the statement ``$n$ is prime,'' where the universe $U$ is $\NN$,
the statement $\exists n P(n)$ is true and the statement $\forall n:P(n)$ is false.
What about the statement $\forall x\,\exists y:y< x$? Well, if the universe $U$ is taken to be $\NN$,
then the statement is false, because setting $x$ equal to $0$, there is no element $y$ of $\NN$ such that
$y<0$. But if $U = \ZZ$, then the statement is true, since for every integer $x$, we can put $y = x-1$,
so that $y<x$.

Now we practise translating converting mathematical statements written in English into formulas
in predicate logic. Suppose we want to write: ``Every integer is even or odd.'' The universe here is
is the set $\ZZ$ of integers. The word ``every'' has the same meaning as ``for all,'' so right
off the bat, we can reexpress the statement as: ``For all $n\in \ZZ$, $n$ is even or $n$ is odd.''
In symbols, this is
$$ \forall n\,(n\ \hbox{even} \vee n\ \hbox{odd}).$$
Lastly, we need to figure out how to express the property of being even or being odd.
An integer $n$ is even if an only if it is a multiple of two; that is, if there is some integer $k$
such that $2k = n$. Likewise, an integer is odd if and only if it is one more than a multiple of two.
The corresponding formula is $\exists k : 2k+1 = n$.
So our statement can be expressed
$$\forall n\, \bigl((\exists k : n = 2k)\vee (\exists k : n = 2k+1)\bigr).$$
The variable $k$ appears twice in this formula, but its first instance is independent of its second instance,
because of the parentheses. (Readers who write computer programs will be familiar with the concept of
a variable ``going out of scope.'') So there is nothing wrong with this formula, but to be extra
clear that the first $k$ is different from the second $k$, why don't we replace it with a different letter?
Thus we arrive at
$$\forall n\, \bigl((\exists k : n = 2k)\vee (\exists l : n = 2l+1)\bigr),$$
a formula in predicate logic that means ``every integer is even or odd.''

\medskip\boldlabel Restrictions using quantifiers. It is not true that every real number has a multiplicative
inverse, since one cannot divide by zero. However, the statement ``every nonzero real number has a multiplicative
inverse'' is true. How should we write this as a formula in predicate logic?
One way is to use the conditional: over the universe $U = \RR$, we could write
$$\forall x : (x\ne 0 \To \exists y : xy = 1).$$
Another is to use subscripts: in the same universe, we write
$$\forall x_{x\ne 0}\, \exists y : xy = 1.$$
Using subscripts is slightly informal, since we didn't formally define above what a subscript is supposed
to mean, but it is something that you might encounter. The last way is to simply restrict the universe itself:
in the universe $U = \RR\setminus\{0\}$, the statement
$$\forall x\, \exists y : xy = 1$$
is true.

\medskip\boldlabel Multiple quantifiers. Withing a formula, quantifiers cannot be interchanged
willy-nilly. The order of $\forall$ and $\exists$ matters! They are read from left to right.
Consider the following examples, over the universe $U = \RR$.
The statement
$$ \forall x\, \exists y : x+y = 0$$
is true, since for each given $x$ we can take $y$ to be $-x$. On the other hand,
$$ \exists y\, \forall x : x+y = 0$$
is false, since it would mean that there is some integer that adds up to zero with {\it any} integer.
Of course, sometimes switching the order of quantifiers, doesn't change the truth value of a statement.
Both
$$\exists y\, \forall x: xy = 0$$
and
$$\forall x\, \exists y: xy = 0$$
are true, since in the first case, we can take $y = 0$, and in the second case, we can set $y$ to $0$ no
matter what $x$ is given.

So we know that the order of $\forall$ and $\exists$ matters in general, but repeated instances
of the {\it same} quantifier {\it can} be interchanged. For instance,
$$\forall x\, \forall y : x^2 + y^4 \ge 0$$
is the same as
$$\forall y\, \forall x : x^2 + y^4 \ge 0,$$
and we can even write $\forall x,y : x^2 + y^4 \ge 0$, to introduce both variables simultaneously.

\medskip\boldlabel Negating quantifiers. The universal quantifier is sort of like a big chain of conjunctions
that goes over the whole of the universe. For example, in the universe $\NN$, the
statement $\forall n : P(n)$ is equivalent to
$$P(1) \wedge P(2) \wedge P(3) \wedge \cdots,$$
if this were a valid propositional formula (it isn't because we don't allow propositional formulas to be
infinite). Likewise, the existential quantifier $\exists n : P(n)$ is equivalent to
$$P(1)\vee P(2) \vee P(3) \vee\cdots.$$
We know, by De Morgan's laws, that negating a big series of conjunctions requires us to
flip all the {\mc AND}s to {\mc OR}s. So, once again abusing notation somewhat, we expect
$$\neg\bigl(P(1) \wedge P(2) \wedge P(3) \wedge \cdots\bigr)
\equiv \neg P(1) \vee \neg P(2)\vee \neg P(3)\vee\cdots.$$
Thus we conclude that
$$\neg\bigl(\forall n : P(n)\bigr) \equiv \exists n : \neg P(n).$$
You can play the same game with the other De Morgan's law to show that
$$\neg \bigl(\exists n : P(n)\bigr) \equiv \forall n : \neg P(n).$$
Going back to our example of $P(n)$ denoting ``$n$ is prime,'' the statement $\neg \bigl( \forall n : P(n)\bigr)$
is true, since not all integers $n$ are prime, and we have just shown that this is equivalent to the
statement $\exists n : \neg P(n)$; that is, there exists $n$ such that $n$ is not prime.

We end this section with a longer example. Let's express the statement, ``There is a nonzero
real number such that every real number is not its inverse or is negative.'' In the universe $\RR$,
the formula corresponding to this statement is
$$ \exists x : ( x\ne 0 \wedge (\forall y : xy\ne 1 \vee y< 0)\bigr) .$$
(Work it out yourself!) Is this statement true or false? It turns out that it is true. You might be able
to stare at the formula long enough to convince yourself of this fact, but another way to see
that it's true is to note that its negation is false. Let's do this now (it's a good excuse to
practise negating a formula). We have
$$\eqalign{
\neg\Bigl(\exists x :( x\ne 0 \wedge (\forall y : x&y\ne 1 \vee y< 0)\bigr)\Bigr) \cr
&\equiv \forall x : \neg ( x\ne 0 \wedge (\forall y : xy\ne 1 \vee y< 0)\bigr) \cr
&\equiv \forall x : \bigr(x= 0 \vee \neg(\forall y : xy\ne 1 \vee y< 0)\bigr) \cr
&\equiv \forall x : \bigr(x= 0 \vee \exists y : \neg (xy\ne 1 \vee y< 0)\bigr) \cr
&\equiv \forall x : \bigr(x= 0 \vee \exists y : (xy = 1 \wedge y\ge 0)\bigr).\cr
}$$
This negated statement is false, since if $x = -2$, then $x=0$ doesn't hold, so the left-hand
side of the {\mc OR} isn't true, and there is no $y$ such that $-2 y = 1$ and $y\ge 0$ are both
true, since the only $y$ satisfying $-2y = 1$ is $-1/2$.

Negating a formula in predicate logic is entirely mechanical.
The $\neg$ symbol moves from left to right like a bulldozer that flips quantifiers and negates
predicates it finds along the way, until eventually its job is done and it disappears.
More broadly, we write mathematical statements in formal logic to make things more precise and mechanical.
This can be useful to humans, since English is often ambiguous whereas the notation we just established
is not. It can be useful to machines as well,
since, as we just saw, manipulating a formula is something that can very easily be automated.

\advsect Proofs

\datestamp{6}{}
We're getting to the fun part of the course now.
Further back in these notes, we already proved a few statements about sets. This was just a taste of
what's to come, as the main focus of this course is to teach you how to prove mathematical statements.
These will all be statements that can ultimately be stated in predicate logic, so using the tools
from the previous section, you can boil them down to their logical skeleton. In this section, we will
formally define what it means to prove a statement in predicate logic.

To prove a statement, we always process the quantifiers from left to right. Whenever we encounter something
of the form $\exists x: P(x)$, we are allowed to choose the value for $x$ (from the given universe), and we
just need to show that $P(x)$ holds for that value of $x$. Here's an example.

\proclaim Proposition \advthm. There exists an integer $m>0$ and an integer $n<0$ such that $m^2 + n^2 = 25$.

\noindent We've written the statement in words, and we will continue to do so for all the propositions
in these notes because we are humans and not cyborgs,
but notice that the underlying predicate logical formula here is
$$\exists m\,\exists n : m>0 \wedge n< 0 \wedge m^2 + n^2 = 25,$$
with $U = \ZZ$. For the remainder of this section, we'll continue to write out the formulaic equivalents
of propositions, to practise converting between the two worlds.

\proof After a moment's contemplation, we notice that $9 + 16 = 25$. So we need a positive integer that
squares to to $9$ and a negative number that squares to $16$. Hence we may pick $m = 3$ and $n=-4$,
and $m^2 + n^2 = 25$.\slug

On the other hand, to prove a statement of the form $\forall x : P(x)$, we are not allowed to choose the
value of $x$. Instead, we imagine it is given to us, and we still have to prove $P(x)$, no matter what
the $x$ might be. Here's what we mean.

\proclaim Proposition \advthm. For all $x\in \QQ$, there exists $y\in \ZZ$ such that
$xy \in \ZZ$.

\noindent The formula this time is
$$\forall x\, \exists y : y\in \ZZ \wedge xy\in \ZZ,$$
over $U = \QQ$.

\proof Let $x \in \QQ$. Then $x$ can be expressed as the ratio of two integers; write $x = m/n$ with
$m,n\in \ZZ$. Then, setting $y=n\in \ZZ$, we have $xy = (m/n)\cdot n = m\in \ZZ$.\slug

Notice how we introduced the variable $x$ in the above proof.
Because we're trying to prove a ``for all'' statement, we use
the word ``let,'' to indicate that $x$ is given to us by some higher power. In fact, we should sort
of view this higher power as possibly being malicious. A very useful way to think about writing proof is
to imagine a game in which you are trying to prove a statement in predicate logic, and a supernatural adversary
is attempting to thwart you. Let's say this predicate has four variables, so the statement is
$$\exists x \, \forall y \, \forall z\, \exists w : P(x,y,z,w).$$
The variables in the statement are introduced from left to right. Each time you see a $\exists$
symbol, it's your turn. In the example above, you get to set the variable $x$ to any element of the
given universe (keeping in mind that your eventual goal is to prove $P(x,y,z,w)$). Each time there
is a $\forall$ symbol, it's the adversary's turn, and you should be prepared for whatever he throws at you.
So in our example, the adversary may set $y$ and $z$ to anything in the universe, and he knows you picked
$x$. Lastly, you get to pick $w$. If $P(x,y,z,w)$ is true, you win, and if not, the adversary wins.
Writing a proof is equivalent to describing a winning strategy for the player against the adversary.

Many mathematical statements introduce some hypotheses, then assert some conclusion. Thus they are some
kind of statement of the form $p\To q$. To prove this kind of statement,
we assume that $p$ holds, then prove that $q$ is true. This is because the only way a statement for $p\To q$
to be false is if $p$ is true and $q$ is false, so we're showing this cannot happen. Take a look at
the following example.

\edef\propoddimpliessquareodd{\the\thmcount}
\proclaim Proposition \advthm. If $n$ is an odd integer, then $n^2$ is also odd.

\noindent
The underlying formula is $\forall n\bigl( (\exists k : n= 2k+1) \To (\exists l : n^2 = 2l+1)\bigr)$,
in the universe $U=\ZZ$.

\proof Let $n$ be an arbitrary odd integer, so that there exists $k$ such that $n = 2k+1$. Then
$$n^2 = (2k+1)^2 = 4k^2 + 4k + 1 = 2(2k^2 + 2k)+1,$$
so setting $l = 2k^2 + 2k$, which is an integer, we have $n^2 = 2l+1$. Hence $n^2$ is odd.\slug

To disprove a statement, we simply prove that its negation is true. For example, suppose we want
to disprove the statement $\exists x\, \forall y : x+y = 0$ (this was a statement we encountered
in the previous section). Its negation is
$$\forall x\, \exists y : x+y \ne 0,$$
and to prove it we let $x$ be given, set $y = -x + 1$, and observe that $x+y = x + (-x+1) = 1$, which is
not equal to $0$.

Now we define two words that sound similar but are very different, logically speaking.
The {\it converse}
of a conditional statement $p\To q$ is the statement $q\To p$. The
statement $p\To q$ and its converse both hold if and only if
the biconditional statement $p\Leftrightarrow q$ holds.
On the other hand, the {\it contrapositive} to $p\To q$ is the statement
$\neg q \To \neg p$. What's the deal with this silly-looking conditional? Well, it turns out to actually
be equivalent to $p\To q$. Check it out:
$$\eqalign{
p\To q &\equiv \neg p \vee q \cr
&\equiv q \vee \neg p \cr
&\equiv \neg (\neg q) \vee \neg p \cr
&\equiv \neg q \To \neg p \cr
}$$
So to prove a statement of the form $p\To q$, it is sufficient to prove $\neg q\To \neg p$. Sometimes,
this can make the proof a lot easier. Here's an example. (It is the converse of
Proposition~{\propoddimpliessquareodd}.)

\edef\propsquareoddimpliesodd{\the\thmcount}
\proclaim Proposition \advthm. If $n$ is an integer such that $n^2$ is odd, then $n$ is odd.

\noindent{\it Proof attempt.}\enspace Assume $n^2$ is odd, so that $n^2 = 2k+1$ for some integer $k$.
Then $n = \sqrt{2k+1}$.
\medskip

Where do we go from here? It's not even clear that $\sqrt{2k+1}$ is an integer, let alone an odd one.
The contrapositive comes to the rescue.

\medskip\noindent{\it Proof of Proposition~{\propsquareoddimpliesodd}.}\enspace
We proceed by contraposition.
Suppose that $n$ is not odd; that is, $n$ is even. So there
exists an integer $k$ such that $n=2k$. Then $n^2 = (2k)^2 = 4k^2$. Setting $l = 2k^2$, we have
$n^2 = 2l$, so $n^2$ is even. Hence $n^2$ is not odd.\slug

\medskip\boldlabel Proofs by contradiction. We now discuss a powerful method of proof, which dates
back to at least the
ancient Greeks and referred to for much of Western history by its Latin name,
{\it reductio ad absurdum}. It proceeds to prove
a statement $p$ by assuming its negation $\neg p$
holds, then deriving a contradiction, i.e., a statement whose truth value is $0$.
By doing this, one will have proved that $\neg p\To 0$, which in turn shows that $p$ is true, since
$$\neg p \To 0 \equiv \neg(\neg p) \vee 0 \equiv p\vee 0 \vee p.$$
Here is an example of a proof by contradiction.

\proclaim Proposition \advthm. There is no least positive rational number.

\noindent In $U = \QQ$, one way to formulate this is
$$\neg \bigl(\exists x: x > 0 \wedge (\forall y : y > 0 \To x \le y)\bigr).$$

\proof Suppose, towards a contradiction, that there exists some rational $x>0$ such that for all $y\in \QQ$
with $y > 0$, $x\le y$. Then, we can apply this property with $y = x/2$ to see that $x \le x/2$.
Since $x$ is positive, we can divide this inequality by $x$ on both sides to obtain $1\le 1/2$. But this
is absurd, since $1/2 < 1$. This contradiction completes the proof.\slug

Here is another classical example of a proof by contradiction. It was known to the Pythagoreans.

\edef\propsqrttwo{\the\thmcount}
\proclaim Proposition \advthm. The number $\sqrt 2$ is irrational.

\proof Suppose, towards a contradiction, that $\sqrt 2$ is rational. Then there are integers $p$ and $q$
such that $\sqrt 2 = p/q$, and furthermore, we can assume that $p$ and $q$ do not have any common factors,
since if they did, we could divide them out and the ratio would remain the same.
Squaring both sides of the equation, we have $2 = p^2/q^2$, or in other words, $2q^2 = p^2$. This implies
that $p^2$ is even, so by (the contrapositive of) Proposition~{\propsquareoddimpliesodd}, $p$ is even as well.
That means we can write $p=2r$ for some integer $r$, and substitute this new information
into the above equation to obtain $2q^2 = (2r)^2 = 4r^2$. Dividing out by $2$ yields $q^2 = 4r^2$,
so $q^2$ is even, and so is $q$.

We have deduced that $p$ and $q$ are both even. On the other hand, we assumed they had no common factors.
This is a contradiction.\slug

Earlier we described the composition of a proof as playing a game with a supernatural adversary.
To prove something by contradiction, then is is akin to starting the game
by letting adversary believe he has already won,
and then working from that assumption to derive an impossibility. Quite the devious strategem.

\bigskip
\begingroup\obeylines\eightssi
\hfill {\eightss Reductio ad absurdum}, which Euclid loved so much,
\hfill is one of a mathematicians finest weapons.
\hfill It is a far finer gambit than any chess gambit:
\hfill a chess player may offer the sacrifice of a pawn
\hfill or even a piece,
\hfill but a mathematician offers the {\eightss game}.
\eightss
\smallskip
\hfill --- G.~H.~HARDY, {\eightssi A Mathematician's Apology} (1940)
\endgroup%\obeylines
\bigskip\goodbreak

\medskip\boldlabel Case analysis.
For all $n\ge 2$ and all propositions $p_1, \ldots, p_n, q$, we have the equivalence
$$(p_1\vee p_2 \vee \cdots \vee p_n) \To q \equiv (p_1\To q) \wedge (p_2\To q) \wedge \cdots \wedge
(p_n\To q).$$
(As an exercise, prove this when $n=2$. Later on, when we learn about mathematical induction,
you'll be able to prove it in general.)

Why is this useful? Well, if we have two or more propositions $p_1, p_2, \ldots, p_n$
such that at least one of them must hold, that is,
$$p_1\vee p_2 \vee \cdots \vee p_n \equiv 1,$$
then if we are able to show that
$$p_1 \To q,\quad p_2\To q,\quad \ldots,\quad\hbox{and}\quad p_n\To q$$
all hold, then we will have shown
$$1\To q \equiv \neg 1 \vee q \equiv 0 \vee q \equiv q.$$
Let's see an example of this in action.

\proclaim Proposition \advthm. For all integers $n\ge 0$, $1 + (-1)^n (2n-1)$ is a multiple
of $4$.

\proof Let an integer $n\ge 0$ be given. We know that $n$ is even or $n$ is odd.

If $n$ is even, then $n = 2k$ for some integer $k$, and
$$1 + (-1)^{2k} \bigl( 2(2k)-1\bigr) = 1 + 1^k (4k-1) = 4k,$$
which is a multiple of $4$.

If $n$ is odd, then $n=2k+1$ for some integer $k$, and
$$1 + (-1)^{2k+1}\bigl(2(2k+1) - 1\bigr) = 1- (4k+2-1) = 1-(4k-1) = -4k,$$
which is also a multiple of $4$.\slug

Sometimes, the cases in which to split the proof are not so obvious.

\proclaim Proposition \advthm. There exist irrational numbers $a$ and $b$ such that $a^b$ is rational.

\proof Consider the number $(\sqrt 2)^{\sqrt 2}$. It is either rational or irrational.

If it is rational, then we can set $a = b = \sqrt 2$, which we already know to be irrational, by
Proposition~{\propsqrttwo}.

If it is irrational, then we can set $a =(\sqrt 2)^{\sqrt 2}$ and $b=\sqrt 2$. We compute
$$a^b = \bigl((\sqrt 2)^{\sqrt 2}\bigr)^{\sqrt 2} = (\sqrt 2)^{\sqrt 2 \cdot \sqrt 2}
= (\sqrt 2)^2 = 2,$$
which is rational.\slug

Note that in this proof, we did not need to know whether $(\sqrt 2)^{\sqrt 2}$ is rational or not;
we just know that it must either be rational or irrational.
(In fact, it is known that $(\sqrt 2)^{\sqrt 2}$ is irrational, but the usual proof of this
requires some notions from complex analysis as well as Galois theory, subjects that are
typically not encountered until at least the second or third year of a mathematics degree.)


\medskip\boldlabel\datestamp{7}{} Mathematical induction. Sometimes we want to prove an infinite
number of statements, indexed by the natural numbers. If the complexity of the statements
sort of ``grows'' in $n$ (a vague notion to be made precise soon), the following theorem holds.

\parenproclaim Theorem {\advthm} (Principle of mathematical induction). Let $P(n)$ be a family of predicates
indexed by $n\in \NN$. Let $m\in \NN$. If
\medskip
\item{i)} $P(m)$ holds; and
\smallskip
\item{ii)} for all $n\ge m$,
$$\bigl( P(m)\wedge P(m+1)\wedge \cdots \wedge P(n) \bigr) \To P(n+1)$$
\medskip
then $P(n)$ holds for all $n\ge m$.

In the following proof, we use the fact that every nonempty subset of the natural numbers has
a least element, the proof of which belongs to a higher-level set theory course.

\proof Assume that (i) and (ii) both hold. Our goal is to prove that $P(n)$ is true for all
$n\ge m$, so we shall suppose, towards a contradiction, that there is some $n\ge m$ such that
$\neg P(n)$ holds. (As an exercise, convince yourself, via symbolic manipulations, that
$$\neg \forall n : n\ge m \To P(n) \equiv \exists n : n\ge m \wedge \neg P(n).)$$
In other words, the set of $n\ge m$ such that $\neg P(n)$ holds is nonempty, so it has a least element,
call it $k$. So $k\ge m$, and we cannot have $k=m$ due to (i), so $k>m$. Furthermore, by the minimality
of $k$, the statements $P(m)$, $P(m+1)$, all the way up to $P(k-1)$ are true. By (ii) though,
this implies that $P(k)$ holds: a contradiction.\slug

When proving something by induction, we need to prove two things: one of the form (i) and another
of the form (ii). The former is called the {\it base case} and the latter the {\it induction step}
or {\it inductive step}. The assumption to the left of the $\To$ symbol in (ii)
is called the {\it induction hypothesis} or {\it inductive hypothesis}.
Technically, the theorem above is called the principle of {\it strong} induction;
the principle of {\it weak} induction has
\medskip
\item{ii)$'$} for all $n\ge m$, $P(n)\To P(n+1)$,
\medskip
instead of the stronger (ii). It turns out that both types of induction are actually equivalent,
so we'll use both interchangeably. More often than not, the hypothesis (ii)$'$ is perfectly sufficient,
and in these cases we'll simply use weak induction so as not to clutter our proofs with lots of unused
hypotheses.

Here is an example. Suppose we want to find (and prove) a formula for the sum of the first $n$ odd numbers.
The first thing to do is to try some small cases. When $n = 1$, the sum is just $1$. When $n=2$,
the sum is $1+3 = 4$, when $n=3$, we have $1+3+5 = 9$, and for $n=4$, we compute $1+3+5+7 = 16$.
So the pattern goes $1,4,9,16,\ldots$, which leads us to conjecture that the sum of the first $n$
odd numbers might equal $n^2$. In fact, this is true, and we shall prove it by induction.

\proclaim Proposition \advthm. For all integers $n\ge 1$,
$$\sum_{i=1}^n (2i-1) = n^2.$$

\proof By induction on $n$. First we prove the base case, $n=1$. We have
$$\sum_{i=1}^1 (2i-1) = 2-1 = 1 = 1^2.$$
Now for the inductive step. Let $n\ge 1$ and assume that
$$ \sum_{i=1}^n (2i-1) = n^2.$$
Then
$$\eqalign{
\sum_{i=1}^{n+1} (2i-1) &= \sum_{i=1}^n (2i-1) + \bigl(2(n+1) - 1\bigr) \cr
&= n^2 + 2n+2 - 1 \cr
&= n^2 + 2n + 1\cr
&= (n+1)^2,\cr
}$$
where it is in the second line that we used the inductive hypothesis.\slug

The first thing to notice about doing a proof by induction is that the proof method itself doesn't
tell you what it is you should prove. You have to guess at the correct statement first. Also,
proofs by induction are often ``unenlightening,'' in that they often don't reveal the fundamental reasons
why something might be true. (The previous proposition can be illustrated by a rather simple
picture, which is not a proof, but is somewhat more elucidating that the induction proof.)

A longer example now. Suppose we have a pile of $n$ stones, with $n\ge 1$. We have a job, which
can be described by a pseudo-algorithm.

\algbegin Algorithm S (Divide stones). The input to
this algorithm is an integer $n\ge 1$, representing a number of stones.
We have a list {\tt PILES} of integers representing a collection
of piles of stones, as well as an integer variable {\tt BALANCE}.
Initialise ${\tt PILES} \Gets 1$ (one pile with $n$ stones), and set ${\tt BALANCE}\gets 0$.
This algorithm splits the stones into $n$ piles of $1$ stone each, accumulating profits
into {\tt BALANCE} along the way.
\algstep S1. [Done?] If every element of {\tt PILES} is $1$, terminate the algorithm and output
{\tt BALANCE}.
\algstep S2. [Choose a pile.] Select some element $m>2$
from {\tt PILES} and remove it from the list. (The variable $m$ is the number of stones in this pile.)
\algstep S3. [Split.] Let $k$ and $l$ be two numbers with $k+l = m$. We append ${\tt PILES}\Gets k$ and
${\tt PILES} \Gets l$, and increment ${\tt BALANCE} \gets {\tt BALANCE} + k\cdot l$. (We have split
pile $n$ into two piles of size $k$ and $l$, and the payout for doing so is $k\cdot l$ dollars.)
\algstep S4. [Loop.] Go to step S1.\slug

This is not, strictly speaking, an algorithm, since we didn't specify how the algorithm
should choose the integers $k$ and $l$ that add up
to $m$ in step S3. However, running through a few instances with, say, $n=6$, on paper, using whatever
choices of split you like in every iteration of step S3, you'll find that the algorithm always terminates
with ${\tt BALANCE} = 15$. Trying a few different starting values of $n$ might lead you to conjecture
the following proposition, which we will prove by (strong) induction.

\proclaim Proposition \advthm. For a given input $n$, Algorithm S always outputs ${\tt BALANCE} = n(n-1)/2$,
regardless of the choice of split at any given iteration of step S3.

\proof By induction on $n$. For the base case $n=1$, note that we immediately output ${\tt BALANCE} = 0$
in the very first step of the algorithm, and $0 = 1(1-1)/2$.

Now for the inductive step, let $n\ge 1$ and suppose that for $1\le k \le n$, the payout for running
algorithm S on input $l$ is $l(l-1)/2$. Suppose we have a pile of $n+1$ stones. We shall divide
it into piles of size $k$ and size $n+1-k$. The total payout will be the pay for this division,
namely $k(n+1-k)$, plus the pay for further subdividing the two piles. Thus by the induction
hypothesis applied twice, the total payout will be
$$\eqalign{
k(n+1-k) + &{k(k-1)\over 2} + {(n+1-k)(n-k)\over 2} \cr
&= {2nk + 2k - 2k^2 + k^2 - k + n^2 + n -2nk - k + k^2\over 2} \cr
&= {n^2 + n\over 2} \cr
&= {(n+1)\bigl((n+1)-1\bigl)\over 2},\cr
}$$
which is the expected formula for $n+1$.\slug

\advsect Functions

A {\it function} $f$ from a set $A$ to a set $B$ is a subset $f\subseteq A\times B$ such that
for every $a\in A$, there is exactly one $b\in B$ such that $(a,b)\in f$. (If there is no $b\in B$,
or more than one, we say that $f$ is not well-defined.
If $(a,b)\in f$,
we write $f(a) = b$. The set $A$ is called the {\it domain} and the set $B$ is the {\it codomain}.
The notation $f:A\to B$ is a way of concisely writing ``$f$ is a function with domain $A$ and
codomain $B$.''

Here are some examples and non-examples.
\medskip
\item{i)} The function $f:\RR\to\RR$ given by $f(x) = x^2$ is
$$f = \bigl\{ (a,b)\in \RR^2 : b = a^2 \bigl\},$$
when written in set-builder notation.
\smallskip
\item{ii)} On the other hand, the set
$$g = \bigl\{ (a,b)\in \RR^2 : a = b^2 \bigl\}$$
is not a function, since $(1,1)$ and $(1,-1)$ are both in $g$. Furthermore,
there is no element of $g$ with $-1$ as its first coordinate.
\smallskip
\item{iii)} If $X = \{1,4,5\}$ and $Y = \{0,1,2,3,4,5\}$, then
$$\bigl\{ (1,0), (2, 4), (5,5)\bigr\}$$
is a function.
\smallskip
\item{iv)} The set
$$h = \bigl\{ (x,y)\in \RR^2 : y = 1/x \bigl\}$$
is not a function, since there is no $y$ such that $(0,y)\in h$. However
if we amend the domain and consider
$$h = \bigl\{ (x,y)\in \bigl(\RR\setminus \{0\}\bigr)\times \RR : y = 1/x \bigl\},$$
then in fact, $h:\RR\setminus \{0\} \to \RR$ is a function.
\medskip

\medskip\boldlabel\datestamp{8}{} Injective and surjective functions.
The {\it range} or {\it image} of a function $f:A\to B$ is
$$f(A) = \bigl\{b\in B : \hbox{there exists}\ a\in A\ \hbox{such that}\ b = f(a)\bigr\}.$$
These are all the values $f$ actually outputs. For instance, if we let $f : \ZZ\to \NN$
be given by $f(n) = n^2$, then
$$\eqalign{
f(\ZZ) &= \{\ldots, f(-2), f(-1), f(0) , f(1), f(2),\ldots\} \cr
&= \{\ldots, 9, 4, 1, 0 1, 4, 9,\ldots\} \cr
&= \{0, 1, 4, 9, 16, 25, \ldots\}. \cr
}$$
A function $f:A\to B$ is called {\it surjective} or {\it onto} if $f(A) = B$, that is,
for every $b\in B$ there exists some $a\in A$ such that $f(a) = b$. An example of a
surjective function is $f:\QQ\to \QQ$ sending $x\mapsto x/2$. This is because for any $q\in \QQ$,
we can set $r = 2q$, and
$$f(r) = {r\over 2} = {2q\over 2} = q.$$
A function $f:A\to B$ is called {\it injective} or {\it one-to-one} if for all $a_1,a_2\in A$
with $a_1 \ne a_2$, we also have $f(a_1)\ne f(a_2)$. Equivalently, $f$ is injective if
$f(a_1) = f(a_2)$ implies that $a_1 = a_2$ for all $a_1, a_2 \in A$. For instance, the function
$f:\ZZ\to \NN$ that sends $n\mapsto n^2$ is not injective since $f(-1) = f(1)$, but $-1 \ne 1$.
On the other hand, if we modify the domain, considering $f:\NN\to \NN$ sending $n\mapsto n^2$,
then now $f$ is injective, since if $f(m) = f(n)$, then $m^2 = n^2$, and there is only one positive
integer that squares to any given integer, so $m=n$.

Hence we see that any function can be transformed into an injective one, in principle, by shrinking its
domain (though this new function might no longer have the properties you liked in the original one),
and any function $f:A\to B$ can be made surjective by changing its codomain to its range, i.e.,
letting $B = f(A)$.

\medskip\boldlabel The pigeonhole principle. We now take a brief pause to introduce one of
the most fundamental laws in discrete mathematics, called the pigeonhole principle.
We begin with the following intuitive theorem.

\proclaim Theorem \advthm. Let $a_1, a_2, \ldots, a_m$ be a finite sequence (repeats allowed) of
real numbers. Let
$$a = {1\over m}\sum_{i=1}^m a_i$$
be the average value of the sequence and let $m$ be the maximum value the sequence attains. Then $m \ge a$.

\proof We have
$$a = {1\over m}\sum_{i=1}^m a_i \le {1\over m} \sum_{i=1}^m m = {m^2\over m} = m.\noskipslug$$

This theorem can be summed up in one sentence: {\sl The maximum is at least the average.}
Don't underestimate this theorem even though its proof was a one-liner! It is often used to prove
highly nontrivial results.
(As an exercise, prove the similar statement: {\sl The minimum is at most the average.})
From here we are now equipped to prove (a general version of) the pigeonhole principle.

\edef\thmgeneralpigeonhole{\the\thmcount}
\proclaim Theorem \advthm. Let $A$ and $B$ be finite sets with $|A| = m$ and $|B|=n$.
For every function $f:A\to B$, then there is some $b\in B$ such that there are at least
$\lceil m/n\rceil$ elements $a\in A$ with $f(a) = b$.

\proof For $1\le i \le n$, let $r_i$ be the number of $a\in A$
such that $f(a) = b_i$. This is a sequence that adds up to $m$, since every $a$ in $A$ maps to
exactly one element of $B$. So the sequence has average $m/n$, and by the previous theorem,
there must be some $j$ such that $r_j \ge m/n$. But the $r_i$ are all actually integers (since
cardinalities of finite sets are integers), meaning that $r_j \ge \lceil m/n \rceil$. Letting
$b = b_j$ completes the proof.\slug

The reason this is called the pigeonhole principle is because of the following special case.

\parenproclaim Corollary {\advthm} (Pigeonhole principle). Let $n\ge 2$. If $n$ pigeons nest in $n-1$ holes,
there is at least one hole that contains at least two pigeons.

\proof Let $A$ be the set of pigeons and $B$ the set of pigeonholes. Let $f$ be any function sending
the set of pigeons to the set of holes. By the previous theorem, there is some hole with at least
$\lceil n/(n-1)\rceil = 2$ pigeons in it. (This is because $1 < n/(n-1) < 2$ for all integers $n\ge 2$.)\slug

\medskip\boldlabel Bijections.
A function $f$ is called {\it bijective} (or a {\it bijection}, or a {\it one-to-one correspondence})
if it is injective and surjective. Bijections are important because of the following proposition.

\proclaim Proposition \advthm. Let $A$ and $B$ be finite sets. Then
\medskip
\item{i)} there exists a bijection $f:A\to B$ if and only if $|A| = |B|$; and
\smallskip
\item{ii)} if $|A| = |B|$ and $f:A\to B$ then $f$ is injective if and only if $f$ is surjective.
\medskip

\proof Suppose $|A| = |B| = n$. Then choose an ordering $a_1, \ldots, a_n$ of $A$ and an ordering
$b_1, \ldots, b_n$ of $B$. Let $f(a_i) = b_i$ for all $1\le i\le n$. By construction, this is a bijection,
proving one direction of (i).

On the other hand, suppose $|A|\ne |B|$ (so we prove this direction by contraposition).
If $A<B$, then $f$ cannot be surjective, since the image of
$f$ has size at most $|A| < |B|$ (at least one element of $B$ must be missed). If $A>B$, then
$|A|/|B| > 1$, so by Theorem~{\thmgeneralpigeonhole}, there is some element $b\in B$
such that the number of $a\in A$ mapping to $b$ is at least $\lceil |A|/|B| \rceil \ge 2$.
This means that $f$ is not injective. We have proved part (i).

To prove part (ii), let $|A| = |B|$ and let $f:A\to B$. First we assume that $f$ is injective.
We enumerate $A = \{a_1, a_2, \ldots, a_n\}$. Then
$$f(A) = \bigl\{ f(a_1), f(a_2), \ldots, f(a_n)\bigr\} \subseteq B.$$
All of the $f(a_i)$ are distinct, since if $f(a_i) = f(a_j)$, then $a_i = a_j$. So
$\bigl| f(A)\bigr| = |A| = n$, and $f(A)$ is a size $n$ subset of $B$, which also has size $n$.
Hence $f(A) = B$; that is, $f$ is surjective.

Lastly, suppose $f$ is not injective (again we are using contraposition). So there
are $a_i$ and $a_j$ such that $a_i\ne a_j$ but $f(a_i) = f(a_j)$. So
$$\bigl|f(A)\bigr| = \bigl|\bigl\{ f(a_1), f(a_2), \ldots, f(a_n)\bigr\}\bigr| < n = |B|,$$
so $f(A) \ne B$ and $f$ is not surjective.\slug

Item (i) of the previous proposition should be entirely intuitive, especially if we use the alternative
nomenclature ``one-to-one correspondence'' instead of ``bijection.'' (In fact, we already implicitly
used (i) in these notes, in the proof of Proposition~{\propcardpowerset}.) Item (ii) is perhaps not
as immediate, but should become clear if you work it out with a picture.

\medskip\boldlabel Bijections.
A function $f:A\to B$ is called {\it invertible} if there exists $g:B\to A$ such that
\medskip
\item{i)} for all $b\in B$, $f\bigl(g(b)\bigr) = b$; and
\smallskip
\item{ii)} for all $a\in A$, $g\bigl(f(a)\bigr) = a$.
\medskip
If $g$ exists, it can be shown that $g$ must be unique, so we write $g = f^{-1}$ and speak of
{\it the} inverse of $f$.

\proclaim Proposition \advthm. Let $f:A\to B$ be a function. Then $f$ is invertible
if and only if $f$ is bijective.

\proof First we assume that $f$ is invertible. So there exists an inverse $g$ of $f$. For each $b\in B$,
setting $a = g(b)$ we have
$$f(a) = f\bigl(g(b)\bigr) = b.$$
This proves that $f$ is surjective. To show that $f$ is injective, suppose that $f(a_1) = f(a_2)$.
By applying $g$ on both sides, we have $g\bigl(f(a_1)\bigr) = g\bigl(f(a_2)\bigr)$, whence
$a_1 = a_2$, by definition of $g$.

Now assume that $f$ is bijective. We need to define $g:B\to A$. Well, gven any $b\in B$, there is
some $a$ such that $f(a) = b$, from surjectivity of $f$, and this $a$ is unique, since $f$ is
injective. So set $g(b) = a$ (and repeat this process for every $b\in B$). We have
$f\bigl(g(b)\bigr) = f(a) = b$, and for every $a\in A$, by definition of $g$ the
element $g\bigl(f(a)\bigr)$ is the unique element in $A$ that gets brought to $f(a)$ by $f$,
has to be $a$ itself.\slug

Sometimes to prove that two sets have the same cardinality, it is easier to prove that there
exists a bijection (as we already saw in the example of Proposition~{\propcardpowerset}),
and sometimes to prove that a function is a bijection, it is easier to show that it has an inverse,
rather than messing around with the definitions of injective and surjective.
Here's an example.

\proclaim Proposition. Let $X$ be a finite nonempty set. Let $E$ be the set of all subsets of $X$ with
even cardinality, and let $F$ be the set of all subsets of $X$ with odd cardinality.
Then $|E| = |F|$.

\proof We shall construct a function $f:E\to F$. Fix one specific $x\in X$;
we can do this because $X\ne \emptyset$. Now, for all $A\in E$, let
$$f(A) = \cases{A\setminus\{x\}, & if $x\in A$;\cr A\cup \{x\}, & if $x\notin A$.\cr}$$
Note that since $|A|$ is even for all $A\in E$, the cardinality of $f(A)$ is odd (in the first
case it is $|A|-1$ and in the second case it is $|A|+1$. This shows that $f$ is indeed a function
with codomain $F$. Now to prove $|E| = |F|$ we will show that $f$ is bijective, which we
do by showing that $f$ is an inverse (as an exercise, you might try to prove bijectivity from
the definitions of injective and surjective).

We define $g : F\to E$ by
$$g(A) = \cases{A\setminus\{x\}, & if $x\in A$;\cr A\cup \{x\}, & if $x\notin A$.\cr}$$
As before, $\bigl|g(A)\bigr|$ is even, since $A$ is assumed to be a member of $F$ now. Now
for any $A\in E$,
$$\eqalign{
g\bigl(f(A)\bigr) &= \cases{g\bigl(A\setminus\{x\}\bigr), & if $x\in A$;\cr
g\bigl(A\cup \{x\}\bigr), & if $x\notin A$\cr} \cr
&= \cases{\bigl(A\setminus\{x\}\bigr)\cup \{x\}, & if $x\in A$;\cr
\bigl(A\cup \{x\}\bigr)\setminus \{x\} , & if $x\notin A$\cr} \cr
&= \cases{A, & if $x\in A$;\cr
A, & if $x\notin A$\cr} \cr
&= A.\cr
}$$
The proof that $f\bigl(g(A)\bigr) = A$ is similar. Thus $g$ is the inverse of $f$.\slug

\advsect Cardinality

\datestamp{9}{}
Earlier, we defined the cardinality of a set to be the number of elements it contains. What, then,
is the cardinality of $\NN$? How about $\RR$? You might say $\infty$, but this is not a number
(at least, it's not an element of $\NN$, the way all cardinalities of finite sets are).
So perhaps we should amend our question to the following: {\sl When do infinite sets have the same size?}
Our experience with functions leads us to the answer: {\sl When there exists a bijection between
them}. We shall say that $A$ and $B$ are {\it equipotent} (or {\it equinumerous}, or {\it have
the same cardinality}) if there exists a bijection between $A$ and $B$. In this case
we write $|A| = |B|$.

As an example, the sets $\NN$ and $\NN\setminus\{0\}$ are equipotent, since $f$
given by $n\mapsto n+1$ is a bijection $\NN\to\NN\setminus \{0\}$. (Check that
$f^{-1}(m) = m-1$ is its inverse.)

It is even possible to remove an infinite number of elements from $\NN$ and end up with
something still equipotent with $\NN$. To see this, let $E$ be the set of nonnegative even integers,
and consider the function $f :\NN\to E$ sending $n\mapsto 2n$. This is injective because if
$2m = 2n$, then dividing out by $2$ on both sides yields $m=n$. It is surjective because if
$n\in E$, then $n=2k$ for some $k\in\NN$, by definition, and $f(k) = 2k = n$.

So we can find subset of $\NN$ equipotent with it. It turns out we can also find supersets of $\NN$ with
the same property.

\proclaim Theorem \advthm. We have $|\NN| = |\ZZ|$.

\proof We define $f : \NN\to \ZZ$ by
$$f(n) = \cases{{n\over 2}, & if $n$ is even;\cr {-{n+1\over 2}}, & if $n$ is odd.}$$
We shall show that $f$ is a bijection.

To show that $f$ is injective, there are four cases. Note first that if $n$ is even, then $f(n)\ge 0$,
and if $n$ is odd, then $f(n) < 0$. So if $f(m) = f(n)$ for some $m,n\in \NN$, then $f(m)$ and $f(n)$ must
either both be negative, or both be nonnegative. Either way, $m$ and $n$
are either both even or they are both odd. If $m$ and $n$ are both even, then from $f(m) = f(n)$ we derive
$${m\over 2} = {n\over 2},$$
whence multiplying by $2$ on both sides we see that $m=n$. If $m$ and $n$ are both odd, then
$$-{m+1\over 2} = -{n+1\over 2},$$
so, multiplying by $-2$ and subtracting $1$ from both sides we have $m=n$ in this case as well.

Now we show that $f$ is surjective. Let $k\in \ZZ$. If $k\ge 0$, then consider $n=2k$. We have
$$f(n) = f(2k) = {2k\over 2} = k.$$
If $k<0$, then consider $n=-2k-1$. (Check that this is an element of $\NN$.) Then
$$f(n) = f(-2k-1) = -{-2k-1+1\over 2} = -{-2k\over 2} = k.$$
This shows that $f$ is surjective, so $f$ is in fact bijective and we conclude that $\NN=\ZZ$.\slug

We say that a set $A$ is {\it countably infinite} if there exists a bijection $f:\NN\to A$, that is,
if $|\NN| = |A|$.
A set is said to be {\it countable} if it is either finite or countably infinite.
Otherwise it is called {\it uncountable}.
The previous theorem shows that $\ZZ$ is countably infinite.
If a set $A$ is countable, then there exists an enumeration
$$A = \{a_0, a_1, a_2, \ldots\}.$$
This is useful in the proof of the following theorem.

\proclaim Theorem \advthm. If $A$ and $B$ are countably infinite sets,
then $A\times B$ is also countably infinite.

\noindent{\it Proof sketch.}\enspace
Enumerate $A = \{a_1, a_2, \ldots\}$ and $B = \{b_1, b_2, \ldots\}$. Now given
$(a_i, b_j)\in A\times B$ for some $i,j\in \NN$, we can let
$$f(a_i, b_j) = {(i+j+1)(i+j+2)\over 2} - i - 1.$$
Work out the values of $f(a_i, b_j)$ for small choices of $i$ and $j$, and draw a picture to convince yourself that
$f : A\times B \to\NN$ is a bijection. (Rigorously proving this fact
is quite tedious, so we omit the details.) This means that $f^{-1}$
is a bijection from $\NN$ to $A\times B$, so $A\times B$ is countably infinite.\slug

\proclaim Corollary \advthm. We have $|\ZZ\times \ZZ| = |\NN|$.

\proof
We proved earlier that $\ZZ$ is countably infinite, so we may apply the previous theorem with $A = B = \ZZ$.\slug

One can also show that $|\QQ| = |\NN|$. We will not include the proof, but the idea is that there
is a surjective map from $\ZZ\times\bigl( \ZZ \setminus \{0\}\bigr)\to \QQ$ given by $(m,n)\mapsto m/n$.
On the other hand, there are infinite sets that are not countably infinite. An {\it infinite binary
string} is a function $f:\NN\to \{0,1\}$. (By listing out
$$ f(0), f(1), f(2), \ldots$$
you get an infinite string of $0$s and $1$.)

\proclaim The set $A$ of all infinite binary strings is uncountable.

\proof Certainly $A$ is not finite, so we need to show that $|\NN|\ne A$.

Let $f : \NN\to A$. We shall show that $f$ is not surjective. For all $m,n\in \NN$, let $a_{m,n}$
be the $n$th bit of $f(m)$. Consider the infinite binary string
$$s = (1-a_{0,0},1-a_{1,1}, 1-a_{2,2}, 1-a_{3,3}, \ldots).$$
This string $s$ cannot equal $f(m)$ for any $m\in \NN$, since given an arbitrary $m$,
the $m$th bit of $f(m)$ is $a_{m,m}$, whereas the $m$th bit of $s$ is $1-a{m,m}$. Hence there is some
$s\in A$ such that $f(m)\ne s$ for all $m\in \NN$. So $f$ is not surjective.

Since there can be no surjection $f:\NN\to A$, {\it a fortiori} we cannot have a bijection $\NN\to A$.
We conclude that $|A| \ne |\NN|$.\slug

The set of all infinite binary strings is in bijection with the set of all real numbers in the interval
$[0,1)$. If we're being extra pedantic, we need to forbid an infinite trailing string of all $1$s,
since, e.g., $0.0111\ldots = 0.1000\ldots$ after carrying the $1$s (akin to the fact
that, e.g., $0.0999\ldots = 0.1$). After dealing with this detail, we will have shown that
$\bigl| [0,1]\bigr| \ne |\NN|$.


\bye
