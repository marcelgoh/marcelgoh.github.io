\input fontmac
\input mathmac

\def\To{\Rightarrow}

\maketitle{Math 242 Tutorial 4}{prepared by}{Marcel Goh}{2 October 2025}


\bigskip

\proclaim Problem 1. For any sets $A, B\subseteq \RR$, we define the {\it difference set}
to be the set
$$A - B = \{ a- b : a\in A, b\in B\}.$$
Prove that for all sets $U,V,W\subseteq \RR$, we have the inequality
$$|U-W| \le {|U-V| \cdot |V-W|\over |V|}.$$

This
is a very fundamental inequality in the field of additive combinatorics. It is called the
{\it Ruzsa triangle inequality}, for its superficial resemblance to the ordinary triangle inequality.
(When you learn about metric spaces, you'll learn that the resemblance is actually more
than superficial.)

\proof What we have to show can be rearranged to
$$ |U-W| \cdot |V| \le |U-V|\cdot |V-W|.$$
We are done then, if we can furnish an injective function from $(U-W)\times V$
to $(U-V)\times (V-W)$.

First we need to do some bookkeeping. For each $x\in U-W$, let
$u(x)\in U$ and $w(x)\in W$ be such that $u(x) - w(x) = x$. (If there are multiple choices for
this pair $\bigl( u(x), w(x)\bigr)$, just pick one.) Now we define a function.
For $(x,v)\in (U-W)\times V$, we let
$$f(x,v) = \bigl( u(x) - v, w(x) - v\bigr).$$
By construction, the output of this function will be an element of $(U-V)\times(V-W)$, so we have
$f : (U-W)\times V \to (U-V)\times (V-W)$.

We now show that $f$ is injective. Suppose that $(x_1, v_1)\in (U-W)\times V$ and
$(x_2, v_2)\in (U-W)\times V$ are such that $f(x_1,v_1) = f(x_2,v_2)$. Then
we have
$$\bigl( u(x_1) - v_1, w(x_1) - v_1 \bigr) = \bigr( u(x_2) - v_2, w(x_2) - v_2\bigr),$$
so subtracting the second element of each pair from the first element, we have
$$ u(x_1) - w(x_1) = u(x_2) - w(x_2).$$
But by definition of the functions $u(x)$ and $w(x)$, we have $u(x_1) - w(x_1) = x_1$
and $u(x_2) - w(x_2) = x_2$. So we have found that $x_1 = x_2$. Lastly, note that
$$u(x_1) - v_1 = u(x_2) - v_2,$$
so since we have found out that $x_1 = x_2$, we must also have $v_1 = v_2$.\slug

Students of algebra will be delighted to realise that this proof did not use anything particular
about $\RR$. The Ruzsa triangle inequality is valid in any group.

\proclaim Problem 2. In class we took the least upper bound property as an axiom (the completeness axiom)
and proved the nested intervals property as well as the Archimedean property.
Assuming the nested intervals property and the Archimedean property as axioms, prove the
least upper bound property.

\proof Let $S\subseteq \RR$ be a nonempty set that is bounded from above. Let $a\in S$ and $b\in \RR$
be an upper bound of $S$. We have $a\le b$, so we can let $a_1 = a$, $b_1 = b$, and call $I_1 = [a_1, b_1]$.
Note that the length of this interval is $|I_1| = a_1 - b_1 = a-b$.

We now define a sequence of nested intervals, starting with $I_1$. For $n\ge 1$, we assume that $a_n$
and $b_n$ are already defined, and let $x_n$ be the midpoint $(b_n - a_n)/2$. There are two possibilities.
\medskip
\item{i)} If $x_n$ is an upper bound of $S$, then we let $a_{n+1} = a_n$ and $b_{n+1} = x_n$.
\smallskip
\item{ii)} If not, then let $a_{n+1} = x_n$ and $b_{n+1} = b_n$.
\medskip
Since $I_{n+1}$ is half the length of $I_n$ for all $n\in \NN$ and we had $|I_1| = a-b$, we have
$|I_n| = (a-b)/2^{n-1}$. Next, since we only set $b_{n+1} = x_n$ if $x_n$ is an upper bound of $S$
and otherwise we have $b_{n+1} = b_n$, we see that $b_n$ is an upper bound of $S$ for all $n\in \NN$.
Lastly, note that $I_n$ always contains some element of $S$. This we show by induction. It is true for $n=1$,
so for the induction step, suppose it's true for some $n\in \NN$. We want to show that $I_{n+1}$ contains
some element of $S$. If $x_n$ is an upper bound of $S$, then $(x_n, b_n]$ does not contain any element
of $S$, so $I_n = [a_n, x_n]$ must. If $x_n$ not an upper bound of $S$, then there must be some $y\in S$
with $y>x$. But $b_n$ is an upper bound of $S$, so then in this case $I_n = [x_n, b_n]$ and $y$ is in
this range.

By the nested interval property, there is an element $s\in \bigcap_{n\in \NN} I_n$. The claim is that
$s$ is the supremum of $S$. There are two things to show: that it is an upper bound of $S$, and that
no smaller element can be an upper bound of $S$.

Suppose, towards a contradiction, that there is some $y\in S$ with $y > s$. Let $\eps = y-s > 0$. By the
Archimedean property, there is some $N\in \NN$ with $N > \log_2\bigl( (b-a)/\eps\bigr) + 1$. This can
be rearranged to get
$$\eps > {b-a \over 2^{N-1}}.$$
Consider $I_N = [a_N , b_N]$, which has $b_N - a_N = (b-a) / 2^{N-1}$. Since $s\in I_N$, we have
$s \ge a_N$, and
$$ y = s+\eps \ge a_N + \eps > a_N + {b-a\over 2^{N-1}} = a_N + (b_N - a_N) = b_N.$$
This contradicts the fact that $b_N$ is an upper bound of $S$. So we see that $s$ is an upper bound of $S$.

Now let $t < s$. We must show that $t$ is not an upper bound of $S$.
This time, set $\eps = s - t > 0$.
By the Archimedean property again, we can pick $N\in \NN$ large enough so that
$$\eps > {b-a \over 2^{N-1}} = b_N - a_N.$$
Again we note that $s\in I_n = [a_N, b_N]$. Hence $s\le b_N$ and we have
$$ t = s-\eps \le b_N - \eps < b_N - (b_N - a_N) = a_N.$$
But since $I_N$ contains an element of $S$, there is some element of $S$ that is at least $a_N$, which
implies that it is greater than $t$. We conclude that $t$ is not an upper bound of $S$, which
settles the proof that $s = \sup S$.\slug

\proclaim Problem 3. Show that the set $[0,1)$ is uncountable.

\proof If we disallow infinite trailing sequences of $9$s, then every element $x\in [0,1)$ has a unique
decimal representation as
$$x = 0.d_1d_2d_3 \cdots,$$
where, for all $n\in\NN$, $d_n$ is a digit from the set $\{0,1,\ldots,9\}$.

Now we show that there is no surjective function $f : \NN\to [0,1)$. Let $f$ be any function
from $f : \NN = [0,1)$. For all $n\in \NN$, let
$$f(n) = 0.d_{n,1} d_{n,2} d_{n,3} \cdots,$$
where we always use a representation of $f(n)$ that doesn't have an infinite trailing sequence of $9$s.
The claim is that there is some $x\in [0,1)$ that differs from every $f(n)$ by at least one digit.

Consider the diagonal sequence $d_{1,1}, d_{2,2}, d_{3,3}, \ldots$. For all $n$, define
$$ c_n = \cases{d_{n,n} - 1,& if $d_{n,n} > 0$;\cr 1,& if $d_{n,n} = 0$.\cr}.$$
Let $x$ be the element of $[0,1)$ with decimal expansion
$$ x = 0.c_1 c_2 c_3 \cdots.$$
Note that none of these digits are equal to $9$, so there is no infinite trailing sequence of $9$s
in the decimal expansion of $x$. Furthermore, $f(n) \ne x$ for all $n\in \NN$, since $f(n)$ has
the digit $d_{n,n}$ in the $n$th decimal place, whereas $x$ has the digit $c_n \ne d_{n,n}$. So
$f$ is not surjective.

We have proved that there is no surjection from $\NN$ to $[0,1)$, so {\it a fortiori} there is no bijection
from $\NN$ to $[0,1)$. We conclude that $[0,1)$ is uncountable.\slug

\proclaim Problem 4. Let $x$ be a real number. Prove that for every $\eps>0$
there exist two rational numbers $q$ and $q'$ such that $q < x < q'$
and $|q-q'|<\eps$.

\proof We know from class that $\QQ$ is dense in $\RR$. We can apply this with
$x-\eps/2$ and $x$ to get $q\in \QQ$ with
$$ x - {\eps \over 2} < q < x.$$
Then we can apply it with $x$ and $x + \eps/2$ to get $q'\in \QQ$ with
$$ x< q' < x + {\eps \over 2}.$$
By construction, we have $q < x < q'$, as desired. It remains to show that
$ |q - q'| < \eps$. Well, we know that $q' < x+\eps/2$ and $q > x-\eps/2$.
Negating the latter, we get $-q < -x + \eps/2$, and adding this to the former,
we obtain
$$|q-q'| = q' - q < {\eps\over 2} + {\eps \over 2} = \eps.$$
So we are done.\slug


\bye
