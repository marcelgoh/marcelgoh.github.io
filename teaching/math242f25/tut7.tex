\input fontmac
\input mathmac

\def\To{\Rightarrow}

\maketitle{Math 242 Tutorial 7}{prepared by}{Marcel Goh}{30 October 2025}

\bigskip

\proclaim Problem \advthm. Let $x_1 > 2$ and $x_{n+1} = 1+\sqrt{x_n - 1}$ for all $n\in \NN$.
\medskip
\item{a)} Show that $(x_n)$ is decreasing and bounded from below.
\smallskip
\item{b)} Find the limit of $(x_n)$.
\medskip

\proof First we show that $x_n\ge 2$ for all $n\in\NN$. The case $n=1$ is
given as a hypothesis. Moreover, if we assume that $x_n\ge 2$ for some $n\in \NN$, then $x_n-1 \ge 1$
and $\sqrt{x_n-1}\ge \sqrt 1 = 1$ as well. Hence
$$x_{n+1} = 1 + \sqrt{x_n - 1} \ge 1 + 1 = 2,$$
and the result holds for all $n\in \NN$ by induction.

Recall that $\sqrt u \le u$ for all $u\ge 1$ (this is immediate from the inequality $u\le u\cdot u$).
Let $n\in \NN$. We showed in the previous paragraph that $x_n - 1 \ge 1$, so we have
$\sqrt{x_n - 1} \le x_n - 1$. But then
$$ x_{n+1} = 1+ \sqrt{x_n -1} \le x_n,$$
so we see that the sequence $(x_n)$ is decreasing. We have settled part (a).

For part (b), we claim that for all $n\ge \NN$,
$$0\le x_n - 2 \le {x_1 - 2\over 2^{n-1}}.$$
The first inequality was already shown in part (a). The second inequality we shall
prove by induction on $n$. It clearly holds (with equality) for $n=1$. Then for $n\ge 1$,
we have
$$
x_{n+1} - 2 = \sqrt{x_n -1} - 1 
= {(x_n - 1) - 1\over (x_n-1)+1} 
= {x_n - 2\over x_n}
\le {x_n - 2\over 2}
\le {x_1 - 2\over 2\cdot 2^{n-1}}
= {x_1 - 2\over 2^n},
$$
where the first inequality we use is $x_n\ge 2$ and the second inequality is the induction hypothesis.
Hence the claim holds for all $n\in \NN$.

From here it's easy to show that $x_n \to 2$. Let $\eps > 0$ and let $N\in \NN$
satisfy $N>\log_2\bigl((x_1 - 2)/\eps\bigr)$, which we can do
by the Archimedean property. Then for all $n\ge N$,
we have
$$ |x_n - 2| \le {x_1 - 2\over 2^n} \le {x_1 - 2\over 2^{\log_2((x_1-2)/\eps)}}
= {x_1 - 2\over (x_1 - 2)/\eps} = \eps,$$
which is what we wanted to show.\slug

Consider the sequence $y_n = x_n - 1$, so that we have $y_n > 1$ and $y_{n+1} = \sqrt{y_n}$ for all
$n$. The exercise above showed that $\lim y_n = 1$, which explains why if you start with a number greater
than $1$ in your calculator and then hit the square root button over and over, you end up getting closer
and closer to $1$.

\proclaim Problem \advthm. Determine the limits of the following sequences (the monotone convergence
can be used to prove that each of these series converges, but we will gloss over this in this
exercise).
\medskip
\item{a)} $\bigl((1+1/n)^{n+1}\bigr)$
\smallskip
\item{b)} $\bigl((1+1/(n+1))^n\bigr)$
\smallskip
\item{c)} $\bigl((1+1/n)^{2n}\bigr)$
\smallskip
\item{d)} $\bigl((1-1/n)^{n}\bigr)$
\medskip

\proof For part (a), note that
$$ \lim_{n\to\infty} \biggl(1+{1\over n}\biggr)^{n+1}
= \lim_{n\to\infty} \biggl(1+{1\over n}\biggr)^n \cdot \biggl(1+{1\over n}\biggr)
= \lim_{n\to\infty} \biggl(1+{1\over n}\biggr)^n \cdot \lim_{n\to\infty} \biggl(1+{1\over n}\biggr)
= e\cdot 1 = e,$$
by the product law for limits. Likewise, for part (b) we have
$$ \lim_{n\to\infty} \biggl(1+{1\over n+1}\biggr)^n
= \lim_{n\to\infty} \biggl(1+{1\over n+1}\biggr)^{n+1} \bigg/ \biggl(1+{1\over n+1}\biggr)
= \lim_{n\to\infty} \biggl(1+{1\over n+1}\biggr)^{n+1} \bigg/ \lim_{n\to\infty} \biggl(1+{1\over n+1}\biggr)
= e.$$
For part (c) we use the product law again to find that
$$\lim_{n\to\infty} \biggl(1+{1\over n}\biggr)^{2n}
= \lim_{n\to\infty} \biggl(1+{1\over n}\biggr)\cdot \biggl(1+{1\over n}\biggr)
= \lim_{n\to\infty} \biggl(1+{1\over n}\biggr)\cdot \lim_{n\to\infty} \biggl(1+{1\over n}\biggr)
= e\cdot e = e^2.$$
Lastly, note that for all $n\in \NN$,
$$\biggl(1-{1\over n}\biggr)^{-n} = \biggl({n-1\over n}\biggr)^{-n}
= \biggl({n\over n-1}\biggr)^n = \biggl(1 + {1\over n-1}\biggr)^n.$$
So
$$\lim_{n\to\infty} \biggl(1-{1\over n}\biggr)^{-n}
= \lim_{n\to\infty} \biggl(1 + {1\over n-1}\biggr)
= \lim_{n\to\infty} \biggl(1 + {1\over n}\biggr)^{n+1} = e,$$
by part (a). Then by the reciprocal law (proved last tutorial), we must have
$$\lim_{n\to\infty} \biggl(1-{1\over n}\biggr)^{-n} = 1/e,$$
settling part (d).\slug

\proclaim Problem \advthm. Using the monotone convergence theorem (and without using the least
upper bound property), prove the Archimedean property as well as the nested interval property.

\proof First we prove the Archimedean property. Suppose, towards a contradiction, that there exists
some $M\in \RR$ such that $n \le M$ for all $n\in \NN$. Consider $\NN$ as a sequence $(x_n)$, where
$x_n = n$ for all $n\in \NN$. Then we have $x_{n+1} = 1+x_n$ for all $n\in \NN$, so $(x_n)$
is monotone increasing, and $x_n$ is bounded above by $M$. By the monotone convergence theorem,
$x_n$ converges to some limit, call it $L$. This $L$ satisfies
$$L = \lim_{n\to\infty} x_{n+1} = \lim_{n\to\infty} (x_n + 1) = 1+ \lim_{n\to\infty} x_n = 1+L,$$
so subtracting $L$ from both sides, we arrive at the nonsense statement $0=1$. We conclude that the
Archimedean property is true.

Now we prove the nested interval property. Let $I_n = [a_n, b_n]$ (where $a_n\le b_n$ for all $n\in \NN$)
be a sequence of intervals with
$$I_1\supseteq I_2 \supseteq I_3 \supseteq \cdots.$$
Since $[a_n,b_n] \supseteq [a_{n+1}, b_{n+1}]$ for all $n\in \NN$, the sequence $(a_n)$ is monotone
nondecreasing, and for all $n\in \NN$ one has $a_n\le b_n \le b_1$, so this sequence is also
bounded from above. By the monotone convergence theorem, $(a_n)$ converges to a limit, call it $x$.

First we show that $x\ge a_n$ for all $n\in \NN$. Suppose, for a contradiction, that there is
some $k\in \NN$ such that $a_k > x$. Let $\eps = a_k - x> 0$, and by the definition of convergence
there is some $N\in \NN$ such that for all $n\ge N$, $|a_n - x| < \eps$. But now, since $a_n$
is monotone nondecreasing, choosing any $n\ge \max\{k+1,N\}$, we have
$$\eps > |a_n - x| \ge a_n - x \ge a_k - x = \eps,$$
which is a contradiction.

Next we show that $x\le b_n$ for all $n\in \NN$. Suppose, for a contradiction, that there is
some $k\in \NN$ such that $b_k < x$. Let $\eps = (x-b_k)/2$, and by the definition of convergence
there is some $N\in \NN$ such that for all $n\ge N$, $|a_n - x| < \eps$. Now, letting
$n\ge \max\{k, N\}$, we have
$$a_n > x-\eps > x-2\eps = b_k \ge b_n,$$
which contradicts our assumption that $a_n\le b_n$.

We have shown that $x\in [a_n,b_n]$ for all $n\in \NN$, so
$$x\in \bigcup_{n\in \NN} I_n,$$
and our proof of the nested interval theorem is complete.\slug

This, combined with Problem~2 of our tutorial on 2~October, shows that the monotone convergence
theorem is logically equivalent to the least upper bound property. (And the nested interval
property combined with the Archimedean property are together equivalent to either of these
as well.)

\bye
