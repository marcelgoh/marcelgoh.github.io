\input fontmac
\input mathmac

\def\To{\Rightarrow}

\maketitle{Math 242 Tutorial 1}{prepared by}{Marcel Goh}{11 September 2025}

\bigskip

\proclaim Problem 1. Let $p$ and $q$ be statements. Show that $p\To q$ is equivalent to $\neg q \To \neg p$.

\proof We have
$$\eqalign{
p\To q &\equiv \neg p \vee q \cr
&\equiv q \vee \neg p \cr
&\equiv \neg (\neg q) \vee \neg p \cr
&\equiv \neg q \To \neg p \noskipslug\cr
}$$

\proclaim Problem 2. Let $n$ be a positive integer. Show that if $n$ is not a perfect square, then $\sqrt n$
is irrational.

\proof By contrapositive. Suppose $\sqrt n$ is rational. Then there exist $a,b\in \ZZ$ (with $b\ne 0$)
such that $\sqrt n = a/b$. Without loss of generality, we can assume that $\gcd(a,b) = 1$. We see that
$n = a^2/b^2$, but since $n$ is an integer, and $a^2$ and $b^2$ have no common factors, we must have
$b^2 = 1$. This forces $n=a^2$, which shows that $n$ is a perfect square.\slug

\proclaim Problem 3. Prove the identity
$$\sum_{k=1}^n k^3 = \biggl({n(n+1)\over 2}\biggr)^2$$
for all integers $n\ge 1$.

\proof By induction on $n$. For the base case $n=1$, we have
$$\sum_{k=1}^1 k^3 = 1^3 = 1^2 = \biggl( {1\cdot 2\over 2}\biggr)^2 = 
\biggl( {1\cdot (1+1)\over 2}\biggr)^2.$$
Now suppose the statement holds for some $n$. For $n+1$ we have
$$\eqalign{
\sum_{k=1}^{n+1} k^3 &= (n+1)^3 + \sum_{k=1}^n k^3 \cr
&= (n+1)^3 + \biggl({n(n+1)\over 2}\biggr)^2 \cr
&= {4(n+1)(n+1)^2 + n^2 (n+1)^2\over 4} \cr
&= {(n^2 + 4n + 4)^2(n+1)^2 \over 4} \cr
&= \biggl({(n+1)\bigl((n+1)+1\bigr)\over 2}\biggr)^2,\cr
}$$
where in the second line, we used the induction hypothesis.\slug

% \proclaim Problem 4. Prove the Cauchy--Schwarz inequality, the statement that for all
% $u_1,\ldots,u_n, v_1,\ldots,v_n\in \RR$,
% $$\biggl(\sum_{i=1}^n u_i v_i\biggr)^2
% \le \biggl( \sum_{i=1}^n {u_i}^2\biggr) \biggl( \sum_{i=1}^n {v_i}^2\biggr).$$
% [{\it Hint:} Recall from your high school days that
% for a quadratic polynomial $ax^2 + bx + c$, the {\it discriminant} is
% defined to be the quantity $b^2 - 4ac$. (It appears under the square-root sign
% in the quadratic formula.) If it is negative, the polynomial has no real roots,
% if it is positive, the polynomial has two real roots, and if it is zero, the polynomial
% has one real root.]
% 
% \proof If $u_i = 0$ for all $1\le i\le n$, then both sides are $0$ and the inequality
% is trivially true. So suppose that at least one of the $u_i$ is nonzero. Then
% $$p(x) = (u_1 x + n_1)^2 + (u_2 x + v_2)^2 + \cdots + (u_n x+v_n)^2$$
% is a quadratic polynomial in the variable $x$. Rewriting $p(x)$ in summation notation,
% we have
% $$\eqalign{
% p(x) &= \sum_{i=1}^n (u_ix + v_i)^2\cr
% &= \sum_{i=1}^n \bigl( u_i^2 x^2 + 2u_i v_i + v_i^2\bigr)\cr
%  &= \biggl(\sum_{i=1}^n u_i^2 \biggr) x^2 +
% 2\biggl(\sum_{i=1}^n u_i v_i\biggr) x + \biggl( \sum_{i=1}^n {v_i}^2\biggr).\cr
% }$$
% Since it is defined as a sum of squares, $p(x)\ge 0$ for all $x\in \RR$, so it has
% at most one real root.
% This means its discriminant is nonpositive; in other words,
% $$4\biggl(\sum_{i=1}^n u_i v_i\biggr)^2
% - 4\biggl( \sum_{i=1}^n {u_i}^2\biggr) \biggl( \sum_{i=1}^n {v_i}^2\biggr) \le 0.$$
% Dividing through by $4$ and then rearranging gives exactly the inequality we sought.\slug

\proclaim Problem 4. Suppose that a local McDonalds branch sells Chicken McNuggets in meals of $4$
or $9$. Prove that for all $n\ge 36$, it is possible to buy exactly $n$ Chicken McNuggets.

\proof Formulated precisely in mathematical language, what we are trying to show is that for every
integer $n\ge 36$, there exist nonnegative integers $a$ and $b$ such that $n=9a + 4b$.
We perform a proof by strong induction. Note first that
$$ 36 = 9\cdot 4 + 4\cdot 0, \qquad 37 = 9\cdot 1 + 4\cdot 7, \qquad
38 = 9\cdot 2 + 4\cdot 5,$$
and
$$39 = 9\cdot 3 + 4\cdot 3.$$
Now let $n\ge 40$ and assume that the statement holds for all integers in the range $[36, n]$.
In particular, it holds for $n-4$. So there exist nonnegative integers $a$ and $b$ such that
$n-4 = 9a + 4b$, and we see that $n = 9a + 4(b+1)$.\slug

\proclaim Problem 5. Show that there exist irrational numbers $a$ and $b$ such that $a^b$ is rational.

\proof Consider the number $(\sqrt 2)^{\sqrt 2}$. It is either rational or irrational.

If it is rational, then we can set $a = b = \sqrt 2$, which we already showed to be irrational in class,
and $a^b$ is rational by the assumption in this case.

If it is irrational, then we can set $a =(\sqrt 2)^{\sqrt 2}$ and $b=\sqrt 2$. We compute
$$a^b = \bigl((\sqrt 2)^{\sqrt 2}\bigr)^{\sqrt 2} = (\sqrt 2)^{\sqrt 2 \cdot \sqrt 2}
= (\sqrt 2)^2 = 2,$$
which is rational.\slug

% \proclaim Problem 7. Let $n\ge 1$ be an integer.
% Prove that the number of subsets of $\{1,2,\ldots,n\}$ is $2^n$.
% 
% \proof By induction. The number of subsets of $\{1\}$ is $2 = 2^1$, since the two subsets are $\emptyset$
% and $\{1\}$. Now assume the statement is true for $n$, and note that every subset $A$ of $\{1,2,\ldots,n,n+1\}$
% either contains $n+1$ or it doesn't. If $A$ doesn't contain $n+1$, then $A\subseteq \{1,2,\ldots,n\}$,
% and there are $2^n$ possibilities for $A$, by the induction hypothesis. If $A$ does contain $n+1$,
% then $A = B\cup\{n+1\}$ for some $B\subseteq \{1,2,\ldots,n\}$, and again there are $2^n$ possibilities
% for $A$ in this case. We conclude that
% there are $2\cdot 2^n = 2^{n+1}$ subsets of $\{1,2,\ldots,n+1\}$ in total.\slug

\bye
