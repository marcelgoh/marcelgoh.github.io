\input fontmac
\input mathmac

\def\To{\Rightarrow}
\def\s{\mathord{*}}

\maketitle{Math 242 Tutorial 9}{prepared by}{Marcel Goh}{13 November 2025}

\bigskip

\proclaim Problem \advthm. Let $A\subseteq \RR$. Show that the boundary of $A$ is equal
to the boundary of the complement of $A$.

\proof Simply note that
$$\eqalign{
\partial A
&= \bigl\{ x\in \RR : \hbox{for all}\ \eps>0,
\ V_\eps(x)\cap A\ne\emptyset\ \hbox{and}\ V_\eps(x)\cap A^c\ne\emptyset\bigr\} \cr
&= \bigl\{ x\in \RR : \hbox{for all}\ \eps>0,
\ V_\eps(x)\cap (A^c)^c \ne\emptyset\ \hbox{and}\ V_\eps(x)\cap A^c\ne\emptyset\bigr\} \cr
&= \partial(A^c). \noskipslug\cr
}$$

\proclaim Problem \advthm. Prove the following:
\medskip
\item{a)} A set $U\subseteq \RR$ is open if and only if $U$ doesn't contain any of
its boundary points.
\smallskip
\item{b)} A set $A\subseteq \RR$ is closed if and only if $A$ contains all of its
boundary points.
\medskip

\proof Suppose that $U$ contains a boundary point $x\in \partial U\cap U$. Then
for this choice of $x\in U$, for any $\eps > 0$, $V_\eps(x)\cap U^c \ne\emptyset$.
This means that $V_\eps(x)\not\subseteq U$ for all $\eps>0$. So $U$ is not open.
Conversely, if $U$ is not open, there is some $x\in U$ such that for all
$\eps > 0$, $V_\eps(x)\not\subseteq U$, meaning that $V_\eps(x)\cap U^c \ne\emptyset$.
We also see that $V_\eps(x)\cap U$ contains $x$, so it is also nonempty.
This means that $x$ is a boundary point of $U$; that is, $x$ contains at least one
of its boundary points.

By the previous paragraph, $A^c$ is open if and only if $A^c$ doesn't
contain any boundary points of $A^c$, and by the previous paragraph
this is true if and only if it doesn't contain
any boundary points of $A$.\slug

\proclaim Problem \advthm. Let $\emptyset \ne S\subseteq \RR$. Prove the following.
\medskip
\item{a)} If $S$ is bounded from above, then $\sup S$ is a boundary point of $S$.
\smallskip
\item{b)} If $S$ is bounded from below, then $\inf S$ is a boundary point of $S$.
\medskip

\proof Let $\eps > 0$. Since $\sup S - \eps$ is not an upper bound of $S$, there must
exist some $s\in S$ with $s>\sup S-\eps$. Since $\sup S$ is an upper bound of $S$,
we find that
$$s\in S\cap (\sup S-\eps, \sup S] \subseteq S \cap V_\eps(\sup S).$$
Also, since
$\sup S$ is an upper bound of $S$, the real number $t = \sup S+\eps/2$ is not a member of $S$.
So
$$t\in S^c \cap (\sup S, \sup S+\eps) \subseteq S^c \cap V_\eps(\sup S).$$

The proof of part (b) is similar.\slug

\medskip\boldlabel The Cantor set.
For any scalar $a\in \RR$ and set $S\subseteq \RR$, define the {\it dilate}
$$a\cdot S = \{ a\cdot s : s\in S\}$$
as well as the {\it translate}
$$a+\cdot S = \{a + s : s\in S\}.$$
For integers $n\ge 0$, define $C_n$ recursively by
$C_0 = [0,1]$ and
$$ C_{n+1} = {1\over 3}\cdot C_n \cup \biggl( {2\over 3} + {1\over 3}\cdot C_n\biggr)$$
for $n\ge 0$.
(In other words, $C_{n+1}$ is obtained by removing the ``open middle third'' from $C_n$.)
The {\it Cantor set} $C$ is defined to be $C = \bigcap_{n=0}^\infty C_n$.

\proclaim Problem \advthm. Prove the following.
\medskip
\item{a)} The Cantor set $C$ is closed.
\smallskip
\item{b)} If $a,b\in \RR$ with $a\le b$ are such that $[a,b]\subseteq C$, then $a=b$.
(Hence the only closed intervals in $C$ are singleton sets $\{a\}$.)
\smallskip
\item{c)} The cardinality of $C$ is equal to that of $[0,1]$. (That is, $C$ is uncountable.)
\medskip

\proof Note that $C_0$ is just a closed interval, so {\it a fortiori} it is a finite union
of closed intervals. Now assume that $C_n$ is a finite union of closed intervals. Then
$(1/3)\cdot C_n$ is also a finite union of (the same number of) closed intervals,
and so is $(2/3)+(1/3) C_n$. Hence $C_{n+1}$ is a finite union of (twice as many) closed intervals.
We see by induction that $C_n$ is closed for all $n\ge 0$, and hence $C = \bigcap_{n=0}^\infty$
is closed, being an intersection of closed sets.

For part (b), first observe that $C_0$ is an interval of length $1 = 1/3^0$. Next, assume
that $C_n$ is composed of (a finite number of) pairwise disjoint
intervals, each of length $1/3^n$. Then
$C_{n+1}$ is obtained by taking the disjoint union of two copies of $C_n$, each dilated by $1/3$, so
we conclude that $C_{n+1}$ consists of pairwise disjoint
intervals of length $1/3^{n+1}$. By induction, for all
integers $n\ge 0$, $C_n$ is composed of disjoint intervals of length $1/3^n$. Now, given $a,b\in \RR$
with $a\le b$. If $a<b$, then $b-a > 0$, so we can pick $N > 1/(b-a)$ by the Archimedean property.
Then $C_N$ is composed of disjoint intervals of length $1/3^N < b-a$. We conclude that
$[a,b]$ is not a subset of $C^N$, hence it cannot be contained in $C$.

Lastly, we show that $C$ is uncountable. Note that every real number $x\in [0,1]$ can
be written as
$$x = \sum_{n=1}^\infty {t_n\over 3^n},$$
where $t_n\in \{0,1,2\}$ for all $n\in \NN$.
This is the representation of
$$x = 0.t_1t_2t_3\ldots$$
in base-$3$ (ternary) notation. Ternary representations aren't unique. For instance,
$0.1 = 0.22222\ldots$, just like $0.1 = 0.999999\ldots$ in decimal notation.
In ternary notation, $0.022222\ldots = 1/3$ and $0.2 = 2/3$, so the
first middle third we removed (in going from $C_0$ to $C_1$) contains all numbers of the form
$0.1\s\s\s\ldots$, where the $\s\s\s\ldots$ part can be anything strictly between $000\ldots$
and $222\ldots$. That is, $C_1$ contains all numbers of the form $0.0\s\s\s\ldots$ and
$0.2\s\s\s\ldots$, with no restriction whatsoever on the digits $\s\s\s\ldots$.

In going from $C_1$ to $C_2$, we remove the middle third of $[0,1/3]$ as well as the middle third
of $[2/3,1]$. By the same logic we applied above, we see that $C_2$ contains all numbers
of the form $0.00\s\s\s\ldots$, $0.02\s\s\s\ldots$, $0.20\s\s\s\ldots$, or $0.22\s\s\s\ldots$, with no
restrictions on the digits $\s\s\s\ldots$. By induction, we find every $x\in C$ can be expressed
as
$$x = \sum_{n=1}^\infty {t_n\over 3^n}$$
where $t_n\in \{0,2\}$ for all $n\in \NN$.

Let $f : C\to [0,1]$ map
$$ x = \sum_{n=1}^\infty {t_n \over 3^n},$$
where $t_n\in \{0,2\}$ for all $n\in \NN$, to
$$ f(x) = \sum_{n=1}^\infty {b_n\over 2^n},$$
where we have set
$$b_n = \cases{0, & if $t_n = 0$;\cr 1, & if $t_n = 2$.}$$
We claim that $f$ is surjective. Every $y\in [0,1]$ has at least one binary representation
$$ y = \sum_{n=1}^\infty {b_n\over 2^n},$$
so letting
$$t_n = \cases{0, & if $b_n = 0$;\cr 2, & if $b_n = 1$}$$
and
$$ x = \sum_{n=1}^\infty {t_n \over 3^n},$$
we find that $f(x) = y$. The fact that there is a surjective function $f:C\to [0,1]$ means that
the cardinality of $C$ is at least that of $[0,1]$. But $C\subseteq [0,1]$, so its cardinality is
at most that of $[0,1]$. Hence $|C| = \bigl|[0,1]\bigr|$, and we are done.\slug

\bye
