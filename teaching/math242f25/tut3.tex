\input fontmac
\input mathmac

\def\To{\Rightarrow}

\maketitle{Math 242 Tutorial 3}{prepared by}{Marcel Goh}{25 September 2025}

\bigskip

\proclaim Problem 1. Show that the supremum and infimum of a set are both unique.

\noindent[Only after we prove this statement does it makes sense to say, e.g.,
{\it the} supremum of a set, rather than just {\it a} supremum. Hence in the proof we can only refer
to something as being {\it a} supremum.]

\proof We present the proof for suprema. The proof for infima is the same, {\it mutatis mutandis}
(do it as an exercise).

Let $S\subseteq \RR$ be a set and let $s$ and $s'$ both be suprema of $S$.
We want to show that $s = s'$.
Suppose, for a contradiction, that $s\ne s'$. Without loss of generality, assume that
$s<s'$ (by switching the roles of $s$ and $s'$ if necessary).
Since $s'$ is a supremum of $S$, for all $y < s'$ there exists some $x\in S$ such that
$x > y$. Well since $s<s'$ we can apply this with $y=s$ to find some
$x\in S$ with $x > s$. This means that $s$ is not an upper bound of the set $S$,
contradicting our assumption that $s$ was a supremum of $S$. We conclude that $s = s'$.\slug

\proclaim Problem 2. Let $S$ be a nonempty subset of $\RR$ that is bounded from below.
Prove that a real number $t$ is the infimum of $S$ if and only if
\medskip
\item{i)} $t$ is a lower bound of $S$; and
\smallskip
\item{ii)} for all $\eps > 0$, the number $t + \eps$ is {\it not} a lower bound of $S$.
\medskip

\noindent[An analogue of this statement holds for suprema as well (see Question 1, part (a), on Assignment 3).
The proof is the same, so we'll just do the infimum version. You should be able to do the other
statement as an easy exercise by making suitable modifications in this proof.]

\proof First suppose that $t$ is the infimum of $S$. Then it is a lower bound of $S$, so (i) is
immediately satisfied. To show part (ii), let $\eps > 0$ be given. The number $t' = t+\eps$
satisfies $t' > t$, so since $t$ is the greatest lower bound of $S$,
we see that $t+\eps = t'$ is not a lower bound of $S$.

Now suppose that $t$ satisfies conditions (i) and (ii) above. We need to show that it is the
greatest lower bound of $S$. It is certainly a lower bound, by condition (i). To show that it is
the greatest lower bound, let $t'\in \RR$ be a number with $t' > t$. We can let $\eps = t'-t > 0$, so
that $t' = t + \eps$. Then condition (ii) tells us $t' = t+ \eps$ that $t'$ is not a lower bound
of $S$. Hence $t$ is the infimum of $S$.\slug

\proclaim Problem 3. Let $S = \bigl\{ 1 - 1/n^2 : n\in \NN\bigr\}$. Prove that $\sup S = 1$
and $\inf S = 0$.

\proof For $n = 1$ we have $1 - 1/n^2 = 1-1 = 0$. So $0\in S$. On the other hand, $0$
is also a lower bound of $S$, since
$$ 1 - {1\over n^2} \ge 1-{1\over 1^2} = 0 $$
for all $n\in \NN$. This shows that $0=\min S$ and hence $0 = \inf S$.

To show that $\sup S = 1$, we need to show that $1$ is an upper bound of $S$ and that $1-\eps$
is not an upper bound of $S$ for any $\eps > 0$. (This is in light of the supremum version
of the previous problem, or Question 1, part (a) of Assignment 3.) To see that $1$ is an upper bound,
note that for any $n\in \NN$, $1/n^2 > 0$, so $1-1/n^2 < 1$. For the second part, let $\eps > 0$ be
given and pick $n>1/\sqrt\eps$. Then
$$ 1- {1\over n^2} > 1 - {1\over 1/(\sqrt\eps)^2} = 1-{1\over 1/\eps} = 1-\eps,$$
so $1-\eps$ is not an upper bound of $S$. Since $\eps$ was arbitrary, we are done.\slug

\proclaim Problem 4. For any nonempty set $A\subseteq \RR$ define the {\it reflection}
$$ -A = \{ - a : a\in A \}.$$
Prove the following.
\medskip
\item{a)} The number $u$ is an upper bound of $A$ if and only if $-u$ is a lower bound of $-A$.
\smallskip
\item{b)} The set $A$ is bounded from above if and only if $-A$ is bounded from below.
\smallskip
\item{c)} If $A$ has a supremum, then $\inf(-A) = -\sup A$.
\smallskip
\item{d)} Any nonempty set $B\subseteq \RR$ that is bounded from below has an infimum.
\medskip

\proof Let $u$ be an upper bound of $A$. This means that $a\le u$ for all $a\in A$. But then
we have $-a \ge -u$ for all $a\in A$, and we see that $-u$ is a lower bound of $-A$. Conversely,
assume that $-u$ is a lower bound of $-A$. Then $-a \ge -u$ for all $a\in A$, so $a\le u$
for all $a\in A$, which implies that $u$ is an upper bound of $A$. [You could also just have
noticed straightaway that each of these implications is an equivalence, and proved the ``if and only if''
all at once.]

By definition, $A$ is bounded from above if and only if there is some $u\in \RR$
that is an upper bound of $A$. But by part (a),
this is true if and only if there is some $u\in \RR$ such that $-u$ is a lower bound of $-A$,
and this is the definition of $-A$ being bounded from below.

Now we prove part (c). To do so, we must show that $-\sup A$ is a lower bound of $-A$ and that
$-\sup A + \eps$ is not a lower bound of $-A$ for any $\eps > 0$. Well, $\sup A$ is an upper bound
of $A$, so by part (a), $-\sup A$ is a lower bound of $-A$. That settles one of the requirements. For
the other one, let $\eps > 0$ be given. Since $\sup A$ is the least upper bound of $A$,
$\sup A - \eps$ is not an upper bound of $A$; that is, there is some $a\in A$ with $a > \sup A - \eps$.
But this implies that $-a < -\sup A + \eps$, so $-\sup A + \eps$ is not a lower bound of $-A$.
Hence $-\sup A$ is the infimum of $-A$.

Let $B$ be nonempty and bounded from below. Let $A = -B$, so that $A$ is also nonempty
and $-A = B$. By part (b), $A$ is bounded from
above, so the least-upper-bound property of $\RR$ implies that $A$ has a supremum. Now by part (c),
$\inf B = \inf(-A) = -\sup A$, and {\it a fortiori} $\inf B$ exists.\slug

\proclaim Problem 5. Prove the Cauchy--Schwarz inequality, the statement that for all
$u_1,\ldots,u_n, v_1,\ldots,v_n\in \RR$,
$$\biggl(\sum_{i=1}^n u_i v_i\biggr)^2
\le \biggl( \sum_{i=1}^n {u_i}^2\biggr) \biggl( \sum_{i=1}^n {v_i}^2\biggr).$$
[{\it Hint:} Recall from your high school days that
for a quadratic polynomial $ax^2 + bx + c$, the {\it discriminant} is
defined to be the quantity $b^2 - 4ac$. (It appears under the square-root sign
in the quadratic formula.) If it is negative, the polynomial has no real roots,
if it is positive, the polynomial has two real roots, and if it is zero, the polynomial
has one real root.]

\proof If $u_i = 0$ for all $1\le i\le n$, then both sides are $0$ and the inequality
is trivially true. So suppose that at least one of the $u_i$ is nonzero. Then
$$p(x) = (u_1 x + v_1)^2 + (u_2 x + v_2)^2 + \cdots + (u_n x+v_n)^2$$
is a quadratic polynomial in the variable $x$. Rewriting $p(x)$ in summation notation,
we have
$$\eqalign{
p(x) &= \sum_{i=1}^n (u_ix + v_i)^2\cr
&= \sum_{i=1}^n \bigl( u_i^2 x^2 + 2u_i v_i + v_i^2\bigr)\cr
 &= \biggl(\sum_{i=1}^n u_i^2 \biggr) x^2 +
2\biggl(\sum_{i=1}^n u_i v_i\biggr) x + \biggl( \sum_{i=1}^n {v_i}^2\biggr).\cr
}$$
Since it is defined as a sum of squares, $p(x)\ge 0$ for all $x\in \RR$, so it has
at most one real root.
This means its discriminant is nonpositive; in other words,
$$4\biggl(\sum_{i=1}^n u_i v_i\biggr)^2
- 4\biggl( \sum_{i=1}^n {u_i}^2\biggr) \biggl( \sum_{i=1}^n {v_i}^2\biggr) \le 0.$$
Dividing through by $4$ and then rearranging gives exactly the inequality we sought.\slug

\bye
