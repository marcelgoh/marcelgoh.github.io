\input fontmac
\input mathmac

\def\To{\Rightarrow}

\maketitle{Math 242 Tutorial 8}{prepared by}{Marcel Goh}{6 November 2025}

\bigskip

\proclaim Problem \advthm. Prove that the Bolzano--Weierstrass theorem implies the monotone
convergence theorem (without using the least upper bound property or anything else we have
shown to be equivalent to it).

\proof Let $(x_n)$ be a monotone nondecreasing sequence that is bounded above; that is, there
is $M\in \RR$ such that $x_n \le M$ for all $n\in \NN$. Then $x_n\in [x_1,M]$ for all
$n\in \NN$ and by the Bolzano--Weierstrass theorem, there is a subsequence $(x_{n_k})$ of
$(x_n)$ as well as some $x\in \RR$ such that $x_{n_k}\to x$ as $k\to\infty$.

First we claim that $x_n \le x$ for all $n\in \NN$. If not, then there is some $r\in \NN$
with $x_r > x$, and letting $\eps = x_r - x$, we see there must be some
$K\in \NN$ such that $\bigl|x_{n_k} - x\bigr| \le \eps$ for all $k\ge K$. Now pick $k\ge K$
large enough so that $r\le n_k$ (this we can do because $n_k$ is an infinite sequence).
For this choice of $k$ we have
$$\eqalign{
x_r - x_{n_k} &= x_{n_r} - x - (x_{n_k} - x)
= \eps - (x_{n_k} - x)
\ge \eps - |x_{n_k} - x|
> \eps - \eps 
= 0,
}$$
which means that $x_r > x_{n_k}$, contradicting the monotonicity of $(x_n)$.

Now we show that $(x_n)$ also converges to $x$.
Let $\eps > 0$. Since the subsequence $(x_{n_k})$ converges to x, there is some $K\in \NN$ such that
for all $k\ge K$,
$$\bigl| x_{n_k} - x\bigr| < \eps.$$
We choose $N = n_k$. Then for all $n\ge N$, we have
$$ x - \eps < x_{n_k} = x_N \le x_n \le x,$$
where $x_N \le x_n$ follows from monotonicity and the last inequality was
proved in the previous paragraph.
Hence in particular, $|x_n - x| < \eps$.\slug

\proclaim Problem \advthm. Cauchy's criterion asserts that every Cauchy sequence converges to a limit.
Assume only Cauchy's criterion and the Archimedean property, and prove the Bolzano--Weierstrass
theorem. (Again, do not resort to the least upper bound property or anything else that we have
shown is equivalent to it.)

\proof Let $(x_n)$ be a sequence that is bounded; that is, there are $a,b\in \RR$ with $a\le b$
such that
$x_n\in [a, n]$ for all $n\in \NN$. We shall build a sequence $(x_{n_k})$ recursively.
For the base case, let
$a_1 = a$ and $b_1 = b$. The interval $[a_1, b_1]$ has infinitely many $x_n$ residing in it.
Let $n_1$ be any integer with $x_{n_1}\in [a_1, b_1]$.

Now for $k\ge 1$, assume that the interval $[a_k, b_k]$ has infinitely many $x_n$ in it. Let
$c_k = (a_k + b_k)/2$ be the midpoint of this interval, and note that (at least) one of
the two intervals $[a_k, c_k]$ and $[c_k, b_k]$ contain infinitely many elements of $(x_n)$.
If it is the first, then let $a_{k+1} = a_k$ and $b_{k+1} = c_k$; otherwise,
let $a_{k+1} = c_k$ and let $b_{k+1} = b_k$. Thus $[a_{k+1}, b_{k+1}]$ contains infinitely
many members of $(x_n)$, and because of this, we can pick $n_{k+1} > n_k$ such that
$x_{n_{k+1}}\in [a_{k+1}, b_{k+1}]$.

It is easy to show by induction that $b_k - a_k \le (b-a)/2^{k-1}$ for all $k\in \NN$.
From this, we can prove
that the sequence $(x_{n_k})$ is Cauchy. Let $\eps > 0$ and pick $K > \log_2 \bigl( (b-a)/\eps\bigr) +1$
by the Archimedean property. Then for any $i,j\ge K$, both $x_{n_i}$
and $x_{n_j}$ are in the interval $[a_K, b_K]$, and consequently
$$\bigl| x_{n_i} - x_{n_j}\bigr| \le b_K - a_K \le {b-a\over 2^{K-1}} < \eps,$$
by our choice of $K$.

But by Cauchy's criterion, every Cauchy sequence is convergent. Hence the subsequence
$(x_{n_k})$ is convergent. This proves the Bolzano--Weierstrass theorem.\slug

\medskip\boldlabel Divergence to $\mathbold\infty$
or $\mathbold-\infty$. We now set up some definitions for the remainder of the tutorial.
We shall say that a sequence $(x_n)$ {\it diverges to $\infty$} if for every
$M\in \RR$ there exists some $N\in \NN$ such that $x_n > M$ for all $n\ge N$.
We say that a sequence $(x_n)$ {\it diverges to $-\infty$} if for all $M\in \RR$ there exists
some $N\in \NN$ such that $x_n < M$ for all $n\ge N$.A

Note that ``diverging to $\infty$'' or ``diverging to $-\infty$'' is not the same as just
``not converging to any $x\in \RR$''. For example, the sequence
$$0,-1,1,0,-2,2,0,-3,3,\ldots$$
does not converge to any $x\in \RR$, nor does it converge to either $-\infty$ or $\infty$. However,
the following result about subsequences is true.

\proclaim Problem \advthm.
Show that for any sequence $(x_n)$ of real numbers, at least one of the following statements holds.
\medskip
\item{a)} There is a subsequence $(x_{n_k})$ of $(x_n)$ that diverges to $\infty$.
\smallskip
\item{b)} There is a subsequence $(x_{n_k})$ of $(x_n)$ that diverges to $-\infty$.
\smallskip
\item{c)} There is some $x\in \RR$ and a subsequence $(x_{n_i})$ of $(x_n)$ such that
$(x_{n_i})$ converges to $x$ as $i\to\infty$.
\medskip

\proof Let $(x_n)$ be any sequence of real numbers.
If (a) holds we are done. If (b) holds we are done. So we assume that (a) and (b)
are both false, and our goal is to prove (c).

Since (a) is false, the sequence $(x_n)$ does not diverge to $\infty$. Hence there exists
$b \in \NN$ such that for all $N\in \NN$, there exists $n\ge N$ with $x_n \le b$.
Let $n_1\ge 1$ such that $x_n\le b$, then let $n_2 > n_1$ such that $x_2\le b$, and so on.
In this manner, we obtain a subsequence $(x_{n_k})$ that is bounded from above by $b$.

Since (b) is false, the sequence $(x_{n_k})$ does not diverge to $-\infty$. So there
is $a\in \NN$ such that for all $K\in \NN$ there is always some $k\ge K$ such that
$x_{n_k} \ge a$. By logic similar to that in the previous paragraph, we see there is a subsequence
$(x_{n_j})$ of $(x_{n_k})$ that is bounded from below by $a$, and hence contained in the
interval $[a,b]$. (We have ``flattened'' out the notation here by simply redefining $n_j = n_{k_j}$.
If we want to be extra precise
with notation, we should have written $(x_{n_{k_j}})$,
since this is a subsequence of $(x_{n_k})$, but this kind of pedantry only creates unreadable nonsense
when we take subsequences of subsequences of subsequences, as in this proof.)

We have found that there is a subsequence $(x_{n_j})$ of $(x_n)$ that is contained in the interval
$[a,b]$. Hence by the Bolzano--Weierstrass theorem, there is a subsequence $(x_{n_i})$
of $(x_{n_j})$ that converges to some limit $x\in \RR$. Of course, $(x_{n_i})$ is a subsequence
of our original sequence $(x_n)$, so we are done.\slug

This proof shows that $\RR\cup \{\infty, -\infty\}$ is a valid {\it compactification} of $\RR$.
This terminology will make more sense once you learn more about topology.

\proclaim Problem 4. Show that $(a^n)$ diverges to $\infty$ for any $a>1$.

\proof Let $M\in \RR$ be given. If $M\le 1$, then since $a^n\ge a>1$ for all $n\in \NN$,
we can simply pick $N=1$ and $a_n> M$ for all $n\ge N$. If $M>1$, then
we can pick $N\in \NN$ satisfying $N>\log_a M > 0$, by using the Archimedean property.
Then for all $n\ge N$, we have
$$a^n \ge a^N > a^{\log_a M} = M.$$
Since $M$ was arbitrary, we see that $(a^n)$ diverges to $\infty$.\slug

\bye
