\input fontmac
\input mathmac

\def\To{\Rightarrow}

\maketitle{Math 242 Tutorial 11}{prepared by}{Marcel Goh}{27 November 2025}

\bigskip

\proclaim Problem \advthm. Let $D\subseteq \RR$ and let $f,g:D\to \RR$ be functions
that are continuous on all of $D$.
\medskip
\item{a)} Show that the function $f-g$ is continuous on all of $D$.
\smallskip
\item{b)} Suppose further that for all $x\in D$ we also have $-x\in D$. (So $D$ is symmetric
about the origin.) Let $h:D\to \RR$ be given by $h(x) = f(-x)$. Show that $h$ is continuous
on all of $D$.
\medskip

\proof Let $p\in D$. Since $f$ and $g$ are both continuous at $p$, we have
$\lim_{x\to p} f(x) = f(p)$ and $\lim_{x\to p} g(x) = g(p)$. Hence
$$\lim_{x\to p} (f-g)(x) = \lim_{x\to p} f(x) - \lim_{x\to p} g(x) = f(p) - g(p),$$
by the algebraic laws for limits of functions.

Now we prove part (b).
Let $\eps > 0$ and let $p\in D$. Since $-p\in D$ and $f$ is continuous on all of $D$,
$f$ is continuous at $-p$. Hence there exists
some $\delta > 0$ such that for all $y\in D$ with $|y-p|< \delta$, one has
$\bigl| f(y) - f(-p)\bigr| < \eps$. Let $x\in D$ be such that $|x-p|<\delta$. Then
$$\bigl|-x-(-p)\bigr| = \bigl|-(x-p)\bigr| = |x-p| < \delta,$$
so letting $y=-x$ above gives tells us that
$$\bigl| h(x) - h(p)\bigr| = \bigl| f(-x) - f(-p)\bigr| < \eps.$$
This shows that $h$ is continuous at $p$, and since $p$ was selected arbitrarily, we conclude that
$h$ is continuous on all of $D$.\slug

These statements generalise to $D\subseteq \RR^n$, a fact you may use in the following problem.

\proclaim Problem \advthm. Let $S^2$ denote a two-dimensional sphere in $\RR^3$,
which can be shifted and scaled so that
$$S^2 = \bigl\{ (x,y,z)\in \RR^3 : x^2 + y^2 + z^2 = 1\bigr\}.$$
Let $f:S^2 \to \RR$ be continuous.
Using the location of roots theorem, show that there exists a point $p\in S^2$
such that $f(-p) = f(p)$.

\proof The set $S^2$, as we have defined it above, is symmetric about the origin of $\RR^3$,
so by the previous problem, the function $g : S^2 \to \RR$ given by $g(v) = f(v)  - f(-v)$ is
continuous. Now fix any point $v\in S^2$ and consider any semicircular arc in $S^2$ from $v$ to $-v$.
Let $\gamma : [0,1]\to \RR^3$ parameterise this arc, so that $\gamma(0) = v$ and $\gamma(1) = -v$ (and
say, $\gamma(1/3)$ is one-third of the way from $v$ to $-v$ along this arc). This is a continuous
function (this technically needs to be proved, but for the purposes of this class, let's just
take it as fact).

Let $h:[0,1]\to \RR$ be given by $h(t) = g\bigl(\gamma(t)\bigr)$. This is continuous, since
it is the composition of continuous functions. First, note that
$$h(0) = g\bigl(\gamma(0)\bigr) = f(v) - f(-v) = -\bigl(f(-v) - f(v)\bigr)
= -g\bigl(\gamma(1)\bigr).$$
If $h(0) = 0$, then $g\bigl(\gamma(0)\bigr) = 0$,
meaning that $g(v) = 0$; that is, $f(v) = f(-v)$, so we can set $p=v$ and we are done.
If $h(0)\ne 0$, then $h(1)\ne 0$ also, and they have opposite signs. Hence by the location
of roots theorem, there is some $t\in (0,1)$ with $h(p) = 0$. Let $p = \gamma(t)$; we have
$$0 = g\bigl(\gamma(t)\bigr) = f(p) - f(-p),$$
which means that $f(p) = f(-p)$, which is what we wanted.\slug

The previous problem shows that if we assume air temperature to be a continuous function on the globe,
then there is some point on Earth that has the same temperature as its antipode (the opposite point
on the globe). In fact, using
a stronger topological
theorem called the {\it Borsuk--Ulam theorem}, one can show that the same thing holds for
{\it pairs} of continuous functions on the sphere. For instance, if we assume barometric pressure
to be continuous on the sphere as well, then there is a point on Earth with the same
air temperature {\it and} the same barometric pressure as its antipode.

\proclaim Problem \advthm. Prove that $\RR$ has the least upper bound property using the
localization of roots theorem (and without using anything else that we have shown to be equivalent
to the completeness axiom).

\proof Suppose, for a contradiction, that there is some $S\subseteq\RR$ that is nonempty and
does not have a least upper bound. We define a function $f:\RR\to \RR$ by
$$f(x) = \cases{1, & if $f$ is an upper bound of $S$;\cr -1, & if $f$ is not an upper bound of $S$.}$$
First we show that $f$ is continuous at every point $c\in \RR$. To do this, let $c\in \RR$
and let $\eps > 0$. There are two cases.

If $f(c) = 1$, then $c$ is an upper bound of $S$. But $S$
does not have a least upper bound, so there is some $b<c$ such that $b$ is also an upper bound of
$c$. In this case, set $\delta = c-b > 0$. For all $x$ with $|x-c|<\delta$, we have $x > b$, so $x$
is an upper bound of $S$ and $f(x) = 1$. We see that
$$\bigl| f(x) - f(c)\bigr| = |1-1| = 0<\eps.$$

If, on the other hand, $f(c) = -1$, then $c$ is not an upper bound of $S$, and there is some $s\in S$
with $s>c$. In this case, set $\delta = s-c > 0$. Then for all $x$ with $|x-c| <\delta$, we must
have $x < s$, so $x$ is not an upper bound of $S$ and $f(x) = 0$. Hence
$$\bigl| f(x) - f(c)\bigr| = \bigl|-1 - (-1)\bigr| = 0 < \eps.$$
We have shown that $f$ is continuous on all of $\RR$.

Now let $s\in S$ and let $t$ be any upper bound of $S$ (so we must have $s\le t)$.
The real number $s-1$ is not an upper bound of $S$, so $f(s-1) = -1$. On the other hand,
$f(t) = 1$. So by the location of roots theorem, there must be some $x\in [s-1, t]$ with
$f(x) = 0$. But this is a contradiction, since $f(x) \in \{-1,1\}$ for all $x\in \RR$.\slug

\proclaim Problem \advthm. Show that a set $f:\RR\to\RR$ is continuous on all of $\RR$
if and only if for every open set $U\subseteq \RR$, the inverse image $f^{-1}(U)$ is open.

\proof Suppose that $f$ is continuous and let $U$ be an open set. We want to show that
$f^{-1}(U)$ is open. Let $x\in f^{-1}(U)$, so that $f(x) \in U$. Since $U$ is open,
there exists some $\eps > 0$ such that $V_\eps\bigl( f(x)\bigr) \subseteq U$. But using this
$\eps$ in the definition of continuity of $f$, there must be some $\delta > 0$
such that for all $y\in V_\delta(x)$, $f(y)$ is in $V_\eps\bigl(f(x)\bigr)$, which is a subset of $U$.
But this implies that $V_\delta(x)\subseteq f^{-1} U$. Since $x$ was arbitrary, we have shown
that $f^{-1}(U)$ is open.

Now suppose that $f:\RR\to\RR$ is a function such that for every open set $U\subseteq \RR$ the set
$f^{-1}(U)$ is also open. Let $x\in \RR$ and $\eps > 0$. The set $U = V_\eps\bigl( f(x)\bigr)$ is open,
so by hypothesis its inverse image $f^{-1}(U)$ is open. Of course, $x$ is in $f^{-1} (U)$,
so there is some $\delta > 0$ such that $V_\delta(x) \subseteq f^{-1}(U)$. Unpacking the definitions
of $V_\delta(x)$ and $U = V_\eps\bigl(f(x)\bigr)$, we see that for this choice of $\delta$, we have
$\bigl| f(x) - f(z)\bigr| < \eps$ whenever $|x-z|<\delta$. Hence $f$ is continuous.\slug

This problem justifies the definition of continuity over $\RR$ that is used in the class, since
over general topological spaces a function is {\it defined} to be continuous if the inverse
image of every open set is open.

\proclaim Problem \advthm. Let $f:\RR\to \RR$ be a continuous function and let $p\in \RR$.
Show that the set
$$S = \bigl\{ x\in \RR : f(x) = p\bigr\}$$
is closed.

\proof The set $\{p\} = [p,p]$ is closed, so its complement $U = \RR\setminus\{p\}$ is open.
Note that $\RR\setminus S = f^{-1}(U)$, so by the previous problem, $\RR\setminus S$ is open.
Hence $S$ is closed.\slug


\bye
