\input fontmac
\input mathmac

\def\To{\Rightarrow}

\maketitle{Math 242 Tutorial 6}{prepared by}{Marcel Goh}{23 October 2025}


\bigskip

\proclaim Problem \advthm. Determine (with proof) the limits of
\medskip
\item{a)} $n^{1/n^2}$; and
\smallskip
\item{b)} $(n!)^{1/n^2}$
\medskip
\noindent as $n\to \infty$. You may use the fact that $\root n\of n \to 1$ as $n\to \infty$; this you
shall prove on your next assignment.

\proof Note that for all $n\in \NN$, $n\ge 1 = 1^{n^2}$, so we have $n^{1/n^2} \ge 1$.
On the other hand, since $n\le n^n$, taking $n^2$th roots we have $n^{1/n^2} \le \root n \of n$. But
$\root n\of n\to 1$, so by the squeeze theorem $n^{1/n^2}\to 1$ too, as $n\to\infty$.

For part (b), observe first that for all $n\in \NN$, $n! \ge 1\ge 1^{n^2}$, so $(n!)^{1/n^2}\ge 1$.
On the other hand, $n! = 1\cdot 2\cdot 3\cdots n \le n^n$, so taking $n^2$th roots again we have
$(n!)^{1/n^2} \le \root n \of n$. Hence
by the squeeze theorem, we see that $(n!)^{1/n^2} \to 1$.\slug

\proclaim Problem \advthm. Let $(a_n)$ be a sequence with $\lim_{n\to\infty} a_n = a$,
and let $(b_n)$ be a sequence with $b_n\ne 0$ for all $n\in \NN$ and $\lim_{n\to\infty} b_n = b\ne 0$.
\medskip
\item{a)} Prove that $\lim_{n\to\infty} 1/b^n = 1/b$.
\smallskip
\item{b)} Conclude that $\lim_{n\to\infty} (a_n/b_n) = a/b$.
\medskip

\proof Let $\eps > 0$. Since $b_n\to b\ne 0$, there exists $N_1\in \NN$ such that
$$|b-b_n|<{|b|^2\eps\over 2}$$
for all $n\ge N_1$. For the same reason, there exists $N_2\in \NN$ such that
$$|b-b_n| < {|b|\over 2}$$
for all $n\ge N_2$. This second bound implies that for all $n\ge N_2$
$$|b_n| > {|b|\over 2}.$$
So, letting $N = \max\{N_1, N_2\}$, we have
$$\biggl| {1\over b_n} - {1\over b}\biggr|
= {|b-b_n|\over |b_n|\cdot |b|} < {2\over |b|^2}\cdot {|b|^2\eps\over 2} = \eps.$$
We see, then, that $1/b_n\to 1/b$ as $n\to \infty$, since $\eps$ was arbitrary.

For part (b), simply let $c_n = 1/b_n$, with limit $c = 1/b$. By the product law for limits (shown in class),
$$\lim_{n\to\infty} {a_n\over b_n} = \lim_{n\to\infty} a_nc_n = ac = {a\over b},$$
and we are done.
\slug

\proclaim Problem \advthm. Let $(a_n)$ be a sequence, let $(b_n)$ be a null sequence, and let $L\in\RR$.
Show that if there exists $K\in \NN$ such that $|a_n - L|\le |b_n|$ for all $n\ge K$,
then $(a_n)$ converges to $L$.

\proof Let $\eps > 0$. Since $(b_n)$ is a null sequence, there is some $N_1\in \NN$
such that for all $n\ge N_1$ we have $|b_n| = |b_n - 0| < \eps$. Let $N = \max\{N_1, K\}$.
Then for all $n\ge N$, we have
$$|a_n - L|\le |b_n| <\eps,$$
so $a_n\to L$ as $n\to\infty$.\slug
\goodbreak

\proclaim Problem \advthm. Let $(x_n)$ be a sequence of nonnegative real numbers that is
{\it subadditive} in the sense that for all $m,n\in \NN$,
$$x_{m+n} \le x_m + x_n.$$
\medskip
\item{a)} Prove that $x_{qk} \le qx_k$ for all integers $k,q\in \NN$.
\smallskip
\item{b)} Explain why $\inf\{ x_n / n : n\in \NN\}$ exists.
\smallskip
\item{c)} Letting $L = \inf\{x_n/ n : n\in \NN\}$, show that $x_n/n \to L$ as $n\to\infty$.
\medskip

\proof For part (a), we perform induction on $q$. The base case $q=1$ is trivial. For $q\ge 1$,
suppose that $n=(q+1)k+r$. Then
$$x_{(q+1)k} = x_{qk + k} \le x_{qk} + x_k \le qx_k + x_k = (q+1)x_k,$$
where the first inequality is a consequence of subadditivity, and the second inequality follows
from the induction hypothesis.

For part (b), we simply note that $x_n\ge 0$ for all $n\in\NN$. Hence $x_n/n\ge 0$ for all $n\in \NN$,
and the infimum exists by completeness of the reals.

Part (c) is where the real work comes in. Let $\eps > 0$.
Since $L$ is the infimum of the set $\{x_n/n : n\in \NN\}$, there exists some $k\in \NN$ such that
$${x_k\over k} \le L+{\eps\over 2}.$$
Let $M = \max\{x_1,x_2,\ldots,x_{k-1}\}$, and pick $N= \max\{k, 2M/\eps\}$. Now let $n\ge N$ be given.
Since $n\ge k$, by the division algorithm we can write $n = qk+r$ for some integers $q\ge 1$ and
$r\in \{0,\ldots,k-1\}$. Then we have
$$x_n = x_{qk+r} \le x_{qk} + x_r \le qx_k + x_r,$$
by subadditivity and part (a). (Technically we didn't define $x_0$, but we can just say $x_0 = 0$,
and the above holds.) This implies that
$${x_n \over n} - {x_k\over k}
\le {qx_k + x_r\over n} - {x_k\over k}
= x_k \biggl({q\over kq+r}-{1\over k}\biggr) + {x_r\over n}.
$$
But since $r\in \{0,\ldots,k-1\}$, we have $x_r \le M$, by definition of $M$. This,
along with our assumption that $n\ge 2M/\eps$, implies that
$${x_r\over n} \le {M\over n} < {M\over 2M/\eps} = {\eps\over 2}.$$
Furthermore, we have
$${q\over qk+r} \le {q\over qk} = {1\over k},$$
so plugging these inequalities in above, we see that
$${x_n\over n} - {x_k\over k} < {\eps\over 2}.$$
Lastly, we bound
$$
\biggl| {x_n\over n} - L\biggr| = {x_n\over n} - L
= {x_n\over n} - {x_k\over k} + {x_k\over k} - L
< {\eps\over 2} + {\eps\over 2}
= \eps,$$
which completes the proof that $x_n/n\to L$ as $n\to\infty$.




\bye
